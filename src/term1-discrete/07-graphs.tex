\Subsection{Определения}
\begin{definition}
    Граф $G$ --- тройка  $(V, E, I)$:
     \begin{enumerate}
        \item $V$ --- конечное множество вершин.
        \item  $E$ --- конечное множество ребер.
        \item  $I\!: E \to \binom{V}{2}$.
    \end{enumerate}
\end{definition}
\begin{definition}
    Концевые вершины ребра --- вершины, которые соединены этим ребром.
\end{definition}
\begin{definition}
    Если два ребра имеют одинаковые концевые вершины, то такие ребра --- кратные (мультиребра).
\end{definition}
\begin{definition}
    Если ребро соединяет вершину с собой, то это ребро --- петля.
\end{definition}
\begin{definition}
    Граф простой --- без петель и мультиребер.
\end{definition}
\begin{definition}
    Степень ребра (валентность) --- количество ребер, исходящих из вершины.
\end{definition}
\begin{theorem}
    Сумма степеней вершин в графе равна удвоенному количеству ребер. То есть $\displaystyle \sum_{v \in V(G)} \deg v = 2 \cdot |E|$
\end{theorem}
\begin{proof}
    Каждое ребро состоит из двух полуребер. Из каждой вершины <<торчит>> $\deg v$ полуребер (принцип биекции). Тогда получили, что  $\sum \deg v = 2|E|$
\end{proof}
\begin{consequence}
    Количество вершин нечетной степени в графе четно.
\end{consequence}
\begin{definition}
    Ребро $e$ инцидентно вершине $u$, если  $u$ --- концевая вершина ребра. 
\end{definition}

\begin{definition}
	Матрица инцидентности --- таблица, где строчки соответствуют вершинам, а столбцы --- рёбрам, а на пересечении столбца и строки стоит 0, если эта вершина не инцидентна этому ребру, иначе то, сколько раз она ему инцидентна (1, если не петля, иначе --- 2).
\end{definition}

Можно заметить, что сумма всех чисел в каждом столбце --- два, а в каждой строке --- степень вершины. Из этого несложно в очередной раз заметить, что суммма удвоенного количества рёбер есть сумма степеней вершин.

\begin{definition}
    Полный граф --- полный простой граф на $n$ вершинах.  $K_n$. Граф, в котором каждая вершина соединена ребром с каждой.
\end{definition}
\begin{definition}
    Дополнением графа $G$ называется граф  $\overline{G}$:  $V(\overline{G}) = V(G)$, а $E(\overline{G}) = \binom{V(G)}{2} \setminus E(G)$ 
\end{definition}

\begin{definition}
    Граф называется двудольным $V(G) = V_1 \cup V_2, V_1 \cap V_2 = \emptyset$. И все ребра ведут из $V_1$ в  $V_2$.
\end{definition}
\begin{definition}
    Полный двудольный граф --- двудольный граф со всеми возможными ребрами. $K_{n, m}$, если в одной доле $n$ вершин, а в другой  $m$.
\end{definition}

\begin{definition}
	Граф <<$k$-мерный куб>> --- $Q_k$, такой граф, что $V$ --- множество бинарных строк длины  $k$.  $E: e = uv \iff$ $u$ и  $v$ отличаются в одном бите. 
\end{definition}
\begin{remark}
    Заметим, что данный граф двудольный: одна доля c четной суммой битов, другая --- с нечетной.
\end{remark}
\begin{definition}
    $P_n$ --- граф <<путь>>. Просто простой путь. Ничего лишнего.
\end{definition}
\begin{definition}
    $C_n$ --- граф <<цикл>>. Простой путь, замкнутый в кольцо.
\end{definition}
\begin{definition}
    Регулярный граф --- граф, в котором степень всех вершин равны. $R$-регулярный граф --- граф, в котором степени всех вершин равны $R$.
\end{definition}
\begin{definition}
    Оргаф (ориентированный граф) $D = (V, E, I)$, где $I: E \to V \times V$.
\end{definition}
\begin{theorem}
    $\displaystyle \sum_{v \in V(G)} \text{indeg}(v) = \sum_{v \in V(G)} = \text{outdeg}(v) = |E(D)|$
\end{theorem}
\begin{proof}
    Очев. Реально очев. Входящих концов у рёбер суммарно столько же, сколько и исходящих.
\end{proof}
\begin{definition}
    Не помню, было ли тут что-нибудь. Если вам кажется, что мы пропустили какое-то определение --- напишите пж.
\end{definition}
\begin{definition}
    Ориентацией графа $G$ называется граф $G$ полученный ориентацией всех ребер графа $G$.
\end{definition}
\begin{definition}
    Граф называется турниром, если он является ориентацией полного графа.
\end{definition}
\begin{definition}
    Две вершины называются смежными, если есть ребро между ними.
\end{definition}

\begin{definition}
    Матрица смежности --- матрица размера $V$ на $V$. $A_{i, j}$ показывает сколько ребер идет из $i$ в $j$.
\end{definition}
\begin{definition}
    Список смежности --- список списков, где для каждой вершины храним выходящие из нее ребра.
\end{definition}
\Subsection{Маршруты, пути, циклы. Связные графы}
\begin{definition}
    Маршрут (walk) --- набор вершин и ребер вида: $v_0, e_0, v_1, e_1, \ldots$, где $v_i$ --- вершины графа, $e_i$ --- ребра графа, причем $e_i = v_{i-1}v_i$
\end{definition}
\begin{definition}
    Путь (trail) --- маршрут без повторяющихся ребер.
\end{definition}
\begin{definition}
    Простой путь (path) --- Путь (сложный) без повтора вершин.
\end{definition}
\begin{definition}
    Вершины $x$ и  $y \in V$ называются связанными, если существует путь, соединяющий $x, y$. 
\end{definition}
\begin{remark}
    Заметим, что свзяность --- отношение эквивалентности. $x \to x$ --- очев, $x\to y = y \to x$. $x\to y, y\to z$, тогда $x\to z = x\to y \cup y\to z$.

    Тогда можно разбить на блоки --- компоненты связности.
\end{remark}
\begin{definition}
    Расстояние $d(x, y)$ --- длина кратчайшего пути из  $x$ в  $y$.
\end{definition}
\begin{definition}
    Диаметр графа --- это расстояние между двумя наиболее удаленными точками.
\end{definition}
\begin{definition}
	$x \in V(G)$ эксцентриситет  $\varepsilon(x) = \max_{y \in V(G)} d(x, y)$
\end{definition}
\begin{definition}
    Радиус $G\ : r(G) = \min_{x \in V(G)} \varepsilon(x)$
\end{definition}
\begin{definition}
    Замкнутый путь --- путь, которого стартовая вершина равна конечной.
\end{definition}
\begin{definition}
    Обхват --- минимальная длина цикла в графе. Если в графе циклов нет, то равен бесконечности.
\end{definition}
\begin{definition}[Теорема Кенига(Kőnig)]
    Граф двудольный $\iff$ в нем нет циклов нечетной длины.
\end{definition}
\begin{proof}
    \slashn
    \begin{itemize}
        \item $\Rightarrow$. Предположим, что есть цикл нечетной длины. Каждое ребро --- переход в другую долю. То есть в стартовую долю мы переходим через четное число ходов. Противоречие.
        \item 
    \end{itemize}
\end{proof}

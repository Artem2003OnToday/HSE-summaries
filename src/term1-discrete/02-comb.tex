\def\multiset#1#2{\ensuremath{\left(\kern-.3em\left(\genfrac{}{}{0pt}{}{#1}{#2}\right)\kern-.3em\right)}}
\Subsection{Сшки}
Есть два способа записи цэшек: $C_{n}^k = \binom{n}{k}= \frac{n!}{k! \cdot (n - k)!}$.
Обычно формулы в комбинаторике используются не для подсчетов, а для определения асимптотики/верней оценки и так далее. Например если взять $n = 100$, то уже проблема: $100!$ --- довольно большое число. Но там еще и деление!!! Короче, может получиться небольшое число при больших числах в подсчетах. 

Давайте забудем эту дурацкую формулу и будем использовать рекурренты: легко считать, пишется в миг. $\binom{n}{k} = \binom{n - 1}{k} + \binom{n}{k - 1}, \binom{0}{0} = 1$.
\begin{proof}
    Пусть есть множество из $n$ элементов.     Разобьем все $k$-элементные подмножества на блоки: в одном все без последнего элемента, в другом все с последним.Тогда в первом блоке тогда есть  $\binom{n - 1}{k}$ элементов. В другом $\binom{n - 1}{k - 1}$ элементов. А значит $\binom{n}{k} = \binom{n - 1}{k - 1} + \binom{n - 1}{k}$
\end{proof}
\slashn
Есть пара граничных случаев: $\binom{n}{0} = 1, \binom{n}{k} (n < k) = 0$.
После этого можно сделать треугольник Паскаля:

\begin{center}
\def\N{6}
\tikz[x=0.75cm,y=0.5cm, 
  pascal node/.style={font=\footnotesize}, 
  row node/.style={font=\footnotesize, anchor=west, shift=(180:1)}]
  \path  
    \foreach \n in {0,...,\N} { 
      %(-\N/2-1, -\n) node  [row node/.try]{Row \n:}
        \foreach \k in {0,...,\n}{
          (-\n/2+\k,-\n) node [pascal node/.try] {%
            \pgfkeys{/pgf/fpu}%
            \pgfmathparse{round(\n!/(\k!*(\n-\k)!))}%
            \pgfmathfloattoint{\pgfmathresult}%
             %$\binom{\n}{\k}=$%  
            \pgfmathresult%
        }}};
\end{center}

Рассмотрим решетчатую плоскость (если вы это читаете это и здесь нет картиночки напишите \texttt{@doktorkrab}, чтобы я добавил картиночку). Какое здесь количество путей? Ну $A{n}^k = A_{n-1}^k + A_{n-1}^{k-1}$. А это Сшки.

Теперь посмотрим на сумму на диагонали. Получаем гипотезу: $\sum{m=0}^n \binom{m}{k}=\binom{k}{k} + \binom{k+1}{k} + \ldots + \binom{n-1}{k} + \binom{n}{k} = \binom{n+1}{k+1}$.
\begin{proof}
    По основному комбинаторному тождеству: $\binom{m + 1}{k + 1} = \binom{m}{k + 1}  \binom{m}{k} \Rightarrow \binom{m}{k} = \binom{m + 1}{k + 1} - \binom{m}{k+1}$. Тогда: \[
        \sum_{m=k}^{n} \binom{m}{k} = \underbrace{\sum_{m=k}^n \binom{m+1}{k+1}}_{\binom{n+1}{k+1} + \sum_{m=k}^{n-1} \binom{m+1}{k+1} } - \underbrace{\sum_{m=k}^{n} \binom{m}{k+1}}_{\sum_{m=k+1}^{n} \binom{m}{k+1}}
    .\] 
    Дальше, если, расписать сумму все получится.

    Пусть хочу набрать $k+1$-элементное подмножество из  $n+1$-элементного множества. Пусть мы выбрали последний элемент, тогда у нас есть $\binom{n}{k}$ способов, а если не выбрали, то  $\binom{n}{k+1}$ способов. А по индукции  $\binom{n}{k+1} = \binom{n-1}{k+1} + \binom{n - 1}{k}$. И так далее.
\end{proof}
\slashn
Рассмотрим $\binom{n+m}{k} = \sum_{i=0}^k \binom{n}{i} \cdot \binom{m}{k-i}$
 \begin{proof}
     Рассмотрим два множества: одно $n$-элементное ("мальчики"), другое  $m$-элементное ("девушки"). Тогда пусть мы выбрали  $i$ мальчиков, тогда нам нужно выбрать $k-i$ девушек. 
\end{proof}
\slashn
Мы здесь применили принцип \texttt{double counting}: если мы посчитали что-то двумя способами, то результаты равны.

\Subsection{Биномиальные коэффициенты}
Подробности на втором курсе.

Рассмотрим бином Ньютона: $(x+y)^n = \sum_{k=0}^n \binom{n}{k} x^k \cdot y^{n-k}$
\begin{proof}
    Раскроем скобки в левой части: $(x+y)(x+y)(x+y)\ldots$. Когда у нас $x^k$? Когда мы ровно в  $k$ скобках выбрали  $x$. Сколько способов? Очевидно  $\binom{n}{k}$.
\end{proof}
Частные случаи:
\begin{itemize}
    \item $x=y=1$. Тогда  $2^n = \sum_{k=0}^n \binom{n}{k}$
    
        Рассмотрим множество $\{x_1,x_2,\ldots,x_n\}$. Каждому числу можно сопоставить 0/1 --- берем/не берем. Тогда количество подмножеств --- количество бинарных строчек длины $n$. Такой метод называется биективным: когда мы доказываем, что один объект является биекцией другого, то их количества равны.
    \item $x = 1, y = -1$. Тогда  $0 = \sum_{k=0}^n (-1)^k \binom{n}{k}$ --- количества способов выбрать подмножество четных длин и нечетных длин равны.
\end{itemize}
\Subsection{Мультимножество}
Хотим посчитать $\multiset{n}{k}$ --- количество  $k$-элементных подмультимножеств.

Пусть $X = [n]$. По принципу биекции найдем сначала $\multiset{n}{k}$ для  $X$, а потом найти для произвольного множества. 

Пусть есть множество  $A$, заменим его на множество  $\{ i + A_i\}$. $\multiset{n}{k} = \binom{n+k-1}{k}$

\Subsection{$k$-перестановки}
\begin{definition}
    Упорядоченные набор из $k$ элементов, где все элементы принадлежат множеству  $X$.
\end{definition}
Если мы считаем, что с повторениям, то ответ $n^k$, а если без то  $n \cdot (n-1) \cdot \ldots \cdot (n-k+1) = (n)_k$. Перестановку можно записать как: $\begin{pmatrix} 1 & 2 & 3 & \ldots & n-1 & n \\ a_1 & a_2 & a_3 & \ldots & a_{n-1} & a_n\end{pmatrix}$. То есть $i$ перешло в  $a_i$. После этого можно композировать перестановки:  $\begin{pmatrix} 1 & 2 & 3 \\ 3 & 2 & 1 \end{pmatrix} \begin{pmatrix} 1 & 2 & 3 \\ 3 & 1 & 2 \end{pmatrix} = \begin{pmatrix} 1 & 2 & 3 \\ 2 & 1 & 3 \end{pmatrix}$.

Заметим, что: 
\begin{enumerate}
    \item Существует нейтральный элемент --- тождественная перестановка $e = \{1,2,\ldots,n-1,n\}$
    \item Существует обратный элемент: $\sigma \sigma^{-1} = \sigma^{-1} \sigma = e$
    \item Ассоциативность: $\sigma \cdot (\tau \cdot \pi) = (\sigma \cdot \tau) \cdot \pi$
\end{enumerate}
\slashn
Значит перестановки с операцией композиции --- группа. Носит название $S_n$. Есть теорема о том, что любая конечная группа представима как подгруппа $S_n$.  

Рассмотрим $(n)_k = n \cdot (n-1) \cdot \ldots \cdot (n-k+1) = \binom{n}{k} \cdot k!$. Тогда $\binom{n}{k} = \frac{(n)_k}{k!}$. Тогда можно заменить  $n$ на  $q, q \in \mathbb{C}$. Тогда  \[
    \binom{q}{k} = \begin{cases} \frac{(q)_k}{k!} & k > 0 \\ 1 & k = 0 \\ 0 & k < 0 \end{cases}
.\]
Пусть $(n)^k = n \cdot (n+1) \cdot \ldots \cdot (n+k-1)$. Тогда $\multiset{n}{k} = \frac{(n)^k}{k!}$
\Subsection{Комбинаторика в схемах и мемах}
Пусть есть $n$ различных предметов. Нужно выбрать $k$ предметов с различными ограничениям: с повторениями/без, упорядоченные/неупорядоченные. 
\begin{center}
    \begin{tabular}{|c|c|c|}
        \hline
        & с повторениями & без повторений \\ \hline
        упорядоченные & $n^k$ &  $(n)_k$ \\ \hline
        неупорядоченные &  $\multiset{n}{k}$ &  $\binom{n}{k}$\\
        \hline
    \end{tabular}
\end{center}
Схема ящиков. 

\begin{center}
    \begin{tabular}{|c|c|c|c|c|}
        \hline
        & $\forall$ & $\le 1$ & $1$ &  $\ge 1$\\ \hline
        ящики+предметы различимы & $n^k$ & $(n)_k$ & $1 / n!$ & $\widehat{S}(n, k)$ \\ \hline
        ящики различимы, а предметы --- нет & $\multiset{n}{k}$ & $\binom{n}{k}$ & $1 / 0$ & $\multiset{n}{k-n}$\\ \hline
        ящики не различимы, а предметы различимы & & & & $S(n, k)$ \\ \hline
        ящики+предметы неразличимы & & & & \\ \hline
    \end{tabular}
\end{center}
Последнюю строчку мы не сможем заполнить на первом курсе, нужны производящие функции. Эта строчка решает множество задач, например, разложение числа на слагаемые.

Отображение $f: X \to Y$ --- такое правило, что $\forall x \in X$  $\exists! y \in Y:$  $y = f(x)$. Количество $k^n$ ($|X| = n, |Y| = k$)

 \begin{definition}
     Отображение --- тройка из $\left(x, y, \Gamma \subseteq X \times Y \right)$, причем каждый  $x_i$ встречается в $\Gamma$ ровно один раз. 
\end{definition}
 \begin{definition}
     Отображение называется иньективным, если $\forall x_1, x_2 \in X$ $f(x_1) \neq f(x_2) \Rightarrow x_1 = x_2$. Их количество --- $(k)_n$
 \end{definition}
 \begin{definition}
     Отображение называется биективным, если $\forall y \in Y$  $\exists! x \in X:$  $y = f(x)$. Количество --- $n!$.
 \end{definition}
 \begin{definition}
     Отображение называется сурьективным, если $\forall y \in Y$  $\exists x \in X:$  $y = f(x)$.
 \end{definition}
 
 Посчитаем количество сурьективных отображений. Пусть $\Im(f) = \{y \in Y \mid \exists x \in X: y = f(x)\}$. Тогда для любого отношения  $f: X \to \Im(f)$ --- сурьективно.

 Пусть $|\Im(f)| = i$, а количество сурьективных отображений ---  $\widehat{S}(n, i)$. Тогда  $\widehat{S}(n, i)\cdot\binom{k}{i}$ --- количество суьективных подмножеств мощности  $k$.

 Тогда  $k^n = \sum_{i=0}^k \binom{k}{i} \widehat{S}(n, i)$
 
 Пусть есть две числовые последовательности  $f_0, f_1,\ldots, f_k, \ldots$ и $g_0, g_1,\ldots, g_k, \ldots$. Причем $g_k = \sum_{i=0}^k \binom{k}{i} f_i$, тогда  $f_k = \sum_{i=0}^{k} (-1)^{k-1} \binom{k}{i}g_i$. Значит  $\widehat{S}(n, k) = \sum_{i=0}^k (-1)^{k-1} \binom{k}{i} \cdot i^n$ 

 Рассмотрим отображение  $\{\{x_1, x_2\}, \{x_3\}, \varnothing\}$. Получение разбиение на блоки. Предположим, что отображение сурьективно, значит получили разбиение $k$ предметов  $n$ ящиков. 

 Предположим, что в первый ящик нужно положить $a_1$ предмет, во второй ---  $a_2$,  и так далее. Тогда количество вариантов: $\sum \binom{n}{a_1}\binom{n-a_1}{a_2}\ldots$. Если взять $\sum_{a_i \ge 0, a_1+\ldots+a_k=n}\binom{n}{a_1}\binom{n-a_1}{a_2} = k^n = \frac{ n!}{a_1!a_2!\ldots a_k!}$. А если $\sum_{a_i > 0, a_1+\ldots+a_k=n}\binom{n}{a_1}\binom{n-a_1}{a_2} = \widehat{S}(n, k)$ 

 Хотим разбить на блоки вида $a_1$ предметов +  $a_2$ предметов +  $a_3$ предметов\ldots Тогда заметим, что это $\sum_{a_i\ge 0, \sum a_i = n} \binom{n}{a_1} \binom{n-a_1}{a_2} \ldots \binom{n-a_1-a_{k-1}}{a_k}$. Заметим, что суммарно это $k^n$, а если строго больше нуля, то  $\widehat{S}(n, k)$. Также можно раскрыть скобки и получить. $\frac{n!}{a_1!a_2!\ldots a_k!}$

 Рассмотрим $\binom{n}{k} = \frac{n!}{k! \cdot (n-k)!} = P(n; k; n - k)$. Комбинаторно они равны через битовые строки.

 Теперь посмотрим на  $\multiset{n}{k} = \frac{(n-1+k)!}{(n-1)! \cdot k!}$, через шары и перегородки.

 Вернемся к $k^n$ --- все отображения,  $\widehat{S}(n, k)$ --- все сюръективные отображение,  $S(n, k)$ --- количество разбиений  $n$-множества на  $k$-подмножества. (Числа Стирлинга второго рода).

 Заметим, что  $S(n, k) \cdot k! = \widehat{S}(n, k)$, так как  в  $S$ с крышечкой это про неупорядоченные.  $S(n, k) = \frac{1}{k!}\sum_{i=0}^k (-1)^{k-i} \binom{k}{i}i^n$. $S(0, 0) = 1$.  $\forall S(n,0) = 0$. $S(n, k) = S(n-1, k-1) + k\cdot S(n-1, k)$. Доказываем так: либо удаляем  ${x_n}$, либо пихаем  ${x_n}$ куда-то.

 \[
     k^n = \sum_{i=0}^n \binom{k}{i} \widehat{S}(n, i) = \sum_{i=0}^n \frac{k!}{i!(k-i!)} S(n, i)
.\] 
Откуда: \[
    x^n = \sum_{i=0}^n (x)_i \cdot S(n, i) \iff (x)_n = \sum_{i=0}^n x^i s(n, i).
.\] 
Где $s(n,i)$ --- числа Стирлинга первого рода.

Решим задачу, где мы хотим разбить $n$ различимых предметов в $k$ различимых ящиков  $B(n, k) = \sum_{i=0}^k S(n, i)$. Причем $B(n, n) = B_n$ --- числа Белла. Количество способов разбить $n$-множество на блоки. 

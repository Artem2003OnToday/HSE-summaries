\Subsection{Метрические и нормированные пространства}
\begin{definition}
    Метрика (расстояние) $\rho\!: X \times X \to [0;+\infty)$, если выполняются следующие условия:
     \begin{enumerate}
         \item $\rho(x, y) = 0 \iff x = y$,
         \item $\rho(x, y) = \rho(y, x)$,
         \item  (неравенство треугольника) $\rho(x, z) \le \rho(x, y) + \rho(y, z)$.
    \end{enumerate}
\end{definition}
\begin{definition}
    Метрическое пространство --- пара $(X, \rho)$.
\end{definition}
\begin{example}
    Дискретная метрика (метрика Лентяя) $\rho(x, y) = \begin{cases} 0, & x = y \\ 1 & x \neq y\end{cases}$
\end{example}
\begin{example}
    На $\R$:  $\rho(x, y) = |x-y|$.
\end{example}
\begin{example}
    На $\R^d$:  $\rho(x, y) = \sqrt{\sum\limits_{k=1}^d (x_k - y_k)^2}$. Неравенство треугольника здесь --- неравенство Минковского.
\end{example}
\begin{example}
    $C[a, b]$.  $\rho(f, g) = \int\limits_a^b |f-g|$.

    Неравенство треугольника:  \[
    \rho(f, h) = \int\limits_a^b |f-h| \le \int_a^b(|f-g|+|g-h|) = \rho(f, g) + \rho(g, h).
    .\] 
\end{example}
\begin{example}
    Манхэтеннская метрика: $\R^2$  $\rho((x_1, y_1), (x_2, y_2)) = |x_1 - x_2| + |y_1 + y_2|$.
\end{example}
\begin{example}
    Французская железнодорожная метрика. $\R^2$. Есть точка  $P$ (Париж), тогда  $\rho(A, B) = AB$, если  $A, B,P$ на одной прямой, иначе  $\rho(A, B) = |AP|+|PB|$. 
\end{example}
\begin{definition}
    $(X, \rho)$ --- метрическое пространство.  $B_r(x) \coloneqq \{y \in X \mid \rho(x, y) < r\}$ --- открытый шар радиуса  $r$ с центром в точке  $x$. 
\end{definition}
\begin{definition}
    $(X, \rho)$ --- метрическое пространство.  $\overline{B}_r(x) \coloneqq \{y \in X \mid \rho(x, y) \le r\}$ --- закрытый шар радиуса  $r$ с центром в точке  $x$. 
\end{definition}
\begin{properties}
    \begin{enumerate}
        \item $B_{r_1}(a) \cap B_(r_2)(a) = B_{\min\{r_1, r_2\}}(a)$.
        \item $x \neq y \implies \exists r > 0\!: B_r(x) \cap B_r(y) = \emptyset \land \overline{B}_r(x) \cap \overline{B}_r(y) = \emptyset$.
    \end{enumerate}
\end{properties}
\begin{proof}
    \begin{enumerate}
        \item $x \in B_{r_1}(a) \cap B_{r_2}(a) \iff \begin{cases} \rho(x, a) < r_1 \\ \rho(x, a) < r_2 \end{cases} \iff \rho(x, a) < \min\{r_1, r_2\} \implies x \in B_{\min\{r_1, r_2\}}(a)$.
        \item $r \coloneqq \frac{1}{3} \rho(x, y) > 0$. Пусть $\overline{B}_r(x) \cap \overline{B}_r(y) \neq \emptyset$. 

            Тогда  $\exists z \in \overline{B}_r(x) \cap \overline{B}_r(y) \implies \rho(x, z) \le r \land \rho(y, z) \le \rho \implies \rho(x, y) \le \rho(x, z) + \rho(z, y) \le 2r = \frac{2}{3} \rho(x, y)$.
    \end{enumerate}
\end{proof}
\begin{definition}
    $A \subset X$.  $A$ --- открытое множество, если  $\forall a \in A \exists B_r(a) \subset A$ ($r > 0$).
\end{definition}
\begin{theorem}[О свойствах открытых множеств]
    \begin{enumerate}
        \item $\emptyset, X$ --- открытые.
        \item Объединение любого числа открытых множеств --- открытое.
        \item Пересечение конечного числа открытых множеств --- открытое.
        \item $B_r(a)$ --- открытое.
    \end{enumerate}
\end{theorem}
\begin{proof}
    \begin{enumerate}
        \item[2.] $A_{\alpha}$ --- открытые,  $\alpha \in I$.  $B \eqqcolon \bigcup\limits_{\alpha \in I}A_{\alpha}$. Берем  $b \in B \implies b \in A_\beta$ для некоторого  $\beta$. Но  $A_\beta$ --- открытое  $\implies \exists r > 0\quad B_r(b) \subset A_\beta \subset B$.
        \item[3.] $A_1, A_2, \ldots, A_n$ --- открытые. $B \coloneqq \bigcap\limits_{k=1}^n A_k$. Берем  $b \in B \implies b \in A_k \forall k=1,2,\ldots,n$. Но $A_k$ --- открытое  $\exists r_k > 0 B_{r_k} \subset A_k$.  $\forall k \implies B_r(b) \subset \bigcap\limits_{k=1}^n A_k = B$.

             $r \coloneqq \min\{r_1, \ldots, r_n\} > 0 \implies B_r(b) \subset B_{r_k}(b) \subset A_k \quad \forall \implies B_r(b) \subset \bigcap\limits_{k=1}^n A_k = B$.
         \item[4.] Картинка :( $\rho(a, x) < R$,  $r \coloneqq R = \rho(a, x) > 0$. Докажем, что  $B_r(x) \subset B_R(a)$. Возьмем  $y \in B_r(x)$, то есть  $\rho(x, y) < r \implies \rho(y, a) \le \rho(y, x) + \rho(x, a) < r + \rho(x, a) = R \implies y \in B_R(a)$.
    \end{enumerate}
\end{proof}
\begin{remark}
    Существенна конечность. $\R$.  $\bigcap\limits_{n=1}^{\infty}(-\frac{1}{n}, 1) = [0, 1)$.
\end{remark}
\begin{definition}
    $A \subset X$,  $a \in A$.  $a$ --- внутренняя точка множества  $A$, если $\exists r > 0\!: B_r(a) \subset A$.
\end{definition}
\begin{remark}
    $A$ --- открытое  $\iff$ все его точки внутренние.
\end{remark}
\begin{definition}
    Внутренность множества $\Int a \coloneqq \{ a \in A\mid a\text{ --- внутренняя точка}\}$.
\end{definition}
\begin{example}
    $A = [0, 1] \subset \R$. Тогда  $\Int A = (0, 1)$.
\end{example}
\begin{properties}[внутренности]
    \begin{enumerate}
        \item $\Int A \subset A$.
        \item  $\Int A$ ---  $\bigcup$ всех открытых множеств, которые содержатся в  $A$.
        \item $\Int A$ --- открытое множество. 
        \item  $A$ ---  открытое $\iff A = \Int A$.
        \item Если $A \subset B$, то $\Int A \subset \Int B$.
        \item $\Int(A \cap B) = \Int A \cap \Int B$
        \item $\Int(\Int A) = \Int A$.
    \end{enumerate}
\end{properties}
\begin{proof}
    $B \coloneqq \bigcup_{\alpha \in I} A_{\alpha}, A_\alpha \subset A$ открытые. 

     $B \subset \Int A$. Берем  $b \in B \implies \exists \beta \in I\!: B \in A_\beta$ --- открытое  $\implies \exists r > 0\!: B_r(b) \subset A_\beta \subset A \implies b$ --- внутренняя точка  $A$  $\implies b \in \Int A$.

      $\Int A \subset B$. Берем  $b \in \Int A \implies \exists r > 0 B_r(b) \subset A$, но  $B_r(b)$ --- открытое множество $\implies $ оно участвует в объединении  $\bigcup\limits_\alpha A_\alpha \implies B_r(b) \subset B \implies b \in B$.

      Докажем пункт 4. $\Rightarrow$: пункт 3.  $\Leftarrow$ всего его точки внутренние  $\implies A = \Int A$.
      Пункт 6. $\subset$:  $A \cap B \subset A, \subset B \implies \Int(A \cap B) \subset \Int A \land \Int(A \cap B) \subset \Int B$.

       $\supset$. Пусть $x \in \Int A \cap \Int B \implies \begin{cases} \exists r_1 > 0 \quad B_{r_1}(x) \subset A \\ \exists r_2 > 0 \quad B_{r_2}(x) \subset B \end{cases} \implies$ если $r = \min \{r_1, r_2 \} \implies B_r(x) \subset A \land B_r(x) \subset B \implies B_r(x) \subset A \cap B \implies x \in \Int(A \cap B)$.

       Пункт 7. $B \coloneqq \Int A$ --- открытое $\implies B = \int B$.
\end{proof}
\begin{definition}
    $A \subset X$.  $A$ --- замкнутое, если  $X \setminus A$ --- открытое.
\end{definition}
\begin{theorem}[о свойствах замкнутых множеств]
    \begin{enumerate}
        \item $\emptyset, X$ --- замкнуты.
        \item Пересечение любого числа замкнутых множеств --- замкнуто. 
        \item Объединение конечного числа замкнутых множеств --- замкнуто.
        \item  $\overline{B}_R(a)$ --- замкнуто.
    \end{enumerate}
\end{theorem}
\begin{proof}
    \begin{enumerate}
        \item[2.] $A_\alpha$ --- замкнуты  $\implies X \setminus A_\alpha$ --- открытые  $\implies \bigcup\limits_{\alpha \in I} X \setminus A_\alpha$ --- открыто  $\implies X \setminus \setminus \bigcup\limits_{\alpha \in I} (X \setminus A_{\alpha}) = \bigcap\limits_{\alpha \in I} A_\alpha$ --- замкнутое.
        \item[4.] $X \in \overline{B}_r(a)$ --- открытое. Берем  $x \notin \overline{B}_R(a)$. Возьмем $r \coloneqq \rho(a, x) - R > 0$. Покажем, что  $B_r(x) \subset x \setminus \overline{B_R}(a)$.

            От противного. Пусть $B_r(x) \cap \overline{B}_R(a) \neq \emptyset$. Берем  $y \in B_r(x) \cap \overline{B}_R(a) \implies \rho(x, y) < r \land \rho(a, y) \le R \implies \rho(a, x) \le \rho(a, y) + \rho(y, x) < R + r = \rho(a, x)$. Противоречие.
    \end{enumerate}
\end{proof}
\begin{remark}
    В 3 важна конечность. $\R$.  $\bigcup\limits_{n=1}^{\infty} [\frac{1}{n}, 1] = (0, 1]$ --- не является замкнутой.
\end{remark}
\begin{definition}
    Замыкание множества $\Cl A$ --- пересечение всех замкнутых множеств, содержащих  $A$.
\end{definition}
\begin{theorem}
    $X \setminus \Cl A = \Int(X \setminus A)$ и  $X \setminus \Int A = \Cl(X \setminus A)$.
\end{theorem}
\begin{proof}
    $\Int(X \setminus A) = \bigcup B_{\alpha}$.  $B_\alpha$ --- открытые,  $B_\alpha \subset X \setminus A \iff X \setminus B_\alpha$ --- замкнутое. $X \setminus B_\alpha \supset A$.

    $\bigcap(X \setminus B_\alpha) = \Cl A \implies X \setminus \bigcap (X \setminus B_\alpha) = X \setminus \Cl A \iff \bigcup(B_\alpha) = \Int(X \setminus A)$.
\end{proof}
\begin{consequence}
    $\Int A = X \setminus Cl(x \setminus A)$ и  $\Cl A = X \setminus \Int(X \setminus A)$.
\end{consequence}
\begin{properties}
    \begin{enumerate}
        \item $\Cl A \supset A$.
        \item  $\Cl A$ --- замкнутое множество. 
        \item $A$ --- замкнуто  $\iff A = \Cl A$.
            \begin{proof}
                $\Leftarrow$ --- пункт 2.  $\Rightarrow A$ --- замкнутое  $\Rightarrow$ оно участвует в пересечении из определения  $\implies \Cl A \subset A \implies \Cl A = A$.
            \end{proof}
        \item $A \subset B \implies \Cl A \subset \Cl B$.
             \begin{proof}
                $X \setminus A \supset X \setminus B \implies \Int(X \setminus A) \supset \Int(C \setminus B) \implies X \setminus \Int(X \setminus A) \subset X \setminus \int(X \setminus B)$
            \end{proof}
        \item $\Cl(A \cup B) = \Cl A \cup B$.
        \item  $\Cl(\Cl A) = \Cl A$.
             \begin{proof}
                $B \coloneqq \Cl A$ --- замкнуто  $\implies \Cl B = B$.
            \end{proof}
    \end{enumerate}
\end{properties}
\begin{exerc}
    $\Cl \Int \Cl \Int \ldots A$. Какое наибольшее количество различных множеств может получиться.
\end{exerc}
\begin{theorem}
    $x \in \Cl A \iff \forall r > 0\quad B_r(x) \cap A \neq \emptyset$.
\end{theorem}
\begin{proof}
    $x \notin \Cl A \iff \exists r > 0 B_r(x) \cap A = \emptyset$. Что означает, что  $x \notin A$? Это значит, что  $x\in X \setminus \Cl A = \Int(X \setminus A) \iff x \in \Int(X \setminus A) \iff x\text{ --- внутренняя точка }X \setminus A \iff \exists r > 0\!: B_r(x) \cap A = \emptyset \iff \exists r > 0\!: B_r(x) \subset X \setminus A$.
\end{proof}
\begin{consequence}
    $U$ --- открытое,  $U \cap A = \emptyset \implies U \cap \Cl A = \emptyset$.
\end{consequence}
\begin{proof}
    Возьмем $x \in U \implies \exists r > 0\!: B_r(x) \subset U \implies B_r(x) \cap A = \emptyset \implies x \notin \Cl A \implies U \cap \Cl A = \emptyset$.
\end{proof}
\begin{definition}
    Окрестностью точки $x$ будем называть шар  $B_r(x)$ для некоторого  $r > 0$. Обозначать будем $U_x$
\end{definition}
\begin{definition}
    Проколотой окрестностью точки $x$ ---  $B_r(x) \setminus \{x\}$. $\dot{U}_x$.
\end{definition}

\begin{definition}
    $x$ --- предельная точка множества  $A$, если  $\forall \dot{U_x}\!: \dot{U_x} \cap A \neq \emptyset$.

    Обозначим через  $A'$ --- множество предельных точек для  $A$.
\end{definition}
\begin{properties}
    \slashn
    \begin{enumerate}
        \item $\Cl A = A \cup A'$.
            \begin{proof}
                $x \in \Cl A \iff \forall U_x \cap A \neq \emptyset \iff \left[ \begin{array}{l} x \in A \\ \forall \dot{U_x} \cap A \neq \emptyset \iff x \in A' \end{array} \right.$
            \end{proof}
        \item $A \subset B \implies A' \subset B'$. Очевидно.
        \item  $A$ --- замкнуто  $\iff A \supset A'$. 
             \begin{proof}
                $A$ --- замкнуто  $\iff = \Cl A \iff A = A \cup A' \iff A \supset A'$.
            \end{proof}
        \item $(A \cup B)' = A' \cup B'$.
             \begin{proof}
                Докажем "$\subset$". Возьмем $x \in (A \cup B)'\!: x \notin A' \implies \exists \dot{U_x}\!: \dot{U_x} \cap A = \emptyset$, но $\dot{U_x} \cap (A \cup B) \neq \emptyset \implies \dot{U_x} \cap B \neq \emptyset \implies x \in B'$.

                Докажем "$\supset$". $A \cup B \supset A \implies (A \cup B)' \supset A'$. Провернем тот же фокус для  $B$, получим  $(A \cup B)' \supset A' \cup B'$.
           \end{proof}
    \end{enumerate}
\end{properties}
\begin{theorem}
    $x \in A' \iff \forall r > 0$  $B_r(x)$ содержит бесконечное количество точек из  $A$.
\end{theorem}
\begin{proof}
    Докажем "$\Leftarrow$". $B_r(x) \cap A$ содержит бесконечное количество точек  $\implies \dot{B_r}(x) \cap A$ содержит бесконечное число точек  $\implies \dot{B_r}(x) \cap A \neq \emptyset \Rightarrow x \in A'$.

     "$\Rightarrow$". Возьмем радиус  $r$. Тогда  $\dot{B_r}(x) \cap A \neq \emptyset \implies \exists x_1 \in A\!: 0 < \rho(x, x_1) < r$. Возьмем $r = \rho(x, x_1)$ $\dot{B_r}(x) \cap A \neq \emptyset \implies \exists x_2 \in A\!: 0 < \rho(x, x_2) < \rho(x, x_1)$. Тогда можно взять $r = \rho(x, x_2)$, и так далее. 

     В итоге получили, что $r > \rho(x, x_1) > \rho(x, x_2) > \rho(x, x_3) > \ldots > 0 \implies$ все $x_n$ различны.
\end{proof}
\begin{consequence}
     Конечно множество не имеет предельных точек.
\end{consequence}
\begin{proof}
     Предположим конечная точка существует $\iff \exists r > 0\!: B_r(x) \cap A$ содержит бесконечное количество точек. Но это невозможно, так как в $A$ конечное число точек. 
\end{proof}

\begin{definition}
     $(X, \rho)$ --- метрическое пространство  $Y \subset X$.

     Тогда  $(Y, \rho \mid_{Y \times Y})$ --- подпространство метрического пространства  $(X, \rho)$.
\end{definition}
\begin{example}
     $(\R, |x-y|)$.  $Y=[a, b] \subset \R$.

      $B_1(1) = (0, 1], B_2(0) = [0, 1]$.  $B_r^Y(a) = Y \cap B_r^X(a)$.
\end{example}
\begin{theorem}[об открытых и замкнутых множества в пространстве и подпространстве]
     $(X, \rho)$ --- метрическое пространство,  $(Y, \rho)$ --- его подпространство,  $A \subset Y$. Тогда
      \begin{enumerate}
          \item $A$ --- открыто в  $Y \iff \exists G$ --- открытое в  $X\!: A = G \cap Y$. 
          \item $A$ --- замкнуто в  $Y \iff \exists F$ --- замкнутое в  $X\!: A = F \cap Y$.
     \end{enumerate}
\end{theorem}
\begin{proof}
     \slashn
     \begin{enumerate}
         \item "$\Rightarrow$" $A$ --- открыто в  $Y \implies \forall x \in A \exists r_x > 0 \!: B_{r_x}^Y(x) \subset A \implies A = \bigcup\limits_{x \in A}B_{r_x}^Y(x)$.

             То есть наше множество будет объединением большего числа шариков (возможно бесконечного). Найдем теперь  $G$:  $G \coloneqq \bigcup\limits_{x \in A} B_{r_x}^X(x)$ --- открыто. Посмотрим теперь на  $G \cap Y = \bigcup\limits_{x \in A}(B_{r_x}^X(x) \cap Y) = \bigcup\limits_{x \in A}B_{r_x}^Y(x) = A$.

         В обратную сторону. Пусть $A = G \cap Y$, где  $G$ открыто в  $X$. Возьмем  $x \in G \cap Y$.  $G$ --- открыто в  $X \implies \forall x \in G \cap Y \exists r > 0\!: B_r^X(x) \subset G \implies B_r^X(x) \cap Y \subset G \cap Y = A \implies B_r^Y(x) \subset A \implies x$ --- внутренняя точка $A \implies A$ --- открыто в  $Y$. 

         \item $A$ --- замкнуто в $Y \iff Y \setminus A$ --- открыто в  $Y \iff \exists G$ --- открытое в  $X$, такое что  $Y \setminus A = Y \cap G \iff A = Y \setminus (Y \cap G) = Y \cap (X \setminus G) \iff \exists G$ --- открытое в  $X$, такое что  $A = Y \cap (X \setminus G) \iff \exists F$ --- замкнуто в  $X$, такое что  $A = Y \cap F$.
     \end{enumerate}
\end{proof}
\begin{example}
     $(\R, |x-y|)$.  $Y = [0, 3)$.  $[0, 1)$ --- открыто в  $[0, 3)$:  $[0, 1) = [0, 3) \cap (-1, -1)$.  $[2, 3)$ --- замкнуто в  $[0, 3)$:  $[2, 3) = [0, 3) \cap [2, 3]$.
\end{example}
 
\begin{definition}
     $X$ --- векторное пространство над  $\R$.

      $||.||\!: X \to \R$ --- норма, если
       \begin{enumerate}
           \item $||x|| \ge 0\quad \forall x \in X$ и $||x|| = 0 \iff x = \overrightarrow{0}$.
           \item  $||\lambda x|| = |\lambda| \cdot ||x||\quad \forall x \in X\ \forall \lambda \in \R$. 
           \item (неравенство треугольник)
      \end{enumerate}
\end{definition}
\begin{example}
     \begin{enumerate}
         \item $|x| \in \R$,
         \item  $||x||_1 = |x_1| + |x_2| + \ldots + |x_d|$ в $\R^d$.
         \item  $||x||_{\infty} = \max\limits_{k=1,2,\ldots, d} |x_k|$. $||x+y||_{\infty} = \max\{|x_k|+|y_k|\} \le \max\{|x_k|\} + \max\{|y_k|\} = ||x||_{\infty} + ||y||_{\infty}$
         \item $||x||_2 = \sqrt{x_1^2 + x_2^2 + \ldots + x_n^2}$.
         \item $||x||_p = \left(\sum\limits_{k=1}^d |x_k|^p\right)^{\frac{1}{p}}$ в $\R^d$ при  $p \ge 1$. Неравенство треугольника --- неравенство Минковского.
         \item $C[a, b]$.  $||f|| = \max\limits_{t \in [a, b]} |f(t)|$. 
     \end{enumerate}
\end{example}
\begin{definition}
    $X$ векторное пространство над  $\R$.  $\langle .,.\rangle\!: X \times X \to \R$ скалярное произведение, если
     \begin{enumerate}
         \item $\langle x, x \rangle \ge 0$ и $\langle x, x \rangle = 0 \iff x = \overrightarrow{0}$.
         \item  $\langle x+y, z\rangle = \langle x, z \rangle + \langle y, z \rangle$
         \item  $\langle x, y \rangle = \langle y, x \rangle$.
         \item  $\langle \lambda x, y \rangle = \lambda \langle x, y \rangle \quad \lambda \in \R$.
    \end{enumerate}
\end{definition}
\begin{example}
    \begin{enumerate}
        \item $\R^d$.  $\langle x, y\rangle = \sum x_iy_i$.
        \item Возьмем $w_1, \ldots, w_d > 0$. Тогда $\langle x, y \rangle = \sum w_i x_i y_i$.
        \item $C[a, b]$.  $\langle f, g \rangle = \int\limits_a^b f(x)g(x) \mathrm{d}x$.
    \end{enumerate}
\end{example}
\begin{properties}
    \begin{enumerate}
        \item Неравенство Коши-Буняковского. $\langle x, y \rangle^2 \le \langle x, x \rangle \cdot \langle y, y\rangle$.
            \begin{proof}
                $f(t) \coloneqq \langle x+ty, x +ty \rangle \ge 0$. $f(t) = \langle x, x \rangle> + t\langle x, y \rangle + t\langle x, y \rangle + t^2 \langle y, y \rangle = t^2 \langle y, y\rangle + 2t\langle x, y \rangle + \langle x, y \rangle$ --- квадратный трехчлен (если $\langle y, y \rangle = 0 \implies y = 0 \implies$ везде нули). Тогда $0 \ge D= (\langle x, y \rangle)^2 - 4 \langle x, x\rangle \cdot \langle y, y \rangle = 4(\langle x, y \rangle^2 - \langle x, x\rangle \cdot \langle y, y \rangle)$. Потому что иначе есть значения меньше нуля.
            \end{proof}
        \item $||x|| \coloneqq \sqrt{\langle x, x \rangle}$ --- норма.
             \begin{proof}
                $||\lambda x|| = \sqrt{\langle \lambda x, \lambda x\rangle} = \sqrt{\lambda^2\langle x, x \rangle} = |\lambda| \sqrt{\langle x, \rangle} = |\lambda| \cdot ||x||$.

                Неравенство треугольника: $\lVert x+y \rVert \le \lVert x \rVert + \lVert y \rVert$. Возведем в квадрат\ldots 

                Я проиграл.
            \end{proof}
        \item $\rho(x, y) = \lVert x - y \rVert$ --- метрика.
            \begin{proof}
                $\rho(x, y) \ge 0$. $\rho(x, y) = 0 \iff \lVert x - y \rVert = 0 \iff x - y = \overrightarrow{0} \iff x = y$.

                 $\rho(y, x) = \lVert y-x \rVert = \lVert (-1)(x-y) \rVert = |-1| \lVert x - y \rVert = \rho(x, y)$.

                  $\rho(x, z) \le \rho(x, y) + \rho(y, z)$: $\lVert (x-y) + (y-z) \rVert = \lVert x-z\rVert \le \lVert x - y \rVert + \lVert y-z \rVert$.
            \end{proof}
        \item $\lVert x - y \rVert \ge |\lVert x \rVert - \lVert y \rVert |$.
            \begin{proof}
                Надо доказать, что $-\lVert x - y \rVert \le \lVert x \rVert - \lVert y \rVert \le \lVert x - y \rVert$.

                $\lVert (y-x) + x \rVert = \lVert y \rVert \le \lVert x \rVert + \lVert x-y \rVert = \lVert x \rVert + \lVert y -x \rVert$.
            \end{proof}
        \item Упражненение. Если норма порождается скалярным произведением $\iff \lVert x+y\rVert^2 + \lVert x-y\rVert^2 = 2\lVert x\rVert^2 + 2\lVert y \rVert^2$. Тождество параллелограмма.
    \end{enumerate}
\end{properties}
\begin{definition}
    $(X, \rho)$ --- метрическое пространство.  $x_1, x_2, \ldots \in X, a \in X$.

    $\lim x_n = a$, если
     \begin{enumerate}
         \item Вне любого открытого шара с центром в точке  $a$ содержится лишь конечное число членов последовательности.
         \item  $\forall \eps > 0 \exists N \forall n \ge N\quad \rho(x_n, a) < \eps \iff x_n \in B_\eps(a)$.
    \end{enumerate}
\end{definition}
\begin{definition}
    $A \subset X$. 

    Тогда  $A$ --- ограничено, если оно содержится в некотором шаре.
\end{definition}
\begin{properties}
    \begin{enumerate}
        \item $a = \lim x \iff \rho(x_n, a) \to 0$.
\begin{proof}
             $\forall \eps > 0 \exists n > N\quad |\rho(x_n, a)| < \eps$ --- предел равен 0.
\end{proof}
        \item Предел единственный. 
            \begin{proof}
                Пусть $\exists a = \lim x_n = b$. Тогда мы, что  $B_r(a) \cap B_r(b) = \emptyset \implies \exists N_1, N_2, \forall n \ge \max\{N_1, N_2\} x_n \in B_r(a) \land x_n \in B_r(b)$.
            \end{proof}
        \item Если $a = \lim x_n, a = \lim y_n$. То для перемешанной последовательности $x_n$ и  $y_n$ предел такой же.
        \item  $a = \lim x_n \implies $ для последовательности, в которой $x_n$ взяты с конечной кратностью, то  $a$ будет пределом. 
        \item Если $a = \lim x_n$, то  $\lim x_{n_k} = a$.
        \item Последовательность имеет предел $\implies$ она ограничена
             \begin{proof}
                $\eps = 1 \exists N \forall n \ge N \rho(x_n, a) < 1$. Тогда $R = \max\{\rho(x_1, a), \ldots, \rho(x_{N-1}, a)\} + 1 \implies x_n \in B_R(a)$.
            \end{proof}
        \item Если $a = \lim x_n$, то последовательность, полученная из  $\{x_n\}$ перестановкой членов имеет тот же предел.
        \item $a$ --- предельная точка  $A \iff \exists a \neq \{x_n\} \in A\!: \lim x_n = a$.

            Более того,  $x_n$ можно выбирать так, что  $\rho(x_n, a)$ строго убывает.
            \begin{proof}
                "$\Rightarrow$" Пусть  $\lim x_n = a$. Возьмем  $B_r(a) \implies \exists N \forall n \ge N x_n \in B_r(a) \implies x_n \dot{B_r(a)} \implies \dot{B_r}(a) \cap A \neq \emptyset \implies$ a --- предельная точка.

                "$\Leftarrow$" Берем  $r_1 = 1$. $\dot{B_{r_1}}(a) \cap A \neq \emptyset$. Берем оттуда точку, называем  $x_1 \neq a$. $r_2 = \frac{\rho(x_1,a)}{2}$. $\dot{B_{r_2}}(a) \cap A \neq \emptyset$. Берем оттуда точку $x_3 \neq a$. $r_3 = \frac{\rho(x_2, a)}{2}$. И так далее.

                Получили: $x_n \neq a$ и  $\rho(x_n, a) < \frac{\rho(x_{n-1}, a)}{2} < \rho(x_{n-1}, a)$. $\rho(x_n, a) < \frac{1}{2^n} \to 0 \implies x_n = a$.
            \end{proof}
    \end{enumerate}
\end{properties}
\begin{theorem}[об арифметических действиях с пределами]
    $X$ --- нормированное пространство,  $x_n, y_n \in X$,  $\lambda_n \in \R$.  $\lim x_n = a, \lim y_n = b, \lim \lambda_n = \mu$. Тогда:
     \begin{enumerate}
         \item $\lim (x_n + y_n) = a+b$.
         \item  $\lim(x_n - y_n) = a-b$.
         \item  $\lim \lambda_nx_n = \mu a$.
         \item  $\lim \lVert x_n\rVert = \lVert a \rVert$.
         \item  Если в  $X$ есть скалярное произведение, то  $\lim \langle x_n, t_n \rangle = \langle a, b \rangle$.
    \end{enumerate}
\end{theorem}
\begin{proof}
    \begin{enumerate}
        \item $\rho(x_n+y_n, a+b) = \lVert (x_n+y_n - (a+b)) \rVert = \lVert (x_n-a) + (y_n-b) \rVert \le \lVert x_n - a \rVert + \lVert y_n - n \rVert = \rho(x_n, a) + \rho(y_n, b) \to 0$.
        \item Аналогично.
        \item $\rho(\lambda_nx_n, \mu a) = \lVert \lambda_n x_n - \mu a\rVert = \lVert \lambda_n x_n - \lambda_n a + \lambda_n a - \mu a \rVert \le \lVert \lambda_n x_n - \lambda_n a \rVert + \lVert \lambda_n a - \mu a \rVert = |\lambda_n| \lVert x_n - a \rVert + |\lambda_n -\mu| \lVert a \rVert \to 0$, так как $|\lambda_n|$ --- ограниченная, $\lVert x_n - a \rVert = \rho(x_n - a) \to 0$,  $|\lambda_n -\mu| \to 0$, $\lVert a \rVert$ --- константа.  
        \item $| \lVert x_n \rVert - \lVert a \rVert| \le \lVert x_n - a \rVert = \rho(x_n, a) \to 0 \implies \lim \lVert x_n \rVert = \lVert a \rVert$
        \item $\langle x, y \rangle = \frac{1}{4}(\lVert x+y \rVert^2 - \lVert x-y \rVert^2) = \frac{1}{4}(\langle x+y, x+y\rangle - \langle x-y, x-y\rangle)$. Тогда получаем $4 \langle x_n, y_n \rangle = \lVert x_n + y_n \rVert^2 - \lVert x_n - y_n \rVert^2 \to \lVert a + b \rVert^2 - \lVert a - b \rVert^2 = 4 \langle a, b \rangle$.
    \end{enumerate}
\end{proof}
\begin{definition}
    $\R^d$ --- пространство с нормой  $\sqrt{x_1^2 + x_2^2 + \ldots + x_d^2}$.
\end{definition}
\begin{definition}
    Покоординатная сходимость в $R^d$:

    $x_n \in \R^d$.  $x_n = (x_n^{(1)}, x_n^{(2)}, \ldots, x_n^{(d)}) \xrightarrow{\text{покоординатно}} a = (a^{(1)}, a^{()2}, \ldots)$.
\end{definition}
\begin{theorem}
    в $\R^d$ сходимость по метрике и покоординатная сходимость совпадает.
\end{theorem}
\begin{proof}
    Метрика $\implies$ покоординатная.  $\rho(x_n, a) \to 0 \implies 0 \le (x_n^{(1)} - a^{(1)})^2 + \ldots + (x_n^{(d)} - a^{(d)}) = \rho(x_n, a)^2 \to 0 \implies \lim (x_n^{(k)} - a^{(k)})^2 = 0 \implies \lim x_n^{(k)} = a^{(k)} \implies$ покоординатная сходимость.

    Покоординатная $\implies$ метрика. Пусть  $|x_n^{(k)} - a^{(k)}| \to 0 \quad \forall k \implies (x_n^{(k)} - a^{(k)})^2 \to 0 \implies \sum\limits_{k=1}^d (x_n^{(k)} - a^{(k)})^2 \to 0$. А так как $(\ldots)^2 = \rho(x_n, a)^2 \implies \rho(x_n, a) \to 0$.
\end{proof}

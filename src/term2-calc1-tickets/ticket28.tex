%BEGIN TICKET 28
\begin{definition}
    $A \subset X$.  $A$ --- замкнутое, если  $X \setminus A$ --- открытое.
\end{definition}
\begin{theorem}[о свойствах замкнутых множеств]
    \begin{enumerate}
        \item $\emptyset, X$ --- замкнуты.
        \item Пересечение любого числа замкнутых множеств --- замкнуто. 
        \item Объединение конечного числа замкнутых множеств --- замкнуто.
        \item  $\overline{B}_R(a)$ --- замкнуто.
    \end{enumerate}
\end{theorem}
\begin{proof}
    \begin{enumerate}
        \item[2.] $A_\alpha$ --- замкнуты  $\implies X \setminus A_\alpha$ --- открытые  $\implies \bigcup\limits_{\alpha \in I} X \setminus A_\alpha$ --- открыто  $\implies X \setminus  \bigcup\limits_{\alpha \in I} (X \setminus A_{\alpha}) = \bigcap\limits_{\alpha \in I} A_\alpha$ --- замкнутое.
        \item[4.] $X \setminus \overline{B}_R(a)$ --- открытое. Берем  $x \notin \overline{B}_R(a)$. Возьмем $r \coloneqq \rho(a, x) - R > 0$. Покажем, что  $B_r(x) \subset X \setminus \overline{B_R}(a)$.

            От противного. Пусть $B_r(x) \cap \overline{B}_R(a) \neq \emptyset$. Берем  $y \in B_r(x) \cap \overline{B}_R(a) \implies \rho(x, y) < r \land \rho(a, y) \le R \implies \rho(a, x) \le \rho(a, y) + \rho(y, x) < R + r = \rho(a, x)$. Противоречие.
    \end{enumerate}
\end{proof}
\begin{remark}
    В 3 важна конечность. $\R$.  $\bigcup\limits_{n=1}^{\infty} [\frac{1}{n}, 1] = (0, 1]$ --- не является замкнутой.
\end{remark}
\begin{definition}
    Замыкание множества $\Cl A$ --- пересечение всех замкнутых множеств, содержащих  $A$.
\end{definition}
\begin{theorem}
    $X \setminus \Cl A = \Int(X \setminus A)$ и  $X \setminus \Int A = \Cl(X \setminus A)$.
\end{theorem}
\begin{proof}
    $\Int(X \setminus A) = \bigcup B_{\alpha}$.  $B_\alpha$ --- открытые,  $B_\alpha \subset X \setminus A \iff X \setminus B_\alpha$ --- замкнутое. $X \setminus B_\alpha \supset A$.

    $\bigcap(X \setminus B_\alpha) = \Cl A \implies X \setminus \bigcap (X \setminus B_\alpha) = X \setminus \Cl A \iff \bigcup(B_\alpha) = \Int(X \setminus A)$.
\end{proof}
\begin{consequence}
    $\Int A = X \setminus Cl(X \setminus A)$ и  $\Cl A = X \setminus \Int(X \setminus A)$.
\end{consequence}
\begin{properties}
    \begin{enumerate}
        \item $\Cl A \supset A$.
        \item  $\Cl A$ --- замкнутое множество. 
        \item $A$ --- замкнуто  $\iff A = \Cl A$.
            \begin{proof}
                $\Leftarrow$ --- пункт 2.  $\Rightarrow A$ --- замкнутое  $\Rightarrow$ оно участвует в пересечении из определения  $\implies \Cl A \subset A \implies \Cl A = A$.
            \end{proof}
        \item $A \subset B \implies \Cl A \subset \Cl B$.
             \begin{proof}
                $X \setminus A \supset X \setminus B \implies \Int(X \setminus A) \supset \Int(C \setminus B) \implies X \setminus \Int(X \setminus A) \subset X \setminus \Int(X \setminus B)$
            \end{proof}
        \item $\Cl(A \cup B) = \Cl A \cup \Cl B$.
        \item  $\Cl(\Cl A) = \Cl A$.
             \begin{proof}
                $B \coloneqq \Cl A$ --- замкнуто  $\implies \Cl B = B$.
            \end{proof}
    \end{enumerate}
\end{properties}
\begin{exerc}
    $\Cl \Int \Cl \Int \ldots A$. Какое наибольшее количество различных множеств может получиться.
\end{exerc}
\begin{theorem}
    $x \in \Cl A \iff \forall r > 0\quad B_r(x) \cap A \neq \emptyset$.
\end{theorem}
\begin{proof}
    Запишем отрицание условия теоремы: $x \notin \Cl A \iff \exists r > 0 B_r(x) \cap A = \emptyset$.

    Что означает, что  $x \notin A$? Это значит, что  $x\in X \setminus \Cl A = \Int(X \setminus A) \iff x \in \Int(X \setminus A) \iff x\text{ --- внутренняя точка }X \setminus A \iff \exists r > 0\!: B_r(x) \cap A = \emptyset \iff \exists r > 0\!: B_r(x) \subset X \setminus A$.
\end{proof}
\begin{consequence}
    $U$ --- открытое,  $U \cap A = \emptyset \implies U \cap \Cl A = \emptyset$.
\end{consequence}
\begin{proof}
    Возьмем $x \in U \implies \exists r > 0\!: B_r(x) \subset U \implies B_r(x) \cap A = \emptyset \implies x \notin \Cl A \implies U \cap \Cl A = \emptyset$.
\end{proof}
%END TICKET 28
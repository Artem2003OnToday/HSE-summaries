%BEGIN TICKET 18
\begin{example}
    $S_p(n) = 1^p + 2^p + \ldots + n^p$, $f(t) = t^p$,  $m = 1$,  $f''(t) = p(p-1)t^{p-2}$.

    $S_p(n) = \frac{1+n^p}{2} + \int\limits_1^n t^p \mathrm{d}t + \frac{1}{2} \int\limits_1^n p(p-1)t^{p-2} \{t\}(1-\{t\}) \mathrm{d}t$.

    При $p \in (-1, 1)$  $\int_1^n t^p \mathrm{d}t = \frac{t^{p+1}}{p+1} \mid_1^n = \frac{n^{p+1}}{p+1} - \frac{1}{p+1} = \frac{n^{p+1}}{p+1} + \mathcal{O}(1)$.\[
        \int_1^n t^{p-2} \underbrace{\{t\}(1-\{t\})}_{\le \frac{1}{4}} \mathrm{dt} \le \frac{1}{4} \int_1^n t^{p-2} \mathrm{d}t = \frac{1}{4} \cdot \frac{t^{p-1}}{p-1} \mid_1^n = \frac{1}{4} \cdot \frac{n^{p-1} - 1}{p - 1} = \mathcal{O}(1)
    .\] 

    То есть $S_p(n) = \frac{n^{p+1}}{p+1} + \frac{n^p}{2} + \mathcal{O}(1)$.

    При $p > 1$  $S_p(n) = \frac{n^{p+1}}{p+1} + \frac{n^p}{2} + \mathcal{O}(n^{p-1})$.
\end{example}
\begin{example}
    Гармонические числа: $H_n \coloneqq 1 + \frac{1}{2} + \frac{1}{3} + \ldots + \frac{1}{n}$. $m = 1, f(t) = \frac{1}{t}, f''(t) = \frac{2}{t^3}$.
    \[
    H_n = \frac{1 + \frac{1}{n}}{2} + \int_1^n \frac{\mathrm{d}t}{t} + \frac{1}{2} \int_1^n \frac{2}{t^3}\{t\}(1-\{t\})\mathrm{d}t
    \] Откуда получаем ($a_n \coloneqq \int\limits_1^n \frac{\{t\}(1-\{t\})}{t^3}$; $\int_1^n \frac{\mathrm{d}t}{t} = \ln t \mid_1^n = \ln n$): \[
H_n = \ln n + \frac{1}{2} + \frac{1}{2n} + a_n
.\] Заметим, что $a_{n+1} = a_n + \int\limits_n^{n+1}\frac{\{t\}(1-\{t\})}{t^3} \mathrm{d}t > a_n$. То есть $a_n\uparrow$. Причем $a_n \le \int\limits_1^n \frac{\mathrm{d}t}{t^3} = -\frac{1}{2t^2} \mid_1^n = \frac{1}{2} - \frac{1}{2n^2} < \frac{1}{2}$. А значит $a_n$ имеет предел, а значит  $a_n = a + o(1)$.
\\
Вывод:  $H_n = \ln n + \gamma + o(1)$, где  $\gamma \approx 0.5772156649$ --- постоянная Эйлера.
\end{example}
\begin{remark}
    $H_n = \ln n + \gamma + \frac{1}{2n} + \mathcal{O}(\frac{1}{n^2})$~--- точная формула.
\end{remark}
%END TICKET 18
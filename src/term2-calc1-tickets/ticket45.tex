%BEGIN TICKET 45
\begin{definition}
    $X$ --- векторное пространство и  $\| . \|$ и  $\|| . |\|$ --- нормы в  $X$. 

    Нормы эквиваленты, если $\exists C_1, C_2 > 0$\!: 
    \[
    C_1 \| x\| \le \|| x |\| \le C_2 \| x \| \quad \forall x \in X
    .\] 
\end{definition}
\begin{remark}
    \begin{enumerate}
        \item Это отношение эквивалентности. (упражнение)
        \item Пределы последовательности для эквивалентных норм совпадают. Док-во:
        Пусть $\lim x_n = a$ по норме $\|.\|$, т.е. $\lim \|x_n - a\| = 0$. А $0 \le \||x_n - a\|| \le C_2 \|x_n - a\| \rightarrow 0$, значит $\lim x_n = a$ и по норме $\||.\||$.
        \item Непрерывность отображений для эквивалентных норм совпадают (записываем по Гейне, а для последовательностей мы всё знаем).
    \end{enumerate}
\end{remark}
\begin{theorem}
   В $\R^d$ все нормы эквивалентны. 
\end{theorem}
\begin{proof}
    $\| x \| = \sqrt{x_1^2+ x_2^2 + \ldots + x_d^2}$. Достаточно доказать, что остальные норма эквиваленты.

    Пусть $p(x)$ --- другая норма в $\R^d$.  $e_k $ --- вектор с нулями и единицей на  $k$-ой позиции.

    $x=(x_1,x_2,\ldots,x_d) = \sum\limits_{k=1}^d x_ke_k$.
     \begin{align*}
         p(x - y) &= p(\sum\limits_{k=1}^d(x_k-y_k)e_k) \overset{(1)}{\le} \sum\limits_{k=1}^d p((x_k - y_k)e_k) = \\ 
                  &= \sum\limits_{k=1}^d |x_k - y_k| p(e_k) \le \text{(Коши-Буняковский)} \left(\sum\limits_{k=1}^d (x_k - y_k)^2\right)^{\frac{1}{2}} \left(\sum\limits_{k=1}^d p(e_k)^2\right)^{\frac{1}{2}} = \\
                  &= \left(\sum\limits_{k=1}^d p(e_k)^2\right)^{\frac{1}{2}} \|x - y \|\ \xRightarrow{(2)} p(x) \le \underbrace{\left(\sum\limits_{k=1}^d p(e_k)^2\right)^{\frac{1}{2}}}_{\coloneqq M} \| x \|.
        \end{align*}
     
     $(1) \iff \|a+b\| \le \|a\| + \|b\|$ и $p(a+b) \le p(a) + p(b)$

     $(2) \iff p(x)$ --- непрерывная функция.

     $S \coloneqq \{ x \in \R^d\!: x_1^2 + x_2^2 + \ldots + x_d^2 = 1\}$ --- компакт $\implies \exists a \in S\!: 0 < p(a) \le p(x) \quad \forall x \in S$.

     $p(x) = p(\frac{x}{\|x\|}\cdot \|x\|) = \|x\| p(\frac{x}{\|x\|}) \ge \|x\| p(a)$, так как норма $\frac{x}{\|x\|}$ будет равна 1.

     Тогда $p(a) \|x\| \le p(x) \le M \|x\| \quad \forall x \in \R^d$.
\end{proof}
%END TICKET 45
%BEGIN TICKET 03
\begin{definition}
    Пусть $f\!: [a, b] \to \R$. Тогда  $f_+, f_-\!: [a, b] \to [0; +\inf)$. Причем  $f_+(x) = \max\{f(x), 0\}$,  $f_- = \max\{-f(x), 0\}$. $f_+$ --- положительная составляющая, а $f_-$ --- отрицательная составляющая.
\end{definition}
\begin{properties}
    \begin{enumerate}
        \item $f = f_+ - f_-$.
        \item  $|f| = f_+ + f_-$
        \item  $f_+ = \frac{f + |f|}{2}$, $f_- = \frac{|f| - f}{2}$. (Сложили и вычли первые два свойства)
        \item Если $f \in C([a, b])$ , то  $f_{\pm} \in C([a, b])$. (Видно из 3-го пункта)
    \end{enumerate}
\end{properties}

\begin{definition}
	Пусть $f\!: [a, b] \to [0; +\infty)$.

    Тогда подграфик $P_f([a; b]) \coloneqq \{(x, y) \in \R^2 \mid x \in [a, b], 0 \le y \le f(x)\}$.
    Подграфик может быть взят и от какого-то подотрезка области определения функции!
\end{definition}
%END TICKET 03
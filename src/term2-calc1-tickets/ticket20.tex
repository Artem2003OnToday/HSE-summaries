%BEGIN TICKET 20
\begin{definition}
    Пусть $-\infty < a < b \le +\infty$ и $f \in C[a, b)$.

    Тогда определим  $\int\limits_a^{\to b} f\coloneqq \lim\limits_{B \to b-} \int\limits_a^B f$.

    Если $-\infty \le a < b < +\infty, f \in C(a, b]$, тогда $\int\limits_{\to a}^b f \coloneqq \lim\limits_{A \to a+} \int\limits_A^b f$.
\end{definition}
\begin{remark}
    Если $b < +\infty$ и  $f \in C[a, b]$, то определение не дает ничего нового:  \begin{align*}
        \int_a^b f &= \lim_{B \to b}f \\
        \left|\int_a^b f - \int_a^B f\right| &= \left| \int_B^b f \right| \le M(b-B) \to 0, M = \max\limits_{x \in [a, b]} f(x)
    .\end{align*}
\end{remark}
\begin{example}
    \begin{enumerate}
        \item $\int\limits_1^{+\infty} \frac{\mathrm{d}x}{x^p} = \lim\limits_{y \to +\infty} \int\limits_a^y \frac{\mathrm{d}x}{x^p} = \lim\limits_{\substack{y \to +\infty \\ \text{при}\ p \neq 1}} -\frac{1}{(p-1)x^{p-1}}\mid_{x=1}^{x=y} = \frac{1}{p-1} - \lim\limits_{y \to +\infty} \frac{1}{(p-1)y^{p-1}} = \frac{1}{p-1}$ при $p > 1$, при $p < 1$ получаем $+\infty$, а при $p = 1$  $\lim\limits_{y \to +\infty} \ln x \mid_1^y = \lim\limits_{y \to +\infty} \ln y = +\infty$
        \item $\int\limits_0^1 \frac{\mathrm{d}x}{x^p} = \lim\limits_{y \to 0+} \int\limits_y^1 \frac{\mathrm{d}x}{x^p} = \lim \limits_{y \to 0+} -\frac{1}{(p-1)x^{p-1}} \mid_{x=y}^{x=1} = -\frac{1}{p - 1} + \lim\limits_{y \to 0+} = \frac{y^{1-p}}{p - 1} = \frac{1}{1 - p}$ при $p < 1$, при  $p > 1$ получаем  $+\infty$, а вот при  $p = 1$  $\lim\limits_{y \to 0+} \ln x \mid_y^1 = \lim\limits_{y \to 0+} - \ln y = +\infty$.

            То есть, при  $p < 1$  $\int\limits_0^1 \frac{\mathrm{d}x}{x^p} = \frac{1}{1 - p}$, \\
            при $p \ge 1$ $\int\limits_0^1 \frac{\mathrm{d}x}{x^p} = +\infty$.
    \end{enumerate}
\end{example}
\begin{remark}
    Если $f \in C[a, b)$ и  $F$ его первообразная, то  $\int\limits_a^b f = \lim\limits_{B \to b-}F(B) - F(a)$.

    Если $f \in C[a, b)$ и  $F$ его первообразная, то  $\int\limits_a^b f = F(b) - \lim\limits_{A \to a+}F(A)$.
\end{remark}
\begin{proof}
    Очевидно по формуле Ньютона-Лейбница.
\end{proof}
\begin{definition}
    $F \Big|_a^b \coloneqq \lim\limits_{B \to b-} F(B) - F(a)$.
\end{definition}

\begin{definition}
    $\int\limits_a^{\to b} f$ сходится, если  $\lim B$ в его определении существует и конечен.
\end{definition}

\begin{theorem}[Критерий Коши]
    Пусть $-\infty < a < b \le +\infty$, $f \in C[a, b)$.

    Тогда $\int\limits_a^b f$ сходится  $\iff \forall \eps \exists c \in (a, b)\!: \forall A, B \in (c, b)\ \left|\int\limits_A^B f\right| < \eps$. 
\end{theorem}
\begin{remark}
    \begin{enumerate}
        \item Если $b = +\infty$ это означает, что  $\forall \eps \exists c > a \forall A, B > c\!: \left| \int\limits_A^B f \right|  < \eps$.
        \item  Если $b < +\infty$ это означает, что  $\forall \eps > 0 \exists \delta > 0 \forall A, B \in (b-\delta; b)\!: \left|\int\limits_A^B f \right| < \eps$.
    \end{enumerate}
\end{remark}
\begin{proof}
    Для $b < +\infty$. 
      \begin{itemize}
          \item "$\Rightarrow$" $\int\limits_a^b f$ сходится  $\implies \exists$ конечный  $I \coloneqq \lim\limits_{B \to b-} \int\limits_a^B f$, обозначим $\int\limits_a^B f$ за  $g(B)$. Воспользуемся критерием Коши для функций:

              \[\forall \eps > 0 \exists \delta > 0 \begin{array}{lc} 
              \forall B \in (b - \delta, b) & |g(B) - I| < \frac{\eps}{2}\\ 
          \forall A \in(b - \delta, b) & |g(A) - I| < \frac{\eps}{2}\end{array} \implies |g(B) - g(A)| \le |g(B) - I| + |I - g(A)| < \eps\]
      \item "$\Leftarrow$" $\int\limits_a^B f \eqqcolon g(B)$.

          $\forall \eps > 0 \exists \delta > 0 \forall A, B \in (b-\delta, b)\!: |g(B)- g(A)| < \eps$ это условие из критерия Коши для $\lim\limits_{B \to b-} g(B)$.
  \end{itemize}
\end{proof}
\begin{remark}
    Если существует $A_n, B_n \in [a, b)\!: \lim A_n = \lim B_n = b\!: \int\limits_{A_n}^{B_n} f \centernot\to 0$, то  $\int\limits_a^b f$ расходится.
\end{remark}
\begin{proof}
    Возьмем $A_{n_k}$ и  $B_{n_k}\!: |\int\limits_{A_{n_k}}^{B_{n_k}} f| \to C > 0 \implies |\int\limits_{A_{n_k}}^{B_{n_k}} f| > \frac{C}{2}$ при больших $k$. Но это противоречит критерию Коши.
\end{proof}
%END TICKET 20
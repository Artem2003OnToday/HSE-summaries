%BEGIN TICKET 30

\begin{definition}
     $(X, \rho)$ --- метрическое пространство  $Y \subset X$.

     Тогда  $(Y, \rho \mid_{Y \times Y})$ --- подпространство метрического пространства  $(X, \rho)$.
\end{definition}
\begin{example}
     $(\R, |x-y|)$.  $Y=[a, b] \subset \R$, например, $Y=[0, 1]$.

      $B_1(1) = (0, 1], B_2(0) = [0, 1]$.  $B_r^Y(a) = Y \cap B_r^X(a)$. ($B^A_r -$ шарик радиуса $r$ на множестве $A$)
\end{example}
\begin{theorem}[об открытых и замкнутых множества в пространстве и подпространстве]
     $(X, \rho)$ --- метрическое пространство,  $(Y, \rho)$ --- его подпространство,  $A \subset Y$. Тогда
      \begin{enumerate}
          \item $A$ --- открыто в  $Y \iff \exists G$ --- открытое в  $X\!: A = G \cap Y$. 
          \item $A$ --- замкнуто в  $Y \iff \exists F$ --- замкнутое в  $X\!: A = F \cap Y$.
     \end{enumerate}
\end{theorem}
\begin{proof}
     \slashn
     \begin{enumerate}
         \item "$\Rightarrow$" $A$ --- открыто в  $Y \implies \forall x \in A \exists r_x > 0 \!: B_{r_x}^Y(x) \subset A \implies A = \bigcup\limits_{x \in A}B_{r_x}^Y(x)$.

             То есть наше множество будет объединением большего числа шариков (возможно бесконечного). Найдем теперь  $G$:  $G \coloneqq \bigcup\limits_{x \in A} B_{r_x}^X(x)$ --- открыто. Посмотрим теперь на  $G \cap Y = \bigcup\limits_{x \in A}(B_{r_x}^X(x) \cap Y) = \bigcup\limits_{x \in A}B_{r_x}^Y(x) = A$.

         В обратную сторону. Пусть $A = G \cap Y$, где  $G$ открыто в  $X$. Возьмем  $x \in G \cap Y$.  $G$ --- открыто в  $X \implies \forall x \in G \cap Y \exists r > 0\!: B_r^X(x) \subset G \implies B_r^X(x) \cap Y \subset G \cap Y = A \implies B_r^Y(x) \subset A \implies x$ --- внутренняя точка $A \implies A$ --- открыто в  $Y$. 

         \item $A$ --- замкнуто в $Y \iff Y \setminus A$ --- открыто в  $Y \iff \exists G$ --- открытое в  $X$, такое что  $Y \setminus A = Y \cap G \iff A = Y \setminus (Y \cap G) \overset{(1)}{=} Y \setminus G \overset{(2)}{=} Y \cap (X \setminus G) \iff \exists G$ --- открытое в  $X$, такое что  $A = Y \cap (X \setminus G) \iff \exists F$ --- замкнуто в  $X$, такое что  $A = Y \cap F$.

             $(1)$ --- Можно забить на пересечение с  $Y$, потому что, если элемент  $G$ не лежит в $Y$, то и в $Y \setminus G$ он участия не принимает.  $(2)$ --- Помним, что  $Y \subset X$, а значит такая операция корректна.
     \end{enumerate}
\end{proof}
\begin{example}
     $(\R, |x-y|)$.  $Y = [0, 3)$.  $[0, 1)$ --- открыто в  $[0, 3)$:  $[0, 1) = [0, 3) \cap (-1, -1)$.  $[2, 3)$ --- замкнуто в  $[0, 3)$:  $[2, 3) = [0, 3) \cap [2, 3]$.
\end{example}
 
%END TICKET 30
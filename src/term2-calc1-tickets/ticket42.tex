%BEGIN TICKET 42
\begin{definition}
    $(X, \rho_X)$ и  $(Y,\rho_Y)$ --- метрические пространства, $E \subset X$,  $a \in E$.\\
     $f\!: E \to Y$,  $f$ непрерывна в точке  $a$, если
      \begin{enumerate}
          \item $a$ не предельная точка или  $a$ --- предельная и  $\lim\limits_{x \to a} f(x) = f(a)$.
          \item По Коши.  $\forall \eps > 0 \exists \delta > 0 \forall x \in E\!: \rho_X(x, a) < \delta \Rightarrow \rho_Y(f(x), f(a)) < \eps$.
          \item С окрестностями.  $\forall B_\eps(f(a)) \exists B_\delta(a)\!: f(B_\delta(a)) \subset B_\eps(f(a))$.
          \item По Гейне:  $\forall x_n \in E\!: \lim x_n = a \implies \lim f(x_n) = f(a)$.
     \end{enumerate}
\end{definition}
\begin{proof}
    Упражнение!
\end{proof}
\begin{theorem}[о непрерывности композиции]
    $(X, \rho_X), (Y, \rho_Y), (Z, \rho_Z)$ ---  $D \subset X, E \subset Y, a \in D, f\!: D \to E, g\!: E \to Z$.
    Если  $f$ непрерывна в точке  $a$, а  $g$ непрерывна в точке  $f(a)$, то $g \circ f$ непрерывна в точке  $a$. 
\end{theorem}
\begin{proof}
    Запишем определения непрерывности для $g$ и  $f$ в терминах окрестностей (в определении для $f$ мы дописали  $\cap E$, но заметим, что это никак не повлияет по определению  $E$):
    \begin{align*}
        &\left. \begin{array}{l}
                \forall B_\eps(g(f(a)))\ \exists B_\delta(f(a))\text{ такой, что }g(B_\delta(f(a)) \cap E) \subset B_\eps(g(f(a)))\\
                \forall B_\delta(f(a))\ \exists B_\gamma(a)\text{ такой, что }f(B_\gamma(a) \cap D) \subset B_\delta(f(a)) \cap E
        \end{array} \right\}\quad
        \\
        &\implies g(f(B_\gamma(a) \cap D)) \subset g(B_\delta(f(a)) \cap E) \subset B_\eps(g(f(a))) \implies g \circ f\ \text{непрерывна в точке}\ a
    \end{align*} 
\end{proof}
\begin{theorem}[Характеристика непрерывности в терминах открытых множеств]
    $f\!: X \to Y$. Тогда

     $f$ непрерывна во всех точках  $\iff \forall U$ --- открытого в $Y$:  $f^{-1}(U) \coloneqq \{ x \in X \mid f(x) \in U\}$ --- открыто в $X$. 
\end{theorem}
\begin{proof}
    $\Rightarrow$. Берем  $a \in f^{-1}(U) \implies f(a) \in U$ -- открыто  $\implies \exists \eps > 0\quad B_\eps(f(a)) \subset U$.

    $f$ непрерывно в точке  $a \implies \exists \delta > 0\!: f(B_\delta(a)) \subset B_\eps(f(a)) \subset U \implies B_\delta(a) \subset f^{-1}(U) \implies a$ --- внутренняя точка  $f^{-1}(U) \implies f^{-1}(U)$ --- открыто.

     $\Leftarrow$.  $U \coloneqq B_\eps(f(a))$ --- открыто  $\implies f^{-1}(B_\eps(f(a)))$ --- открыто и  $a \in f^{-1}(B_\eps(f(a))) \implies \exists \delta > 0\quad B_\delta(a) \subset f^{-1}(B_\eps(f(a))) \implies f(B_\delta(a)) \subset B_\eps(f(a)) \implies f$ непрерывна в точке  $a$.
\end{proof}
%END TICKET 42
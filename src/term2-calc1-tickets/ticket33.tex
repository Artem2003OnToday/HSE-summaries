%BEGIN TICKET 33
\begin{theorem}[об арифметических действиях с пределами]
    $X$ --- нормированное пространство,  $x_n, y_n \in X$,  $\lambda_n \in \R$.  $\lim x_n = a, \lim y_n = b, \lim \lambda_n = \mu$. Тогда:
     \begin{enumerate}
         \item $\lim (x_n + y_n) = a+b$.
         \item  $\lim(x_n - y_n) = a-b$.
         \item  $\lim \lambda_nx_n = \mu a$.
         \item  $\lim \lVert x_n\rVert = \lVert a \rVert$.
         \item  Если в  $X$ есть скалярное произведение, то  $\lim \langle x_n, y_n \rangle = \langle a, b \rangle$.
    \end{enumerate}
\end{theorem}
\begin{proof}
    \begin{enumerate}
        \item $\rho(x_n+y_n, a+b) = \lVert (x_n+y_n - (a+b)) \rVert = \lVert (x_n-a) + (y_n-b) \rVert \le \lVert x_n - a \rVert + \lVert y_n - b \rVert = \rho(x_n, a) + \rho(y_n, b) \to 0$.
        \item Аналогично.
        \item $\rho(\lambda_nx_n, \mu a) = \lVert \lambda_n x_n - \mu a\rVert = \lVert \lambda_n x_n - \lambda_n a + \lambda_n a - \mu a \rVert \le \lVert \lambda_n x_n - \lambda_n a \rVert + \lVert \lambda_n a - \mu a \rVert = |\lambda_n| \lVert x_n - a \rVert + |\lambda_n -\mu| \lVert a \rVert \to 0$, так как $|\lambda_n|$ --- ограниченная, $\lVert x_n - a \rVert = \rho(x_n - a) \to 0$,  $|\lambda_n -\mu| \to 0$, $\lVert a \rVert$ --- константа.  
        \item $| \lVert x_n \rVert - \lVert a \rVert| \le \lVert x_n - a \rVert = \rho(x_n, a) \to 0 \implies \lim \lVert x_n \rVert = \lVert a \rVert$
        \item $\langle x, y \rangle = \frac{1}{4}(\lVert x+y \rVert^2 - \lVert x-y \rVert^2) = \frac{1}{4}(\langle x+y, x+y\rangle - \langle x-y, x-y\rangle)$. Тогда получаем $4 \langle x_n, y_n \rangle = \lVert x_n + y_n \rVert^2 - \lVert x_n - y_n \rVert^2 \to \lVert a + b \rVert^2 - \lVert a - b \rVert^2 = 4 \langle a, b \rangle$.
    \end{enumerate}
\end{proof}
\begin{definition}
    $\R^d$ --- пространство с нормой  $\sqrt{x_1^2 + x_2^2 + \ldots + x_d^2}$.
\end{definition}
\begin{definition}
    Покоординатная сходимость в $\R^d$:

    $x_n \in \R^d$.  $x_n = (x_n^{(1)}, x_n^{(2)}, \ldots, x_n^{(d)}) \xrightarrow{\text{покоординатно}} a = (a^{(1)}, a^{(2)}, \ldots, a^{(d)})$.
\end{definition}
\begin{theorem}
    в $\R^d$ сходимость по метрике и покоординатная сходимость совпадает.
\end{theorem}
\begin{proof}
    Метрика $\implies$ покоординатная.  $\rho(x_n, a) \to 0 \implies 0 \le (x_n^{(1)} - a^{(1)})^2 + \ldots + (x_n^{(d)} - a^{(d)})^2 = \rho(x_n, a)^2 \to 0 \implies \lim (x_n^{(k)} - a^{(k)})^2 = 0 \implies \lim x_n^{(k)} = a^{(k)} \implies$ покоординатная сходимость.

    Покоординатная $\implies$ метрика. Пусть  $|x_n^{(k)} - a^{(k)}| \to 0 \quad \forall k \implies (x_n^{(k)} - a^{(k)})^2 \to 0 \implies \sum\limits_{k=1}^d (x_n^{(k)} - a^{(k)})^2 \to 0$. А так как $(\ldots)^2 = \rho(x_n, a)^2 \implies \rho(x_n, a) \to 0$.
\end{proof}
%END TICKET 33
%BEGIN TICKET 22
\begin{theorem}
    Пусть  $f \in C[a, b)$ и  $f \ge 0$. 

    Тогда $\int\limits_a^b f$ сходится  $\iff F(y) \coloneqq \int\limits_a^y f$ ограничена сверху.
\end{theorem}
\begin{proof}
    $f \ge 0 \implies F$ монотонно возрастает. $\int\limits_a^b f$ сходится  $\iff \exists$ конечный  $\lim\limits_{y \to b-}F(y) \iff F$ ограничена сверху. 
\end{proof}
\begin{remark}
    $f \in C[a; b), f\ge 0$. Если $\int\limits_a^b f$ расходится, это означает, что  $\int\limits_a^b f = +\infty$.
\end{remark}
\begin{consequence}[Признак сравнения]
    $f, g \in C[a, b)$,  $f, g \ge 0$ и $f \le g$.
    \begin{enumerate}
        \item Если $\int\limits_a^b g$ сходится, то и $\int\limits_a^b f$ сходится.
        \item Если  $\int\limits_a^b f$ расходится, то и $\int\limits_a^b g$ расходится.
    \end{enumerate}
\end{consequence}
\begin{proof}
    $F(y) \coloneqq \int\limits_a^y f$ и  $G(y) \coloneqq \int\limits_a^y g$.
     \begin{enumerate}
         \item Пусть $\int\limits_a^b g$ сходится  $\implies$  $G(y)$ ограничена, но  $F(y) \le G(y) \implies F(y)$ ограничена $\implies \int\limits_a^b f$ сходится.
         \item От противного. Пусть $\int\limits_a^b g$ сходится $\Rightarrow$ см. первый пункт --- противоречие.
    \end{enumerate}
\end{proof}
\begin{remark}
    \begin{enumerate}
        \item Неравенство $f \le g$ может выполняться лишь для аргументов, близких к $b$.
        \item Неравенство  $f \le g$ можно заменить на $f = \mathcal{O}(g)$.

            $f = \mathcal{O}(g) \implies f \le cg$. $\int\limits_a^b g$ сходится  $\implies \int\limits_a^b cg $ сходится  $\implies \int\limits_a^b f$ -- сходится.
        \item Если  $f = \mathcal{O}(\frac{1}{x^{1 + \eps}})$ для $\eps > 0$, то  $\int\limits_a^{+\infty} f$ --- сходится.

            $f \in C[a, +\infty),  g(x) = \frac{1}{x^{1+\eps}}$ и можно считать, что $a \ge 1$ $\int\limits_1^{+\infty} g(x) \mathrm{d}x$ --- сходится.
    \end{enumerate}
\end{remark}
\begin{consequence}
    $f, g \in C[a, b)$,  $f, g \ge 0$ и $f(x) \sim g(x)$ при $x \to b-$. Тогда  $\int\limits_a^b f$ и  $\int\limits_a^b g$ ведут себя одинаково (либо оба сходятся, либо оба расходятся).
\end{consequence}
\begin{proof}
    $f \sim g \implies f = \vphi \cdot g$, где  $\vphi(x) \xrightarrow{x \to b-} 1 \implies$ в окрестности $b$  $\frac{1}{2} \le \vphi \le 2 \implies f \le 2g \land g \le 2f$ в окрестности $b$  $\implies$ из сходимости  $\int\limits_a^b g$ следует сходимость $\int\limits_a^b f$, и наоборот. 
\end{proof}
%END TICKET 22
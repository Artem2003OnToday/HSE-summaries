%BEGIN TICKET 11
\begin{example}
    \begin{align}
    H_j \coloneqq \frac{1}{j!} \int\limits_0^{\frac{\pi}{2}}\left(\left(\frac{\pi}{2}\right)^2 - x^2\right)^j \cos x \mathrm{d}x.
    \end{align}
    \textbf{Свойство 1.} $0 < H_j \le \frac{1}{j!}\left(\frac{\pi}{2}\right)^{2j} \int\limits_0^{\frac{\pi}{2}} \cos x \mathrm{d}x = \frac{\left(\frac{\pi}{2}\right)^{2j}}{j!}$.\\
    \textbf{Свойство 2.} $\forall c > 0\!: c^j \cdot H_j \xrightarrow{j \to \infty} 0$.  $0 < c^j H_j \le \frac{\left(\frac{\pi}{2}\right)^{2j} \cdot c^j}{j!} = \frac{\left(\frac{\pi^2}{4}c\right)^j}{j!} \to 0$.\\
    \textbf{Свойство 3.} $H_0 = 1$,  $H_1 = 2$ (\textit{упражнение}).\\
    \textbf{Свойство 4.} $H_j = (4j - 2) H_{j-1} - \pi^2 H_{j-2}$, при  $j \ge 2$.
\end{example}
\begin{proof}
\begin{align}
    j! H_j = \int_0^{\frac{\pi}{2}} \left(\left(\frac{\pi}{2}\right)^2 - x^2\right)^j (\sin x)' \mathrm{d}x
\end{align}
Заметим, что $\left(\left(\left(\frac{\pi}{2}\right)^2 - x^2\right)^j\right)' = j \left(\left(\frac{\pi}{2}\right)^2 - x^2\right)^{j-1} \cdot (-2x)$. Тогда: {\small \begin{align*}
    (3) &= \underbrace{\left(\left( \frac{\pi}{2} \right)^2 - x^2\right)^j \sin x \mid_{x = 0}^{x = \frac{\pi}{2}}}_{=0} + 2j \int_{0}^{\frac{\pi}{2}}\left(\left(\frac{\pi}{2}\right)^2 - x^2\right)^{j-1} x \underbrace{\sin x}_{=(-\cos x)'} \mathrm{d} x = \\
	&= 2j \left(\underbrace{\left(\left(\frac{\pi}{2}\right)^2 - x^2\right)^{j-1} \cdot x \cdot (- \cos x) \mid_{x=0}^{x=\frac{\pi}{2}}}_{=0} - \int\limits_0^{\frac{n}{2}} \left( (j-1)\left( \left(\frac{\pi}2 \right)^2 - x^2  \right)^{j-2}(-2x)x + \left( \left(\frac{\pi}2 \right)^2 - x^2  \right)^{j-1}  \right)(-\cos x)  \mathrm{d} x\right) \\
	&= 2j\left( (j-1)!H_{j-1} - 2(j-1)\int_0^{\frac{\pi}2} \left( \left( \frac{\pi}2\right)^2 - x^2  \right)^{j-2}x^2 \cos x dx \right)
.\end{align*} }
В процессе мы дважды интегрировали по частям, а теперь нужно избавиться во втором слагаемом от $x^2$. Для этого заметим, что $x^2 = \left( \frac{\pi}2 \right)^2 - \left( \left( \frac{\pi}2 \right)^2 - x^2  \right)$, подставим и разобьём интеграл на два, которые есть $H_{j-2}$ и $H_{j-1}$ с нужными коэффициентами:

$$
j! H_j = 2j(j-1)! H_{j-1} - 4j(j-1)\left(((j-2)!\left( \frac{\pi}2  \right)^2) H_{j-2} - (j-1)! H_{j-1} \right)
$$

Откуда с легкостью получаем $j! H_j = 2j! H_{j-1} - \pi^2 j! H_{j-2} + 4(j-1)j! H_{j-1} \iff H_j = (4j-2)H_{j-1} - \pi^2 H_{j-2}$.\\
\textbf{Свойство 5.} Существует многочлен $P_n$ с целыми коэффициентами степени $\le n$, такой что $H_j = P_j(\pi^2)$.
\begin{proof}
    $P_0 \equiv 1, P_1 \equiv 2, P_n(x) = (4n-2)P_{n-1}(x) - xP_{n-2}(x)$.
\end{proof}
\end{proof}
\begin{theorem}[Ламберта, доказательство: Эрмит]
    Числа $\pi$ и  $\pi^2$ иррациональные.
\end{theorem}
\begin{proof}
    От противного. Пусть $\pi^2$ --- рационально. Тогда пусть  $\pi^2 = \frac{m}{n}$. Тогда $H_j = P_j(\frac{m}{n}) = \frac{\text{целое число}}{n^j} > 0$.\\
    $n^j H_j = \text{целое число} > 0 \Rightarrow n^j H_j \ge 1$

    Но, по свойству 2, при $j \to +\infty\ \ n^j H_j \to 0$, противоречие.
\end{proof}
%END TICKET 11
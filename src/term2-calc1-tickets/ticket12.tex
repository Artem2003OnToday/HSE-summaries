%BEGIN TICKET 12
\begin{definition}
    $f\!: E \subset \R \to \R$ равномерно непрерывна на  $E$, если  $\forall \eps > 0 \exists \delta > 0 \forall x, y \in E\!: |x-y| < \delta \Rightarrow |f(x)-f(y)| < \eps$
\end{definition}
\begin{definition}
    $f$ непрерывна во всех точках из  $E$:\\
    $\forall x \in E \forall \eps > 0 \exists \delta > 0 \forall y \in E\!: |x-y| < \delta \Rightarrow |f(x)-f(y)| < \eps$
\end{definition}
Концептуальное отличие в том, что в первом случае у нас $\delta(\varepsilon)$, а во втором --- $\delta(x, \varepsilon)$, т.е. при равномерной непрерывности у нас есть общая дельта по эспилону на всю область, а при непрерывности во всех точках для каждой точки своё $\delta$ по $\varepsilon$
\begin{example}
    $\sin x$ и  $\cos x$ равномерно непрерывны на  $\R$.

     $|\sin x - \sin y| \le |x-y| \Rightarrow \delta = \eps$ подходит. $|\cos x - \cos y| \le |x-y|$.
\end{example}
\begin{example}
    $f(x) = x^2$ не является равномерно непрерывной на $\R$. Рассмотрим  $\eps = 1$, никакое  $\delta > 0$ не подходит.  $x$ и  $x + \frac{\delta}{2}$. $f(x + \frac{\delta}{2}) - f(x)  = (x+\frac{\delta}{2})^2 - x^2 = \ldots = \delta x + \frac{\delta^2}{4} > \delta x$.
    При $x = \frac{1}{\delta}$ противоречие. 
\end{example}
\begin{theorem}[Теорема Кантора]
    Пусть $f \in C[a, b]$, тогда $f$ равномерно непрерывна на  $[a, b]$.
\end{theorem}
\begin{proof}
    Берем $\eps > 0$ и предположим, что  $\delta = \frac{1}{n}$ не подходит, то есть $\exists x_n, y_n \in [a, b]\!: |x_n - y_n| < \frac{1}{n}$ и $|f(x_n) - f(y_n)| \geq \eps$. По теореме Больцано-Вейерштрасса у последовательности $x_n$ есть сходящаяся последовательность  $x_{n_k} \to c$, то есть  $\lim x_{n_k} = c \in [a, b]$.

    $\underbrace{x_{n_k} - \frac{1}{n_k}}_{\to c} < y_{n_k} < \underbrace{x_{n_k} + \frac{1}{n_k}}_{\to c} \implies \lim y_{n_k} = c$. Но $f$ непрерывна в точке  $c$  $\implies \lim f(x_{n_k}) = f(c) = \lim f(y_{n_k}) \implies \lim (f(x_{n_k}) - f(y_{n_k})) = 0$, но $|f(x_{n_k}) - f(y_{n_k})| \ge \eps$.
\end{proof}
\begin{remark}
    Для интервала или полуинтервала неверно. $f(x) = \frac{1}{x}$ на $(0; 1]$. Докажем, что нет равномерной непрерывностью на  $(0; 1]$. 

    Пусть  $\eps = 1$ и  $\delta > 0$. Пусть  $0 < x < \delta$,  $y = \frac{x}{2}$, $|x-y| = \frac{x}{2} < \delta$. Тогда $f(y) - f(x) = \frac{2}{x} - \frac{1}{x} = \frac{1}{x} > 1$.
\end{remark}

%END TICKET 12
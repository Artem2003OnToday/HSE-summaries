%BEGIN TICKET 31
\begin{definition}
     $X$ --- векторное пространство над  $\R$.

      $\|.\|\!: X \to \R$ --- норма, если
       \begin{enumerate}
           \item $\|x\| \ge 0\quad \forall x \in X$ и $\|x\| = 0 \iff x = \overrightarrow{0}$.
           \item  $\|\lambda x\| = |\lambda| \cdot \|x\|\quad \forall x \in X\ \forall \lambda \in \R$. 
           \item (неравенство треугольника): $\forall x, y\!: \|x + y\| \le \|x\| + \|y\|$.
      \end{enumerate}
\end{definition}
\begin{example}
     \begin{enumerate}
         \item $|x|$ в $\R$,
         \item  $\|x\|_1 = |x_1| + |x_2| + \ldots + |x_d|$ в $\R^d$.
         \item  $\|x\|_{\infty} = \max\limits_{k=1,2,\ldots, d} |x_k|$. \\ Неравенство треугольника: $\|x+y\|_{\infty} = \max\{|x_k|+|y_k|\} \le \max\{|x_k|\} + \max\{|y_k|\} = \|x\|_{\infty} + \|y\|_{\infty}$
         \item $\|x\|_2 = \sqrt{x_1^2 + x_2^2 + \ldots + x_n^2}$.
         \item $\|x\|_p = \left(\sum\limits_{k=1}^d |x_k|^p\right)^{\frac{1}{p}}$ в $\R^d$ при  $p \ge 1$. Неравенство треугольника --- неравенство Минковского.
         \item $C[a, b]$.  $\|f\| = \max\limits_{t \in [a, b]} |f(t)|$. 
     \end{enumerate}
\end{example}
\begin{definition}
    $X$ векторное пространство над  $\R$.  $\langle .,.\rangle\!: X \times X \to \R$ скалярное произведение, если
     \begin{enumerate}
         \item $\langle x, x \rangle \ge 0$ и $\langle x, x \rangle = 0 \iff x = \overrightarrow{0}$.
         \item  $\langle x+y, z\rangle = \langle x, z \rangle + \langle y, z \rangle$
         \item  $\langle x, y \rangle = \langle y, x \rangle$.
         \item  $\langle \lambda x, y \rangle = \lambda \langle x, y \rangle \quad \lambda \in \R$.
    \end{enumerate}
\end{definition}
\begin{example}
    \begin{enumerate}
        \item $\R^d$.  $\langle x, y\rangle = \sum x_iy_i$.
        \item Возьмем $w_1, \ldots, w_d > 0$. Тогда $\langle x, y \rangle = \sum w_i x_i y_i$.
        \item $C[a, b]$.  $\langle f, g \rangle = \int\limits_a^b f(x)g(x) \mathrm{d}x$.
    \end{enumerate}
\end{example}
\begin{properties}
    \begin{enumerate}
        \item Неравенство Коши-Буняковского. $\langle x, y \rangle^2 \le \langle x, x \rangle \cdot \langle y, y\rangle$.
            \begin{proof}
                $f(t) \coloneqq \langle x+ty, x +ty \rangle \ge 0$. $f(t) = \langle x, x \rangle + t\langle x, y \rangle + t\langle x, y \rangle + t^2 \langle y, y \rangle = t^2 \langle y, y\rangle + 2t\langle x, y \rangle + \langle x, y \rangle$ --- квадратный трехчлен (если $\langle y, y \rangle = 0 \implies y = 0 \implies$ везде нули). Тогда $0 \ge D= (\langle x, y \rangle)^2 - 4 \langle x, x\rangle \cdot \langle y, y \rangle = 4(\langle x, y \rangle^2 - \langle x, x\rangle \cdot \langle y, y \rangle)$. Потому что иначе есть два корня и где-то есть отрицательное значение, а $f(t) \geq 0$.
            \end{proof}
        \item $\|x\| \coloneqq \sqrt{\langle x, x \rangle}$ --- норма.
             \begin{proof}
                $\|\lambda x\| = \sqrt{\langle \lambda x, \lambda x\rangle} = \sqrt{\lambda^2\langle x, x \rangle} = |\lambda| \sqrt{\langle x, x \rangle} = |\lambda| \cdot \|x\|$.

                Неравенство треугольника: $\lVert x+y \rVert \le \lVert x \rVert + \lVert y \rVert$.
                Возведем в квадрат, получим $\langle x + y, x + y\rangle \le \langle x, x\rangle + \langle y, y\rangle + 2\sqrt{\langle x, x\rangle\langle y, y\rangle}$, но теперь вспомним, что $\langle x + y, x + y\rangle = \langle x, x\rangle + \langle y, y\rangle + 2\langle x, y\rangle$.
                А, сократив общие слагаемые, получим доказанное неравенство Коши-Буняковского.
            \end{proof}
        \item $\rho(x, y) = \lVert x - y \rVert$ --- метрика.
            \begin{proof}
                $\rho(x, y) \ge 0$. $\rho(x, y) = 0 \iff \lVert x - y \rVert = 0 \iff x - y = \overrightarrow{0} \iff x = y$.

                 $\rho(y, x) = \lVert y-x \rVert = \lVert (-1)(x-y) \rVert = |-1| \lVert x - y \rVert = \rho(x, y)$.

                  $\rho(x, z) \le \rho(x, y) + \rho(y, z)$: $\lVert (x-y) + (y-z) \rVert = \lVert x-z\rVert \le \lVert x - y \rVert + \lVert y-z \rVert$.
            \end{proof}
        \item $\lVert x - y \rVert \ge |\lVert x \rVert - \lVert y \rVert |$.
            \begin{proof}
                Надо доказать, что $-\lVert x - y \rVert \le \lVert x \rVert - \lVert y \rVert \le \lVert x - y \rVert$.

                Левое: $\lVert y \rVert = \lVert (y - x) + x \rVert \le \lVert y - x \rVert + \lVert x \rVert$

                Правое: $\lVert x \rVert = \lVert (x - y) + y \rVert \le \lVert x - y \rVert + \lVert y \rVert$

            \end{proof}
        \item Упражненение. Если норма порождается скалярным произведением $\iff \lVert x+y\rVert^2 + \lVert x-y\rVert^2 = 2\lVert x\rVert^2 + 2\lVert y \rVert^2$. Тождество параллелограмма.
    \end{enumerate}
\end{properties}
%END TICKET 31
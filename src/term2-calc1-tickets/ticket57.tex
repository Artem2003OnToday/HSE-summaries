%BEGIN TICKET 57
\begin{theorem}[Признак Коши]
    Пусть $a_n \ge 0$.
    \begin{enumerate}
        \item Если $\sqrt[n]{a_n} \le q < 1$, то ряд сходится.
        \item $\sqrt[n]{a_n} > 1$, то ряд расходится.
        \item  Пусть $\varlimsup \sqrt[n]{a_n} \eqqcolon q^*$. Если  $q^* > 1$, то ряд расходится, если  $q^* < 1$, то ряд сходится.
    \end{enumerate}
\end{theorem}
\begin{remark}
    Если $q^* = 1$, то ряд может сходиться, а может расходиться.  $\sum\limits_{n=1}^\infty \frac{1}{n(n+1)}$ --- сходится, $\sqrt[n]{\frac{1}{n(n+1)}} \to 1$.

    $\sum\limits_{n=1}^\infty \frac{1}{n}$ --- расходится. $\sqrt[n]{a_n} = \frac{1}{\sqrt[n]{n}} \to 1$.
\end{remark}
\begin{proof}
    \begin{enumerate}
        \item $\sqrt[n]{a_n} \le q < 1 \implies a_n \le q^n$. По признаку сравнения с геометрической прогрессией $\sum\limits_{n=1}^\infty q^n$ --- сходится.
        \item  $\sqrt[n]{a_n} \ge 1 \implies a_n \centernot \to 0 \implies $ расходится.
        \item Если $q^* > 1$. Найдется  $n_k\!: \sqrt[n_k]{a_{n_k}} \to q^*  > 1$ (по определению верхнего предела)  $\implies$ начиная с некоторого номера $\sqrt[n_k]{a_{n_k}} > 1 \implies a_{n_k} > 1 \implies a_n \centernot \to 0$ и ряд расходится.

            Если $q^* < 1$,  $q^* = \lim\limits_{n \to \infty} \sup\limits_{k \ge n} \sqrt[k]{a_k} \implies$ для больших $n$  $\sup_{k \ge n} \sqrt[k]{a_k} < q < 1$. Но при этом $\sqrt[n]{a_n} \le \sup\limits_{k \ge n}\sqrt[k]{a_k}$, а значит $\sqrt[n]{a_n} < q$ при больших  $n \implies$ ряд сходится.
    \end{enumerate}
\end{proof}
%END TICKET 57
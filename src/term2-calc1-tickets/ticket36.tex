%BEGIN TICKET 36
\begin{theorem}
    $K_\alpha$ --- семейство компактов, такое что пересечение любого конечного числа из них непусто. Тогда  $\bigcap_{\alpha \in I} K_\alpha \neq \emptyset$.
\end{theorem}
\begin{consequence}
    $K_1 \subset K_2 \subset K_3 \subset \ldots$ непустые компакты. Тогда $\bigcap\limits_{n=1}^\infty K_n \neq \emptyset$.
\end{consequence}
\begin{proof}[Доказательство теоремы.]
    От противного. Пусть $\bigcap\limits_{\alpha \in I} K_\alpha = \emptyset$. Зафиксируем компакт  $K_0 \implies K_0 \cap \bigcap\limits_{\alpha \in I} K_\alpha = \emptyset \implies K_0 \subset X \setminus \bigcap\limits_{\alpha \in I} K_\alpha = \bigcup_{\alpha \in I} X \setminus K_\alpha$ --- открытое покрытие  $K_0$. Выделим конечное подпокрытие  $K_0 \subset \bigcup\limits_{i=1}^n X \setminus K_{\alpha_i} = X \setminus \bigcap\limits_{i=1}^n K_{\alpha_i} \implies K_0 \cap \bigcap\limits_{i=1}^n K_{\alpha_i} = \emptyset$.??!
\end{proof}
%END TICKET 36
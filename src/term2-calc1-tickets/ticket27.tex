%BEGIN TICKET 27
\begin{definition}
    $A \subset X$,  $a \in A$.  $a$ --- внутренняя точка множества  $A$, если $\exists r > 0\!: B_r(a) \subset A$.
\end{definition}
\begin{remark}
    $A$ --- открытое  $\iff$ все его точки внутренние.
\end{remark}
\begin{definition}
    Внутренность множества $\Int a \coloneqq \{ a \in A\mid a\text{ --- внутренняя точка}\}$.
\end{definition}
\begin{example}
    $A = [0, 1] \subset \R$. Тогда  $\Int A = (0, 1)$.
\end{example}
\begin{properties}[внутренности]
    \begin{enumerate}
        \item $\Int A \subset A$.
        \item  $\Int A$ ---  $\bigcup$ всех открытых множеств, которые содержатся в  $A$.
        \item $\Int A$ --- открытое множество. 
        \item  $A$ ---  открытое $\iff A = \Int A$.
        \item Если $A \subset B$, то $\Int A \subset \Int B$.
        \item $\Int(A \cap B) = \Int A \cap \Int B$
        \item $\Int(\Int A) = \Int A$.
    \end{enumerate}
\end{properties}
\begin{proof}
    \slashn
    \begin{enumerate}
        \item[2.] $B \coloneqq \bigcup_{\alpha \in I} A_{\alpha}, A_\alpha \subset A$, $A_\alpha$ открытые.

      $B \subset \Int A$. Берем  $b \in B \implies \exists \beta \in I\!: b \in A_\beta$ --- открытое  $\implies \exists r > 0\!: B_r(b) \subset A_\beta \subset A \implies b$ --- внутренняя точка  $A$  $\implies b \in \Int A$.

      $\Int A \subset B$. Берем  $b \in \Int A \implies \exists r > 0 B_r(b) \subset A$, но  $B_r(b)$ --- открытое множество $\implies $ оно участвует в объединении  $\bigcup\limits_\alpha A_\alpha \implies B_r(b) \subset B \implies b \in B$.

      \item[4.] $\Leftarrow$: пользуемся пунктом 3.  \\$\Rightarrow:$ всего его точки внутренние  $\implies A = \Int A$.

      \item[6.] $\subset$:  $A \cap B \subset A, \subset B \implies \Int(A \cap B) \subset \Int A \land \Int(A \cap B) \subset \Int B$.

      $\supset$. Пусть $x \in \Int A \cap \Int B \implies \begin{cases} \exists r_1 > 0 \quad B_{r_1}(x) \subset A \\ \exists r_2 > 0 \quad B_{r_2}(x) \subset B \end{cases} \implies$ если $r = \min \{r_1, r_2 \} \implies B_r(x) \subset A \land B_r(x) \subset B \implies B_r(x) \subset A \cap B \implies x \in \Int(A \cap B)$.

      \item[7.] $B \coloneqq \Int A$ --- открытое $\implies B = \Int B$.
    \end{enumerate}
\end{proof}
%END TICKET 27
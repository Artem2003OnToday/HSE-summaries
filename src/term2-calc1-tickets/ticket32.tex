%BEGIN TICKET 32
\begin{definition}
    $(X, \rho)$ --- метрическое пространство.  $x_1, x_2, \ldots \in X, a \in X$.

    $\lim x_n = a$, если
     \begin{enumerate}
         \item Вне любого открытого шара с центром в точке  $a$ содержится лишь конечное число членов последовательности.
         \item  $\forall \eps > 0 \exists N \forall n \ge N\quad \rho(x_n, a) < \eps \iff x_n \in B_\eps(a)$.
    \end{enumerate}
\end{definition}
\begin{definition}
    $A \subset X$. 

    Тогда  $A$ --- ограничено, если оно содержится в некотором шаре.
\end{definition}
\begin{properties}
    \begin{enumerate}
        \item $a = \lim x \iff \rho(x_n, a) \to 0$.
\begin{proof}
             $\forall \eps > 0 \exists n > N\quad |\rho(x_n, a)| < \eps$ --- предел равен 0.
\end{proof}
        \item Предел единственный. 
            \begin{proof}
                Пусть $a = \lim x_n$ и $b = \lim x_n$. Тогда возьмем шарики такие, что  $B_r(a) \cap B_r(b) = \emptyset \implies \exists N_1, N_2, \forall n \ge \max\{N_1, N_2\}\ x_n \in B_r(a) \land x_n \in B_r(b)$ --- противоречие.
            \end{proof}
        \item Если $a = \lim x_n, a = \lim y_n$. То для перемешанной последовательности $x_n$ и  $y_n$ предел такой же.
        \item  $a = \lim x_n \implies $ для последовательности, в которой $x_n$ взяты с конечной кратностью, $a$ будет пределом.
        \item Если $a = \lim x_n$, то  $\lim x_{n_k} = a$.
        \item Последовательность имеет предел $\implies$ она ограничена
             \begin{proof}
                $\eps = 1 \exists N \forall n \ge N \rho(x_n, a) < 1$. Тогда $R = \max\{\rho(x_1, a), \ldots, \rho(x_{N-1}, a)\} + 1 \implies x_n \in B_R(a)$.
            \end{proof}
        \item Если $a = \lim x_n$, то последовательность, полученная из  $\{x_n\}$ перестановкой членов имеет тот же предел.
        \item $a$ --- предельная точка  $A \iff \exists \{x_n\} \neq a \in A\!: \lim x_n = a$.

            Более того,  $x_n$ можно выбирать так, что  $\rho(x_n, a)$ строго убывает.
            \begin{proof}
                "$\Leftarrow$" Пусть  $\lim x_n = a$. Возьмем  $B_r(a) \implies \exists N \forall n \ge N x_n \in B_r(a) \implies \exists x_n \in \dot{B_r}(a) \implies \dot{B_r}(a) \cap A \neq \emptyset \implies a$ --- предельная точка.

                "$\Rightarrow$" Берем  $r_1 = 1$. $\dot{B_{r_1}}(a) \cap A \neq \emptyset$. Берем оттуда точку, называем  $x_1 \neq a$. $r_2 = \frac{\rho(x_1,a)}{2}$. $\dot{B_{r_2}}(a) \cap A \neq \emptyset$. Берем оттуда точку $x_3 \neq a$. $r_3 = \frac{\rho(x_2, a)}{2}$. И так далее.

                Получили: $x_n \neq a$ и  $\rho(x_n, a) < \frac{\rho(x_{n-1}, a)}{2} < \rho(x_{n-1}, a)$. $\rho(x_n, a) < \frac{1}{2^n} \to 0 \implies x_n = a$.
            \end{proof}
    \end{enumerate}
\end{properties}
%END TICKET 32
%BEGIN TICKET 25
\begin{definition}
    Метрика (расстояние) $\rho\!: X \times X \to [0;+\infty)$, если выполняются следующие условия:
     \begin{enumerate}
         \item $\rho(x, y) = 0 \iff x = y$,
         \item $\rho(x, y) = \rho(y, x)$,
         \item  (неравенство треугольника) $\rho(x, z) \le \rho(x, y) + \rho(y, z)$.
    \end{enumerate}
\end{definition}
\begin{definition}
    Метрическое пространство --- пара $(X, \rho)$.
\end{definition}
\begin{example}
    Дискретная метрика (метрика Лентяя) $\rho(x, y) = \begin{cases} 0, & x = y \\ 1 & x \neq y\end{cases}$
\end{example}
\begin{example}
    На $\R$:  $\rho(x, y) = |x-y|$.
\end{example}
\begin{example}
    На $\R^d$:  $\rho(x, y) = \sqrt{\sum\limits_{k=1}^d (x_k - y_k)^2}$. Неравенство треугольника здесь --- неравенство Минковского.
\end{example}
\begin{example}
    $C[a, b]$.  $\rho(f, g) = \int\limits_a^b |f-g|$.

    Неравенство треугольника:  \begin{align*}
        \rho(f, h) = \int\limits_a^b |f-h| &\overset{(*)}{\le} \int_a^b(|f-g|+|g-h|) = \rho(f, g) + \rho(g, h).\\
        (*) \iff |f(x)-h(x)| \le |f(x)-g(x)| &+ |g(x) - h(x)|\ \text{--- неравенство треугольника для }(\R, \left| x - y \right|)
    .\end{align*}
\end{example}
\begin{example}
    Манхэтеннская метрика: $\R^2$,  $\rho((x_1, y_1), (x_2, y_2)) = |x_1 - x_2| + |y_1 - y_2|$.
\end{example}
\begin{example}
    Французская железнодорожная метрика. $\R^2$. Есть точка  $P$ (Париж), тогда  $\rho(A, B) = AB$, если  $A, B,P$ на одной прямой, иначе  $\rho(A, B) = |AP|+|PB|$. 
\end{example}
\begin{definition}
    $(X, \rho)$ --- метрическое пространство.  $B_r(x) \coloneqq \{y \in X \mid \rho(x, y) < r\}$ --- открытый шар радиуса  $r$ с центром в точке  $x$. 
\end{definition}
\begin{definition}
    $(X, \rho)$ --- метрическое пространство.  $\overline{B}_r(x) \coloneqq \{y \in X \mid \rho(x, y) \le r\}$ --- закрытый шар радиуса  $r$ с центром в точке  $x$. 
\end{definition}
\begin{properties}
    \begin{enumerate}
        \item $B_{r_1}(a) \cap B_{r_2}(a) = B_{\min\{r_1, r_2\}}(a)$.
        \item $x \neq y \implies \exists r > 0\!: B_r(x) \cap B_r(y) = \emptyset \land \overline{B}_r(x) \cap \overline{B}_r(y) = \emptyset$.
    \end{enumerate}
\end{properties}
\begin{proof}
    \begin{enumerate}
        \item $x \in B_{r_1}(a) \cap B_{r_2}(a) \iff \begin{cases} \rho(x, a) < r_1 \\ \rho(x, a) < r_2 \end{cases} \iff \rho(x, a) < \min\{r_1, r_2\} \implies x \in B_{\min\{r_1, r_2\}}(a)$.
        \item $r \coloneqq \frac{1}{3} \rho(x, y) > 0$. Пусть $\overline{B}_r(x) \cap \overline{B}_r(y) \neq \emptyset$. 

            Тогда  $\exists z \in \overline{B}_r(x) \cap \overline{B}_r(y) \implies \rho(x, z) \le r \land \rho(y, z) \le r \implies \rho(x, y) \le \rho(x, z) + \rho(z, y) \le 2r = \frac{2}{3} \rho(x, y) \implies 1 \le \frac{2}{3}$. Противоречие.

            При этом, $B_r(x) \subset \overline{B}_r(x) \implies \exists r\!: B_r(x) \cap B_r(y) = \emptyset$.
    \end{enumerate}
\end{proof}
%END TICKET 25
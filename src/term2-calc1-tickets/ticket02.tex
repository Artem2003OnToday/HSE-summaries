%BEGIN TICKET 02
\begin{example}
    \slashn
     \begin{enumerate}
         \item $\sigma_1(E) = \inf \left\{\sum\limits_{k=1}^n |P_k|\!: P_k\text{ --- прямоугольник}, \bigcup\limits_{k=1}^n P_k \supset E\right\}$.
         \item $\sigma_2(E) = \inf \left\{\sum\limits_{k=1}^n |P_k|\!: P_k\text{ --- прямоугольник}, \bigcup\limits_{k=1}^\infty P_k  \supset E\right\}$.
    \end{enumerate}
\end{example}
\begin{exerc}
    \slashn
    \begin{enumerate}
        \item Доказать, что $\forall E\ \ \sigma_1(E) \ge \sigma_2(E)$.
        \item $E = \left([0, 1] \cap \Q\right) \times \left([0, 1] \cap \Q\right)$. Доказать, что  $\sigma_1(E) = 1, \sigma_2(E) = 0$.
    \end{enumerate}
\end{exerc}
\begin{theorem}
    \slashn
     \begin{enumerate}
         \item $\sigma_1$ --- квазиплощадь.
         \item Если  $E'$ --- сдвиг  $E$, то  $\sigma_1(E) = \sigma_1(E')$.
     \end{enumerate}
\end{theorem}
\begin{proof}
    \slashn
    \begin{enumerate}
        \item[2.] $E'$ --- сдвиг  $E$ на вектор  $v$. Пусть  $P_k$ --- покрытие  $E \iff P'_k$ --- покрытие  $E'$. Знаем, что площади прямоугольников не меняются при сдвиге, а значит:
            $\sigma_1(E) = \inf\{\sum\limits_{k=1}^n|P_k|\} = \inf\{\sum |P'_k|\} = \sigma_1(E')$.
        \item[1.] $\Rightarrow$ монотонность. Пусть есть  $E \subset \widetilde{E}$. Тогда возьмем покрытие  $P_k$ для  $\widetilde{E}$.  $E \subset \widetilde{E} \subset \bigcup\limits_{k=1}^n P_k$. 

		А теперь заметим, что $\sigma_1$ ---  $\inf$, и любое покрытие для $\widetilde{E}$ является покрытием и для $E$, т.е. все суммы из $\sigma_1(\widetilde{E})$ есть в $\sigma_1(E)$, а значит $\sigma_1(E) \le \sigma_1(\widetilde{E})$ как инфинум по более широкому множеству. 

        \item[1'.] Докажем теперь аддитивность. 

            <<$\le$>>. $\sigma_1(E) = \sigma_1(E_-) + \sigma_1(E_+)$. Пусть $P_k$ --- покрытие  $E_-$,  $Q_j$ --- покрытие  $E_+$.  $\bigcup\limits_{k=1}^n P_k \cup \bigcup\limits_{j=1}^m Q_j \supset E_- \cup E_+ = E$. А значит  $\sigma_1(E) \le \inf \left\{ \sum\limits_{k=1}^n |P_k| + \sum\limits_{j=1}^n |Q_j|\right\} = \inf\{\sum |P_k|\} + \inf\{\sum |Q_j|\} = \sigma_1(E_-) + \sigma(E_+)$. Заметим, Что переход с разделением инфинумов  возможен, так как $P$ и  $Q$ выбираются независимо.

	    <<$\ge$>>. Пусть $P_k$ --- покрытие  $E$. Тогда можно пересечь прямой (покрытие и само $E$) и разбить $P_k$ на $P_k^-$ и $P_k^+$, а тогда: $|P_k| = |P^-_k| + |P^+_k|$,  $\sum |P_k| = \sum |P^-_k| + \sum |P^+_k|$. $\sum |P_k^-| \ge \sigma_1(E_-), \sum |P_k^+| \ge \sigma_1(E^+) \Rightarrow \sum |P_k| \ge \sigma_1(E_-) + \sigma_1(E_+)$ для любого покрытия $P_k$, а значит и $\sigma_1(E) \ge \sigma_1(E_-) + \sigma_1(E_+)$

	    Таким образом $\sigma_1(E) = \sigma_1(E_-) + \sigma_1(E_+)$
        \item[1''.] Проверим, что сама площадь прямоугольника не сломалась: $\sigma_1([a, b] \times [c, d]) = (b-a)(d-c)$. Заметим, что  $\sigma_1(P) \le |P|$, т.к. прямоугольник можно покрыть им самим.

		Тогда посмотрим на $P_k$. Проведем прямые содержащие все стороны прямоугольников из покрытия (и $P$). Заметим, что такими прямыми каждый прямоугольник разбивается на подпрямоугольники, сумма площадей которых равна площади исходного прямоугольника. Тогда заметим, что и площадь $P$ это сумма <<кусочков из нарезки>> $P$, и некоторые части разбиения встречаются в  $P_k$ несколько раз. А значит выкинув все лишнее мы как раз получим  $|P|$, а значит  $\sigma_1(P) \ge |P|$.
		Таким образом $\sigma_1(P) = |P|$
    \end{enumerate}
\end{proof}
%END TICKET 02
%BEGIN TICKET 37
\begin{definition}
    $K$ --- секвенциально компактное множество, если из любой последовательности точек из $K$ можно выделить подпоследовательность, которая сходится к какой-то точке из  $K$.
\end{definition}
\begin{example}
    $[a, b] \in \R$ секвенциально компактно.

     $x_n \in [a; b] \xRightarrow{\text{Т. Б-В}} \exists$ подпоследовательность $x_{n_k}$, имеющая предел  $\implies \lim x_{n_k} \in [a, b]$, так как неравенства сохраняются.
\end{example}
\begin{theorem}
    Бесконечное подмножество компакта имеет предельную точку.
\end{theorem}
\begin{proof}
    $K$ --- компакт.  $A \subset K$. Пусть  $A'$ (предельные точки) $= \emptyset$. Тогда  $A$ --- замкнуто  $\implies A$ --- компакт и ни одна из его точек не является предельной.  $a \in A$ не предельная  $\implies \exists r_a > 0\ \dot{B_{r_a}}(a) \cap A = \emptyset \implies B_{r_a}(a) \cap A = \{a\}$. Рассмотрим открытое покрытие  $A \subset \bigcup_{a \in A} B_{r_a}(a)$, но из этого покрытия нельзя убрать ни одного множества, так как мы выбрали радиусы так, что каждый шар в пересечении с $A$ дает только одну точку $\implies$ нет конечного подпокрытия  $\implies$ противоречие.
\end{proof}
\begin{consequence}
    Компактность $\implies$ секвенциальная компактность. 
\end{consequence}
\begin{proof}
    $x_1, x_2, \ldots \in K$. $D = \{ x_1, x_2, x_3,\ldots\}$ --- множество значений последовательности. 

    \begin{enumerate}
        \item $|D| < +\infty \implies$ в последовательности есть элемент, повторяющийся бесконечно много раз, оставим только его --- это нужная подпоследовательность.
        \item $|D| = +\infty \implies$ у  $D$ есть предельная точка.

            Пусть  $a$ --- предельная точка  $D \implies$ найдутся различные $y_1, y_2, \ldots \in D$, такие что $\lim y_n = a$. 

            Но $y_i$ --- это какой-то  $x_{n_i}$ и $\lim x_{n_i} = a$. Осталось переставить  $x_{n_i}$ так, что получится подпоследовательность. Ну, а так как  $K$ --- замкнуто, то  $a \in K$.
    \end{enumerate}
\end{proof}
%END TICKET 37
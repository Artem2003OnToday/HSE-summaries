%BEGIN TICKET 05
\begin{theorem}[Аддитивность интеграла]
    Пусть $f\!: [a, b] \to \R$,  $c \in [a, b]$.

    Тогда  $\int\limits_a^b f = \int\limits_a^c f + \int\limits_c^b f$.
\end{theorem}
\begin{proof}
	$\int\limits_a^b f = \sigma(P_{f_+}([a, b])) - \sigma(P_{f_-}([a, b]))$. Разделим наш $[a, b]$ и соответствующие множества вертикальной прямой $x=c$. Тогда $\sigma(P_{f_+}[a, b]) - \sigma(P_{f_-}[a, b]) = \sigma_{P_{f_+}[a, c]} + \sigma_{P_{f_+}[c, b]} - \sigma(P_{f_-}[a, c]) - \sigma(P_{f_-}[c, b]) = \int_a^c f + \int_c^b f$
\end{proof}
\begin{theorem}[Монотонность интеграла]
    Пусть $f, g\!: [a, b] \to \R$ и  $\forall x \in [a, b]\!: f(x) \le g(x)$.

    Тогда $\int\limits_a^b f \le \int\limits_a^b g$. 
\end{theorem}
\begin{proof}
    $f_+ = \max\{f, 0\} \le \max\{g, 0\} = g_+ \Rightarrow P_{f_+} \subset P_{g_+} \Rightarrow \sigma(P_{f_+}) \le \sigma(P_{g_+})$.

    $f_- = \max\{-f, 0\} \ge \max\{-g, 0\} = g_- \Rightarrow P_{f_-} \supset P_{g_-} \Rightarrow \sigma(P_{f_-}) \ge \sigma(P_{g_-})$.
\end{proof}
\begin{consequence}
    \begin{enumerate}
        \item $|\int\limits_a^b f| \le \int\limits_a^b |f|$ 
        \item $(b-a)\min\limits_{x \in [a, b]} f(x) \le \int\limits_a^b f \le (b-a)\max\limits_{x \in [a, b]} f(x)$.
    \end{enumerate}
\end{consequence}
\begin{proof}
    \begin{enumerate}
        \item $-|f| \le f \le |f| \Rightarrow \int\limits_a^b -|f| \le \int\limits_a^b f \le \int\limits_a^b |f| \Rightarrow |\int\limits_a^b f| \le \int\limits_a^b |f|$
	\item $m \coloneqq \min\limits_{x \in [a, b]} f(x)$,  $M \coloneqq \max\limits_{x \in [a, b]} f(x)$.  $m \le f(x) \le M \Rightarrow \int\limits_a^b m \le \int\limits_a^b f \le \int\limits_a^b M$.
    \end{enumerate}
\end{proof}
\begin{theorem}[Интегральная теорема о среднем]
    Пусть $f \in C([a, b])$.

    Тогда  $\exists c \in (a, b)\!: \int\limits_a^b f = (b-a)f(c)$.
\end{theorem}
\begin{proof}
	$m \coloneqq \min f = f(p), M \coloneqq \max f = f(q)$ (по теореме Вейерштрасса). Тогда  $f(p) \le \frac{1}{b-a}\int\limits_a^b f \le f(q) \xRightarrow{\text{т. Б-К}} \exists c \in (p, q)  \text{или $(q, p)$}\!: f(c) = \frac{1}{b - a}\int\limits_a^b f$.
\end{proof}
\begin{definition}
    $I_f \coloneqq \frac{1}{b-a} \int\limits_a^b f$ --- среднее значение $f$ на отрезке  $[a, b]$.
\end{definition}
%END TICKET 05
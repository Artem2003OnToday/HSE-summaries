%BEGIN TICKET 34
\begin{definition}
    $x_n \in X$ --- фундаментальная последовательность, если
     \[
    \forall \eps > 0 \exists N \forall m, n \ge N\!: \rho(x_n, x_m) < \eps
    .\] 
\end{definition}
\begin{properties}
    \begin{enumerate}
        \item Сходящаяся последовательность фундаментальна.
        \item Фундаментальная последовательность ограничена.
        \item Если у последовательности есть сходящаяся подпоследовательность, то последовательность имеет предел.
    \end{enumerate}
\end{properties}
\begin{proof}
    Упражнение! Утверждается, что так же, как и в пределах.
\end{proof}
\begin{definition}
    $(x, \rho)$ --- метрическое пространство --- полное, если любая фундаментальная последовательность имеет предел.
\end{definition}
\begin{example}
    $\R\!:$,  $\rho(x, y) = |x-y|$ --- полное.
\end{example}
\begin{exerc}
    $(X, \rho)$ --- полное метрическое пространство  $X \supset Y$ замкнуто. Доказать, что  $(Y, \rho)$ --- полное.
\end{exerc}
\begin{exerc}
    $(0, 1)$ не полное.  $x_n = \frac{1}{n}$ --- фундаментальная, но $\lim \frac{1}{n} = 0 \notin (0; 1)$.
\end{exerc}
\begin{theorem}
    $\R^d$ --- полное.
\end{theorem}
\begin{proof}
    Пусть $x_n$ --- фундаментальная, то есть
     \[
    \forall \eps > 0 \exists N \forall m,n \ge N\!: \rho(x_n, x_m) = \sqrt{(x_n^{(1)}-x_m^{(1)})^2 + \ldots + (x_n^{(d)}-x_m^{(d)})^2} < \eps
    .\] 

    Но мы знаем, что $\rho(x_n, x_m) \ge |x_n^{(k)} - x_m^{(k)}|$, так как, могут быть еще координаты, а значит еще неотрицательные слагаемые. 

    Тогда заметим, что $x_n^{(k)}$ --- фундаментальная  $\implies \exists a^{(k)} = \lim\limits_{n \to \infty} x_n^{(k)}$. Значит и  $x_n$ сходится к  $a$ покоординатно  $\implies \rho(x_n, a) \to 0 \implies x_n$ сходится к  $a$ по метрике.
\end{proof}
%END TICKET 34
%BEGIN TICKET 29
\begin{definition}
    Окрестностью точки $x$ будем называть шар  $B_r(x)$ для некоторого  $r > 0$. Обозначать будем $U_x$
\end{definition}
\begin{definition}
    Проколотой окрестностью точки $x$ ---  $B_r(x) \setminus \{x\}$. Обозначать будем $\dot{U}_x$.
\end{definition}

\begin{definition}
    $x$ --- предельная точка множества  $A$, если  $\forall \dot{U_x}\!: \dot{U_x} \cap A \neq \emptyset$.

    Обозначим через  $A'$ --- множество предельных точек для  $A$.
\end{definition}
\begin{properties}
    \slashn
    \begin{enumerate}
        \item $\Cl A = A \cup A'$.
            \begin{proof}
                $x \in \Cl A \iff \forall U_x\!: U_x \cap A \neq \emptyset \iff \left[ \begin{array}{l} x \in A \\ \forall \dot{U_x} \cap A \neq \emptyset \iff x \in A' \end{array} \right.$
            \end{proof}
        \item $A \subset B \implies A' \subset B'$. Очевидно.
        \item  $A$ --- замкнуто  $\iff A \supset A'$. 
             \begin{proof}
                $A$ --- замкнуто  $\iff A = \Cl A \iff A = A \cup A' \iff A \supset A'$.
            \end{proof}
        \item $(A \cup B)' = A' \cup B'$.
             \begin{proof}
                Докажем "$\subset$". Возьмем $x \in (A \cup B)'\!: x \notin A' \implies \exists \dot{U_x}\!: \dot{U_x} \cap A = \emptyset$, но $\dot{U_x} \cap (A \cup B) \neq \emptyset \implies \dot{U_x} \cap B \neq \emptyset \implies x \in B'$.

                Докажем "$\supset$". $A \cup B \supset A \implies (A \cup B)' \supset A'$. Провернем тот же фокус для  $B$, получим  $(A \cup B)' \supset A' \cup B'$.
           \end{proof}
    \end{enumerate}
\end{properties}
\begin{theorem}
    $x \in A' \iff \forall r > 0$  $B_r(x)$ содержит бесконечное количество точек из  $A$.
\end{theorem}
\begin{proof}
    Докажем "$\Leftarrow$". $B_r(x) \cap A$ содержит бесконечное количество точек  $\implies \dot{B_r}(x) \cap A$ содержит бесконечное число точек  $\implies \dot{B_r}(x) \cap A \neq \emptyset \Rightarrow x \in A'$.

     "$\Rightarrow$". Возьмем радиус  $r = 1$. Тогда  $\dot{B_r}(x) \cap A \neq \emptyset \implies \exists x_1 \in A\!: 0 < \rho(x, x_1) < 1$. Возьмем $r = \rho(x, x_1)$ $\dot{B_r}(x) \cap A \neq \emptyset \implies \exists x_2 \in A\!: 0 < \rho(x, x_2) < \rho(x, x_1)$. Тогда можно взять $r = \rho(x, x_2)$, и так далее. 

     В итоге получили, что $r > \rho(x, x_1) > \rho(x, x_2) > \rho(x, x_3) > \ldots > 0 \implies$ все $x_n$ различны.
\end{proof}
\begin{consequence}
     Конечное множество не имеет предельных точек. (Потому что их должно быть $\infty$)
\end{consequence}
\begin{proof}
     Предположим предельная точка существует $\iff \exists r > 0\!: B_r(x) \cap A$ содержит бесконечное количество точек. Но это невозможно, так как в $A$ конечное число точек.
\end{proof}
%END TICKET 29
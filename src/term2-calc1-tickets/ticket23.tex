%BEGIN TICKET 23
\begin{definition}
    $f \in C[a, b)$. $\int\limits_a^b f$ абсолютно сходится, если  $\int\limits_a^b |f|$ сходится.
\end{definition}
\begin{theorem}
    $\int\limits_a^b f$ сходится абсолютно  $\implies \int\limits_a^b f$ сходится.
\end{theorem}
\begin{proof}
    $f = f_+ - f_-$,  $|f| = f_+ + f_-$.  $|f| \ge f_\pm \ge 0$. Если $\int\limits_a^b f$ сходится абсолютно  $\implies \int\limits_a^b |f|$ сходится  $ \Rightarrow \int\limits_a^b f_{\pm}$ сходится  $\implies \int\limits_a^b f = \int\limits_a^b f_+ - \int\limits_a^b f_-$ сходится.
\end{proof}
\begin{theorem}[Признак Дирихле]
    $f, g \in C[a, +\infty)$. Если
    \begin{enumerate}
        \item $f$ имеет ограниченную первообразную на  $[a, +\infty)$ (то есть  $\left| \int\limits_a^y f(x) \mathrm{d}x \right| \le K \quad \forall y$)
        \item $g$ монотонна
        \item  $\lim\limits_{x \to +\infty} g(x) = 0$
    \end{enumerate} $\Rightarrow$ то $\int\limits_a^{+\infty} f(x)g(x) \mathrm{d}x$ сходится.
\end{theorem}
\begin{proof}
    Только для случая $g \in C^1[a; +\infty)$. 

    Надо доказать, что  $\exists$ конечный  $\lim\limits_{y \to +\infty} \int\limits_a^y f(x)g(x) \mathrm{d}x$,  $F(y) \coloneqq \int\limits_a^y f(x)\mathrm{d}x$. 
    \begin{align*}
        \int\limits_a^y f(x)g(x)\mathrm{d}x = \int\limits_a^y F'(x)g(x) \mathrm{d}x = F(x)g(x) \Big|_a^y - \int_a^y F(x) g'(x) \mathrm{d}x = F(y)g(y) - \int_a^y F(x)g'(x) \mathrm{d}x.
    \end{align*}

    Чтобы доказать существование предела у разности каких-то штук, нужно доказать, что он существует у них по отдельности.

    $\lim\limits_{y \to +\infty} F(y)g(y) = 0$ --- произведение бесконечно малой и ограниченной функции.

    Хотим показать, что $\int\limits_a^y F(x)g'(x)\mathrm{d}x$ имеет конечный  $\lim$, то есть  $\int\limits_a^{+\infty} F(x)g'(x) \mathrm{d}x$ сходится.

    Тогда докажем, что он абсолютно сходится.  $\int\limits_a^{+\infty} |F(x)| |g'(x)| \mathrm{d}x$, $|F(x)||g'(x)| \le K|g'(x)| = Kg'(x)$. (считаем, что $g(x)$ возрастает) $\int_a^{+\infty} g'(x) \mathrm{d}x = g \mid_a^{+\infty} = \lim\limits_{y \to +\infty} g(y) - g(a) = -g(a) \implies$ сходится.
\end{proof}
\begin{theorem}[Признак Абеля]
    $f, g \in C[a, +\infty)$, Если 
     \begin{enumerate}
         \item $\int\limits_a^{+\infty} f(x) \mathrm{dx}$ сходится
         \item  $g$ монотонна
         \item  $g$ ограничена
    \end{enumerate}
    $\Rightarrow$ то $\int\limits_a^{+\infty} f(x)g(x) \mathrm{d}x$ сходится.
\end{theorem}
\begin{proof}
    $2)+3) \implies g$ имеет конечный предел $l \in \R \coloneqq \lim\limits_{x \to +\infty} g(x)$.

    Пусть $\widetilde{g}(x) \coloneqq g(x) - l \implies \lim\limits_{x \to +\infty}\widetilde{g}(x) = 0$ и  $\widetilde{g}$ монотонна.

    Пусть  $F(x) \coloneqq \int\limits_a^x f(t) \mathrm{d}t$. Тогда $1) \iff$ существует конечный предел  $\lim\limits_{x \to +\infty} F(x) \implies$ $F$ ограничена.

    Тогда  $f$ и  $\widetilde{g}$ удовлетворяют условиям признака Дирихле $\implies \int\limits_a^{+\infty} f(x) \widetilde{g}(x) \mathrm{d}x$ --- сходится. Тогда: \[
    \int\limits_a^{+\infty} fg = \int\limits_a^{+\infty} f(\widetilde{g}+l) = \int\limits_a^{+\infty} f\widetilde{g} + l \int\limits_a^{+\infty} f
    .\] 
    Где $\int\limits_a^{+\infty} f\widetilde{g}$ сходится по доказанному, а  $\int\limits_a^{+\infty} f$ --- по условию.
\end{proof}
%END TICKET 23
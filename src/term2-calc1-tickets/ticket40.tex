%BEGIN TICKET 40
\begin{theorem}[Хаусдорфа]
    \begin{enumerate}
        \item Компактное множество вполне ограничено.
        \item Если $(X, \rho)$ --- полное метрическое пространство, то замкнутое вполне ограниченное подмножество  $X$ --- компактно.
    \end{enumerate}
\end{theorem}
\begin{proof}
    \begin{enumerate}
        \item Берем $\eps > 0$  $K \subset \bigcup_{x \in K} B_\eps(x)$ --- открытое покрытие. Выделим конечное подпокрытие  $\implies K \subset \bigcup\limits_{i=1}^n B_\eps(x_i) \implies x_1, \ldots, x_n$  --- $\eps$-сеть.
        \item Проверим секвенциальную компактность. Берем  $x_1, x_2, \ldots \in K$. Возьмем $1$-сеть  $K \subset \bigcup_{i=1}^{n_1} B_1(y_{1i})$.

            В каком-то шарике $B_1(z_1)$ бесконечное число членов последовательности. Выкинем все, кроме них, останутся  $x_{11}, x_{12}, x_{13},\ldots$. Возьмем $\frac{1}{2}$-сеть. $K \subset \bigcup\limits_{i=1}^{n_2} B_{\frac{1}{2}}(y_{2i})$. В каком-то шарике $B_{\frac{1}{2}(z_2)}$ бесконечное число членов последовательности\dots

            На $j$-ом шаге  $K \subset B_{\frac{1}{j}}(y_{ji})$. Пусть на каждом шаге выбирали шарик $B_{\frac{1}{i}}(z_i)$.

            В итоге получили:
            \[
            \begin{array}{cccccc}
                x_{11} & x_{12} & x_{13} & x_{14} & \ldots & B_1(z_1)\\
                x_{21} & x_{22} & x_{23} & x_{24} & \ldots & B_{\frac{1}{2}}(z_2)\\
                x_{31} & x_{32} & x_{33} & x_{34} & \ldots & B_{\frac{1}{3}}(z_3)\\
                x_{41} & x_{42} & x_{43} & x_{44} & \ldots & B_{\frac{1}{4}}(z_4)\\
            \end{array}            \] 
            Воспользуемся диагональным методом Кантора. Пусть $a_n \coloneqq x_{nn}$. Заметим, что  $a_n, a_{n+1}, a_{n+2},\ldots$ --- подпоследовательность  $x_{n1}, x_{n 2}, x_{n 3},\ldots \implies$ все лежат в $B_{\frac{1}{n}}(z_n) \implies \rho(a_i, a_j) \le \rho(a_i, z_n) + \rho(a_j, z_n) < \frac{1}{n} + \frac{1}{n} = \frac{2}{n}$, при $i, j \ge n \implies a_i$ --- фундаментальная $\implies$ у нее есть предел  $\implies a = \lim a_n \in K$,  так как $K$ --- замкнуто $\implies K$ --- секвенциально компактно.
    \end{enumerate}
\end{proof}
\begin{consequence}[Характеристика компактов в $\R^d$]
    $K \subset \R^d. K$ --- компакт  $\iff$  $K$ --- замкнуто и ограничено. 
\end{consequence}
\begin{proof}
    $\Rightarrow$ верна всегда и доказана выше.

    А вот  $\Leftarrow$ верна не всегда. Поэтому докажем эту штуку для  $\R^d$. Мы знаем, что  $\R^d$ --- полное. А еще мы знаем, что в  $\R^d$ ограниченность  $\implies$ вполне ограниченность, а значит понятно, что  $K$ --- компакт.
\end{proof}
\begin{exerc}
    $(K, \rho)$ --- метрическое пространство,  $K$ --- компакт. Доказать, что  $(K, \rho)$ --- полное.
\end{exerc}

\begin{theorem}[Теорема Больцано-Вейерштрасса в $\R^d$]
    Из любой ограниченной последовательности в $\R^d$ можно выбрать сходящуюся подпоследовательность.
\end{theorem}
\begin{proof}
    $\{x_n\}$ --- ограничено $\implies \exists R\quad x_n \in B_R(a) \subset \overline{B}_R(a)$ --- замкнуто и ограничено $\implies$ компактно  $\implies$ секвенциально компактно $\implies x_n$ --- последовательность точек секвенциального компакта  $\implies$ у нее есть сходящаяся подпоследовательность.
\end{proof}
%END TICKET 40
%BEGIN TICKET 01
Пусть $\mathcal{F}$ --- совокупность (множество) ограниченных плоских фигур. 
\begin{definition}
    Площадь: $\sigma\!: \mathcal{F} \to [0; +\infty)$, причём 
     \begin{enumerate}
         \item $\sigma([a; b] \times [c, d]) = (b - a)(d - c)$
         \item (Аддитивность).  $\forall E_1, E_2 \in \mathcal{F}\!: E_1 \cap E_2 = \emptyset \Rightarrow \sigma(E_1 \cup E_2) = \sigma(E_1) + \sigma(E_2)$
    \end{enumerate}
\end{definition}
\begin{property}[Монотонность площади]
    $\forall E, \widetilde{E}\!: E \subset \widetilde{E} \Rightarrow \sigma(E) \le \sigma(\widetilde{E})$.
\end{property}
\begin{proof}
    $\widetilde{E} = E \cup (\widetilde{E} \setminus E) \Rightarrow \sigma(\widetilde{E}) = \sigma(E) + \sigma(\widetilde{E} \setminus E)$.
\end{proof}
\begin{definition}
    Псевдоплощадь: $\sigma\!: \mathcal{F} \to [0; +\infty]$, причём
     \begin{enumerate}
         \item $\sigma([a; b] \times [c, d]) = (b - a)(d - c)$,
         \item $\forall E, \widetilde{E} \in \mathcal{F}\!: E \subset \widetilde{E} \Rightarrow \sigma(E) \le \sigma(\widetilde{E})$, 
         \item Разобьем $E$ вертикальной или горизонтальной прямой, в том числе теми прямыми, которые правее или левее $E$. Тогда  $E = E_- \cup E_+, E_- \cap E_+ = \emptyset$ и  $\sigma(E) = \sigma(E_-) + \sigma(E_+)$.
    \end{enumerate}
\end{definition}
\begin{properties}
    \begin{enumerate}
        \item Подмножество вертикального или горизонтального отрезка имеет нулевую площадь.
        \item В определении $E_-$ и  $E_+$ не важно  куда относить точки из  $l$.
            \begin{proof}
                Заметим, что $\sigma(E_- \cup (E \cap l)) = \sigma(E_- \setminus l) + \underbrace{\sigma(E \cap l)}_{=0} \Rightarrow$ вообще не имеет разницы куда относить точки из  $l$.
            \end{proof}
    \end{enumerate}
\end{properties}
%END TICKET 01
%BEGIN TICKET 19
\begin{example}[Формула Стирлинга]
    $m = 1, f(t) = \ln t, f''(t) = -\frac{1}{t^2}$.

    \[
        \ln n! = \sum_{k=1}^n \ln k = \underbrace{\frac{\ln 1 + \ln n}{2}}_{= \frac{1}{2} \ln n} + \underbrace{\int_1^n \ln t \mathrm{d}t}_{\mathclap{=t\ln t-t\mid_1^n = n\ln n - n + 1}} - \underbrace{\frac{1}{2} \int_1^n \frac{\{t\}(1-\{t\})}{t^2} \mathrm{d}t}_{\coloneqq b_n} \Rightarrow \ln n! = \frac{1}{2} \ln n + n \ln n - n + 1 - b_n
    .\] 
    Посмотрим на $b_n$:  \[
        b_n \le \frac{1}{2} \int_1^n \frac{\mathrm{d}t}{t^2} = \frac{1}{2} (-\frac{1}{t}) \mid_1^n = \frac{1}{2} (1-\frac{1}{n}) < \frac{1}{2} \implies b_n = \underbrace{b}_{=\lim b_n} + o(1)
    .\] 

    А значит $\ln n! = n \ln n - n + \frac{1}{2} \ln n + (1-b) + o(1)$. \\
    Можем найти $b$, для этого представим обе части как экспоненты: $n! = n^n e^{-n} \sqrt{n} e^{1-b} e^{o(1)} \sim n^n e^{-n} \sqrt{n} C$.

    Вспомним (из следствия формулы Валлиса): $\binom{2n}{n} \sim \frac{4^n}{\sqrt{\pi n}}$. А еще знаем, что $\binom{2n}{n} = \frac{(2n)!}{(n!)^2} \sim \frac{(2n)^{2n}e^{-2n}\sqrt{2n}C}{(n^n e^{-n}\sqrt{n}C)^2} = \frac{4^n \sqrt{2}}{\sqrt{n}C}$.

    Тогда получаем, что $\frac{4^n}{\sqrt{\pi n}} \sim \frac{4^n \sqrt{2}}{\sqrt{n}C} \implies C \sim \frac{4^n \sqrt{2}}{\sqrt{n}} \cdot \frac{\sqrt{\pi n}}{4^n} = \sqrt{2\pi}$.

    Итоговый результат (Формула Стирлинга): \begin{align*}
        n! &\sim n^n e^{-n} \sqrt{2\pi n}\\
        \ln n! &= n \ln n - n + \frac{1}{2} \ln (2\pi n) + o(1)
    .\end{align*}
\end{example}
\begin{remark}
    Если посчитать точнее, то получим $\ln n! = n \ln n - n + \frac{1}{2} \ln(2 \pi n) + \mathcal{O}(\frac{1}{n})$.
\end{remark}
%END TICKET 19
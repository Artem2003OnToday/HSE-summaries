%BEGIN TICKET 06
\begin{definition}
    $f\!: [a, b] \to \R$. Интеграл с переменным верхним пределом  $\Phi(x) \coloneqq \int\limits_a^x f$, где  $x \in [a, b]$.
\end{definition}
\begin{definition}
    $f\!: [a, b] \to \R$. Интеграл с переменным нижним пределом  $\Psi(x) \coloneqq \int\limits_x^b f$, где  $x \in [a, b]$.
\end{definition}
\begin{remark}
    $\Phi(x) + \Psi(x) = \int\limits_a^b f$.
\end{remark}
\begin{theorem}[Теорема Барроу]
	Пусть  $f \in C([a, b])$. Тогда  $\Phi'(x) = f(x)\quad  \forall x \in[a, b]$. То есть  $\Phi$ --- первообразная функции  $f$.
\end{theorem}
\begin{proof}
	Надо доказать, что $\lim\limits_{y \to x} \frac{\Phi(y) - \Phi(x)}{y-x} = f(x)$. Проверим для предела справа (слева аналогично, но, возможно, с чуть другим порядком точек).

    Тогда $\Phi(y) - \Phi(x) = \int\limits_a^y f - \int\limits_a^x f = \int\limits_x^y f$.

    Тогда  $\frac{\Phi(y) - \Phi(x)}{y-x}=\frac{1}{y-x}\int\limits_x^y f = f(c)$ для некоторого $c \in (x, y)$ по интегральной теореме о среднем.

    Проверяем определение по Гейне. Берем  $y_n > x$ и  $y_n \to x$. Тогда  $\frac{\Phi(y_n)-\Phi(x)}{y_n - x} = f(c_n)$, где $c_n \in (x, y_n)$,  $x < c_n < y_n \to x \Rightarrow c_n \to x \Rightarrow$ в силу непрерывности $f$  $f(c_n) \to f(x)$.
\end{proof}
\begin{consequence}
    $\Psi'(x) = -f(x)\quad \forall x\in [a, b]$.
\end{consequence}
\begin{proof}
    $\Psi(x) = \int\limits_a^b f - \Phi(x) = C - \Phi(x) \Rightarrow \Psi' = (C - \Phi(x))' = -\Phi'(x) = -f(x)$.
\end{proof}
\begin{theorem}
    Непрерывная на промежутке функция имеет первообразную.
\end{theorem}
\begin{proof}
    $f\!: \langle a, b \rangle \to \R$. 

    Возьмём $c \in (a, b)$
    Рассмотрим  $F(x) \coloneqq \begin{cases} \int\limits_c^x f & \text{при } x \ge c \\ -\int\limits_x^c f & \text{при } x \le c \end{cases}$.

    Утверждаем, что $F(x)$ --- первообразная $f(x)$.
    Если $x > c$, то  $F'(x) = f(x)$. 
    Если $x < c$, то $F'(x) = -(-f(x)) = f(x)$
    Если $x = c$, то, так как производные слева и справа считаются правильно и равны, то и в этой точке производная есть $f(x)$.
\end{proof}

\begin{theorem}[Формула Ньютона-Лейбница]
    $f\!: [a, b] \to \R$ и  $F$ -- её первообразная. Тогда  $\int\limits_a^b f = F(b) - F(a)$.
\end{theorem}
\begin{proof}
	$\Phi(x) = \int\limits_a^x f$ --- первообразная и  $F(x) = \Phi(x) + C$ (знаем, что две первообразные отличаются на константу)

    Тогда  $F(b) - F(a) = (\Phi(b) + C) - (\Phi(a) + C) = \Phi(b) - \Phi(a) = \int\limits_a^b f$
\end{proof}

И ровно в этот момент мы поняли, что от выбора псевдоплощади не зависим, поскольку первообразные от них не зависят (отсылка к первому билету/началу конспекта про псевдоплощади)

\begin{definition}
    $F\mid_a^b \coloneqq F(b) - F(a)$
\end{definition}
%END TICKET 06
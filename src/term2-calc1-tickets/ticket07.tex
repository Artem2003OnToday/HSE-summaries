%BEGIN TICKET 07
\begin{theorem}[Линейность интеграла]
    $\int\limits_a^b(\alpha f + \beta g) = \alpha \int\limits_a^b f + \beta \int\limits_a^b g$.
\end{theorem}
\begin{proof}
    $F, G$ --- первообразные для  $f, g$.

    Тогда  $\alpha F + \beta G$ --- первообразная для  $\alpha f + \beta g$. Тогда воспользуемся формулой Ньютона-Лейбница:  \[
        \int_a^b \alpha f + \beta g = \alpha F + \beta G \mid_a^b = \alpha F(b) + \beta G(b) - \alpha F(a) - \beta G(a)
    .\] 
\end{proof}
\begin{theorem}[Формула интегрирования по частям]
    Пусть $f, g \in C^{1}[a, b]$.

    Тогда  $\int\limits_a^b fg' = fg \mid_a^b - \int\limits_a^bf'g$.
\end{theorem}
\begin{proof}
    Докажем при помощи формулы Ньютона-Лейбница. Пусть $H$ --- первообразная  $f'g$. Тогда  $fg - H$ --- первообразная для $fg'$.

    Проверим данный факт: $\left(fg - H\right)' = f'g + fg' - f'g = fg'$. А значит нам можно воспользоваться формулой Ньютона-Лейбница.

    $\int\limits_a^b fg' = \left(fg - H\right) \mid_a^b = fg \mid_a^b - H\mid_a^b = fg\mid_a^b - \int_a^b f'g$.
\end{proof}

\begin{remark}[Соглашение]
    Если $a>b$, то  $\int\limits_a^b f \coloneqq -\int\limits_b^a f$.

    Мотивация: Если  $F$ --- первообразная, то  $\int\limits_a^b f = F \mid_a^b$.
\end{remark}
\begin{theorem}[Формула замены переменной]
    Пусть $f \in C[a, b]$, $\vphi\!: [c,d] \to [a,b]$, $\vphi \in C^1[c,d], p, q \in [c, d]$.

    Тогда  $\int\limits_a^b f(\vphi(t)) \vphi'(t) \mathrm{d}t = \int\limits_{\vphi{p}}^{\vphi{q}} f(x) \mathrm{d}x$.
\end{theorem}
\begin{proof}
    Пусть $F$ --- первообразная  $f$. Тогда  $\int\limits_{\vphi(p)}^{\vphi(q)}f(x) \mathrm{d}x = F \mid_{\vphi(p)}^{\vphi(q)} = F_0\vphi\mid_p^q$, где $F_0\vphi$ --- первообразная для  $f(\vphi(t))\vphi'(t)$.

    Проверим данные факты:  $\left(F(\vphi(t))\right)' = F'(\vphi(t)) \cdot \vphi'(t) = f(\vphi(t))\vphi'(t)$.

    Тогда интеграл равен $\int\limits_{p}^qf(\vphi(t))\vphi'(t) \mathrm{d}t$
\end{proof}
\begin{example}
\begin{align}
    \int_0^{\frac{\pi}{2}} \frac{\sin 2t}{1 + \sin^4 t} \mathrm{d}t 
.\end{align}
Произведем замену $\vphi(t) = \sin^2 t$,  $f(x) = \frac{1}{1+x^2}$, $\vphi'(t) = 2 \sin t \cos t = \sin 2t$, $\vphi(0) = 0, \vphi(\frac{\pi}{2}) = 1$:
\begin{align*}
    (1) = \int_{0}^{\frac{\pi}{2}} \frac{\vphi'(t)}{1 + (\vphi(t))^2} = \int_{\vphi(0)}^{\vphi(\frac{\pi}{2})} f(x)\mathrm{d}x = \int_0^1 \frac{\mathrm{d}x}{1+x^2} = \arctg x \mid_0^1 = \frac{\pi}{4}.
\end{align*}
\end{example}
%END TICKET 07
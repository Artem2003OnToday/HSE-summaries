%BEGIN TICKET 49
\begin{definition}
    $A$ --- связное множество, если  $\forall$ покрытие из $U, V$  $A \subset U \cup V, U \cap V = \emptyset \implies $ либо  $A \subset U$, либо  $A \subset V$, где $U, V$ --- открытые.
\end{definition}
\begin{example}
    \begin{enumerate}
        \item $[a, b]$ --- связное множество в  $\R$.
        \item  $\Q$ --- несвязное множество в  $\R$. Пример  $\Q \subset (-\infty; \sqrt{2}) \cup (\sqrt{2}; +\infty)$.
    \end{enumerate}
\end{example}
\begin{theorem}
    Непрерывный образ связного множества --- связное множество.
\end{theorem}
\begin{proof}
    $A$ --- связное,  $f\!: A \subset X \to Y$ непрерывное. Хотим показать, что $f(A) \subset U, V$ --- открытых множеств в $Y$, причем  $U \cap V = \emptyset$, то тогда образ лежит либо в $U$, либо в  $V$. Так как множества открытые, то и $f^{-1}(U), f^{-1}(V)$ будут открытыми, причем  $A \subset f^{-1}(U) \cup f^{-1}(V)$ и пересечения прообразов будет пустым.

    Так как $A$ связно, то оно будет лежать ровно в одном из прообразов, а значит и образ будет лежать ровно в одном множестве.
\end{proof}
\begin{consequence}[Теорема Больцано-Коши]
    Пусть $A$ --- связное, $a, b \in A$. $f\!: A \to \R$ непрерывная. 

    Тогда  $f$ принимает все промежуточные значения, лежащие между  $f(a)$ и  $f(b)$.
\end{consequence}
\begin{proof}
    От противного. Пусть $f(a) < C < f(b)$ и $C$ --- не значение. Тогда  $f(A) \subset (-\infty, C) \cup (C, +\infty)$. Заметим, что данные множества открытые и не пересекаются. Тогда получили противоречие со связностью  $f(A)$.
\end{proof}
\begin{theorem}
    $\langle a, b \rangle$ --- связное подмножество  $\R$,  $a, b \in \overline{\R}$.
\end{theorem}
\begin{proof}
    От противного. Пусть $\langle a, b \rangle \subset U \cap V$,  $U \cap V = \emptyset$. 

    Пусть  $f\!: \langle a, b \rangle \to \R = f(x) = \begin{cases} 0 & x \in \langle a, b \rangle \cap U \neq \emptyset \\ 1 & x \in \langle a, b \rangle \cap V \neq \emptyset \end{cases}$ --- непрерывная функция. Её прообраз:  $\emptyset, \langle a, b \rangle, \langle a, b \rangle \cap U, \langle a, b \rangle \cap V$ --- открытые в  $\langle a, b \rangle$ множества, но значения  $\frac{1}{2}$ не принимаются, а значения 0 и 1 точно принимаются, так как иначе бы $\langle a, b \rangle$ лежал бы ровно в 1 множестве. 
\end{proof}
\begin{definition}
    $A$ --- линейно связно, если  $\forall u, v \in A \exists \gamma\!: [a, b] \to A\!: \gamma(a) = u, \gamma(b) = v$.
\end{definition}
\begin{theorem}
    Линейно связное множество связно.
\end{theorem}
\begin{proof}
    $A$ --- линейно связно, пусть оно не связно $\implies A \subset U \cup V\quad U \cap V = \emptyset$.  $A \cap U \neq \emptyset$ и  $A \cap V \neq \emptyset$. 

    Возьмем  $u \in A \cap U, v \in A \cap V$ и соединим их путем  $\gamma$.  $\gamma[a,b]$ --- связное (как образ отрезка),  $\gamma[a, b] \subset A \subset U \cup V \implies \gamma[a, b] \subset U$ или $\gamma[a, b] \subset V$. Противоречие. 
\end{proof}
\begin{definition}
    Область --- открытое, линейно связное множество.
\end{definition}
\begin{remark}
    Если $A$ открыто, то  $A$ --- связно  $\iff A$ --- линейно связное.
\end{remark}
%END TICKET 49
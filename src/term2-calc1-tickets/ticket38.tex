%BEGIN TICKET 38
\begin{lemma}[Лемма Лебега]
    $K$ --- секвенциальный компакт,  $K \subset \bigcup\limits_{\alpha \in I} U_\alpha$ --- открытое покрытие.

    Тогда  $\exists r > 0\!: \forall x \in K\quad B_r(x)$ целиком покрывается каким-то  $U_\alpha$.
\end{lemma}
\begin{proof}
    От противного. Тогда $r = \frac{1}{n}$ не подходит $\implies \exists x_n \in K\!: B_{\frac{1}{n}}(x_n)$ не содержится целиком ни в каком $U_\alpha$.

    Выберем подпоследовательность $x_{n_k}$, такую что  $\lim x_{n_k} = a \in K$.

    Тогда  $a \in U_\beta$ для некоторого  $\beta \in I \implies \exists B_\eps(a) \subset U_\beta$. Возьмем  $N_1\!: \forall k \ge N_1\quad \rho(x_{n_k}, a) < \frac{\eps}{2}$. А еще можно взять $N_2\!: \forall k \ge N_2\quad \frac{1}{n_k} < \frac{\eps}{2}$. А значит $B_{\frac{1}{n_{k}}}(x_{n_k}) \subset B_\eps(a) \subset U_\beta$ при $k \ge \max\{N_1, N_2\}$?!!

    Докажем $B_{\frac{1}{n_{k}}}(x_{n_k}) \subset B_\eps(a)$: Если $x \in B_{\frac{1}{n_k}}(x_{n_k})$  $\rho(x_{n_k}, x) < \frac{1}{n_k} < \frac{\eps}{2} \land \rho(x_{n_k}, a) < \frac{\eps}{2} \implies \rho(x, a) \le \rho(x_{n_k}, x) + \rho(a, x_{n_k}) < \eps$ 
\end{proof}
\begin{theorem}
    Компактность = секвенциальная компактность.
\end{theorem}
\begin{proof}
    $\Leftarrow$ Пусть  $K \subset \bigcup\limits_{\alpha \in I} U_\alpha$ --- открытое покрытие. Возьмем $r > 0$ из леммы Лебега. Рассмотрим открытое покрытие  $K \subset \bigcup\limits_{x \in K} B_r(x)$.

    Достаточно из него выделить конечное подпокрытие. Возьмем  $x_1 \in K$. Если $B_r(x_1) \supset K$, то выбрали конечное покрытие. Иначе берем $x_2 \in K \setminus B_r(x_1)$. Если объединение шариков $\supset K$, то выбрали конечное подпокрытие. Иначе продолжаем процесс:  $x_n \in K \setminus \bigcup_{i=1}^{n-1}B_r(x_i)$. Если процесс оборвался, то выделили конечное подпокрытие.

    Если он не оборвался, то мы построили последовательность $x_1, x_2,\ldots$. Причем  $\rho(x_n, x_k) \ge r \forall n > k \implies \rho(x_i, x_j) \ge r \forall i \ne j$. Из такой последовательности не выбрать сходящуюся подпоследовательность, так как любая подпоследовательность не фундаментальная, - противоречие секвенциальной компактности.
\end{proof}
%END TICKET 38
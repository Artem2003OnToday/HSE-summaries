\begin{definition}
    Определение в кавычках. $f \in K[x]$,  $K$ --- кольцо.  $f'(x)$ --- $\frac{f(x) - f(y)}{x - y}$ в точке  $y=x$.
\end{definition}
\begin{example}
    $f=x^n$  $(x^n)' = \frac{x^n - y^n}{x-y} = \frac{(x-y)(x^{n-1} + x^{n-2}y + \ldots + y^{n-1})}{x-y} = x^{n-1} + x^{n-2}y+\ldots+y^{n-1} = x^{n-1} + x^{n-1} + \ldots = nx^{n-1}$
\end{example}
\begin{definition}
    $f = a_n x^n + a_{n-1} x^{n-1} + \ldots + a_1 x + a_0$.
\end{definition}
\begin{proof}
    По определению получаем $f' = n a_n x^{n-1} + (n-1)a_{n-1}x^{n-2} + \ldots + a_1$.
\end{proof}
\begin{properties}
    \begin{enumerate}
        \item $(f+g)' = f' + g'$
        \item  $(kf)' = kf'$,  $k \in K$
        \item  $(fg)' = f'g + fg'$
    \end{enumerate}
\end{properties}
\begin{remark}
    $D\!: K[x] \to K[x], D(f) = f'$

    Любой $D\!: A\to A$ удовлетворяющий свойствам 1-3 называется оператором дифференцирования. 

    В случае  $K[x] = A$  "обычное" дифференцирование --- единственное.
\end{remark}
\begin{proof}[Проверка свойств]
    \slashn
     \begin{enumerate}
         \item по вычислительным определениям --- упражнение.
         \item по определению 1.
         \item Как в матане  $(fg)'(x) = \frac{(fg)(x) - (fg)(y)}{x-y} = \frac{f(x) g(x) - f(x)g(y) + f(x)g(y) - f(y)g(y)}{x-y} = f(x) \frac{g(x) - g(y)}{x-y} + g(y) \frac{f(x) -f(y)}{x-y} = f(x)g'(x) + g(x) f'(x)$
    \end{enumerate}
\end{proof}
\begin{theorem}
    Пусть $f \divby (x-a)^k \land f \centernot \divby (x-a)^{k+1}$ и  $k \neq 0$ в $K$ ($f \in K[x], K$ ---  поле).

    Тогда $f' \divby (x-a)^{k-1}, f' \centernot \divby (x-a)^k$.
\end{theorem}
\begin{definition}
    Такое $k$ называется кратностью корня $a$ в  $f$. 
\end{definition}
Тогда $k \neq 0$: кратность уменьшается на 1 при дифференцировании.  $k = 0$, кратность уменьшается не более, чем на 1.

 \begin{example}
     $K = \Z / p\Z$,  $f = x^p$, 0 --- корень кратности  $p$.  $f' = p x^{p-1} = 0$, $0$ --- корень бесконечной кратности.
\end{example}
\begin{proof}
    $f = (x-a)^k \cdot g\ g \centernot \divby (x-1) \Rightarrow f' = ((x-a)^k g)' = ((x-a)^k)' \cdot g + (x-a)^k g' = k(x-a)^{k-1} \cdot g + (x-a)^k \cdot g' = (x-a)^{k-1} (kg + (x-a)g') \divby (x-a)^{k-1}$. 

    Если $k \neq 0 \Rightarrow (x-a)g' \divby x-a$. $g \centernot \divby x-a \Rightarrow k \cdot g \centernot \divby x-a \Rightarrow kg + (x-a)g' \centernot \divby x-a \Rightarrow f' \centernot \divby (x-a)^k$. 
\end{proof}
\begin{lemma}
    \begin{align*}    
        ((x-a)^k)' &= k(x-a)^{k-1}\\
        (x-a)^k &= \sum_{i=0}^k \binom{k}{i} x^i (-a)^{k-i} \\
        k(x-a)^{k-1} &= \sum_{i=0}^{k-1} \binom{k-1}{i} kx^i(-a)^{k-1-i}\\
        (\binom{k}{i} x^i(-a)^{k-i})' &= ix^{i-1} \binom{k}{i} (-a)^{k-i} = \\
        = i x^{i-1} \binom{k}{i}(-a)^{k-1 - i + 1} &= \binom{k-1}{i-1}kx^{i-1}(-1)^{(k-1) - (i 0 1)}
    \end{align*}
    Получили, что $i$-ое слагаемое в первой сумме --- $i$-ое слагаемое во 2-ой сумме.
\end{lemma}
\begin{statement}
    $f \in K[x], \deg f = n$,  $a \in K$. Тогда  $\exists c_0, c_1,c_2,\ldots c_n$, такой что $f = c_0 + c_1(x-a) + c_2(x-a)^2 + \ldots + c_n(x-a)^n$.
\end{statement}
\begin{proof}
    $x-a = t, x = t+a$. Дальше раскрыть скобки \ldots Хотя хуй знает я банан)
\end{proof}
\Subsection{Характеристика поля}
$K$ --- поле,  $1 \in K$

Рассмотрим последовательность $1$, $1 + 1$, $1 + 1 + 1$, $\cdots$

\begin{definition}
	Есть два варианта: все элементы последовательности попарно различны или последовательность периодична с периодом  $C$, где  $C = \ord(1)$ относительно  $+$. Тогда $C$ --- характеристика поля. ($C = \charr K$ )
\end{definition}
Если $\ord 1 = \infty$, $\charr K = 0$. Тогда $\underbrace{1 + 1 + \cdots + 1}_n = 0$ в  $K \iff n \divby \charr K$.

 \begin{lemma}
    $\charr K$ --- ноль или простое число.
\end{lemma}
\begin{proof}
    Пусть $\charr K = m \cdot n$,  $m, n > 1$.  $\underbrace{1+\ldots+1}_{mn} = 0 = \underbrace{(1 + \ldots + 1)}_m + \ldots + \underbrace{(1 + \ldots + 1)}_m$. Получили $(1 + 1 + 1 + \ldots + 1)(1 + 1 + 1 + \ldots + 1) = 0$, где в одной скобке $m$ единиц, а в другой ---  $n$. Тогда, поскольку в поле нет делителей нуля, $m = 0 \lor n = 0$. Противоречие с минимальностью.
\end{proof}
\begin{consequence}
    $\charr K = 0 \Rightarrow K$ содержит копию $\Q$.

     $\charr K = p \Rightarrow K$ содержит копию  $\Z / p \Z$.
\end{consequence}
\Subsection{Формула Тейлора}
$D((x-a)^k) = k (x-a)^{k-1}$, тогда определим $D^2(f) = D(D(f))$, $D^{(l)}(f) = D(D^{(l - 1)}(f))$

Известно, что  $D(kf) = kD(f), k \in K$. Тогда  $D^{(2)}((x-a)^k) = D(k(x-a)^{k-1}) = k(k-1)(x-a)^{k-2}$. Тогда в  $l$-ой производной:  $k(k-1)(k-2)\ldots(k - l + 1)(x-a)^{k-l}$, если $l \le k$. При $l > k$ получаем 0.

Вычислим значение  $l$-ой производной в точке  $a$ --- $D^{(l)}f(a) = D^{(l)}\left(\sum a_k(x-a)^k\right)(a) = \sum a_kD^{(l)}((x-a)^k)(a) = a_l D^{l}((x-a)^l)(a) = a_l \cdot l!$ Объяснение последнего перехода: все члены $a_i, i < l$ отпали, поскольку для них $l$-ая производная есть константный ноль. Все члены $a_i, i > l$ отпали, поскольку в них осталась скобка $(x-a)$, которая в точке $x=a$ обращается в ноль. А значит осталось только слагаемое при $a_i, i = l$, для которого мы умеем считать $l$-ую производную.

Предполагая, что $\charr K = 0 \lor \charr K > \deg f$, знаем, что $1!, 2!, 3!, \ldots (\deg f)! \neq 0$ в $K$. Тогда имеем $a_l = \frac{D^{l}(f)(a)}{l!}$. А ещё есть следующее обозначение: $D^{(l)}(f) = f^{(l)}$.
\begin{theorem}[Формула Тейлора]
    \[
        f = \sum_{l=0}^{\deg f} \frac{f^{(l)}(a)}{l!}(x-a)^l
    .\] 
\end{theorem}

Рассмотрим $a^2 - b^2 = 15^{2021} \iff (a-b)(a + b) = 3^{2021} \cdot 5^{2021} \Rightarrow \begin{cases} a+b=3^k \cdot 5^l \\ a-b=3^{2021-k} \cdot 5^{2021-l} \end{cases} \Rightarrow a = \frac{3^k \cdot 5^l + 3^{2021-k} \cdot 5^{2021-l}}{2}$.  

Уравнение $81a^2-169b^2=15^{2021}$ --- тоже решается. А вот $a^2-2b^2 = 15^{2021} \iff (a-\sqrt{2}b)(s+\sqrt{2}b) = 3^{2021}5^{2021}$ уже не решается в целых чисел. Если вылезать, то надо расписывать разложение $a+\sqrt{2}b$, "3", "5" и единственность разложения на множители.

Еще один пример:  $a^2+b^2=15^{2021}$. Посмотрим на остатки от деления на 4:  $a^2, b^2 \mod 4 \in \{0, 1\}, 15^{2021} \mod 4 = 3$. Но для этого нам нужно понимать что-то по кольцо вычетов по модулю.

\Subsection{Группы}
 \begin{definition}
     Группой называется пара $(G, \ast)$, где  $G$ --- множество, а  $\ast: G \to G$ --- бинарная операция, так что выполнены свойства:
     \begin{enumerate}
         \item $\forall a,b,c \in G:$  $(a \ast b) \ast c = a \ast (b \ast c)$. Ассоциативность.
         \item $\exists e \in G:$ $a \ast e = e \ast a = a$. Существование нейтрального элемента.
         \item $\exists a^{-1}:$  $a \ast a^{-1} = a^{-1} \ast a = e$. Существование обратного элемента.
     \end{enumerate}
 \end{definition}
\slashn
Несколько примеров:
\begin{enumerate}
    \item $(\Z, +)$.  $e=0, a^{-1}=-a$.
    \item  $(\Q \setminus 0, \cdot)$,  $e=1, a^{-1}= \frac{1}{a}$.
    \item $(2^M, \bigtriangleup)$ $e=\varnothing, A^{-1} = A$.
\end{enumerate}
\begin{definition}
    Группа $G$ называется абелевой, если  $\forall x, y \in G:$ $x \ast y = y \ast x$.
\end{definition}
\begin{example}[Главный пример группы]
    Пусть $G=S(M) = \{f: M \to M \mid f\text{ --- биекция}\}$
    \begin{itemize}
    \item Ассоциативность --- упражнение.
    \item Нейтральный элемент --- $f(x) = x$, тождественное отображение.
    \item  $f^{-1}=$ обратная функция. Она существует, так как $f$ --- биекция. 
    \end{itemize}
    \slashn
    Получили группы по композиции.
\end{example}
\begin{example}
    $M=\{1,2,3\}$.  $f_1, f_2: M \to M$ --- биекция.  $f_1$ --- меняет местами  1 и 2: $1 \to 2, 2 \to 1, 3 \to 3$,  $f_2$ переставляет по циклу: $1 \to 2, 2 \to 3, 3 \to 1$. $f_2 \circ f_1: 1 \to 3, 2\to 2, 3\to 1$. $f_1 \circ f_2: 1 \to 1, 2 \to 3, 3 \to 2$. Ну значит группа не абелева.
\end{example}
\slashn
Докажем простейшие свойства групп:
\begin{enumerate}
\item $\exists!$ нейтральный элемент.

    \textbf{Доказательство:} заметим, что $e_1=e_1 \ast e_2 = e_2$
\item $\exists!$ обратный элемент. 

    \textbf{Доказательство:} пусть $b, c$ --- обратные к  $a$. Тогда  $(b\ast a)\ast c = e \ast c = c$, но при этом $b \ast (a \ast c) = b \ast e = b$. Значит  $b=c$.
\item $a \ast b = b \ast c \iff a = c$

    \textbf{Доказательство:} $a \ast b = a \ast c \iff (a^{-1} \ast a) \ast b = (a^{-1} \ast a) \ast c \iff e \ast b = e \ast c \iff b = c$
\end{enumerate}
\Subsection{Кольца}
\begin{definition}
    Кольцо --- тройка $(R, +, \cdot)$ ($R$ --- множество,  $+, \cdot: R \times R \to R$), такая что:
     \begin{enumerate}
         \item[1--4.] $(R, +)$ --- абелева группа. Нейтральный элемент обозначается $0$, обратный к  $a$ ---  $-a$.
         \item[5.] $a\cdot(b+c) = a \cdot b + a \cdot c$ и  $(b+c) \cdot a = b \cdot a + b \cdot c$. Дистрибутивность.
    \end{enumerate}
\end{definition}
\begin{definition}
    Кольцо $R$ называется ассоциативным, если выполнено 
    \begin{itemize}
        \item[6.] $a \cdot (b \cdot c) = (a \cdot b) \cdot c$.
    \end{itemize}
\end{definition}
\begin{definition}
    Кольцо $R$ называется коммутативным, если
    \begin{itemize}
        \item[6.] $a \cdot b = b \cdot a$
    \end{itemize}
\end{definition}
\begin{definition}
    Кольцо $R$ называется кольцом с 1, если  
    \begin{enumerate}
        
        \item[7.] $\exists 1 \in R: 1 \cdot a = a \cdot 1 = a$
    \end{enumerate}
\end{definition}
\slashn
\begin{example}
    $(\Z, +, \cdot)$ --- коммутативное ассоциативное кольцо с 1.
\end{example}
\begin{definition}
    Коммутативное ассоциативное кольцо с 1 называется полем, если выполнена 
    \begin{enumerate}
        \item[8.] $\forall a \in R \ \{0\}$  $\exists b \in R$  $ab = 1 \land 1 \neq 0$
    \end{enumerate}
\end{definition}
\begin{example}
    $(\Q, +, \cdot)$ --- поле, а вот  $(\Z, +, \cdot)$ --- не поле.
\end{example}

\Subsection{Построение кольца вычетов}
\begin{definition}
Пусть $a, b \in \Z$, говорят, что  $a$ сравнимо с  $b$ по модулю  $n$ ($a \equiv b \pmod{n}$), если $n \mid a - b$. Эквивалентное определение:  $a$ и  $b$ имеют одинаковые остатки по модулю  $n$.
\end{definition}
Докажем, что сравнимость по модулю --- отношение эквивалентности.
\begin{itemize}
    \item $a \equiv a \pmod{n} \iff n \mid 0$
    \item $n | a - b \iff n | b - a \Rightarrow a \equiv b \pmod{n} \iff b \equiv a \pmod{n}$.
    \item Транзитивность...
\end{itemize}
\slashn
Наблюдение.  $a \in \Z \rightarrow \overline{a} = \{b \mid a \equiv b\} = \{a + kn \mid k \in \Z\}$. $\Z = \overline{0} \cup \overline{1} \ldots$

\begin{definition}
    Фактор множества по отношению $\equiv$ обозначается  $\Z / n\Z$.
\end{definition}

$\Z \to \Z / n\Z$.Элементы $\Z / n\Z$ называются классами вычетами по модулю.

 \begin{enumerate}
     \item $a \equiv b \pmod{n} \land c \equiv d \pmod{n} \iff a+c \equiv b+d \pmod{n} \land ac \equiv bd \pmod{n}$. 

         Доказательство  $(a+c) - (b+d) = \underbrace{(a-b)}_{\vdots n} - \underbrace{(d-c)}_{\vdots n} \vdots n$. 

         Доказательство $ac - bd = ac - bc + bc - bd = c (a-b) + b(c-d) \vdots n$.

         Значит класс суммы и произведения зависит только от классов множителей и слагаемых.
\end{enumerate}
\begin{theorem}
    Пусть $n \in \N$. Тогда класс $(\Z / n\Z, +, \cdot)$, где $\overline{a}+\overline{b} = \overline{a+b} \land \overline{a} \cdot \overline{b} = \overline{a \cdot b}$ --- ассоциативное коммутативное кольцо с единицей.
\end{theorem}
\begin{proof}
    Все аксиомы --- следствия из $\Z$. Докажем для примера  $(\overline{a} + \overline{b}) + \overline{c} = \overline{a} + (\overline{b} + \overline{c}) = \overline{a+b} + \overline{c} = \overline{(a+b)+c} = \overline{a + (b+c)} = \overline{a} + \overline{(b+c)} = \overline{a} + (\overline{b} + \overline{c}).$
\end{proof}
\slashn
Закон сокращения не очень работает в кольце вычетов по модулю: $2 \cdot 1 = 2 \cdot 4$ ($\pmod 6$), но  $1 \neq 4 \pmod 6$.

\begin{definition}
    Пусть $R$ --- коммутативное ассоциативное кольцо с единицей. Тогда  $\forall a \in R: a\text{ --- делитель} \Rightarrow \exists B \neq 0: ab = 0$.
\end{definition}
\begin{example}
    $n$ --- составное  $n=p_1p_2$ в $\Z / n\Z \overline{p_1} \overline{p_2}=\overline{n}=0$. Значит $p_1,p_2$ --- делители числа.
\end{example}
\begin{lemma}
    $\forall a, b, c \in R ab = ac \land a\text{ --- не делитель 0} \Rightarrow b = c$. 
\end{lemma}
\begin{proof}
    $ab=ac$:  $ab - ac = 0 \iff a(b-c) = 0$.  $a$ --- не делитель 0  $\Rightarrow b-c=0 \iff b = c$.
\end{proof}
\begin{lemma}
    $a \in R a$ --- обратим $\Rightarrow a$ --- не делитель 0.
\end{lemma}
\begin{proof}
    Пусть $ab=0 \Rightarrow a^{-1}(ab) = a^{-1} \cdot 0 = (a^{-1}a)b = 0 \Rightarrow b =0$.
\end{proof}
\begin{remark}
    Обратное неверно: в $\Z$ 2 -- не делитель нуля, но  $\frac{1}{2} \notin \Z$ .
\end{remark}
\begin{theorem}
    $\forall a \in Z: \overline{a} \in \Z / n\Z$. Тогда:  
    \begin{enumerate}
        \item $\overline{a}$ --- обратим $\iff (a, n) = 1$
        \item  $\overline{a}$ --- делитель нуля $\iff (a, b) \neq 1$.
    \end{enumerate}
\end{theorem}
\begin{proof}
    $\overline{a}$ --- обратим  $\iff \exists \overline{b}: \overline{a} - \overline{b} = \overline{1} \iff \exists b: ab = 1 \pmod{n} \iff \exists b: ab - 1 \vdots n \iff \exists b, k: ab-1=nk \iff \exists b, k: ab-nk=1 \iff (a,n)  = 1$.

    $(a, n) = 1 \Rightarrow \overline{a}\text{ --- обратим} \Rightarrow$ не делитель нуля.

    $(a, n) = d > 1, a = dx$. Тогда  $\overline{a} \cdot \frac{\overline{n}}{\overline{d}} = \overline{d}x \frac{\overline{n}}{\overline{d}} = \overline{nx} = 0$ и $\frac{\overline{n}}{\overline{d}} \neq 0$. Значит  $9 < |\frac{n}{d}| < n$.
\end{proof}
\begin{consequence}
    $n$ --- простое  $\Rightarrow \Z / n\Z$ --- поле.
\end{consequence}
\begin{proof}
    Достаточно проверить существование обратного. $\overline{a} \neq \overline{0} \iff a \not\vdots n \iff (a, n) = 1 \iff a$ --- обратим.
\end{proof}
\begin{definition}
    $\forall $ ассоциативного кольца с 1 $R$:  $R$ --- называется кольцом без делителей 0 (область целостности), если делитель 0 только 0.  $ab = 0 \iff a = 0 \lor b = 0$  .
\end{definition}
 \begin{remark}
     $R$ --- область  $\Rightarrow ax_1=ax_2 \Rightarrow x_1=x_2$ ($a \neq 0$).
\end{remark}
\slashn
Вернемся к диофантову уравнению $ax+by=1$, $(a, b) = 1$. Тогда $ax = c \pmod{b}$ и  $by = c \pmod{a}$. Тогда  $\overline{a}\overline{x}=\overline{c}$ в  $\Z / n\Z \xRightarrow{(a,b)=1} \overline{x} = \overline{a}^{-1}\overline{c} \pmod{b}$. Тогда $x = x_0+kb$.

\Subsection{Квадратное уравнение}
Посмотрим на $x^2+px+q=0$ в  $\Z / n\Z$. Работает ли  $x_{1,2} = \frac{-p \pm \sqrt{p^2 - 4q}}{2}$. Есть проблемки:
 \begin{enumerate}
     \item $p^2 - 4q$ --- не квадрат в  $\Z / n\Z$ (не решений).
     \item $2 = 0$. Или  $\nexists 2^{-1}$ (нельзя поделить на два).
     \item  $n$ --- не простое. Тогда  $(x-x_1)(x-x_2)\ldots=0$. Тогда не следует, что $x = x_1 \lor x = x_2$. Пример: $x^2-1=0 \pmod{8}$
\end{enumerate}
\Subsection{Китайская теорема об остатках}
Чтобы решать такие уравнения можно свести к простым модулям при помощи китайской теоремы об остатках.

Вопрос такой: как связаны $\Z / n\Z, \Z / m\Z, \Z / mn\Z$. Пусть $P_m: \Z \mapsto \Z / m\Z$, а $P_{mn} \Z \mapsto \Z / mn\Z$. 

 \begin{definition}
     Гомоморфизмом колец $f: R_1 \mapsto R_2$ называется такое отображение, что $\forall r_1, r_1 \in R_1: f(r_1 + r_2) = f(r_1) + f(r_2), f(r_1r_2)=f(r_1)\cdot f(r_2), f(1) = 1$.
\end{definition}
\begin{definition}
    Гомоморфизмом группы $f: G_1 \mapsto G_2$ называется такое отображение, что $\forall g_1, g_2: f(g_1g_2) = f(g_1) \cdot f(g_2)$.
\end{definition}
\begin{remark}
    $f$ --- гомоморфизм групп  $G_1, G_2 \Rightarrow f(e_{G_1}) = e_{G_2}$.  В частности  $f$ --- гомоморфизм колец  $R_1,R_2 \Rightarrow f(0_{R_1}) = 0_{R_2}$.
\end{remark}
\begin{proof}
    $f(e_{G_1}) = f(e_{G_1} \cdot e_{G_1}) = f(e_{G_1}) \cdot f(e_{G_1})$. Дальше сокращаем.
\end{proof}
\slashn

Существует $P_{mn,m}: P_{mn, m} \cdot P_{mn} = P_{m}$. 
 \begin{proof}
     $P_{mn, m}(\overline{a_{mn}}) = \overline{a_m}$.
\end{proof}
\begin{proof}[Корректность]
    $\overline{a_m} = \overline{b_mn} \iff a \equiv b \pmod{mn} \iff a-b \vdots mn \Rightarrow a-b \vdots m \Rightarrow \overline{a_m} = \overline(b_m)$
\end{proof}
\slashn Аналогично существует гомоморфизм $P_{mn, n}$. То есть $\overline{a_{mn}} \rightarrow (\overline{a_m}, \overline{a_n})$ --- отображение. То есть $\Z / mn \Z \mapsto \Z / m\Z \times \Z / n \Z$.
Отступление.
\begin{definition}
    $R_1, R_2$ --- кольца. Рассмотрим  $(R_1 \times R_2, +, \cdot): (r_1, r_2) +_{R_1\times R_2} (r_1'r_2') \coloneqq (r_1+_{R_1}r_2, r_2+_{R_2}r_2')$. Тоже самое для умножения. Тогда $R_1 \times R_2$ --- тоже кольцо.
\end{definition}
\slashn
Итак мы построили гомоморфизм $\Z / mn \Z \mapsto \Z / m\Z \times \Z / n \Z$. Подумаем про его свойства. Во-первых заметим, что слева $mn$ элементов, но и справа $mn$ элементов!

\begin{definition}
    Биективный гомоморфизм (групп, колец, ...) (называется изоморфизмом, $\cong$) если каждым $a_i$ задано ровно одно  $b_j$ и наоборот.
\end{definition}
\begin{theorem}[Китайская теорема об остатках]
    Пусть $(m, n)=1$, тогда $\Z / mn \Z \cong \Z / m\Z \times \Z / n \Z$.
\end{theorem}
\begin{proof}
    \slashn
    \begin{enumerate}
        \item $i_{m,n}$ --- инъективно. Пусть $i_{m,n}(\overline{a_{m,n}}) = (\overline{a_m}, \overline{a_n})$,  $i_{m,n}(\overline{b_{n, m}}) = (\overline{b_m}, \overline{b_n}) \Rightarrow  a-b \vdots m \land a-b\vdots n \xRightarrow{(n, m) = 1} a - b \vdots mn$.
        \item $i_{m, n}: a \mapsto B$ инъективно: $|A| = |B| \Rightarrow i_{m, n}$ --- сюръективно.  
    \end{enumerate}
\end{proof}
\begin{theorem}[КТО 2]
    $m_1,m_2,m_3,\ldots,m_n \in \Z \land (m_i, m_j) = 1 \Rightarrow \Z / m_1,m_2,\ldots,m_n \Z \mapsto \Z / m_1\Z \times \Z / m_2\Z \ldots$ - изоморфизм колец. 
\end{theorem}
\begin{theorem}[КТО без колец]
    $\forall m_1, \ldots, m_n \in \Z$, $\forall a_1, \ldots, a_n$ $(m_i, m_j) = 1 \Rightarrow \exists x_0 \in Z x \equiv a_1 \pmod{m_1} \land \ldots \land x \equiv a_n \pmod{m_n} \iff x \equiv x_0 \pmod {\prod_i m_i}$
\end{theorem}
\slashn
То есть по факту мы хотим получить обратную функцию к $f_{m_1,m_2,\ldots}: \overline{a_{m_1m_2m_3}} \mapsto (\overline{a_{m_1}}, \overline{a_{m_2}}, \overline{a_{m_3}})$. Пусть тогда $g=f^{-1}$. Заметим, что  $g$ --- гомоморфизм колец. Раз  $g$ сохраняет операции, то  $g(\overline{x}, \overline{y}, \overline{z}) = g(\overline{x}, 0, 0) + g(0, \overline{y}, 0) + g(0, 0, \overline{z}) = \overline{x}g(1, 0, 0) + \overline{y}g(0, 1, 0) + \overline{z}g(0, 0, 1)$.

Пусть  $x=g(1, 0, 0) \iff \begin{cases} x \equiv 1 \pmod{m_1} \\ x \equiv 0 \pmod{m_2} \\ x \equiv 0 \pmod{m_3} \end{cases} \iff \begin{cases} x \equiv 1 \pmod{m_1} \\ x \equiv 0 \pmod{m_2m_3} \end{cases}$.

В группе $\forall a \neq e\; \forall x: ax \neq x$. Тогда посмотрим группу  $(\Z / m\Z \times \Z / n\Z) \supset \{(a, 0) \mid a \in \Z / m\Z\} \cong \Z / m\Z$.
 
Тогда для любого $n \in \N: n = p_1^{\alpha_1} p_2^{\alpha_2}\ldots p_n^{\alpha_3}$  $\Z / n\Z \cong \Z / p_1^{\alpha_1}\Z \times \ldots \times \Z / p_n^{\alpha_n}$.
\begin{example}
    Для того, чтобы решить $b^2 = a$ надо решить  $b_i^2 = a$ для все состовляющих.
\end{example}
\begin{definition}
    Пусть $C$ --- группа ($a \in C$), тогда порядок элемента $a$:  $\ord(a) = \{\min k \in \N \mid a^k = 1\}$. А если такого  $k$ нет, то  $\ord(a) = \infty$
\end{definition}
\begin{lemma}
    Пусть $G$ --- группа ($a \in G$). $\langle a \rangle = \{a, a^2,\ldots; a^{-1}, (a^{-1})^2, \ldots, e\} = \{a^k \mid k \in \Z\}$.Тогда $(\langle a \rangle, *)$ --- группа.
\end{lemma}
\begin{proof}
    Проверим замкнутость относительно операций: 0-рной ($\{\dot\} \to e$), унарной $a \to a^{-1}$, бинарной  $(a, b) \to a * b$.
    \begin{itemize}
        \item $e = a^0 \in \langle a \rangle$
        \item  $b \in \langle a \rangle. b = a^k \Rightarrow b^{-1} = a^{-k} \in \langle a \rangle$.
        \item  $b, c \in \langle a \rangle$.  $b = a^k, c = a^l \Rightarrow bc = a^{k+l} \in \langle a \rangle$.
    \end{itemize}
\end{proof}
\begin{definition}
    $\langle a \rangle$ называется циклической группой, порожденной  $a$.  $G$ --- циклическая группа  $ \iff \exists a \in G G \cong \langle a \rangle$
\end{definition}
\begin{theorem}
    $\ord a = \infty \Rightarrow \langle a \rangle \cong (\Z, +)$.  $\ord a = k \in \N \Rightarrow \langle a \rangle \cong (\Z / k\Z, +)$
\end{theorem}
\begin{proof}
    $f: (\Z, +) \to \langle a \rangle$. То есть  $k \mapsto a^k$.  $f(k+l) = a^{k+l} = a^k \cdot a^l = f(k) + f(l)$. Тогда $f$ --- сюръекция по определению циклической группы.

    Докажем инъективность. Пусть $a^k = a^l \iff a^{k-l} \cdot a^l = e a^l \iff a^{k-l} = e$. Но  $\ord a = \infty$! Значит $k-l=0$.  

    Теперь $\ord a \neq \infty$. Тогда построим  $f: \Z / k \Z \to \langle a \rangle$, то есть  $\overline{m_k} \mapsto a^m$.

    Корректность:  $\overline{m_k} = \overline{n_k} \Rightarrow (m - n) \vdots k$. То есть $m = n + k \cdot l$. Значит  $a^m = a^{n + k \cdot l} \iff a^m = a^n \cdot a^{kl} = a^m$.

    Суръективность/инъективность: смотри выше. Или ниже, ну тут бан короче.
\end{proof}
\slashn
Простыми словами, если $\ord a = \infty \Rightarrow$ в последовательности $\{a^i\}$ - элементы не повторяются. А если  $\ord a \neq \infty$, то элементы повторяются с периодом $k$, а внутри элементы не повторяются. 
\begin{theorem}[Теорема Лангранжа]
    Пусть $G$ --- группа.  $\forall G$ ---  $n$-элементная группа, тогда  $\forall a \in G: n \vdots \ord a$
\end{theorem}
\begin{proof}
    Пусть $\ord a = k$.
    Рассмотрим отображение $m_a(x) = ax$. $m_a G \to G$. Нарисуем граф отображений (вершины --- элементы $G$, ребра (стрелки) ---  $x \to a_x$). $x \to ax \to a^2x \to a^3x \to \ldots \to a^kx \to x$, так как для $\forall i, j \le k: a^i x = a^j x \Rightarrow a^i = a^j$. 

    Значит все элементы $G$ разбиваются на циклы длины  $k$. Следовательно $n \vdots k$.
\end{proof}
\begin{consequence}
    $G$ --- конечная группа ($a \in G$) $\Rightarrow a^{|G|} = e$
\end{consequence}
\begin{proof}
    $\ord a = k$.  $n = k \cdot l$ по теореме Лагранжа. Тогда  $a^n = a^{k\cdot l} = \left(a^k\right)^l = e^l = e$
\end{proof}
\begin{example}
    $(\Z / p\Z, +)$. $\overline{a}^x = \underbrace{\overline{a} + \overline{a} + \overline{a} + \overline{a}}_{x\text{ раз}} = \overline{x}\overline{a}$.
\end{example}
\begin{example}
    $p$ --- простое.\\
    $G \coloneqq (\Z / p\Z \setminus \{0\}, \cdot) $. $|G| = p - 1$. Тогда $a^{p-1} = 1$. Малая теорема Ферма. 
\end{example}
\slashn
На языке сравнений: $a \in \Z, a \vdots p \Rightarrow a^{p-1} - 1 \vdots p \iff a^{p-1} \equiv 1 \pmod{p}$. 
 \begin{example}
     $(\Z / p\Z, +)$ --- циклическая группа. А вот с  $G$ из предыдущего пункта --- тоже, если $p$ --- простое. Но не очев.  
\end{example}
\begin{statement}
    $G$ --- группа ($|G|=n$). $G$ --- циклическая  $\iff \exists a \in G: \ord a = n$.  
    МТФ: $\overline{a}, \overline{a}^2,\ldots$ --- периодична с периодом $p-1$. Утверждение:  $\exists \overline{a}: p-1$ --- наименьший период этой последовательности.
\end{statement}
\begin{remark}
    Пусть $G$ --- группа,  $|G| = p$ --- простое. Тогда  $G \cong (\Z / p\Z, +)$.  $G$ --- циклическая.
\end{remark}
\begin{proof}
    Возьмем $a \neq e$. Тогда  $p \vdots \ord(a) \Rightarrow \ord(a) = 1 \lor \ord(a) = p \Rightarrow a = e \lor \langle a \rangle = G \Rightarrow$
    $G$ --- циклическая  $\Rightarrow G \cong (\Z / p \Z, +)$. 
\end{proof}
\begin{definition}
    $R$ --- ассоциативное кольцо, тогда  $R^* = \{a \in R | \exists a^{-1}\}$ --- группа обратимых элементов.
\end{definition}
Проверим, что $R^*$ --- группа. 
 \begin{itemize}
     \item Проверим замкнутость. $a, b \in R^* \Rightarrow \exists a^{-1}\; \exists b^{-1}: (ab)^{-1} = b^{-1} a^{-1}$.
     \item  $1 \in R^*$. 
     \item $a \in R^*: \exists a^{-1} \Rightarrow \exists \left(a^{-1}\right)^{-1} = a$, значит $a^{-1} \in R^*$. 
\end{itemize}
\begin{remark}
    $a^n = 1 \Rightarrow \in R^*$. Т.к. тут записано, что  $a \cdot a^{n-1} = 1$ --- то есть он обратим.
\end{remark}
\slashn
Рассмотрим $R = \Z / n\Z$. Тогда  $R^* = \{ \overline{a} \in \Z / n\Z \mid \exists \overline{b}: \overline{a} \overline{b} = 1\} = \{\overline{a} \in \Z / n\Z \mid (a, n) = 1\}$. Тогда $|R^*| = \varphi(n)$ --- функция Эйлера.
 \begin{theorem}[Теорема Эйлера]
     $\forall b \in \left(\Z / n\Z\right)^* = b^{\varphi(n)} = 1$
\end{theorem}
\begin{theorem}[Теорема Эйлера]
    $\forall a \in \Z: (a, n) = 1 \Rightarrow a^{\varphi(n)} \equiv 1 \pmod{n}$ 
\end{theorem}
\slashn
Эффективно вычислим $\varphi(n)$:
 \begin{enumerate}
     \item $n = p^k$, $p$ --- простое.

         $\varphi(n) = \{x \in \{1,\ldots,p^k\} \mid (x, p^k)=1\} = \{x \in \{1,\ldots, p^k\} \mid x \not \vdots p\} = p^k - |\{p, 2p,..,p^k\}| = p^k - p^{k-1}$.
     \item $n$ --- составное. $n = p_1^{\alpha_1}p_2^{\alpha_2}\ldots p_k^{\alpha_k}$

         По КТО:  \[\Z / n\Z \cong (\Z / p_1^{\alpha_1} \Z) \times \ldots \times (\Z / p_k^{\alpha_k} \Z).\].
         Тогда заметим, что \[
             (\Z / p_1^{\alpha_1} \Z \times \ldots \times \Z / p_k^{\alpha_k} \Z)^* = (\Z / p_1 \Z)^* \times \ldots \times (\Z / p_k^{\alpha_k} \Z)^*
         .\] Так как если $(x_1,\ldots,x_k)$ --- обратим, то $x_i$ --- обратимы.

         Из этого получаем, что  \[\varphi(n) = |(\Z / n\Z)^*| = |(\Z / p_1^{\alpha_1} \Z \times \ldots \times \Z / p_k^{\alpha_k} \Z)^*| = \prod_{i=1}^k\ |(Z / p_i^{\alpha_i} \Z)^*|.\]
         Получили формулу из а). Применим её: \[
             \varphi(n) = (p_1^{\alpha_1} - p_1^{\alpha_1 - 1})\ldots(p_k^{\alpha_k} - p_k^{\alpha_k - 1}) = n \cdot (1 - \frac{1}{p_1})\cdot\ldots\cdot(1-\frac{1}{p_k})
         .\] 
 \end{enumerate}
 \begin{theorem}[Теорема о первообразном корне]
     $p \in \Z$ --- простое  $\Rightarrow (\Z / p\Z)^*$ --- циклическая.
 \end{theorem}
 \begin{proof}
     В ноябре.
 \end{proof}
 \slashn
 Посмотрим на устройство $\Z / p\Z$.  $\exists a \in Z: \{\overline{a}, \overline{a^2},\ldots,\overline{a^{p-1}}\} = \{\overline{1}, \ldots, \overline{p-1}\}$.

 Тогда как устроены $(\Z / n \Z)^*$ в общем случае?

 Отступление: группа, порожденная множеством.

 \begin{definition}
     $G$-группа  $S \subset G$ --- подгруппа, порожденная множеством  $S$.
    \begin{enumerate}
        \item Наименьшая (по включению) подгруппа $G$, содержащая  $S$.
        \begin{remark}
            $H$ --- подгруппа  $G$:  $H \le G$.
        \end{remark}
        \begin{remark}
            $\{H_i\}_{i \in I}: H_i \le G \Rightarrow \bigcap H_i \le G$.
        \end{remark}
    \item (Явное описание). $\langle S \rangle  = \{a_1^{\varepsilon_1}a_2^{\varepsilon_2}\ldots a_k^{\varepsilon_k} \mid a_i \in S, \varepsilon = \pm 1\}$.
    \end{enumerate}
 \end{definition}
 \slashn Докажем, что 1) равно 2). 
 \begin{proof}
     \slashn
     \begin{enumerate}
         \item Пусть $a_1, a_2, \ldots, a_k \in S$. Тогда для любой $H \le G$ $h \supset S$ верно:
              \begin{enumerate}
                  \item $a_i \in H$.
                  \item  $a_i^{\varepsilon_i} \in H$, так как  $H$ замкнута относительно $^{-1}$
                  \item  $a_1^{\varepsilon_1} a_2^{\varepsilon_2} \ldots a_k^{\varepsilon_k} \in H$, так как $H$ замкнуто относительно  $\cdot$.
             \end{enumerate}

             Значит $H \supset \langle S \rangle \Rightarrow \langle S \rangle \subset \prod_{H \le G \land H \supset S}$.
        
             С другой стороны $H = \langle S \rangle \Rightarrow H \supset S \land H \le G$. $\langle S \rangle$ --- подгруппа:  \[
                 (a_1^{\varepsilon_1} \ldots a_k^{\varepsilon_k}) \cdot (b_i ^ {\mu_i}) = \prod_i ??? 
             .\] 
     \end{enumerate}

 \end{proof}

  \begin{theorem}
      $(\Z / n\Z)^*$ --- циклическая  $\iff \begin{cases} n=p^k & p>2\text{ --- простое} \\ n = 2 p^k & \text{см. выше} \\ n = 2 \lor n = 4\end{cases}$.
 \end{theorem}
 \slashn

    $n = p_1^{\alpha_1} \ldots p_k^{\alpha_k}$. Тогда $(\Z / n \Z)^* = (\Z / p_1^{\alpha_1} \Z)^* \times \ldots \times (\Z / p_k ^{\alpha_k})^*$. 

    Общее утверждение:  $G_1, G_2, G$ --- группы (конечные). 
    \begin{enumerate}
        \item $G \cong G_1 \times G_2$. $(|G_1|, |G_2|) \neq 1 \Rightarrow G$ --- не циклическая.
        \item $(|G_1|, |G_2|) = 1$ и $G_1, G_2$ --- циклическая $\Rightarrow G_1 \times G_2$ --- циклическая. (КТО).
    \end{enumerate}

    Тогда $\forall a \in G_1, b \in G_2 a^{|G_1|} = e_{G_1} \land b^{|G_2|} = e_{G_2} \Rightarrow (a, b)^{\lcm(|G_1|, |G_2|)} = (e, e) \Rightarrow \forall x \in G_1 \times G_2: ord(x) \le \lcm(|G_1|, |G_2|) < |G_1| \cdot |G_2| = |G_1\times G_2| \Rightarrow G_1 \times G_2$ --- не циклическая.
\begin{remark}
    $a^{\varphi(n)} = 1$. Точна ли оценка  $\varphi(n)$? Если  $(\Z / n\Z)^*$ --- циклическая (например, $n$ --- простое). Тогда да. Иначе пусть $n = pq$,  $p,q$ --- простые. Тогда  по Эйлеру $a^{(q-1)(p-1)} = 1$, а на самом деле  $a^{\frac{(q-1)(p-1)}{2}} = 1$. 
\end{remark}
\begin{proof}
    $n = p_1^{\alpha_1} \ldots p_k^{\alpha_k}$. Тогда $|(\Z / p_i^{\alpha_i} \Z)^*| = p_i^{\alpha_i} - p_i^{\alpha_i - 1} \vdots 2$, кроме случая  $p_i = 2, a_i = 1$. Поэтому, если  $k > 2$ или  $k = 2$ $p_1^{\alpha_1}, p_2^{\alpha_2} \neq 2^1 \Rightarrow p_i^{\alpha_i} - p_i^{\alpha_i - 1}$ --- не циклическая  $\Rightarrow$  $(\Z / n\Z)^*$ --- не простое. Остались случаи  $k=1, n=p^a$,  $k=2 n = 2 \cdot p^a$.

    Случай  $n = 2p^a, p \neq 2$.  $(\Z / n \Z)^* \times (\Z / p^a \Z)^* = (\Z / p^a \Z)^*$ --- свели к случаю  $1$.

    Пусть  $n = p^a$.  $p=2$,  $a = 1,2$ --- очев.  $a>2 \Rightarrow (\Z / 2^a \Z)^*$ --- не циклическая. Пусть циклическая, тогда $(\Z / 2^a \Z)^* = \langle x \rangle, \ord x = 2^{a-1}$. Тогда  в $(\Z / 2^a \Z)^*$:  $y^2 = 1 \iff \exists k (x^k)^2 = 1 \iff x^{2k} = 1$. $2k \vdots 2^{a-1} \land 2k \vdots 2^{a-2} \Rightarrow k = 0 \lor k = 2^{a-2}$. $y^2$ --- имеет два решения.
\end{proof}

\begin{theorem}
    $a \in (\Z / p \Z)^*$. Тогда  $x^2 = a$ имеет решение  $\iff a^{\frac{p-1}{2}} = 1$
\end{theorem}
\begin{proof}
    \slashn
    \begin{itemize}
        \item $\Rightarrow$.  $a = x^2 \Rightarrow a^{\frac{p-1}{2}} = (x^2)^{\frac{p-1}{2}} = x^{p-1} = 1$ (МТФ).
        \item $\Leftarrow$.  $a^{\frac{p-1}{2}} = 1$. $\exists c: (\Z / p \Z)^* = \langle c \rangle$.  $\exists k: a = c^k$. Тогда  $a^{\frac{p-1}{2}} = (c^k)^{\frac{p-1}{2}} \iff c^{\frac{k(p-1)}{2}} = 1$ Та как $\ord \frac{k(p-1)}{2} \vdots p - 1$. Тогда  $\frac{k}{2}\in \Z$, то есть $k = 2l$.  $a = c^{2l} = (c^l)^2$.
    \end{itemize}
\end{proof}

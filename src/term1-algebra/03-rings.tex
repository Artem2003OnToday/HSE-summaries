Рассмотрим $a^2 - b^2 = 15^{2021} \iff (a-b)(a + b) = 3^{2021} \cdot 5^{2021} \Rightarrow \begin{cases} a+b=3^k \cdot 5^l \\ a-b=3^{2021-k} \cdot 5^{2021-l} \end{cases} \Rightarrow a = \frac{3^k \cdot 5^l + 3^{2021-k} \cdot 5^{2021-l}}{2}$.  

Уравнение $81a^2-169b^2=15^{2021}$ --- тоже решается. А вот $a^2-2b^2 = 15^{2021} \iff (a-\sqrt{2}b)(s+\sqrt{2}b) = 3^{2021}5^{2021}$ уже не решается в целых чисел. Если вылезать, то надо расписывать разложение $a+\sqrt{2}b$, "3", "5" и единственность разложения на множители.

Еще один пример:  $a^2+b^2=15^{2021}$. Посмотрим на остатки от деления на 4:  $a^2, b^2 \mod 4 \in \{0, 1\}, 15^{2021} \mod 4 = 3$. Но для этого нам нужно понимать что-то по кольцо вычетов по модулю.

\Subsection{Группы}
 \begin{definition}
     Группой называется пара $(G, \ast)$, где  $G$ --- множество, а  $\ast: G \to G$ --- бинарная операция, так что выполнены свойства:
     \begin{enumerate}
         \item $\forall a,b,c \in G:$  $(a \ast b) \ast c = a \ast (b \ast c)$. Ассоциативность.
         \item $\exists e \in G:$ $a \ast e = e \ast a = a$. Существование нейтрального элемента.
         \item $\exists a^{-1}:$  $a \ast a^{-1} = a^{-1} \ast a = e$. Существование обратного элемента.
     \end{enumerate}
 \end{definition}
\slashn
Несколько примеров:
\begin{enumerate}
    \item $(\Z, +)$.  $e=0, a^{-1}=-a$.
    \item  $(\Q \setminus 0, \cdot)$,  $e=1, a^{-1}= \frac{1}{a}$.
    \item $(2^M, \bigtriangleup)$ $e=\varnothing, A^{-1} = A$.
\end{enumerate}
\begin{definition}
    Группа $G$ называется абелевой, если  $\forall x, y \in G:$ $x \ast y = y \ast x$.
\end{definition}
\begin{example}[Главный пример группы]
    Пусть $G=S(M) = \{f: M \to M \mid f\text{ --- биекция}\}$
    \begin{itemize}
    \item Ассоциативность --- упражнение.
    \item Нейтральный элемент --- $f(x) = x$, тождественное отображение.
    \item  $f^{-1}=$ обратная функция. Она существует, так как $f$ --- биекция. 
    \end{itemize}
    \slashn
    Получили группы по композиции.
\end{example}
\begin{example}
    $M=\{1,2,3\}$.  $f_1, f_2: M \to M$ --- биекция.  $f_1$ --- меняет местами  1 и 2: $1 \to 2, 2 \to 1, 3 \to 3$,  $f_2$ переставляет по циклу: $1 \to 2, 2 \to 3, 3 \to 1$. $f_2 \circ f_1: 1 \to 3, 2\to 2, 3\to 1$. $f_1 \circ f_2: 1 \to 1, 2 \to 3, 3 \to 2$. Ну значит группа не абелева.
\end{example}
\slashn
Докажем простейшие свойства групп:
\begin{enumerate}
\item $\exists!$ нейтральный элемент.

    \textbf{Доказательство:} заметим, что $e_1=e_1 \ast e_2 = e_2$
\item $\exists!$ обратный элемент. 

    \textbf{Доказательство:} пусть $b, c$ --- обратные к  $a$. Тогда  $(b\ast a)\ast c = e \ast c = c$, но при этом $b \ast (a \ast c) = b \ast e = b$. Значит  $b=c$.
\item $a \ast b = b \ast c \iff a = c$

    \textbf{Доказательство:} $a \ast b = a \ast c \iff (a^{-1} \ast a) \ast b = (a^{-1} \ast a) \ast c \iff e \ast b = e \ast c \iff b = c$
\end{enumerate}
\Subsection{Кольца}
\begin{definition}
    Кольцо --- тройка $(R, +, \cdot)$ ($R$ --- множество,  $+, \cdot: R \times R \to R$), такая что:
     \begin{enumerate}
         \item[1--4.] $(R, +)$ --- абелева группа. Нейтральный элемент обозначается $0$, обратный к  $a$ ---  $-a$.
         \item[5.] $a\cdot(b+c) = a \cdot b + a \cdot c$ и  $(b+c) \cdot a = b \cdot a + b \cdot c$. Дистрибутивность.
    \end{enumerate}
\end{definition}
\begin{definition}
    Кольцо $R$ называется ассоциативным, если выполнено 
    \begin{itemize}
        \item[6.] $a \cdot (b \cdot c) = (a \cdot b) \cdot c$.
    \end{itemize}
\end{definition}
\begin{definition}
    Кольцо $R$ называется коммутативным, если
    \begin{itemize}
        \item[6.] $a \cdot b = b \cdot a$
    \end{itemize}
\end{definition}
\begin{definition}
    Кольцо $R$ называется кольцом с 1, если  
    \begin{enumerate}
        
        \item[7.] $\exists 1 \in R: 1 \cdot a = a \cdot 1 = a$
    \end{enumerate}
\end{definition}
\slashn
\begin{example}
    $(\Z, +, \cdot)$ --- коммутативное ассоциативное кольцо с 1.
\end{example}
\begin{definition}
    Коммутативное ассоциативное кольцо с 1 называется полем, если выполнена 
    \begin{enumerate}
        \item[8.] $\forall a \in R \ \{0\}$  $\exists b \in R$  $ab = 1 \land 1 \neq 0$
    \end{enumerate}
\end{definition}
\begin{example}
    $(\Q, +, \cdot)$ --- поле, а вот  $(\Z, +, \cdot)$ --- не поле.
\end{example}

\Subsection{Построение кольца вычетов}
\begin{definition}
Пусть $a, b \in \Z$, говорят, что  $a$ сравнимо с  $b$ по модулю  $n$ ($a \equiv b \pmod{n}$), если $n \mid a - b$. Эквивалентное определение:  $a$ и  $b$ имеют одинаковые остатки по модулю  $n$.
\end{definition}
Докажем, что сравнимость по модулю --- отношение эквивалентности.
\begin{itemize}
    \item $a \equiv a \pmod{n} \iff n \mid 0$
    \item $n | a - b \iff n | b - a \Rightarrow a \equiv b \pmod{n} \iff b \equiv a \pmod{n}$.
    \item Транзитивность...
\end{itemize}
\slashn
Наблюдение.  $a \in \Z \rightarrow \overline{a} = \{b \mid a \equiv b\} = \{a + kn \mid k \in \Z\}$. $\Z = \overline{0} \cup \overline{1} \ldots$

\begin{definition}
    Фактор множества по отношению $\equiv$ обозначается  $\Z / n\Z$.
\end{definition}

$\Z \to \Z / n\Z$.Элементы $\Z / n\Z$ называются классами вычетами по модулю.

 \begin{enumerate}
     \item $a \equiv b \pmod{n} \land c \equiv d \pmod{n} \iff a+c \equiv b+d \pmod{n} \land ac \equiv bd \pmod{n}$. 

         Доказательство  $(a+c) - (b+d) = \underbrace{(a-b)}_{\vdots n} - \underbrace{(d-c)}_{\vdots n} \vdots n$. 

         Доказательство $ac - bd = ac - bc + bc - bd = c (a-b) + b(c-d) \vdots n$.

         Значит класс суммы и произведения зависит только от классов множителей и слагаемых.
\end{enumerate}
\begin{theorem}
    Пусть $n \in \N$. Тогда класс $(\Z / n\Z, +, \cdot)$, где $\overline{a}+\overline{b} = \overline{a+b} \land \overline{a} \cdot \overline{b} = \overline{a \cdot b}$ --- ассоциативное коммутативное кольцо с единицей.
\end{theorem}
\begin{proof}
    Все аксиомы --- следствия из $\Z$. Докажем для примера  $(\overline{a} + \overline{b}) + \overline{c} = \overline{a} + (\overline{b} + \overline{c}) = \overline{a+b} + \overline{c} = \overline{(a+b)+c} = \overline{a + (b+c)} = \overline{a} + \overline{(b+c)} = \overline{a} + (\overline{b} + \overline{c}).$
\end{proof}
\slashn
Закон сокращения не очень работает в кольце вычетов по модулю: $2 \cdot 1 = 2 \cdot 4$ ($\pmod 6$), но  $1 \neq 4 \pmod 6$.

\begin{definition}
    Пусть $R$ --- коммутативное ассоциативное кольцо с единицей. Тогда  $\forall a \in R: a\text{ --- делитель} \Rightarrow \exists B \neq 0: ab = 0$.
\end{definition}
\begin{example}
    $n$ --- составное  $n=p_1p_2$ в $\Z / n\Z \overline{p_1} \overline{p_2}=\overline{n}=0$. Значит $p_1,p_2$ --- делители числа.
\end{example}
\begin{lemma}
    $\forall a, b, c \in R ab = ac \land a\text{ --- не делитель 0} \Rightarrow b = c$. 
\end{lemma}
\begin{proof}
    $ab=ac$:  $ab - ac = 0 \iff a(b-c) = 0$.  $a$ --- не делитель 0  $\Rightarrow b-c=0 \iff b = c$.
\end{proof}
\begin{lemma}
    $a \in R a$ --- обратим $\Rightarrow a$ --- не делитель 0.
\end{lemma}
\begin{proof}
    Пусть $ab=0 \Rightarrow a^{-1}(ab) = a^{-1} \cdot 0 = (a^{-1}a)b = 0 \Rightarrow b =0$.
\end{proof}
\begin{remark}
    Обратное неверно: в $\Z$ 2 -- не делитель нуля, но  $\frac{1}{2} \notin \Z$ .
\end{remark}
\begin{theorem}
    $\forall a \in Z: \overline{a} \in \Z / n\Z$. Тогда:  
    \begin{enumerate}
        \item $\overline{a}$ --- обратим $\iff (a, n) = 1$
        \item  $\overline{a}$ --- делитель нуля $\iff (a, b) \neq 1$.
    \end{enumerate}
\end{theorem}
\begin{proof}
    $\overline{a}$ --- обратим  $\iff \exists \overline{b}: \overline{a} - \overline{b} = \overline{1} \iff exists b: ab = 1 \pmod{n} \iff \exists b: ab - 1 \vdots n \iff \exists b, k: ab-1=nk \iff \exists b, k: ab-nk=1 \iff (a,n)  = 1$.

    $(a, n) = 1 \Rightarrow \overline{a}\text{ --- обратим} \Rightarrow$ не делитель нуля.

    $(a, n) = d > 1, a = dx$. Тогда  $\overline{a} \cdot \frac{\overline{n}}{\overline{d}} = \overline{d}x \frac{\overline{n}}{\overline{d}} = \overline{nx} = 0$ и $\frac{\overline{n}}{\overline{d}} \neq 0$. Значит  $9 < |\frac{n}{d}| < n$.
\end{proof}
\begin{consequence}
    $n$ --- простое  $\Rightarrow \Z / n\Z$ --- поле.
\end{consequence}
\begin{proof}
    Достаточно проверить существование обратного. $\overline{a} \neq \overline{0} \iff a \not\vdots n \iff (a, n) = 1 \iff a$ --- обратим.
\end{proof}
\begin{definition}
    $\forall $ ассоциативного кольца с 1 $R$:  $R$ --- называется кольцом без делителей 0 (область целостности), если делитель 0 только 0.  $ab = 0 \iff a = 0 \lor b = 0$  .
\end{definition}
 \begin{remark}
     $R$ --- область  $\Rightarrow ax_1=ax_2 \Rightarrow x_1=x_2$ ($a \neq 0$).
\end{remark}
\slashn
Вернемся к диофантову уравнению $ax+by=1$, $(a, b) = 1$. Тогда $ax = c \pmod{b}$ и  $by = c \pmod{a}$. Тогда  $\overline{a}\overline{x}=\overline{c}$ в  $\Z / n\Z \xRightarrow{(a,b)=1} \overline{x} = \overline{a}^{-1}\overline{c} \pmod{b}$. Тогда $x = x_0+kb$.

\Subsection{Квадратное уравнение}
Посмотрим на $x^2+px+q=0$ в  $\Z / n\Z$. Работает ли  $x_{1,2} = \frac{-p \pm \sqrt{p^2 - 4q}}{2}$. Есть проблемки:
 \begin{enumerate}
     \item $p^2 - 4q$ --- не квадрат в  $\Z / n\Z$ (не решений).
     \item $2 = 0$. Или  $\nexists 2^{-1}$ (нельзя поделить на два).
     \item  $n$ --- не простое. Тогда  $(x-x_1)(x-x_2)\ldots=0$. Тогда не следует, что $x = x_1 \lor x = x_2$. Пример: $x^2-1=0 \pmod{8}$
\end{enumerate}
\Subsection{Китайская теорема об остатках}
Чтобы решать такие уравнения можно свести к простым модулям при помощи китайской теоремы об остатках.

Вопрос такой: как связаны $\Z / n\Z, \Z / m\Z, \Z / mn\Z$. Пусть $P_m: \Z \mapsto \Z / m\Z$, а $P_mn \Z \mapsto \Z / mn\Z$. 

 \begin{definition}
     Гомоморфизмом колец $f: R_1 \mapsto R_2$ называется такое отображение, что $\forall r_1, r_1 \in R_1: f(r_1 + r_2) = f(r_1) + f(r_2), f(r_1r_2)=f(r_1)\cdot f(r_2), f(1) = 1$.
\end{definition}
\begin{definition}
    Гомоморфизмом группы $f: G_1 \mapsto G_2$ называется такое отображение, что $\forall g_1, g_2: f(g_1g_2) = f(g_1) \cdot f(g_2)$.
\end{definition}
\begin{remark}
    $f$ --- гомоморфизм групп  $G_1, G_2 \Rightarrow f(e_{G_1}) = e_{g_2}$.  В частности  $f$ --- гомоморфизм колец  $R_1,R_2 \Rightarrow f(0_{R_1}) = 0_{R_2}$.
\end{remark}
\begin{proof}
    $f(e_{G_1}) = f(e_{G_1} \cdot e_{G_1}) = f(e_{G_1}) \cdot f(e_{G_1})$. Дальше сокращаем.
\end{proof}
\slashn

Существует $P_{mn,m}: P_{mn, m} \cdot P_{mn} = P_{m}$. 
 \begin{proof}
     $P_{mn, m}(\overline{a_{mn}}) = \overline{a_m}$.
\end{proof}
\begin{proof}[Корректность]
    $\overline{a_m} = \overline{b_mn} \iff a \equiv b \pmod{mn} \iff a-b \vdots mn \Rightarrow a-b \vdots m \Rightarrow \overline{a_m} = \overline(b_m)$
\end{proof}
\slashn Аналогично существует гомоморфизм $P_{mn, n}$. То есть $\overline{a_{mn}} \rightarrow (\overline{a_m}, \overline{a_n})$ --- отображение. То есть $\Z / mn \Z \mapsto \Z / m\Z \times \Z / n \Z$.
Отступление.
\begin{definition}
    $R_1, R_2$ --- кольца. Рассмотрим  $(R_1 \times R_2, +, \cdot): (r_1, r_2) +_{R_1\times R_2} (r_1'r_2') \coloneqq (r_1+_{R_1}r_2, r_2+_{R_2}r_2')$. Тоже самое для умножения. Тогда $R_1 \times R_2$ --- тоже кольцо.
\end{definition}
\slashn
Итак мы построили гомоморфизм $\Z / mn \Z \mapsto \Z / m\Z \times \Z / n \Z$. Подумаем про его свойства. Во-первых заметим, что слева $mn$ элементов, но и справа $mn$ элементов!

\begin{definition}
    Биективный гомоморфизм (групп, колец, ...) (называется изоморфизмом, $\cong$) если каждом $a_i$ задано ровно одно  $b_j$ и наоборот.
\end{definition}
\begin{theorem}[Китайская теорема об остатках]
    Пусть $(m, n)=1$, тогда $\Z / mn \Z \cong \Z / m\Z \times \Z / n \Z$.
\end{theorem}
\begin{proof}
    \slashn
    \begin{enumerate}
        \item $i_{m,n}$ --- инъективно. Пусть $i_{m,n}(\overline{a_{m,n}}) = (\overline{a_m}, \overline{a_n})$,  $i_{m,n}(\overline{b_{n, m}}) = (\overline{b_m}, \overline{b_n}) \Rightarrow  a-b \vdots m \land a-b\vdots n \xRightarrow{(n, m) = 1} a - b \vdots mn$.
        \item $i_{m, n}: a \mapsto B$ инъективно: $|A| = |B| \Rightarrow i_{m, n}$ --- суръективно.  
    \end{enumerate}
\end{proof}
\begin{theorem}[КТО 2]
    $m_1,m_2,m_3,\ldots,m_n \in \Z \land (m_i, m_j) = 1 \Rightarrow \Z / m_1,m_2,\ldots,_n \Z \mapsto \Z / m_1\Z \times \Z / m_2\Z \ldots$ - изоморфизм колец. 
\end{theorem}

Теперь мы многочлены будем рассматривать как самостоятельные элементы, а не как функции, ведь сами многочлены можно складывать и умножать! Причем свойства умножения и сложения удовлетворяет требованием кольца! Получили \textbf{Кольцо многочленов над кольцом $\R$}.

Но сначала рассмотрим немного другую штуку: \textbf{кольцо формальных степенных рядов} (отличие будет позже).
\begin{definition}
	Пусть $R$ --- ассоциативное коммутативное кольцо. Тогда кольцо формальных степенных рядов  $R[[x]]$ -- тройка  $(R^{\Z_{\ge 0}}, +, \cdot)$.

    $+$: $(a_0, a_1, a_2, \ldots) +  (b_0, b_1, b_2, \ldots) \coloneqq (a_0 + b_0, a_1 + b_1, \ldots)$

    $\cdot$ (Правило свертки):  $(a_0, a_1, a_2,\ldots) \cdot (b_0, b_1,b_2,\ldots) = (a_0 b_0, a_0b_1+b_1b_0,\ldots)$, по факту: $(a_i)\cdot(b_i) = (c_i), c_n \coloneqq  \sum_0^n a_kb_{n-k}$

    Так же можно представлять $(a_0, a_1, a_2,\ldots) \iff a_0 + a_1x + a_2x^2+\ldots$. То есть, если неформально, то правило свертки --- обычное раскрытие скобок.
\end{definition}
\begin{definition}
    $R^{\Z_{\ge 0}} = \{ f: \Z_{\ge 0} \to R\} = \{(a_0, a_1, \ldots) | a_i \in R\}$
\end{definition}

\begin{theorem}
	$R[[x]]$ --- ассоциативное, коммутативное кольцо. Причем, если  $R$ с единицей, то  $R[[x]]$ --- кольцо с единицей.
\end{theorem}
\begin{proof}
	Заметим, что все аксиомы доказываются супер просто, ведь сложение у нас просто по координатам. Тогда получили очевидность коммутативности и ассоциативности $+$ (следует из коммутативности и ассоциативности $R$). В качестве нуля берется  $0 =(0, 0, 0, 0,\ldots)$. Обратный элемент --- $-(a_0, a_1,a_2,\ldots) = (-a_0, -a_1,-a_2\ldots)$

    Дистрибутивность --- упражнение (из дистрибутивности $R$).
    
    Коммутативность произведения: $c_n = \sum_0^n a_k b_{n-k} = \sum a_k b_l$, где  $k, l \ge 0 \land k+l=n$. Тогда $c_n = \sum_{l=0}^n a_{n-l} b_l = \sum_{l = 0}^m b_l a_{n-l}$ --- формула свертки для $b \cdot a$.

    Если  $\exists 1_R$, то  $(1_R, 0_R, 0_R, \ldots)$ --- нейтральный относительно $\cdot$ в  $R[[x]]$ (упражнение).

    Ассоциативность (упражнение на смирение духа):  $\forall f, g, h \in R[[x]] (f\cdot g) \cdot h = f \cdot (g \cdot h)$. Введем много обозначений: $f=(a_n), g=(b_n), h=(c_n), f \cdot g = (d_n), g \cdot h = (e_n), (f \cdot g) \cdot h = k_n, f \cdot (g \cdot h) = (l_n)$

    Хотим доказать, что $k_n = l_n\ \forall n \in \Z_{\ge 0}$. Тогда \[
        k_n = \sum_{i=0}^n d_i c_{n-i} = \sum_{i=0}^n (\sum_{j=0}^i a_jb_{i-j}) c_{n-i}
    .\]  
    Воспользуемся дистрибутивностью: \[
        k_n = \ldots = \sum_{\mathclap{\substack{0 \le i \le n \\ 0 \le j \le i}}} a_j b_{i-j} c_{n-i}
    .\] 
    Определим $s \coloneqq i - j, t \coloneqq n - i$, тогда  \[
    k_n = \ldots = \sum_{\mathclap{\substack{j, s, t \ge 0 \\ j+s+t = n}}} a_jb_sc_t.
    .\] 
    Аналогично для $l_n$:  \[
    l_n = \ldots = \sum_{\mathclap{\substack{j, s, t \ge 0 \\ j+s+t = n}}} a_jb_sc_k.
    .\] 
\end{proof}
\begin{remark}
   Если $R$ --- не коммутативное кольцо, то стоит различать  $ax^2, x^2a, xax$. 
\end{remark}
\begin{remark}
    Существует инъективный гомоморфизм колец $i: R \to R[[x]]$: $a \to (a, 0, 0, 0,\ldots)$. \textit{Это можно проверить}. 

    Тогда не умаляя общности считаем, что $R$ содержится в  $R[[x]]$ (в качестве подкольца).
\end{remark}
\begin{remark}
    Положим по определению $x \coloneqq (0, 1, 0, 0, 0, \ldots)$. 

    Тогда (упражнение на индукцию) $x^n \coloneqq (0, 0,\ldots, \overbrace{1}^n, 0, 0, \ldots)$ (1 стоит на $n$-ой позиции в \textbf{нумерации с нуля}) 

    Тогда, если $f= (a_0, a_1,a_2,\ldots,a_n, 0, 0, 0)$ ($a_i$ при  $i > n$ равно 0). 

    Тогда $f = a_0 + a_1 \cdot x + a_2 \cdot x^2 + \ldots + a_n \cdot x^n$. 
\end{remark}
\begin{remark}
    $(a_0, a_1, a_2,\ldots) \cdot \underbrace{(0, 1, 0,\ldots)}_{x} = (0, a_0, a_1, \ldots)$
\end{remark}
\begin{consequence}
    $f \divby x$.  $f = (a_i) \land a_0 = 0 \Rightarrow 1 \centernot \divby f$. 
\end{consequence}
\begin{theorem}
	$f = (a_i)$.  $f \in R^*[[x]] \iff a_0 \in R^*$. В частности: $R$ --- поле  $\Rightarrow f$ --- обратим $\iff f \centernot \divby x$. 
\end{theorem}
\begin{proof}
    \slashn
     \begin{itemize}
         \item $\Rightarrow$. $(b_n)$ --- обратный к  $(a_n)$. Тогда,  $(a_0, a_1, \ldots) \cdot (b_0, b_1, \ldots) = (1, 0, 0,\ldots)$.

             $1 = a_0b_0 \Rightarrow a_0\in R^*$.
         \item $\Leftarrow$: будем вычислять последовательность  $(b_0, b_1,\ldots)$. $a_0 \in R^*$, тогда: 

             $a = a_0b_0 \Rightarrow b_0 = a_0^{-1} = \frac{1}{a_0}$. $0 = a_0b_1 + a_1b_0 \Rightarrow \frac{-a_1b_0}{a_0}$. И так далее.

             $0 = \sum_{i=0}^n a_i b_{n-i}$.  $b_n = (-\sum_{i=1}^n a_i b_{n-i})a_0^{-1}$.  

             Построили метод построения $b$, причем все хорошо!
    \end{itemize}
\end{proof}
\begin{example}
    $f = (1, 1, 1, 1, \ldots) = 1+x+x^2+x^3+\ldots$. Тогда $\frac{1}{1+x+x^2+\ldots} = 1-x$. Тогда $1+x+x^2+x^3+\ldots = \frac{1}{1-x}$.
\end{example}
\begin{theorem}
    $R[x]\ (\subset R[[x]]) = \{(a_0,a_1,\ldots \mid \exists N \forall n > N\!: a_n=0\}$ --- финитные последовательности, образуют подкольцо с единицей, называемое \textbf{кольцом многочленов} (вот и то самое отличие от формальных степенных рядов)).
\end{theorem}
\begin{proof}
    Замкнутость по $+$:  $a_n = 0$ при  $n > N_1$ и  $b_n = 0$ при $n > N_2$. Тогда при $n > \max(N_1, N_2) a_n+b_n = 0$.

    Замкнутость по $\cdot$:  $a_n = 0, n > N_1$ и $b_n = 0, n > N_2$. Тогда при  $n > N_1+N_2:$ $c_n = \sum_{i+j=n} a_ib_j = 0$. Так как при  $i + j = N > N_1+N_2 \Rightarrow i > N_1 \lor j > N_2$.

    $1 \in K[x]$!!! (что это значит и что хотел сказать автор --- я не понял)
\end{proof}
\begin{definition}
    $f \in K[x]$ степенью  $f$ называется  $\deg f = \{\max k: a_k \neq 0\}$. Причем $\deg 0 = -\infty$
\end{definition}
\begin{properties}
    \slashn
    \begin{enumerate}
        \item $\deg (f+g) \le \max(\deg f, \deg g)$. Причем $\deg f \neq \deg g \to \deg(f+g) = \max(\deg f, \deg g)$. 
        \item $\deg(f\cdot g) \le \deg f + \deg g$, а если $R$ --- область целостности, то  $\deg (fg) = \deg f + \deg g$.
    \end{enumerate}
\end{properties}
\begin{consequence}
    $R$ --- область целостности  $\Rightarrow R[x]$ --- область целостности.
\end{consequence}
\slashn
Теперь у нас $K$ --- поле.
\begin{theorem}[О делении с остатком]
    $f, g \in K[x]$ $g \neq 0$. Тогда  $\exists! q, r \in K[x]: f = g\cdot q + r, \deg r < \deg g$.
\end{theorem}
\begin{proof}
    Существование докажем индукцией по $n = \deg(f)$. 

    База: все $n < \deg(g)$, для них мы можем положить $q = 0, r = f$.

    Переход: $n \to n + 1, n + 1 \ge \deg(g)$. Пусть $\deg(g) = k$. Тогда $g = ax^k + g_1$, причём $a \neq 0$ и $\deg(g_1) < k$. Аналогично $f = bx^{n + 1} + f_1, \deg(f_1) < n + 1$. Понизим степень $f$ за счёт $g$: пусть $f_2 = f - \frac{b}a x^{n+1-k}g = bx^{n+1} + f_1 - bx^{n+1} - \frac{b}a x^{n + 1 - k}g_1 = f_1 - \frac{b}a x^{n + 1 - k}g_1$. Проверим понижение: $\deg(f_2) \le \max(f_1 - \frac{b}a x^{n + 1 - k}g_1) \le \max(n, n + 1 - k + k - 1) = n$, что меньше $\deg(f) = n + 1$. 

    По индукционному предположению, $f_2$ на $g$ делить умеем: найдутся $q_1, r$ такие, что $f_2 = gq_1 + r, \deg(r) < \deg(g)$. Тогда умеем делить и $f$: $f = f_2 + \frac{b}a x^{n + 1 - k}g = g(q_1 + \frac{b}a x^{n + 1 - k} + r$ --- ровно то, что надо. Существование доказали.

    Докажем единственность (по сути, как в $\Z$): пусть $\exists f, g: f = gq+r$ и $f = gq_1 + r_1, \deg(r), \deg(r_1) < \deg(g)$. Приравняем правые части и перегруппируем: $g(q - q_1) = r_1 - r$. Если обе части равенства не равны 0 в $K[x]$, то $\deg(g(q - q_1)) = \deg(g) + \deg(q - q_1)) \ge \deg(g)$, но $\deg(r_1 - r) \le \max(\deg(r), \deg(r_1)) < \deg(g)$. Противорчение, поскольку $g(q - q_1)$ и $r_1 - r$ --- один и тот же многочлен. Значит $r_1 - r = 0$, значит $r = r_1, q = q_1$, поскольку $K[x]$ --- область целостности.
\end{proof}
\begin{consequence}
    $R$ --- коммутативное, ассоциативное кольцо  $a \in R$. Тогда $\exists$ гомоморфизм колец  $R[x] \to R: a_0 + a_1 x + \ldots + a_n x^n \mapsto a_0 + a_1 \cdot a + \ldots + a_n a^n$ --- гомоморфизм эвалюации. 

    С другой стороны $f \in R[x]$ --- полиномиальная функция.  $F_f: R \to R$  $a \mapsto \text{ev}_a(f)$.
\end{consequence}
\begin{consequence}
    $f, g \xrightarrow[\text{Евклида}]{\text{Алгоритм}} h = (f, g)\ h= u_1f + u_2g$. А значит, у $\gcd$ корнями будут общие корни  $f$ и  $g$.
\end{consequence}
\begin{definition}
    $f \in R[x]$.  $a\in R$ --- корень  $f$, если  $F_f(a) = 0$.
\end{definition}
\begin{theorem}[Безу]
    $K$ --- поле.  $f \in K[x]$. $a \in K$. $f = (x-a)g + r$ --- деление с остатком.
    \begin{enumerate}
        \item $r = f(a)$.
	\item  $r = 0 \iff f \divby (x-a)$ (тут $r$ можно заменить на $f(a)$, сути не меняет)
    \end{enumerate}
\end{theorem}
\begin{proof}
    $f = (x-a) \cdot g + r$,  $\deg r < \deg (x-a) = 1 \Rightarrow \deg r = 0 \lor \deg r = -\infty \iff r = c \in K$.

    $F_f(a) = F_{x-a}(a)F_g(a) + F_r(a)$.  $f(a) = (a - a)g(a) + r \iff r = f(a)$. 
\end{proof}
\begin{consequence}
    $\deg f = n, f \in K[x], f \neq 0 \Rightarrow$ существует не более  $n$ корней  $f$ в  $K$.
\end{consequence}
\begin{proof}
    По индукции по $n$.
     \begin{itemize}
         \item База $n = 0$  $f=r \neq 0$ --- 0 корней.
         \item Переход  $n \to n+1$:

              $\deg f = n + 1$. Нет корней  $\Rightarrow 0 \le n + 1$.

              Существует $a$ --- корень.  $f = (x-a) \widetilde{f}, \deg \widetilde{f} = n$. У $\widetilde{f}$ не более  $n$ корней  $\Rightarrow$ у  $f$ не более  $n+1$ корня.

              С другой стороны $b$ --- корень  $f \Rightarrow f(b) = 0$. $(b-a) \widetilde{f}(b) = 0 \xRightarrow{k\text{ --- о. ц. }} b - a = 0 \lor \widetilde{f} = 0 \iff b = a \lor b$ --- корень $\widetilde{f}$. Таких не более $n$, а значит у  $f$ не более  $n+1$ корня.
    \end{itemize}
\end{proof}
\slashn
$f \in K[x]$.  $f \leadsto F_f\!: K \to K$ --- полиномиальная функция. Верно ли  $F_f = F_g \Rightarrow f=g$? 
\begin{theorem}[Теорема о формальном и функциональном равенстве]
    Пусть $K$ --- поле,  $f, g \in K[x]$,  $|K| > \max(\deg f, \deg g)$, например,  $K$ --- бесконечно. Тогда  $F_f = F_g \Rightarrow f=g$.
\end{theorem}
\begin{proof}
	$F_f = F_g \Rightarrow f(k) = g(k)\ \forall k \in K \Rightarrow (f-g)(k)=0\ \forall k \in K$. По свойствам степени знаем, что $\deg (f - g) \le \max(\deg f, \deg g) < |K|$, а значит $(f-g)$, многочлен степени меньше $|K|$ имеет $|K|$ корней, т.е. количество корней больше степени многочлена, а значит $(f-g)$ --- константный ноль, т.е. $f=g$
\end{proof}
\begin{remark}
	Для $K = \Q, \R$ из функционального равенства следует равенство формы ($f=g$)
\end{remark}
\begin{remark}
	$K = \Z / 8 \Z$. Тогда у $x^2 - 1 = 0$ есть 4 корня: $\overline{1}, \overline{3}, \overline{5}, \overline{7}$. И у $x^2 - 2x = 0$ 4 корня: $\overline{0}, \overline{2}, \overline{4}, \overline{6}$. Тогда, т.к. при всех $x \in K$, то $(x^2-1)(x^2-2x) = 0; x^4 + 2x = 2x^3 + x^2$ как функции. При этом $\max(\deg) = 4 < 8$, и как многочлены они не равны!
\end{remark}
\begin{remark}
	$(\Z / p \Z)^*$: $x^p = x$, т.е. $x^{p - 1} = 1$. Всё нормально, т.к. у нас многочлен степени не меньше, чем мощность множества: $p - 1 \nless |(\Z / p \Z)^*| = p - 1$. (На самом деле написанное --- один из вариантов интерпретации исходного текста, который не сохранился. Если у вас есть какая-либо другая информация по данному пункту --- сообщите кому-нибудь из нас)
\end{remark}
\begin{remark}
    Рассмотрим $\Z / p \Z[x]$ --- бесконечное кольцо.  $f \leadsto F_f\!: \Z / p \Z \to \Z / p \Z$ --- не более $p^p$ отображений. Докажем:  $\Z / p\Z[x]_{p-1} \coloneqq \{f \mid \deg f \le p-1\}$, а таких --- $p^p$

    $\Z / p \Z[x]_{p-1} \Leftrightarrow \{\text{отображения}\  \Z / p\Z \to \Z / p\Z \}$, а значит и таких отображений тоже не более чем $p^p$.
\end{remark}
\Subsection{Интерполяция}
\begin{definition}
    Интерполяционная задача в поле $K$ --- набор данных  $x_1, x_2, \ldots, x_n \in K\ (x_i \neq x_j)$, $y_1, y_2,\ldots,y_n \in K$.
    
    Задача заключается в поиске $f \in K[x]\!: f(x_i) = y_i\ \forall i \in 1..n$. 

     $x_i$ --- узлы интерполяции. 
\end{definition}
\begin{theorem}
    В поле любая интерполяционная задача с $n$ узлами имеет единственное решение  $f_0$ среди многочленов степени $<n$.
\end{theorem}
\begin{proof}
    \slashn
    \begin{itemize}
    \item Единственность. Пусть $f_0, f_1$ --- два решения.  $\deg(f_i) < n$.  

        $f_0(x_c) = y_c = f_1(x_c)$. Тогда возьмем $g \coloneqq f_0 - f_1$. Заметим, что у него  $n$ корней, но  $\deg g< n$. Значит  $f_0=f_1$
\item Существование: рассмотрим задачу $\chi_i\!: \chi_i(x_i) = 1, \chi_i(x_j) = 0$, если  $i \neq j$. Её решение:  $L_i = \frac{(x-x_1)(x-x_2)\ldots(x - x_{i-1})(x - x_{i+1})\ldots(x - x_n)}{(x_i - x_1)(x_i - x_2)\ldots(x_i - x_{i - 1}) (x_i - x_{i+1}) \ldots (x_i - x_n)}$ --- в числителе условие на 0, в знаменателе на 1.

	Тогда $f_0 \coloneqq y_1 L_1 + y_2 L_2 + \ldots +y_nL_n$. Тогда для $\forall i\!: f_0(x_i) = y_1L_1(x_i) + \ldots$ во всех слагаемых, кроме $y_i \cdot L_i(x_i)$ равно 0, а данное слагаемое равно  $y_i$. $\deg f_0 \le \max(\deg(L_i)) = n - 1 < n$
    \end{itemize}
\end{proof}
\begin{definition}
Интерполяционный полином Лагранжа: \[f_0 = \sum_i \frac{y_i \prod_{j \neq i}(x - x_j)}{\prod_{j \neq i}(x_i - x_j)},\ \deg f_0 < \max(\deg L_i) = n-1\]
\end{definition}

\Subsection{Закрываем долг}
\begin{theorem}
    $(\Z / p \Z)*$ --- циклическая группа, то  есть  $\exists a \in \Z / p \Z\!: \{a, a^2,\ldots,a^{p-1}\} = \{\overline{1}, \overline{2},\ldots, \overline{p-1}\}$, то есть $\ord a = p - 1$,  $a$ --- первообразный корень.
\end{theorem}
\begin{lemma}
    Пусть $a \in G$ --- группа,  $\ord a = d$. Тогда  $\ord(a^k) = \frac{d}{(d, k)}$
\end{lemma}
\begin{proof}
    Пусть $l = \ord a^k$.

	$(a^k)^l = e \iff a^{kl} = e \iff kl \divby \ord(a) = d$. Тогда, если $k = (d, k) \cdot k'$  и  $d = (d, k) d'$, то  $(d, k) \cdot k' \cdot l \divby (d, k) \cdot d' \iff k' \cdot l \divby d' \xLeftrightarrow{(k', d') = 1} l \divby d' = \frac{d}{(d, k)}, \min l = \frac{d}{(d, k)}$ 
\end{proof}
\begin{lemma}
    $\forall n \in \N\ \sum_{d \mid n} \varphi(d) = n$
\end{lemma}
\begin{proof}
    Пусть $d_1, d_2, \ldots, d_k$ --- все натуральные делители $n$.  $n = |\{1, 2, \ldots, n\}| \eqqcolon |A|$.

    Хотим разбить множество $A = A_1 \cup A_2 \cup \ldots \cup A_k$, причем $A_i \cap A_j = \varnothing$ и  $|A_i| = \varphi(d_i)$, этим мы докажем лемму.

    $A_i = \{ a \in A \mid (a, n) = \frac{n}{d_i}\}$. Заметим, $d_1, \ldots d_k$ --- все делители $n \Rightarrow \frac{n}{d_1}, \ldots \frac{n}{d_k}$ --- все делители $n$. И понятно, что  $\forall a\ (a, n)$ --- какой-то делитель  $n$. 

     Поэтому $A = A_1 \cup A_2 \cup \ldots \cup A_k\, A_i \cap A_j = \varnothing$.

     $a \in A_i \iff (a, n) = \frac{n}{d_i}$. Значит $a = \frac{n}{d_i}k$, $(\frac{n}{d_i}k, n) = \frac{n}{d_i} \iff (\frac{n}{d_i}k, \frac{n}{d_i}d_i) = \frac{n}{d_i} \iff (k, d_i) = 1$. 

     Тогда $|A_i| = |\{ k \mid k \le d_i \land (k, d_i) = 1\}| = \varphi(d_i)$.
\end{proof}
\begin{lemma}
    Количество элементов порядка $d$ в  $(\Z / p \Z)*$ равно либо 0, либо  $\varphi(d)$. 
\end{lemma}
\begin{proof}
    Например, $p-1 \centernot \divby d \Rightarrow $ кол-во равно  $0$. 

    Пусть  $\exists a\!: \ord a = d\ a^d = 1$, $a, a^2, \ldots, a^d = 1$ --- различные элементы. Тогда $\forall k=1..d\ (a^k)^d = (a^d)^k = 1$, то есть это  $d$ решений  $x^d = 1$. Других решений нет, так как  $x^d - 1$ имеет  $\le d$ корней.

    Пусть $\ord b = d \Rightarrow b^d = 1 \Rightarrow b = a^k$,  $k = 1..d$. Тогда по предыдущей лемме  $\ord a^k = \frac{d}{(d, k)} \Rightarrow (d, k) = 1$. 

    Тогда $(k, d)= 1 \Rightarrow \ord(a^k) = d$. То есть все элементы порядка  $d$ это  $\{a^k \mid 1 \le k \le d \land (k, d) = 1 \}$.
\end{proof}
\begin{proof}[Доказательство теоремы] 
    $B_d \subset (\Z / p\Z)*$, такие что $B_d = \{x \in (\Z / p \Z)* \mid \ord x = d\}$. 

    Тогда получится, что  $(\Z / p \Z)^* = B_{d_1} \cup \ldots \cup B_{d_k}$, $d_i$ --- делители $p - 1$. 

    $p-1 = |(\Z / p \Z)^*| = \sum |B_{d_i}|$ по лемме 3  каждое слагаемое  $0$ или  $\varphi(d_i)$, а по лемме 2 $p - 1 = \sum_{i = 1}^k \varphi(d_i)$. А значит в первой сумме каждое слагаемое  $\varphi(d_i)$. 

    В том числе  $|B_{p-1}| = \varphi(p - 1) \neq 0$, то есть  $\exists $ элементы порядка  $p-1$.
\end{proof}
\begin{remark}
    $K$ --- не область целостности  $\Rightarrow$ не выполняется ОТА для многочленов.
    
    $\Z / 8\Z\!: x^2-1 = (x-1)(x+1) = (x-3)(x+3)$
\end{remark}

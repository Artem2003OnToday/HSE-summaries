Рассмотрим простейшую систему уравнений: 
\begin{align*}
    \begin{cases}
        ax + by = e \\
        cx + dy = f.
    \end{cases} \iff x \cdot\binom{a}{c} + y \cdot \binom{b}{d} = \binom{e}{f}
\end{align*}
По сути задача: выразить $\binom{e}{f}$ через $\binom{a}{c},\ \binom{b}{d}$: так как  $x \cdot \binom{a}{c} = \binom{xa}{xc}$, тогда  $\binom{xa}{xc} + \binom{yb}{yd} = \binom{xa + yb}{xc + yd}$.

\begin{definition}
 $x\binom{a}{c} + y\binom{b}{d}$ --- линейная комбинация  $\binom{a}{c}$ и  $\binom{b}{d}$.
\end{definition}
\begin{definition}
    $\left\{x \binom{a}{c} + y \binom{b}{d}\right\}$ --- линейная оболочка  $\binom{a}{c}$ и  $\binom{b}{d}$. Она обозначается  $\langle \binom{a}{c}, \binom{b}{d} \rangle$.
\end{definition}
\begin{definition}
    Пусть $R$ --- кольцо. \\
    Множество $\left\{ \begin{pmatrix} a_1 \\ a_2 \\ \vdots \\ a_n \end{pmatrix} \mid a_1, a_2, \ldots, a_n \in R\right\}$ --- называется $n$-мерным арифметическим (координатным) пространством (пространством столбцов) над $R$, обозначается $R^n$.

    на котором мы ещё определяем операции сложения и умножения на скаляр:
    \begin{itemize}
        \item $\begin{pmatrix} a_1 \\ a_2 \\ \vdots \\ a_n \end{pmatrix} + \begin{pmatrix} b_1 \\ b_2 \\ \vdots \\ b_n \end{pmatrix} = \begin{pmatrix} a_1 + b_1\\ a_2 + b_2 \\ \vdots \\ a_n + b_n \end{pmatrix}$ 
        \item $r \cdot \begin{pmatrix} a_1 \\ a_2 \\ \vdots \\ a_n \end{pmatrix} = \begin{pmatrix} ra_1 \\ ra_2 \\ \vdots \\ ra_n \end{pmatrix} \forall r \in R$  
    \end{itemize}
\end{definition}
\begin{definition}
	Аналогично пространству столбцов, можно определить пространство строк. Всё ровно аналогично, но теперь элементы расположены в строку. Обозначается ${}^n K$
\end{definition}
\begin{definition}
    Пусть $K$ --- поле. Векторное пространство над $K$ --- тройка  $\left( V, +, \cdot \right)$, где  $V$ --- множество,  $+\!: V \times V \to V$,  $\cdot\!: K \times V \to V$. Причем:
     \begin{enumerate}
         \item [1--4] $(V, +)$ --- абелева группа.
     \setcounter{enumi}{0}
     \item $a + b = b+a\ \forall a, b \in V$.
     \item  $(a+b)+c = a+(b+c) \forall a, b, c \in V$
     \item  $\exists \overline{0}\!: a + \overline{0} = a \forall a \in V$
     \item  $\forall a \in V\ \exists (-a) \in V \!: a+(-a) = \overline{0}$
     \item $(k_1k_2)v = k_1(k_2v)$ (ассоциативность умножения на скаляр)
     \item $(k_1+k_2)v = k_1v+k_2v$ (дистрибутивность умножения вектора на скаляр относительно сложения скаляров)
     \item $k(v_1 + v_2) = kv_1 + kv_2$ (дистрибутивность умножения вектора на скаляр относительно сложения векторов)
     \item $1_K \cdot v = v$ (унитарность, единица поля $K$ является единицей и относительно умножения вектора на скаляр)
    \end{enumerate}

    Здесь и далее (и немного ранее) скалярами называются элементы поля $K$, а векторами --- элементы множества $V$.
\end{definition}
\begin{remark}
    $V$ --- векторное пространство над  $K$. Тогда: 
     \begin{itemize}
         \item $0 \cdot v = \overline{0}\ \forall v \in V$.
         \item  $k \cdot \overline{0} = \overline{0}\ \forall k \in K$.
         \item $(-1)\cdot v = -v\ \forall v \in V$.
    \end{itemize}
\end{remark}
\begin{remark}
    Из определений 2--8 следует 1.
\end{remark}
\begin{definition}
    Пусть $R$ --- кольцо. \\
    Тройка  $\left(V, +, \cdot\right)$ с аксиомами 1-8 называется модулем над $R$.
\end{definition}
\begin{remark}
	Абелевы группы($V$ --- абелева, а умножение на скаляр выкинули) $\implies$ модули над  $\Z$.
\end{remark}
\begin{definition}
    $V$ --- векторное пространство над  $K$. $v_1, v_2, \ldots, v_n \in V$. $a_1, a_2, \ldots, a_n \in K$. Тогда $\sum a_i v_i$ --- линейная комбинация  $v_1, v_2, \ldots, v_n$.
\end{definition}
\begin{definition}
    Пусть $M$ --- множесто векторов: $M \subset V$, тогда $\langle M \rangle = \left\{ a_1v_1 + a_2v_2 + \ldots + a_k v_k \mid a_i \in K, v_i \in M\right\}$ называется линейной оболочкой множества $M$.
\end{definition}
\begin{definition}
    Подпространство $V$ --- подмножество  $U \subset V$, такое что  $(U, +_V, \cdot_V)$ --- векторное пространство. 
\end{definition}
\begin{statement}
    $U \subset V$ --- подпространство  $\iff$ $U$ --- замкнуто, т.е. все операции с элементами  $U$ лежат в $U$.
\end{statement}
\begin{example}
    $\prescript{n}{}{K}$ --- арифметическое пространство строк.

    Пусть $v_1, v_2, \ldots, v_m \in K^n$, т.е. векторы в пространстве столбцов. Рассмотрим $x_1v_1+x_2v_2+\ldots+x_mv_m = 0 = \begin{pmatrix} 0 \\ 0 \\ \vdots \\ 0 \end{pmatrix}$ --- однородную систему линейных уравнений ($x_i$ --- неизвестные). Рассмотрим всё множество решений --- строк $(x_1, x_2, \ldots, x_m) \in \prescript{n}{}{K}$. Утверждение --- оно является подпространством в $\prescript{n}{}{K}$. Для этого нужно просто проверить, что сумма двух решений --- тоже решение, и решение, умноженное на какой-либо скаляр всё ещё остаётся решением. Доказывается просто расписав почленно $x_i v_i + y_i v_i = (x_i + y_i) v_i$ и получив итоговое равенство 0. Домножение на скаляр очевидно. 

    Полученное пространство не является всем пространством строк (${}^n K$), но является его подпространством, как мы только что доказали.
\end{example}
\slashn

Обозначение: $U$ --- подмножество  $V$:  $U \le V$.

\begin{statement}
    $V_1, V_2, \le V \implies V_1 \cap V_2 \le V$.
\end{statement}
\begin{proof}
    Очевидно!
\end{proof}
\begin{definition}
   Сумма по Минковскому: $A, B \subset V\!: A + V \coloneqq \{a + b \mid a \in A \land b \in B\}$. 
\end{definition}
\begin{statement}[Сумма по Минковскому]
    $V_1, V_2 \le V \implies V_1 + V_2 \le V$.
\end{statement}
\begin{proof}
    \slashn
    \begin{itemize}
	    \item $x, y \in V_1 + V_2 \iff x = v_1 + v_2, y = v'_1 + v'_2$, где $v_1, v'_1 \in V_1$, $v_2, v'_2 \in V_2$. Тогда $x + y = (v_1 + v'_1) + (v_2 + v'_2), (v_1 + v'_1) \in V_1, (v_2 + v'_2) \in V_2 \implies x + y \in V_1 + V_2$
        \item $k \cdot x$ --- очевидно.
    \end{itemize}
\end{proof}
\begin{remark}
    $M \subset V$,  $\langle M \rangle = \bigcap\limits_{\mathclap{\substack{U \le V \\ U \supset M}}} U$, доказывается как аналогичное утверждение из первого семестра.
\end{remark}
\begin{definition}
    $V_1, V_2$ --- векторные пространствами над $K$. Тогда  $f\!: V_1 \to V_2$ --- гомоморфизм (линейное отображение), если 
    \begin{enumerate}
        \item $f(v_1+v_2) = f(v_1) + f(v_2)\ \forall v_1, v_2 \in V_1$.
        \item $f(kv) = k f(v)$.
    \end{enumerate}

    Если при этом $f$ --- биекция, то  $f$ --- изоморфизм.
\end{definition}
\begin{definition}
    Координатизация --- сопоставление элементам векторного пространства координат пространства, являющимся изоморфным этому пространству, ака построение гомоморфизма:

$$
\forall v \in V, v \to \begin{pmatrix} k_1 \\ k_2 \\ \vdots \\ k_n \end{pmatrix} , k_i \in K
$$
\end{definition}

Верно ли, что любое векторное пространство изоморфно какому-то $K^n$? Да, если правильно понимать, что за $n$, и вообще, мы это чуть позже докажем.

\begin{example}[векторных пространств]
\slashn
\begin{enumerate}
	\item $K$ --- векторное пространство над $K$ (следует из аксиом поля)
     \item Векторы над плоскостью/пространством.
     \item $K\left[x\right]_n = \{ f \in K[x] \mid \deg f \le n \}$. Тогда $K[x]_n \cong K^{n+1}$. 
     \item $M$ --- множество,  $K$ --- поле. Тогда  $V = \{ f\!: M \to K\}$ (множество функций из $M$ в $K$) --- векторное пространство:
         \begin{itemize}
            \item $(f_1 + f_2)(m) \coloneqq f_1(m) + f_2(m)\ \forall m \in M$.
            \item $(kf)(m) = k \cdot f(m)\ \forall k \in K$.
         \end{itemize}

	 По сути, каждая такая функция задаётся значениями в каждой точке $M$, и тогда получаем $f$ --- $\{ f(m) \in K | m \in M\}$, что есть, по сути, $K^{|M|}$
 \item[4'.] $M=K=\R$,  $C_0(\R)$ --- непрерывные функции  $\R \to \R$.  $C_0(\R) \le (a_0, a_1, \ldots)$. Значения во всех рациональных точках. (Любая такая функция задаётся своими значениями во всех рациональных точках, а все рациональные точки можно пронумеровать и составить последовательность, и тогда каждая такая фукнция задаётся последовательностью значений во всех своих рациональных точках)
     \item Последовательность фиббоначиевого типа: $a_n = a_{n-1} + a_{n-2}$. Тогда множество таких последовательностей --- векторное пространство  $\cong \R^2$
     \item $M$ --- множество.  $V = 2^M$,  $K = \Z / 2 \Z$, $M_1 + M_2 \coloneqq M_1 \bigtriangleup M_2,\ 0 \cdot M = \emptyset, 1 \cdot M = M$. Тогда $V$ --- векторное пространство над  $\Z / 2 \Z$,  $V \cong \prescript{n}{}{(\Z / 2 \Z)}$, а координатизация тут --- битовая строка из 0 и 1.
\end{enumerate}
\end{example}

Но в любом ли векторном пространстве есть координатизация? Да, это мы докажем, но чуть позже, смотри дальше.
\begin{definition}
	$V$ --- векторное пространство над  $K$.  $\{ v_i \}_{i \in I}$ (множество векторов) называется базисом  $V$, если  $\forall v \in V\ \exists! \{a_i\}_{i \in I}\text{(множество коэффициентов)}, a_i \in K\!: v = \sum_{i \in I} a_i v_i$, из которых почти все (т.е. все, кроме какого-то конечного числа) $a_i = 0$
\end{definition}
\begin{remark}
    В терминах этого определения $I = \{1, 2, \ldots, n\}$ $V \leftrightarrow \begin{pmatrix} a_1 \\ a_2 \\ \vdots \\ a_n \end{pmatrix}$, то есть $V \cong K^n$.
\end{remark}
\begin{definition}
    $V$ --- векторное пространство над полем  $K$, тогда  $\{v_i\}_{i \in I}$ называется линейно независимой системой (ЛНЗ), если выполнено одно из равносильных утверждений:
     \begin{enumerate}
         \item $\nexists i \in I\!: v_i = \sum\limits_{j \neq i} a_jv_j$ 
         \item $\forall \{a_i\} \in K\!: \sum a_i v_i = 0 \implies a_i = 0\ \forall i \in I$.
    \end{enumerate}
\end{definition}
\begin{proof}
	$2 \implies 1$.  Пусть  $\exists i\!: v_i = \sum a_j v_j \implies \sum a_j v_j - v_i = 0 \xRightarrow{a_i = -1}$ не выполняется второе (не все коэффициенты равны нулю).

    $1 \implies 2$. Пусть  $a_i v_i = 0$, причем  $\exists a_j \neq 0$. Тогда перенесём $a_j v_j$ в левую часть и разделим на $-a_j$ и получить  $v_j = \sum\limits_{i \neq j} b_i v_i$, т.е. выразили, противоречие с первым пунктом.
\end{proof}
\begin{theorem}[Равносильное определение базиса]
    $\{v_i\}_{i \in I}, v_i \in V$,  $V$ --- векторное пространство над $K$.
     \begin{enumerate} 
         \item $\{v_i\}$ --- базис.
         \item $\{v_i\}$ --- линейно независимая система и  $\langle \{v_i\} \rangle = V$.  $\{v_i\}$ --- порождающая система.
	 \item  $\{v_i\}$ --- максимальная линейно независимая система, т.е. $\forall v \in V\!: \{v_i\}_{i \in I} \cup \{v\}$ --- линейно зависимая.
         \item  $\{v_i\}$ --- минимальная порождающая система. То есть выкидывание любого вектора делает систему не порождающей.
    \end{enumerate}
\end{theorem}
\begin{proof}
    \slashn
     \begin{itemize}
	     \item $1 \implies 2$.  $\{v_i\}$ --- базис  $\implies \{v_i\}$ порождающая по определению. Причем если $\sum a_i v_i = 0$, то $a_i = 0$, иначе получили два разложения для нуля (всегда есть разложение со всеми нулевыми коэффициентами), тогда получили, что $\{v_i\}$ --- Л.Н.С.
         \item $2 \implies 1$.  $\forall v \in V\!: v = \sum a_i v_i$, так как $\{v_i\}$ --- порождающая. Тогда докажем единственность: пусть существуют $\sum a'_i v_i = v = \sum a_i v_i$. Тогда возьмем разность: $0 = \sum (a_i - a'_i) v_i \iff a_i - a'_i = 0 \iff a_i = a'_i$.
	 \item  $3 \implies 2$. Нужно доказать, что $\{v_i\}$ --- порождающая система. Рассмотрим произвольный $v \in V$. Знаем, что ${v_i} \cup {v}$ --- линейно зависимая, значит $\exists a: \sum a_i v_i + av = 0$ и не все $a$ равны нулю. Легко понять, что $a\neq 0$, иначе исходная система линейно зависимая, а тогда можно выразить вектор $v$ --- $v = \sum\frac{a_i}{-a}v_i$, умеем выражать любой вектор --- значит мы порождающая система.
         \item $2 \implies 4$. Пусть наша  $\{v_i\}$ --- ЛНЗ и порождающая, хотим доказать, что тогда она минимальная порождающая. Пусть это не так, тогда если она не минимальная порождающая, то убрав один вектор, мы сможем его получить при помощи других наших векторов  $\Rightarrow$ исходная система линейно зависима, противоречие.
    \end{itemize}
\end{proof}

\begin{definition}
    $V$ --- векторное пространство над  $K$.

    $V$ называется конечномерным, если  $\exists$ конечная порождающая система, т.е. $V = \langle v_1, v_2, \cdots, v_n \rangle$.
\end{definition}
\begin{lemma}
    Из любой конечной порождающей системы $V = \langle v_1, v_2, \ldots, v_n \rangle$ можно выбрать базис.
\end{lemma}
\begin{proof}
    Во-первых, если она линейно независима, то все очевидно, вот и базис.

    Иначе, пусть $\exists v_i = \sum\limits_{j \neq i}a_j b_j$. Тогда заметим, что система никак не пострадает, если убрать $V_i$ из системы: мы все равно можем его получить при помощи остальных векторов. 

    Теперь можно продолжить этот процесс до момента, когда эта система станет линейно независимой. Так как система была конечной, то этот процесс когда-либо закончится (например, если выкинем все вектора).
\end{proof}

\begin{remark}
    Пример пространства с пустым базисом: у множества $V = \{ 0 \}$ базис равен  $\emptyset$.
\end{remark}
\begin{consequence}
    В любом конечном пространстве есть базис.
\end{consequence}
\begin{remark}
    В любом пространстве есть базис.
\end{remark}
\begin{example}
    \begin{align*}
        K[x] = \langle 1, x, x^2,  \ldots \rangle \\
        K[[x]] = \langle ??? \rangle
    \end{align*}

    У $K[[x]]$ есть базис, но на человеческом нельзя задать.
    
    У $\R$ тоже есть базис, но как его задать --- вопрос.
\end{example}
\begin{definition}
    Размерность пространства $\dim V$ --- количество элементов в базисе.
\end{definition}

Это хорошо, но непонятно, почему это определение корректное, т.е. почему во всех базисах пространста одинаковое количество элементов.

\begin{theorem}
    Все базисы имеют поровну элементов.
\end{theorem}
\begin{lemma}[Лемма о линейной зависимости линейных комбинаций]
    Пусть $u_1, \ldots, u_n \in \langle v_1, v_2, \ldots, v_n \rangle$, $m > n$. Тогда  $u_1, \ldots, u_m$ линейно зависима.
\end{lemma}
\begin{proof}
	\begin{lemma}[О замене]
		$\langle v_1, v_2, \ldots v_n \rangle = \langle \sum a_i v_i, v_2, \ldots, v_n \rangle$, т.е. можно заменить элемент на линейную комбинацию элементов без изменения линейной оболочки, если $v_1 \neq 0$ (?)
    \end{lemma}
    \begin{proof}
	    $\sum a_i v_i, v_2, \ldots, v_n \in \langle v_1, v_2, \ldots, v_n \rangle$, это очевидное доказательство в одну сторону. А для другой стороны заметим, что $v_2, v_3, \cdots, v_n \in \lbrace v_1, v_2, \cdots, v_n \rbrace$, а $v_1 = \frac{(\sum a_i v_i) - a_2 v_2 - \cdots - a_n v_n}{a_1}$
    \end{proof}
    
    Доказательство ЛЗЛК:

    Рассмотрим $u_1$. Если $u_1 = \overline{0}$, то система сразу линейно зависимая, конец. Иначе можно представить $u_1 = \sum a_i v_i$, и $\exists i : a_i \neq 0$ По лемме произведем замену $u_1$ на сумму. Получили $\lbrace u_1, v_2, \cdots, v_n \rbrace = \lbrace v_1, v_2, \cdots, v_n \rbrace$. Давайте продолжать подобную операцию: на $k$-ом шаге заменяем/выражаем $u_k = \sum_{i \ge k } a_i v_i + \sum_{i < k} b_i u_i$ Если все $a_i = 0$, то мы выражаем $u_k$ через остальные $u$, т.е. получили линейную зависимость. Иначе будем там заменять, и через $n$ шагов получим:

    $u_{n+1}, \cdots, u_m \in \lbrace v_1, v_2, \cdots, v_n \rbrace = \lbrace u_1, u_2, \cdots, u_n, \rbrace$, т.е. умеем выражать остальные $u$ через первые $n$, т.е. система всё же линейно зависимая.
\end{proof}

Из доказанной леммы очевидно следует теорема о равенстве количеств элементов во всех базисах оного пространства, а значит и корректность определения.

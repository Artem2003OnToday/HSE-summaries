\begin{definition}
    Пусть $R$ --- кольцо,  $I, J$ --- конечные множества.

    Тогда матрица $A$ над  $R$ --- отображение  $I \times J \to R$.

    Обычно  $I = \{1,\ldots, m\}, J = \{1, \ldots, n\}, (i, j) \mapsto a_{ij} \in R$.

    Тогда матрица $m \times n\!: (a_{ij})_{\substack{i=1..m \\ j = 1..n}}$
\end{definition}
\begin{definition}
    Множество матриц $M_{m, n}(R)$.

    При $I = J$ мы называем квадратными  $M_n(R)$.
\end{definition}

Рассмотрим матрицу $A \in M_{m, n}$. Её можно разбить на  $n$ столбцов  $(c_1 \mid c_2 \mid \ldots \mid c_n), c_i \in K^m$ и $m$ строчек:  $\begin{pmatrix} r_1 \\ r_2 \\ \ldots \\ r_n \end{pmatrix}, r_i \in \prescript{n}{}{K}$.

Также заметим, что $M_{m, n}(k)$ --- векторное пространство над  $K$. Ясно, что  $M_{m, 1}(K) \cong K^m$ и  $M_{1, m}(K) \cong \prescript{n}{}{K}$.

 \begin{definition}
     Умножение матриц: $M_{m, n} \times K^n \to K^m$: $(a_{ij}, \begin{pmatrix} x_1 \\ x_2 \\ \vdots \\ x_n \end{pmatrix}) \mapsto \begin{pmatrix} y_1 \\ y_2 \vdots \\ y_n \end{pmatrix}$, где $u_i = a_{i 1} x_1 + a_{i 2}x_2 + \ldots + a_{in}x_n = \sum\limits_{k=1}^n a_{ik}x_k$.

     Можно определить умножение строки на столбец:  $\begin{pmatrix} a_1 & a_2 & \ldots & a_m \end{pmatrix} \cdot \begin{pmatrix} x_1 \\ x_2 \\ \vdots \\ x_n\end{pmatrix} \leadsto \sum a_i x_i$.  Тогда умножение матрицы на столбец можно записать как $\begin{pmatrix} c_1 \\ c_2 \\ \vdots \\ c_m \end{pmatrix} \cdot X \leadsto \begin{pmatrix} c_1 \cdot X \\ c_2 \cdot X \\ \vdots \\ c_m \cdot X \end{pmatrix}$.

     Тогда заметим, что умножение матриц: $A \cdot B = \begin{pmatrix} Ac_1 & Ac_2 & \ldots  & A_l \end{pmatrix}$.
\end{definition}
\begin{example}[.СЛУ]
    Системы линейных уравнений можно записывать как матрицу. Дальше я не успел. 
\end{example}
\begin{remark}
    $A \in M_{m, n}(K)$. Рассмотрим  $\mathcal{A}\!: K^n \to K^m$,  $X \mapsto A \cdot X$.
\end{remark}
\begin{statement}
    $\mathcal{A}$ --- линейное отображение. 
\end{statement}
\begin{proof}
    $\mathcal{A}(X+Y) = A \cdot X + A \cdot Y\quad \forall x, y$,  $\mathcal{A}(kX) = kA\cdot X$,  $k \in K$.
\end{proof}

Однородная СЛУ: $AX = 0$.  $A = \begin{pmatrix} c_1 & c_2 & \ldots & c_n \end{pmatrix}$. Эта система имеет тривиальное решение $x = \begin{pmatrix} 0 \\ 0 \\ \vdots \\ 0\end{pmatrix} \iff c_1, \ldots, c_n$ --- ЛНЗ. Тогда заметим, что если $n > m$, то  $AX=0$ имеет нетривиальное решение.
\Subsection{Структура линейных отображений}
\begin{example}
   $V=R^2$. Поворот вокруг $O$ --- линейное отображение. Симметрия относительно прямой --- линейное отображение, если  $0 \in l$. Проекция на  $l$ --- линейное отображение.
\end{example}
\begin{properties}
    \begin{enumerate}
        \item $\mathcal{A}(0) = 0, x = 0 \Rightarrow \mathcal{A}(0) = 0$.
        \item $\mathcal{A}$ --- инъекция  $\iff \mathcal{A}(x) = 0 \Rightarrow x = 0$.
        \item $x_1, x_2, \ldots, x_n$ --- ЛЗ $\Rightarrow \mathcal{A}(x_i)$ --- ЛЗ.
        \item[3'.] $\mathcal{A}$ --- инъекция:  $\{x_i \}$ --- ЛНЗ  $\Rightarrow \{\mathcal{A}(x_i)\}$ --- ЛНЗ.
        \item  $u_1, u_2, \ldots, u_m$ --- базис $U$.  $v_1, v_2, \ldots, v_n \in V$. Тогда $\exists! A\!: i \to v$ такое что  $\mathcal{A}(u_i) = v_i$.
    \end{enumerate}
\end{properties}
\begin{proof}
     \begin{enumerate}
         \item $\mathcal{A}(0) = \mathcal{A}(0+0) = \mathcal{A}(0) + \mathcal{A}(0)$ 
         \item $\Rightarrow\!:$  $\mathcal{A}$ --- инъекция,  $\mathcal{A}(x) = 0, \mathcal{A}(0)=0 \Rightarrow x = 0$. 

             $\Leftarrow\!:$ От противного, пусть $x \neq y \in U : \mathcal{A}(x) = \mathcal{A}(y) \Rightarrow \mathcal{A}(x) - \mathcal{A}(y) = \mathcal{A}(x - y) = 0 \Rightarrow x - y = 0$. Противоречие.
         \item  $\sum a_i x_I = 0 \implies \sum a_i \mathcal{A}(x_i) = \sum \mathcal{A}(a_i x_i) = \mathcal{A}(\sum a_i x_i) = \mathcal{A}(0) = 0$.
         \item[3'.] Пусть $\sum a_i x_i = 0 \Rightarrow \sum a_i \mathcal{A}(x_i) = \mathcal{A}(\sum a_i x_i) = 0 \implies a_i x_i = 0 \rightarrow a_i = 0$.
         \item Определим $A$: пусть  $u \in U$.  $\exists! \{a_i\}\!:\ u = \sum a_i u_i$. Положим,  $\mathcal{A}(u) = \sum a_i v_i$.  $\mathcal{A}$ --- линейно (очевидно/упражнение).

             Единственность: пусть  $\mathcal{A}_2(u_i) = v_i$, тогда  по линейности  $\mathcal{A}_2(\sum a_i u_i) = \sum a_i \mathcal{A}_2(u_i) = \sum a_i v_i = \mathcal{A}(\sum a_i u_i)$.
    \end{enumerate}
\end{proof}
\begin{definition}
    $\mathcal{A}\!: U \to V$ --- линейное отображение.

    Тогда  $\ker \mathcal{A} = \{ u \in U \mid \mathcal{A}(u) = 0\}$ --- ядро  $\mathcal{A}$.  $\Im A = \{ v \in V \mid \exists u: \mathcal{A}(u) = v\}$.
\end{definition}
\begin{properties}
    \begin{enumerate}
        \item $\ker \mathcal{A} \le U$, $\Im \mathcal{A} \le U$.
        \item $\Im \mathcal{A} = V \iff \mathcal{A}$ --- сюръекция.
        \item $\ker \mathcal{A} = \{0\} \iff \mathcal{A}$ --- инъекция.
    \end{enumerate}
\end{properties}
\begin{proof}
    \begin{enumerate}
        \item Нам нужно собственно проверить замкнутость $\ker \mathcal{A}$. Пусть $x, y \in \ker \mathcal{A} \Rightarrow \mathcal{A}(x + y) = \mathcal{A}(x) + \mathcal{A}(y) = 0$ по определению ядра. Осталось проверить замкнутость домножения на скаляр. Ну действительно, пусть $x \in \ker \mathcal{A} \Rightarrow \mathcal{A}(kx) = k \cdot \mathcal{A}(x) = 0$.\\
        $\Im \mathcal{A} \le U$ аналогично. Пусть $X, Y \in \Im \mathcal{A} \Rightarrow \exists x, y \in U : \begin{cases}\mathcal{A}(x) = X\\
        \mathcal{A}(y) = Y\end{cases} \Rightarrow \mathcal{A}(x + y) = X + Y$. Замкнутость по домножению на скаляр: пусть $x \in \Im \mathcal{A} \Rightarrow \exists x \in U : \mathcal{A}(x) = X \Rightarrow \mathcal{A}(kx) = kX$. 
        \item Это абсолютно тривиально -- просто перефразирования одного и того же: если достигаются все значения, то у каждого значения хотя бы один достигающий его аргумент и наоборот.
        \item $\Leftarrow\!:$ Очевидно, так как тогда только $\mathcal{A}(0) = 0$.\\
        $\Rightarrow\!:$ Предположим от противного: $x \neq y \in U : \mathcal{A}(x) = \mathcal{A}(y) \Rightarrow \mathcal{A}(x - y) = 0 \Rightarrow 0 \neq x - y \in \ker \mathcal{A}$. Противоречие.
    \end{enumerate}
\end{proof}
\begin{theorem}[О ядре и образе]
    $\mathcal{A}$ --- линейное отображение.
     \begin{enumerate}
         \item $\exists$ базис  $u_1, u_2, \ldots, u_k, u_{k+1}, u_n$. Причем $u_1, \ldots, u_k$ --- базис $\ker \mathcal{A}$, а  $\mathcal{A}(u_{k+1}), \ldots, \mathcal{A}(u_n)$ --- базис $\Im \mathcal{A}$.
         \item  $\dim (\ker \mathcal{A}) + \dim (\Im \mathcal{A}) = \dim U$.
    \end{enumerate}
\end{theorem}
\begin{proof}
    Рассмотрим базис $u_1, u_2, \ldots, u_k$ --- базис $\ker \mathcal{A}$. По лемме эту систему можно дополнить до базиса  $U$. Рассмотрим $u_{k+1}, \ldots, u_n$ из нового базиса. 

    Хотим доказать, что $\mathcal{A}(u_{k+1}), \ldots, \mathcal{A}(u_n)$ --- базис $\Im \mathcal{A}$. Докажем по определению, доказав линейную независимость и порождение всех векторов в пространстве.

    \begin{itemize}
        \item ЛНЗ-ть: Пусть  $\sum a_{k + i} \mathcal{A}(u_{k+i}) = 0 \Rightarrow \mathcal{A}(\sum a_{k+i} u_{k+i}) = 0 \Rightarrow \sum a_{k+i} u_{k+i} \in \ker A \Rightarrow \exists a_1, a_2, ..., a_k : \sum a_{k + i} u_{k + i} = \sum\limits_{i=1}^k (-a_i) u_i \Rightarrow \sum\limits_{i=1}^n a_i u_i = 0$. Противоречие, так как $u_1, u_2, ..., u_n$ -- базис в $U$.
        \item Порождение: Возьмём какой-нибудь $u \in U$. Докажем, что $\mathcal{A}(u)$ выражается через базис. Разложим $u$ через базис: $u = \sum\limits_{i = 1}^n a_i u_i \Rightarrow \mathcal{A}(u) = \mathcal{A}\left(\sum\limits_{i = 1}^n a_i u_i\right) = \mathcal{A}\left(\sum\limits_{i=1}^k a_i u_i\right) + \sum\limits_{i=k+1}^n a_i \mathcal{A}(u_i)$. Но $\mathcal{A}\left(\sum\limits_{i=1}^k a_i u_i\right) = 0$, так как оно лежит в ядре. Значит действительно $\mathcal{A}(u) = \sum\limits_{i=k+1}^n a_i \mathcal{A}(u_i)$.
    \end{itemize}
\end{proof}

\begin{definition}
    Пусть $U, V$ --- векторные пространства над полем $K$. 

    Тогда  $U \oplus V \coloneqq U \times V$ как множества. То есть  $(u_1, v_1) + (u_2,v_2) = (u_1+u_2, v_1 + v_2)$, $k(u, v) \coloneqq (ku, kv)$.
\end{definition}
\begin{remark}
    Пусть $\small \begin{array}{cc} u_1,\ldots, u_n &\text{--- Базис } U \\ v_1,\ldots, v_m &\text{--- Базис } V \end{array}$, $\widetilde{v_i} = (0, v_i), \widetilde{u_i} = (0, u_i)$,

    Тогда $\{\widetilde{u_i}\} \cup \{\widetilde{v_i}\}$ --- Базис $U \oplus V$.
\end{remark}
\begin{proof}
    $\forall (u, v) \in U \oplus V\!: \exists!(a_i) \exists!(b_i)\ (u, v) = (\sum a_i u_i, \sum b_i v_i)$. 

    $(u, 0) = (\sum a_i u_i)$,  $(0, v) = (\sum b_i v_i)$.
\end{proof}
\begin{remark}
    $i_u\!: U \to U \oplus V$, $u \mapsto (u, 0)$, $i_v$ --- аналогично. Инъективный гомоморфизм векторных пространств. 

    $P_u\!: U \oplus V \to U, (u, v) \mapsto u$ --- проекция.  $\Im P_u = U, \ker P_u = \Im i_v$.

    Диаграмма прямой суммы:  $U \xleftrightarrow[i_u]{P_u} U \oplus V \xleftrightarrow[i_v]{P_v} V$.  $P_ui_u = \id_u$, $P_vi_v = \id_v$,  $P_vi_u= 0_u$,  $p_u i_v = 0_v$,  $i_uP_u + i_v p_v = ???$
\end{remark}

\begin{theorem}[Формулаа Грассмана]
   Пусть $U, V \le W$, $U, V$ --- конечномерные.

   $\dim(U + V) = \dim U + \dim V - \dim(U \cap V)$
\end{theorem}
\begin{proof}
    Построим линейное $f\!: U \oplus V \to W, (u, v) \mapsto u + v$. $f$ --- линейное (очев/упражнение).

    Заметим, что $\Im f = U+V$,  $\ker f = \{(u, -u) \mid \begin{array}{r} u \in U \\ -u \in V \end{array} \} = \{ (u, -u) \mid u \in U \cap V \}$.

    Очевидно, что $\ker f \cong U \cap V \implies \dim (\ker f) = \dim (U \cap V)$. А по теореме о размерности ядра и образа:  $\dim V + \dim U = \dim(U \oplus V) = \dim(U + V) + \dim(U \cap V)$
\end{proof}
\begin{example}
    $K^n$, $U = \left\{ \begin{pmatrix} x_1 \\ \vdots \\ x_n \end{pmatrix} \mid a_1 x_1 + \ldots + a_n + x_n = 0\right\}$ --- гиперплоскость $\dim U = n - 1$. 

    СЛУ ---  $m$ уравнений,  $m$ гиперплоскостей  $--- u_1, u_2, \ldots, u_m$. Ответ --- $\bigcap\limits_{i=1}^n u_i$.

    $\dim u_1 = n - 1$. $\dim (u_1 \cap u_2) = \dim u_1 + \dim u_2 - \dim (u_1 + u_2) \ge n - 1 + n - 1 - n \ge n - 2$. Можно продолжить процесс.
\end{example}
\begin{consequence}
    Множество решений однородной СЛУ ($n$ неизвестных,  $m$ уравнений) --- пространство размерности $\ge n - m$.
\end{consequence}
\begin{remark}
    Аналогия $(+, \cap)$ c  $(\cup, \cap)$ --- неполная:  $(V_1 + V_2) \cap V_3 \neq (V_1 \cap V_3) + (V_2 \cap V_3)$, пример: три прямые на плоскости.
\end{remark}

Пусть $A \in M_{m, n}(K), A = (a_{ij})_{\substack{i = 1..m \\ j = 1..n}}$,  $\mathcal{A}\!: \begin{array}{c} K^n \to K^m \\ x \to A \cdot x \end{array}$ --- линейное отображение.

$K^n - \langle e_1, \ldots, e_n \rangle, K^m = \langle \widetilde{e_1}, \ldots \widetilde{e_n} \rangle$, где $e_i$ --- вектор нулей с 1 на  $i$ строчке. Тогда 
$A e_j = \begin{pmatrix} a_{1j} \\ a_{2j} \\ \vdots \\ a_{mj} \end{pmatrix}$. 
Тогда можно сказать, что $A = \left(\begin{array}{c|c|c|c} \mathcal{A}(e_1) & \mathcal{A}(e_2) & \ldots & \mathcal{A}(e_n) \end{array} \right)$ 

\begin{consequence}
    $A, B \in M_{m, n}(K)$.  $\mathcal{A}, \mathcal{B}$ --- линейные отображения,  $\mathcal{A} = \mathcal{B} \implies A = B$.
\end{consequence}
\begin{statement}
    $A \in M_{m, n}(K)$,  $\mathcal{A}$ --- соответствующее отображение.

    $\ker \mathcal{A}$ --- множество решений однородной СЛУ с матрицей  $A$\\
    $\Im \mathcal{A}$ --- линейная оболочка столбцов $A$.
\end{statement}
\begin{proof}
    $A = \left( \begin{array}{c|c|c} & &  \\ C_1 & C_2 & \ldots C_n \\ & & \end{array} \right) = \left(\begin{array}{c|c|c} & &  \\ \mathcal{A}(e_1) & \mathcal{A}(e_2) & \ldots \mathcal{A}(e_3) \\ & & \end{array} \right)$ 

    $\langle C_1, C_2, \ldots, C_n \rangle = \langle \{A e_i \} \rangle = \Im \mathcal{A}. \sum a_i \mathcal{A}(e_i) = \mathcal{A}(\sum a_i e_i) = \mathcal{A}(V)$. $V$ --- вектор  $\in K^n$.
\end{proof}
\begin{definition}
    $A = \left( \begin{array}{c|c|c} & &  \\ C_1 & C_2 & \ldots C_n \\ & & \end{array} \right)$.

    $\dim \langle C_1, \ldots, C_n \rangle$ --- называется рангом матрицы.

    Обозначение $\rank A, \rk A, \rg A$.
    
    При $n=m$  $n-\rk A$ называется  дефектом матрицы. Дефект $\dim \ker A$.
\end{definition}
\begin{theorem}[Принцип Дирихле для векторных пространств]
    $\mathcal{A}\!: V \to V$ --- линейное множество.  $V$ --- конечномерное.

    Тогда  $\mathcal{A}$ --- инъекция  $\iff \mathcal{A}$ --- сюръекция.
\end{theorem}
\begin{proof}
    $\dim V = n$,  $\mathcal{A}$ --- инъекция  $\iff \ker \mathcal{A} = 0 \iff \dim \ker A = 0 \iff \dim \Im A = n - 0 = n \iff \Im A = V \iff \mathcal{A}$ --- сюръекция.
\end{proof}

\begin{definition}
    Неоднородная система: $AX=B$,  $A \in M_{m, n}(K), B \in K^m$.
\end{definition}
\begin{theorem}
    Решение неоднородной и соответствующей ей однородной системы связаны:

\end{theorem}
\begin{proof}
    Пусть $X_0$ --- решение $AX=B$, тогда  $AX=B \land AX_0=B \iff A(X-X_0) = 0 \iff x - x_0 \in \ker \mathcal{A}$ --- решения соответствующей однородной $A$.
    
    $x = x_0 + v, v \in \ker \mathcal{A} $. Множество решений $x_0 + \ker \mathcal{A} = x_0 + \ker A = x_9 + \mathcal{A}^{-1}\{0\}$. 

    $\mathcal{A}^{-1}(\{\mathcal{A} X_0\}) = X_0 + \mathcal{A}^-1{0}$.
\end{proof}

\begin{theorem}[Альтернатива Фредгольма]
    $\forall n = m$. СЛУ: $n$ уравнений, $n$ неизвестных,  $AX=B$. Пусть  $A$ --- фиксировано,  $B$ --- нефиксировано.

    Тогда верно одно из двух: 
     \begin{enumerate}
         \item однородное СЛУ имеет только тривиальное решение и неоднородное СЛУ имеет единственное решение.
         \item $AX=0$ имеет бесконечно много решений, тогда 0 или бесконечное множество решений.
    \end{enumerate}
\end{theorem}
\begin{theorem}
    $A \in M_{n, m}(K), B \in M_{l, n}(K)$. Причем  $K^m \xrightarrow[\mathcal{B}]{} B K^n \xrightarrow[\mathcal{A}]{} K^l$.

    Рассмотрим  $C = A \cdot B$,  $\mathcal{C}\!: K^m \to K^l$.  $C \coloneqq A \cdot B$, тогда  $\mathcal{C}$ --- отображения домножения  $C$.
\end{theorem}
\begin{proof}
    Докажем, что $C_1(X) = (A \cdot B) \cdot X$,  $C(X) = A \cdot (B \cdot X)$. Достаточно проверить для какого-то базиса  $K^m$.

    $e_i$ --- все нули, но на  $i$-ой строчке единица,  без волны в $K^m$, с --- в  $K^n$.  Тогда  $Be_i = \begin{pmatrix} B_{1i} \\ B_{2i} \\ \vdots \\ B_{ni} \end{pmatrix} = \sum_i b_{ki} \widetilde{e_k}$. 

    Тогда $A(Be_i) = A(\sum b_{ki} \widetilde{e_i}) = \sum b_{ki}(A\widetilde{e_k}) = \sum b_{ki} \begin{pmatrix} a_{1k} \\ a_{2k} \\ \vdots \\ a_{lk} \end{pmatrix} = \begin{pmatrix} \sum_k a_{1k} b_{ki} \\ \sum a_{2k} b_{ki} \\ \vdots \\ \sum a_{lk} b_{ki} \end{pmatrix}$, где $i$ --- фиксированный столбец. 
\end{proof}

\begin{consequence}
    Умножение матриц ассоциативно:

    $A \in M_{k, l}(K), B \in M_{l, m}(K), C \in M_{m, n}(K)$.

    Тогда $(AB)C = A(BC)$.
\end{consequence}
\begin{definition}
    При $m = n$,

    $M_n(K)$ --- кольцо квадратных матриц. Ассоциативное, но не коммутативное кольцо.
\end{definition}
\begin{example}
    $\begin{pmatrix} 0 & 0 \\ 1 & 0 \end{pmatrix} \cdot \begin{pmatrix} 1 & 0 \\ 0 & 0 \end{pmatrix} = \begin{pmatrix} 0 & 0 \\ 1 & 0 \end{pmatrix}$\\
    $\begin{pmatrix} 1 & 0 \\ 0 & 0 \end{pmatrix} \cdot \begin{pmatrix} 0 & 0 \\ 1 & 0 \end{pmatrix} = \begin{pmatrix} 0 & 0 & 0 & 0 \end{pmatrix}$.
\end{example}
\Subsection{Матрица линейного отображения}

$U, V$ --- векторные пространства над  $K$.  $\mathcal{A}\!: U \to V$ --- линейное.  $u_1, \ldots, u_n$ --- базис $U$,  $v_1, \ldots, v_n$ --- базис $V$. 

$\mathcal{A}(u_i) \in V \Rightarrow \mathcal{A}(u_i)$ --- линейная комбинация  $\{ v_i \}$.

Тогда  $(a_{ij})$ --- матрица линейного отображения  $\mathcal{A}$ в базисах  $\{u_i\}, \{v_i\}$.  $A = [ \mathcal{A}]_{\{u_i\}, \{v_i\}}$. Столбцы  $A$ --- столбцы координат  $\mathcal{A}(u_i)$ в базисе  $\{v_i\}$.

\begin{statement}
    $u \in U$,  $u$ --- столбец координат в базисе  $\{u_i\}$.

    Тогда  $A \cdot u$ --- столбец координат  $\mathcal{A}(u)$ в базисе  $\{v_i\}$.
\end{statement}
\begin{proof}
    Для $u_i$ это так по определению, а для остальных векторов по дистрибутивности/линейности. 
\end{proof}
\begin{remark}
    $\{u_i\}$ задает изоморфизм  $u \xrightarrow[f_u]{} K^n$,  $v \xrightarrow[f_v] K^m$.

    \[
    \begin{matrix}
        U & \xrightarrow[\mathcal{A}]{} & V \\
        f_u \downarrow & & \downarrow f_v \\
        K^n & \xrightarrow[\cdot A]{} & K^m
    \end{matrix}
    .\] 
\end{remark}
\Subsection{Матрица перехода и формулы пересчета}

$V$ --- векторное пространство.  $\id_v$ --- линейное.  $\id_v\!: V \to V$,  $\{v_i\}$ --- базис.  $[\id]_{\{v_i\}, \{u_i\}} = \begin{pmatrix} 1 & \ldots & 0 \\ \vdots & \ddots & \vdots \\ 0 & \ldots & 1 \end{pmatrix} = E_n$. Причем  $E_n$ --- единица в кольце матриц.


Пусть теперь $\{u_i\}, \{v_i\}$ --- базисы  $V$. Тогда  $[\id]_{\{v_i\}, \{u_i\}} = C = (c_{ij})$. $u_i = \sum_j c_{ji} v_{ji}$.  $\begin{pmatrix} u_1, \ldots, u_n\end{pmatrix} = \begin{pmatrix}v_1, \ldots, v_j \end{pmatrix} \cdot C$.

$x\in V$,  $x = \sum a_i u_i$,  $x = \begin{pmatrix} a_1 \\ \vdots \\ a_n \end{pmatrix} = (x)_{u_i}$.

$\chi \coloneqq \begin{pmatrix} u_1 & \ldots & u_n \end{pmatrix} \cdot X= ???$

$C$ --- матрица перехода.
 \begin{remark}
\begin{itemize}
    \item $C_{\{u_i\}, \{v_i\}} = E$
    \item $\{u_i\}, \{v_i\}, \{w_i\}$ --- базисы. Тогда $C_{\{u_i\}\{w_i\}} = C_{\{u_i\}\{v_i\}} \cdot C_{\{v_i\}\{w_i\}}$.
    \item  $C_1 = C_{\{u_i\}, \{v_i\}}, C_2 = C_{\{v_i\}, \{u_i\}}$. \\
        $C_1 \cdot C_2 = C_2 \cdot C_1 = E$. $C_1, C_2$ --- взаимообратные.
\end{itemize}
\end{remark}

Мотивация: Решим систему $\begin{cases} ax+by=e \\ cx+dy=f \end{cases}$. Получим решения  $\begin{cases} x = \frac{ed - bf}{ad - bc} \\ y = \frac{af - ec}{ad - bc} \end{cases}$. 

В чем смысл $ad - bc$. Возьмем вектора  $\begin{pmatrix} a \\ c \end{pmatrix}$ и $\begin{pmatrix} b \\ d \end{pmatrix}$. Тогда площадь параллелограмма, натянутого на эти вектора $S = |ad-bc|$ и $ad - bc$.

Пусть $\widehat{S}(\Phi)$ --- ориентированная площадь,  $|\widehat{S}(\Phi)| = S(\Phi)$.  $\widehat{S}(\Phi) > 0$, если поворот от первого вектора ко второму против часовой стрелки.

\begin{properties}
    \begin{enumerate}
        \item $\widehat{S}(V_1, V_2) = -\widehat{S}(V_2, V_1)$.
        \item $\widehat{S}(kv_1, kv_2) = k \widehat{S}(v_1, v_2)$.
        \item $S(v_1, v'_2 + v''_2) = S(v1, v'_2) + S(v_1, v''_2)$
    \end{enumerate}
\end{properties}

\textbf{\huge Общий случай.}

\begin{definition}
    $V_1, V_2, \ldots, V_n$, $V$ ---  векторные пространства над $K$.

    Отображение  $\mathcal{A}\!: V_1 \times V_2 \times \ldots \times V_n \to V$ называется полилинейным, если $\forall i \forall v_i \in V_i \mathcal{A}_{v_1, \ldots, v_n}\!: V_i \to V$ --- линейно. То есть, если закрепить все переменные, кроме одной, то отображение будет линейно.
\end{definition}

Будем изучать $V = K^n, \omega\!: (K^n)^m \to K$.
 \begin{lemma}
    $e_1, \ldots, e_n$ --- базис $K^n$.  $\omega\!: (K^n)^m \to K$ --- полилинейно, тогда $\omega(v_1, v_2, \ldots, v_m) = \sum\limits_{\mathclap{\{i_1, \ldots, i_m\} \in \{1..n\}^n}} a_{i_1 1}\cdot a_{i_2 2}\ldots a_{i_m m} \omega(e_{i_1}, e_{i_2},\ldots, e_{i_m})$.
\end{lemma}
\begin{proof}
    $\omega(v_1, v_2, \ldots, v_m) = \omega(\sum a_{j_1} e_j, \sum a_{j_2} e_j, \ldots) = \sum\limits_{j=1}^n a_{j_1} \cdot \omega(e_j, \sum \ldots)$ и дальше по линейности.
\end{proof}

\begin{definition}
    Кососимметрической $n$-формой называется полилинейное отображение  $\omega\!: (K^n)^n \to K$, такое что  $(\exists i, j\!: v_i = v_j \implies \omega(v_1, \ldots, v_n) = 0)$.
\end{definition}
\begin{statement}
    \begin{enumerate}
        \item $\omega$ --- кососимметрична  $\implies$  $\forall i \neq j\!: \omega(v_1, v_2, \ldots, v_i, \ldots, v_j, \ldots, v_n) = - \omega(v_1,\ldots, v_j, \ldots, v_i, \ldots v_n)$.
        \item $\charr k \neq 2 \implies $ верно и обратное. То есть, если выполняется равенство то и  $\omega$ --- кососимметричная. 
    \end{enumerate}
\end{statement}
\begin{proof}
    \begin{enumerate}
        \item фиксируем все, кроме $v_i$. Тогда  $f(x, y) = \omega(v_1, .., v_{i-1}, x, v_{i+1}, \ldots, y, \ldots, v_n)$. $f$ --- полилинейно. Надо доказать?? 
    \end{enumerate}
\end{proof}
\begin{statement}
    $\exists$ не более одной кососимметричной  $n$-формы с точностью до линейного множества. 
\end{statement}
\begin{proof}
    $\omega$ определено однозначено значениями
     \begin{enumerate}
         \item На базисе: $\omega(e_{i_1}, e_{i_2}, \ldots, e_{e_n}) = \widetilde{\omega}(e_{i_1}, e_{i_2}, \ldots, e_{i_n})$. Из леммы известно, что тогда $\omega = \widetilde{\omega}$. 
         \item $\omega$ --- кососимметрична, среди  $i_1, \ldots, i_n$ есть одинаковые. Тогда $\omega(e_{i_1},\ldots) = 0$.
         \item Осталось изучить перестановки $e$. Пусть $\pi$ перестановка. Тогда  $e_{\pi(1)}, e_{\pi(2)}, \ldots$ --- правильный порядок. Каждая перестановка двух элементов меняет знак у $\omega$. То есть  $\omega(e_{\pi(1)}, e_{\pi(2)}, \ldots) = (-1)^k \omega(e_1, e_2, \ldots, e_n)$, где $k$ --- количество сделанных ходов.
    \end{enumerate}
    Из 1,2,3 следует, что  $\omega(e_1, e_2, \ldots, e_n) = \widetilde{\omega}(e_1, \ldots, e_n)$ и они кососимметричны, то они равны. 

    Тогда потребуем, чтобы $\omega$ была кососимметричной  $n$-формой,  $\omega(e_1, e_2,\ldots, e_n) = 1$, где $e_1, e_2, \ldots, e_n$ базис в $K^n$. Тогда таких функция  $\le 1$.
\end{proof}
\begin{definition}
    Такие функции называются определителем порядка $n$.
\end{definition}

\begin{definition}
    $S_n$ --- группа перестановок:  $S_n = \{ f\! :\{1, 2,\ldots, n\} \to \{1, 2,\ldots, n\} \mid f\text{ --- биекция}\}$. С операцией $\cdot$ --- композиция.
\end{definition}
\begin{definition}
    $t_{ij} \in S_n$ --- транспозиция.  $t_{ij}(i) = j, t_{ij}(j) = i, t_{ij}(k) = k$, при $i \neq j$. 
\end{definition}
\begin{statement}
    $\langle\{t_{ij}\}_{\substack{i=1..n\\j=1..n}}\rangle = S_n$.
\end{statement}
\begin{proof}
    Индукция по $n$.
\begin{itemize}
    
    \item База $n=2$:  $S_2 = \langle id, t_{12}\rangle$. 
    \item Переход. $n \to n +1$.  Пусть $\pi \in S_{n+1}$.  $\exists i = n + 1$. Рассмотрим  $t_{i, n+1}$, то есть $t_{i, n + 1}(i) = n + 1$ и $t_{i, n + 1}(n+1) = i$. $\pi \circ t_{i, n + 1}(n+1) = \pi(i) = n + 1$. 

        Тогда сузим  $\pi$ до  $n$. Получим  $\widetilde{\pi}=\{1\ldots n\} \to \{1 \ldots n\}$. Тогда заметим, что \[\pi = \underbrace{t_{i_1, j_1} \circ t_{i_2, j_2} \circ \ldots}_{\text{образующие для} \widetilde{\pi}} \circ (t_{i, n+1})^{-1}.\]
\end{itemize}
\end{proof}
\begin{definition}
    Перестановка $\pi$ называется четной (нечетной), если выполнено одно из равносильных условий:
    \begin{enumerate}
        \item $\pi = t_{i_1, j_1} \ldots t_{i_{2k}, j_{2k}}$ (соответственно $2k+1$).
        \item $\#\{(i, j) \mid \substack{i \in \{1..n\} \\ j \in \{1..n\}} \land \begin{cases} i < j \\ \pi(i) > \pi(j) \end{cases} \}$ --- четно/нечетно соответственно.
        \item  $\prod\limits_{i<j} \frac{\pi(i) - \pi(j)}{i-j} = 1$ (соответственно -1).
    \end{enumerate}

\end{definition}
\begin{consequence}
    1. корректно (то есть четность чила множителей не зависит от разложения).
\end{consequence}
\begin{proof}
    Докажем, что $1 \iff 2$.

    Количество из определения 2 называется число инверсий. Надо доказать, что $\pi = \prod\limits_{l = 1}^k t_{i_l, j_l} \implies$ количество инверсий $\equiv k \pmod 2$. 

     Индукция по $k$:
      \begin{itemize}
          \item База. $k = 0, \pi = \id$. 0 инверсий.
          \item Переход: надо доказать, что  $\pi \in S_n \forall i, j$ четности числа инверсий для  $\pi$ и  $t_{ij} \pi$ --- разные.
            
              Какие пары поменяли статус при применении $t_{ij}$:  $(\pi(i), t_l), (\pi(j), t_l), (\pi(i), \pi(j))$. Первого типа --- $k$, второго --- $k$, третьего ---  $1$. А значит четность изменилась.
      \end{itemize}
\end{proof}
\begin{definition}
    $A \in M_n(K)$. Определителем  $A$ называется число  \[det A = |A| = \sum\limits_{\pi \in S_n} \eps(\pi) a_{1\pi(1)}a_{2\pi(2)}\ldots a_{n \pi(n)}\], где $\eps(\pi) = 1$ --- $\pi$ четна,  $-1$ иначе. 
\end{definition}
\begin{remark}
    $t_{12}$.  $\pi \leftrightarrow t_{12} \circ \pi$ --- биекция между четными и нечетными.
\end{remark}
\begin{statement}
    Функция $\omega\!: (K^n)^n \to K$, $\omega(c_1, \ldots, c_n) = \det(C_1 \mid C_2 \mid\ldots\mid C_n)$ --- полилинейная кососимметрическая форма.
\end{statement}
\begin{proof}
    Полилинейность --- очев. 

    Кососимметричность: $\omega(c_1, \ldots, c_i, \ldots,c_j, \ldots) = 0$, если $c_i = c_j$. Докажем, что все слагаемые в формуле разбиваются на пары вида  $(x, -x)$ для  $\det$.

    Рассмотрим $\eps(\pi) a_{1\pi(1)} a_{2\pi(2)}\ldots a_{n\pi(n)}$. Здесь $a_{ki}$ и  $a_{lj}$.

    Тогда пусть $\pi(k) = i, \pi(l) = j$.

    Рассмотрим  $\eps(t_{ij}\pi)$. Здесь все будет так же, за исключением $a_{ki} = a_{kj}, a_{li} = a_{lj}, \eps(t_{ij}\pi) = -\eps(\pi) \implies A+B=0$.
\end{proof}
\begin{theorem}
    $\det(A) = \det(A^T)$.
\end{theorem}
\begin{consequence}
    Любое свойство $\det$ про столбцы  $\leadsto$ такое же свойство про строки.
\end{consequence}
\begin{proof}
    \begin{align*}
    \det A &= \sum\limits_{\pi \in S_n} a_{1\pi(1)}\ldots a_{n\pi(n)} \cdot \eps(\pi).\\
    \det A^T &= \sum\limits_{\pi \in S_n} a_{\pi(1)1} a_{\pi(2)2}\ldots a_{\pi(n)n} \cdot \eps(\pi) = \\
             &= \sum\limits_{\pi \in S_n} a_{\pi(1), \pi^{-1}(\pi(1))} a_{\pi(2), \pi^{-1}(\pi(2))} \ldots a_{\pi(n), \pi^{-1}(\pi(n))} \eps(\pi)\\
             &\stackrel{(*)}{=} \sum\limits_{\sigma \in S_n} a_{1\sigma(1)}\ldots a_{n, \sigma(n)} \eps(\sigma) = \det A
    \end{align*}
    Здесь $\sigma = \pi^{-1}$.  $\eps(\pi) = \eps(\sigma)$ так как количество транспозиций у них равно.
\end{proof}
\begin{theorem}
    $A \in M_n(K)$. 
     \begin{enumerate}
         \item $\det (t_{ij}(a)A) = \det(A t_{ij}(a)) = \det A$.
         \item $\det(m_i(a)A) = \det(A \cdot m_i(a)) = a \det A$
         \item  $\det(s_{ij} A) = \det(A s_{ij}) = -\det A$.
    \end{enumerate}
\end{theorem}
\begin{proof}
    \begin{enumerate}
        \item[3.] кососимметричность. (Второе определение).
        \item[2.] Линейность по $i$-ой строке (столбцу).
        \item[1.]  $\det(t_{ij}A) = \det(A)$.  Пусть $r_i$ ---  $i$-ая строка.

            Тогда  $\det \left( \begin{array}{c} A_1 \\\hline r_i + r_j \cdot a \\\hline B_1 \end{array}\right) = \det \left( \begin{array}{c} A_1 \\\hline r_i \\\hline B_1 \end{array}\right) + \det \left( \begin{array}{c} A_1 \\\hline r_j \cdot a \\\hline B_1 \end{array}\right) = \det \left( \begin{array}{c} A_1 \\\hline r_i\\\hline B_1 \end{array}\right) + a \cdot \det \left( \begin{array}{c} A_1 \\\hline r_j\\\hline \vdots \\ r_j \\\hline \vdots \end{array}\right) = \det A$. Последний переход за счет определения кососимметрической формы.
    \end{enumerate}
\end{proof}
\begin{remark}
    Определитель --- сумма произведения элементов $A$ по ладейной расстановке.
\end{remark}

Умеем приводить матрицу к треугольному виду. Тогда заметим, что единственная ненулевая перестановка --- $\id$. А значит  $\det$ треугольной матрицы --- произведение элементов матрицы на диагонали. А дальше надо домножить на  $-1$ в степени количества перестановок. 

\begin{theorem}
    $A$ --- обратима  $\iff \det A \neq 0$.
\end{theorem}
\begin{proof}
    $A$ --- обратима  $\iff e_1\ldots e_kA =$ треугольная матрица $B$ --- обратима.

    $\det B \neq 0 \iff $ все $a_i \neq 0$.

     $B$ --- обратима  $\iff a_i \neq 0$ (доказывали).
\end{proof}
\begin{theorem}
    $A, B \in M_n(K)$. Тогда 
     \begin{enumerate}
         \item $\det(AB) = \det(A) \cdot \det(B)$
         \item если  $\exists A^{-1}$, то  $\det(A^{-1}) = \frac{1}{\det A}$.
         \item $\det E = 1$.
    \end{enumerate}
    То есть $\det\!: GL(n, k) \to K^*$ --- гомоморфизм групп (единственный нетривиальный).
\end{theorem}
\begin{proof}
    \begin{enumerate}
        \item[3.] $E$ --- частный случай треугольной. Очев
        \item[2.] Следует из $1$ и 3:  $1 = \det E = \det(A \cdot A^{-1}) = \det(A) \cdot \det(A^{-1})$.
        \item[1.] Представим  $B$ как набор столбцов $c_1$ --- переменная, а $A = \text{const}$.

             $A \cdot B = \left(A \cdot C_1 \mid A \cdot C_2 \mid \ldots \mid A \cdot C_n\right)$. $B \mapsto \det(AB)$ --- кососимметрическая полилинейная форма от  $C_1, C_2, \ldots, C_n$. Воспользуемся кососимметричностью: $C_i = C_j \implies A C_i = A\cdot C_j$ --- два одинаковых столбца в  $AB \implies \det AB = 0$.

              $B' = (C_1' \mid C_2 \mid \ldots \mid C_n), B'' = (C_1'' \mid C_2 \mid \ldots \mid C_n)$. Тогда $AB=(A(C_1' + C_1'') \mid C_2 \mid \ldots) = (AC_1' + AC_2'' \mid \ldots)$.

              $(AB') = ()$
    \end{enumerate}
\end{proof}
\begin{theorem}[Определитель блочной матрицы]
    \begin{enumerate}
        \item $A = \left( \begin{array}{c|c} A_1 & \ast\\ \hline 0 & A_2\end{array}\right) \implies \det A = \det(A_1)\det(A_2)$.
        \item Если блоков  $k$, то  $\det A = \prod \det(A_i)$. ($A_i$ --- квадратные блоки).
    \end{enumerate}
\end{theorem}
\begin{proof}
    Второй пункт из первого по индукции (упражнение).

    \begin{enumerate}
        \item $\det \left( \begin{array}{c|c} E_x & \ast \\ \hline 0 & E_y \end{array}\right) = 1$. Так как треугольная матрица.
        \item Зафиксируем  $B$.  $\det \left( \begin{array}{c|c} A_1 & B \\ \hline 0 & E\end{array} \right)$ --- полилинейная и кососимметричная относительно столбцов  $A_1$.

            Поставим $A_1 = E \implies \text{const} = 1$ по пункту 1.
        \item fix $B, A_1$  $\det\left(\begin{array}{c|c} a_1 & b \\ \hline 0 & a_2 \end{array} \right) = c_{A_1, B} \cdot \det(A_2)$. Подставляем  $A_2 = E$:  $\det A_1 = \det\left(\begin{array}{c|c} A_1 & B \\ \hline 0 & E \end{array} \right) = c_{A_1, B} \cdot 1 \implies c_{A_1, B} = \det A_1$. А значит $\det\left(\begin{array}{c|c} a_1 & b \\ \hline 0 & a_2 \end{array} \right) = \det A_1 \det A_2$.
    \end{enumerate}
\end{proof}
\begin{theorem}[Разложение по строкам/столбцам]
    $A = (a_{ij})$.  $A_{ij} = \det$ матрицы, полученной удалением  $i$-ой строки и  $j$-го столбца.

    Тогда  $\forall i \det A = \sum{j=1}^n (-1)^{i+j} a_{ij} A_{ij}$ --- разложение по строке.


    Сила в том, что определитель  $n$-го порядка можно свести к  $\det$  $(n-1)$-го порядка.
\end{theorem}
\begin{proof}
    Для строк. Пусть $r_i = (a_{i1}, a_{i 2}, \ldots, a_{in}) = \sum a_{ik}f_k$, где $f_k$ --- строка с  $1$ на  $k$ строке.

    По полилинейности  $\det A = \sum a_{ik} \det A_k$.  $A_k = \begin{pmatrix} r_1 \\ r_2 \\ r_{i-1} \\ 0\ \ldots\ 1\ 0\ \ldots\ 0 \\ \vdots \\ r_n \end{pmatrix}$.

        Тогда $\det A_k = (-1)^{i+k} \cdot \det B$. Где  $B$ мы просто перенесли  $i$-ой строки и  $k$-го столбца на первые места. Тогда получилась блочная матрица.  $= A_{ik} \cdot (-1)^{i+k}$.
\end{proof}
\begin{consequence}
    $\sum_{i=1}^n (-1)^{i+j} a_{ij} \cdot A_{kj} = 0$ ($k \neq i$).
\end{consequence}
\begin{proof}
    Эта сумма по предыдущей теореме равна $\pm \det$ матрицы вида  $k \begin{pmatrix} r_1 \\ r_2 \\ \vdots \\ r_n \end{pmatrix}$ (разложение по $k$-ой) строке, а он равен 0.
\end{proof}

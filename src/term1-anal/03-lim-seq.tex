 \begin{definition}
   $f: \; \N \to \R$ 
\end{definition}.
\slashn
Способы задания последовательностей
\begin{enumerate}
    \item Формулой. $f_n \coloneqq \frac{\sin n}{n^n}$ 
    \item Рекуррентой: $f_1 = 1, f_2=2, f_{n+2} = f_n + f_{n+1}$.
\end{enumerate}
Способы визуализации:
\begin{enumerate}
    \item Можно ставить точки на прямой. Но если последовательность, например, $a_n \coloneqq \sin(\frac{n \pi}{2})$, то получится кукож.
    \item График. Считаем значения в натуральных точках.
\end{enumerate}
\begin{definition}
    Последовательность $a_n$ ограничена сверху, если  $\exists C: \; \forall n \in \N: \; a_n \le c$.
\end{definition}
\begin{definition}
    Последовательность $a_n$ ограничена снизу, если  $\exists C: \; \forall n \in \N: \; a_n \ge c$.
\end{definition}
\begin{definition}
    Последовательность $a_n$ ограничена, если она ограничена и сверху, и снизу.
\end{definition}
\begin{definition}
    Последовательность $a_n$ монотонно возрастает, если  $a_1 \le a_2 \le a_3 \le \ldots$.
\end{definition}
\begin{definition}
    Последовательность $a_n$ строго монотонно возрастает, если  $a_1 < a_2 < \ldots$.
\end{definition}
\begin{definition}
    Последовательность $a_n$ монотонно убывает, если  $a_1 \ge a_2 \ge a_3 \ge \ldots$.
\end{definition}
\begin{definition}
    Последовательность $a_n$ строго монотонно убывает, если  $a_1 > a_2 > a_3 > \ldots$.
\end{definition}
\begin{definition}[Нетрадиционное определение предела]
    $l = \lim a_n \iff$ вне любого интервала, содержащего $l$ находится конечное число членов последовательности. 
\end{definition}
\begin{remark}
    Мы можем смотреть только на симметричные относительно точки $l$ интервалы. Если он не симметричен, то можно большую границу уменьшить. Так можно сделать, так как мы знаем, что вне меньшего конечное число точек, то и снаружи большего точно конечное число точек. Тогда наш интервал выглядит как $(l - \epsilon; l + \epsilon)$
\end{remark}
\begin{remark}
    Конечное число точек снаружи интервала $\iff$ начиная с некоторого номера все попали в интервал, так как возьмем последнюю точку вне интервалов, и взяли её номер + 1.
\end{remark}
\begin{definition}[Традиционное определение предела]
    $l = \lim a_n \iff \forall \epsilon > 0: \; \exists N: \; \forall n\ge N: \;  |a_n-l| < \epsilon$
\end{definition}
\begin{enumerate}
    \item Предел единственный. Пусть $l$ и  $l'$ единственный. \emph{(Картинка)}. Рассмотрим интервал содержащий  $l$, но не  $l'$. Снаружи конечное число точек, теперь наоборот, там тоже конечное число точек. Тогда последовательность конечна.
    \item Если из последовательности выкинуть какое-то число членов, то предел не изменится. Доказательство через картинку.
    \item Если как-то переставить члены последовательности, то предел не изменится. Ну очевидно, что количество членов не изменилось, точки не поменяли своё местоположение.
    \item Если члены последовательности записать с какой-то кратностью (конечной), то предел не изменится.
    \item Если добавить к последовательности конечное число членов, то наличие/отсутствие предела и значение предела, если он существует, не поменяется. Доказательство по картинке.
    \item Изменение конечного числа членов в последовательности не меняет предел.
\end{enumerate}
\begin{example}
    $\lim \frac{1}{n} = 0$. Мы знаем, что найдется такой номер, что $\frac{1}{n} < \beta$, тогда при $n \ge N$ $0 < \frac{1}{n} \le \frac{1}{N} < \beta$
\end{example}
\begin{example}
    $a_n = (-1)^n$ не имеет предела.
\end{example}
\begin{proof}
    Посмотрим на картинку. Возьмем сначала точку не равную $\pm 1$. Тогда можно выбрать интервал, которые не содержит $\pm 1$. То есть интервал не содержит бесконечное число точек.

    Для $x=1$ можно взять  $(0;2)$, для  $x=-1$ можно взять  $(-2;0)$.
\end{proof}
\begin{lemma}
    $\forall a, b, x_n, y_n, \epsilon > 0: a = \lim x_n \land b = \lim y_n \Rightarrow \exists N: \forall n \ge N: |x_n-a| < \epsilon \land |y_n-b| < \epsilon$ 
\end{lemma}
\begin{proof}
    Запишем определения пределов: $\forall \epsilon > 0 \exists N_1 \forall n \ge N_1 |x_n-a| < \epsilon$ и $\forall \epsilon > 0 \exists N_2 \forall n \ge N_2 |y_n-b| < \epsilon$. Тогда просто возьмем $N=\max(N_1, N_2)$.
\end{proof}
\begin{theorem}[Предельный переход в неравенствах]
    $\forall x_n, y_n (x_i < y_i \; \forall i)\; a = \lim x_n \land b = \lim y_n \Rightarrow a \le b$
\end{theorem}
\begin{proof}
    Докажем от противного. Пусть $a>b$. Посмотрим картиночку. Пусть $\epsilon \coloneqq \frac{a-b}{2}$. По лемме $\exists N: \forall n \ge N: |x_n-a|<\epsilon \land |y_n-b|<\epsilon$. Заметим, что $x_n-a| < \epsilon \Rightarrow x_n > a - \epsilon$, а $|y_n-b|<\epsilon \Rightarrow y_n < b + \epsilon \Rightarrow x_n > a - \epsilon = b + \epsilon > y_n$. Противоречие.
\end{proof}
\begin{remark}
    Строгий знак может не сохраняться. Пример: $x_n = -\frac{1}{n} < y_n = \frac{1}{n}$, но предел и там, и там 0. Т.к. $\forall \epsilon > 0 \exists N: \forall n \ge N: \frac{1}{n} = |y_n| = |x_n| < \epsilon$ 
\end{remark}
\begin{consequence}
   Три пункта:
   \begin{enumerate}
       \item $\forall n x_n \le b \land \lim x_n = a \Rightarrow a \le b$.
       \item $\forall n a \le y_n \land \lim y_n = b \Rightarrow a \le b$.
       \item $\forall n x_n \in [a;b] \land \lim x_n = l \Rightarrow l \in [a, b]$.
   \end{enumerate}
\end{consequence}
 \begin{proof}
     Константу можно заменить на последовательность $z_n = \text{const}$
\end{proof}
\begin{theorem}[Теорема о двух милиционерах(теорема о сжатой последовательности)]
    Пусть $\forall n: x_n \le y_n \le z_n \land \lim x_n = \lim z_n \eqqcolon l$, тогда $\lim y_n = l$.
\end{theorem}
\begin{proof}
    Возьмем $\epsilon > 0$. По лемме:  $\exists N: \forall n \ge N: |x_n-l| < \epsilon \land |z_n-l| < \epsilon$, откуда $x_n > l - \epsilon$ и  $z_n < l + \epsilon$. Тогда  $l - \epsilon < x_n \le y_n \le z_n < l + \epsilon \Rightarrow l - \epsilon < y_n < l + \epsilon$, то есть $|y_n - l| < \epsilon$.
\end{proof}
\begin{consequence}
     Если $\forall n |y_n| \le z_n \land \lim z_n = 0 \Rightarrow \lim y_n = 0$
\end{consequence}
\begin{proof}
    $x_n \coloneqq -z_n$. Тогда  $|y_n| \le z_n \iff -z_n \le y_n \le z_n$. Ну тогда и $\lim y_n = 0$
\end{proof}
\begin{theorem}[Теорема Вейерштрасса для монотонной последовательности]
    Три пункта:
    \begin{enumerate}
        \item $\forall x_n x_n \uparrow \land x_n\text{ --- ограничена сверху} \Rightarrow \exists a = \lim x_n$.
        \item $\forall x_n x_n \downarrow \land x_n\text{ --- ограничена снизу} \Rightarrow \exists a = \lim x_n$. 
        \item Монотонная последовательность имеет предел $\iff$ она ограничена.
    \end{enumerate}
\end{theorem}
\begin{proof}[Пункт 1.]
    $b \coloneqq \sup \{x_1,x_2,\ldots\}$ --- существует, т.к. $x_n$ --- ограничено сверху. Теперь докажем, что  $\lim x_n = b$, возьмем  $\epsilon > 0$. $b$ --- наименьшая верхняя граница  $\Rightarrow \forall \epsilon > 0 b - \epsilon$ --- не верхняя граница. То есть  $\exists N: x_N > b - \epsilon$. Проверим, что такое  $N$ подходит: при  $n \ge N$ $b - \epsilon < x_N < x_{N+1} < \ldots x_n \le b \le b + \epsilon \Rightarrow b - \epsilon < x_n < b + \epsilon$.
\end{proof}
\begin{proof}[Пункт 3.]
    Докажем отдельно в каждую сторону:
    \begin{itemize}
        \item[$\Leftarrow$]Если $\uparrow$, то пункт 1, иначе  пункт 2.
        \item[$\Rightarrow$]Докажем это утверждение для любой последовательности.

            Пусть $\lim x_n = a$. Возьмем  $\epsilon = 1$, тогда  $\exists N: \forall n > N: |x_n-a| < 1 \Rightarrow a-1<x_n<a+1$. Ну тогда верхняя граница $\max\{a+1,x_1,x_2,\ldots, x_{N+1}\}$, а нижняя $\min\{a-1,\ldots\}$.
    \end{itemize}
\end{proof}
\begin{remark}
    В 1: $\lim x_n = \sup\{x_1,x_2,\ldots\}$, во 2: $\lim x_n = \inf\{x_1,x_2,\ldots\}$.
\end{remark}
\begin{theorem}[О арифметичеких операциях с пределами]
    $\forall x_n, y_n a = \lim x_n \land \lim y_n = b$. Тогда: 
     \begin{enumerate}
         \item $x_n+y_n$ имеет предел и он равен  $a+b$
         \item $x_n-y_n$ имеет предел и он равен  $a-b$
         \item $x_n*y_n$ имеет предел и он равен  $a*b$
         \item $|x_n|$ имеет предел и он равен  $|a|$
         \item $\frac{x_n}{y_n}$ имеет предел, если $b \neq 0 \land \forall n y_n \neq 0$ и он  равен  $\frac{a}{b}$
    \end{enumerate}
\end{theorem}
\begin{proof}
    \slashn
    \begin{enumerate}
        \item Возьмем $\epsilon > 0$ и найдем  $N$ из леммы для  $\frac{\epsilon}{2}$. Тогда $\forall n \ge N: |x_n-a| < \frac{\epsilon}{2} \land |y_n-a| <  \frac{\epsilon}{2} \Rightarrow |(x_n+y_n) - (a+b)| \le |x_n - a| + |y_n-b| < \frac{\epsilon}{2} + \frac{\epsilon}{2} = \epsilon$
        \item Так же.
        \item Поскольку $\lim y_n = b$, то $y_n$ --- ограничена, а значит  $\exists M: |y_n| \le M$. Рассмотрим $|x_ny_n - ab| = |x_ny_n - ay_n + ay_n - ab| \le |x_ny_n - ay_n| + |ay_n - ab| = |y_n| |x_n-a| + |a| |y_n-b| \le M |x_n-a| + |a| |y_n|-b$. $M |x_n - a| < \frac{\epsilon}{2} \iff |x_n - a| < \frac{\epsilon}{2M}$. Значит $\exists N_1$ при котором $\forall n > N_1$ выполнено.  $|a| |y_n - b| < \frac{\epsilon}{2} \Leftarrow |y_n - b| < \frac{\epsilon}{2|a|+1}$. Тогда найдется $N_2$, такой что  $\forall n \ge N_2$ это выполнено. Такой что $N = \max{N_1, N_2}$.
        \item $||x|-|a|| \le |x_n-a| \iff -|x_n-a| \le |x_n| - |a| \le |x_n-a|$, а в правой части написано, что $|x_n| = |(x_n-a)+a| \le |x_n-a| + |a|$. Понятно, что это выполняется при любых $x_n, a$. 

            Возьмем  $N$, для которого  $\forall n > N: |x_n - a| < \epsilon$. Тогда  $\forall n \ge N: ||x_n|-|a|| \le |x_n-a| < \epsilon$
        \item Докажем, что $\lim \frac{1}{y_n} = \frac{1}{b}$.  Возьмем  $|\frac{1}{y_n} - \frac{1}{b}| = \frac{|y_n-b|}{|y_n||b|} \iff (1)$. Посмотрим на картинку: возьмем $\epsilon = \frac{b}{2}$. Получим интервал $(\frac{b}{2}; \frac{3b}{2})$. Тогда берем $N_1: \forall n \ge N |y_n-b| < |b|/2 \Rightarrow |y_n| > \frac{|b|}{2}$. Тогда $(1) \iff \frac{|y_n-b|}{\frac{|b|}{2}|b|} = \frac{2}{|b|^2}|y_n-b|<\epsilon \iff |y_n-b| < \epsilon \cdot \frac{|b|}{2}$. Поэтому $\exists N_2: \forall n \ge N_2$ такой, что это выполняется. Ну тогда $N=\max{N_1, N_2}$.
    \end{enumerate}
\end{proof}
\begin{consequence}
   Если $\lim x_n = a$, то  $\lim cx_n = ca$. 
\end{consequence}
\begin{consequence}
    Если $\lim x_n=a \land \lim y_n = b$, то $\lim(cx_n+dy_n) = ca+db$
\end{consequence}
 \begin{remark}
    Если $\lim y_n = b \neq 0$, то начиная с некоторого  $N$,  $y_n \neq 0$
\end{remark}
\begin{example}
    $\lim \frac{n^2+2n-3}{4n^2-5n+6} = \frac{1 + \frac{2}{n} - \frac{3}{n^2}}{4-\frac{5}{n}+\frac{6}{n^2}} = \frac{\lim(1+\frac{2}{n}-\frac{3}{n^2})}{4 - \frac{5}{n}+\frac{6}{n^2}} = \frac{1}{4}$
\end{example}
\Subsection{Бесконечно большие и бесконечно малые}
\begin{definition}
    Последовательность $x_n$ называется бесконечной малой, если $\lim x_n=0$.
\end{definition}
\begin{statement}
    $\forall x_n, y_n: x_n\text{ --- бесконечно мала последовательность} \land y_n\text{ ограничена}, то x_ny_n$ --- бесконечно малая последовательность.
\end{statement}
\begin{proof}
    $y_n$ --- ограничена  $\Rightarrow \exists M: \forall n: |y_n| \le M$. Возьмем $\epsilon > 0$ и подставим в определение  $\lim x_n = 0$. Тогда найдется  $N: \forall n \ge N: |x_n| < \frac{\epsilon}{M}$. Следовательно $x_ny_n \le M |x_n| < M \frac{\epsilon}{M} = \epsilon \Rightarrow \lim x_n y_n = 0$.
\end{proof}
\begin{definition}
    $\lim x_n = +\infty$ означает то, что вне любого луча вида  $(E; +\infty)$ лежит лишь конечное число членов последовательности. Или:  $\forall E \exists N: \forall n \ge N x_n > E$.
\end{definition}
\begin{definition}
    $\lim x_n = -\infty$ означает то, что вне любого луча вида  $(-\infty, E)$ лежит лишь конечное число членов последовательности. Или:  $\forall E \exists N: \forall n \ge N x_n < E$.
\end{definition}
\begin{definition}
    $\lim x_n = \infty$ означает то, что в любом промежутке содержится конечное число членов последовательности. Или:  $\forall \exists N \forall n \ge N |x_n| > E$.
\end{definition}
\begin{remark}
    $\lim x_n = \infty \iff \lim |x_n| = +\infty$
\end{remark}
\begin{remark}
    $\lim x_n = +\infty$(или $0\infty$)$\Rightarrow \lim x_n=\infty$. Но $\not \Leftarrow$! Пример  $x_n = (-1)^n \cdot n$.
\end{remark}
\begin{remark}
    $\lim x_n = \infty \Rightarrow x_n$ --- неограниченная последовательность. Но наоборот неверно. Пример: $x_n = \begin{cases} n & n\text{ --- четно} \\ 0 & n\text{ --- нечетно}\end{cases}$.
\end{remark}
\begin{definition}
    $x_n$ называется бесконечно большой, если  $\lim x_n = \infty$.
\end{definition}
\begin{theorem}
    $\forall x_n: \forall n x_n \neq 0 \Rightarrow x_n\text{ --- бесконечно малая} \iff \frac{1}{x_n}$ --- бесконечно большая.
\end{theorem}
\begin{proof}
    Докажем в каждую сторону отдельно:
    \begin{itemize}
        \item[$\Rightarrow$] $x_n$ --- бесконечно малая  $\iff \lim x_n=0$. Возьмем  $E$ из определения бесконечно большой и  $\varepsilon = \frac{1}{E}$, подставим в предел. Тогда $\exists N: \forall n \ge N|x_n| < \varepsilon = \frac{1}{E} \Rightarrow |\frac{1}{x_n}| > E$.
        \item[$\Leftarrow$] $\frac{1}{x_n}$ --- бесконечно большая $\Rightarrow \frac{1}{x_m} = \infty$. Возьмем $\varepsilon > 0$ из определения бесконечно малой и  $E = \frac{1}{\varepsilon}$ и подставим в $\lim$. Тогда $\exists N, \forall n \ge N: |\frac{1}{x_n}| >E=\frac{1}{\varepsilon} \Rightarrow |x_n| < \varepsilon$
    \end{itemize}
\end{proof}
\begin{definition}
    $\overline{\R} = \R \cup {\pm \infty}$
\end{definition}
\begin{theorem}
    В $\overline{\R}$ предел единственен.
\end{theorem}
\begin{proof}
    Пусть $\lim x_n = a \in \overline{\R}$ и  $\lim x_n = b \in \overline{\R}$. Если  $a, b \in \R$, то знаем. Иначе рассмотрим случаи:
     \begin{itemize}
         \item $a = \pm \infty, b \in \R$. Картинка.
         \item  $a = +\infty, b = -\infty$. Ну такого быть не может, смотри картинку.
    \end{itemize}
\end{proof}
\begin{theorem}[о стабилизации знака]
    Если $\lim x_n = a \in \overline{\R} \land a \neq 0 \Rightarrow \exists N: \forall n \ge N$ все члены последовательности имеют тот же знак, что и $a$.
\end{theorem}
\begin{proof}
    Несколько случаев:
     \begin{itemize}
         \item $a \in \R$. Картинка. Начиная с некоторого номер все  $x_n \in (0; 2a)$ или  $x_n \in (2a; 0)$.
         \item  $a = +\infty$. Картинка. Возьмем  $E=0$, начина с некоторого номера все члены попали в этот луч.
         \item  $a = \infty$. Аналогично.
    \end{itemize}
\end{proof}
\begin{theorem}[предельный переход в неравентсве $\overline{\R}$]
    $\forall n: x_n \le y_n \land \lim x_n = a \in \overline{\R} \land \lim y_n = b \in \overline{\R} \Rightarrow a \le b$.
\end{theorem}
\begin{proof}
    Если $a, b \in \R$, то уже есть. Иначе предположим противное:
     \begin{itemize}
         \item $a = +\infty$ и  $b \in \R$. Картинка...
    \end{itemize}
\end{proof}

\begin{theorem}[Теорема о двух миллиционерах]
    \slashn
    \begin{enumerate}
        \item $\forall x_n, y_n: x_n \le y_n \land \lim x_n = +\infty \Rightarrow \lim y_n = +\infty$
        \item $\forall x_n, y_n: x_n \le y_n \land \lim y_n = -\infty \Rightarrow \lim x_n = -\infty$
    \end{enumerate}
\end{theorem}
\begin{proof}
    \slashn
    \begin{enumerate}
        \item $\forall E: \lim x_n = +\infty \Rightarrow \exists N : n \ge N x_n > E$, но $y_n \ge x_n > E$.
        \item Упражнение для читателя.
    \end{enumerate}
\end{proof}
\begin{theorem}[О арифметических действиях с бесконечно большими]
    \slashn
    \begin{enumerate}
        \item $\forall x_n, y_n \lim x_n = +\infty, y_n\text{ --- ограничена снизу} \Rightarrow \lim(x_n+y_n)=+\infty$
        \item $\forall x_n, y_n \lim x_n = -\infty, y_n\text{ --- ограничена сверху} \Rightarrow \lim(x_n+y_n)=-\infty$
        \item $\forall x_n, y_n \lim x_n = \infty, y_n\text{ --- ограничена} \Rightarrow \lim(x_n+y_n)=\infty$
        \item $\forall x_n, y_n \lim x_n = \pm\infty \land \exists C: \forall n: y_n \ge C > 0  \Rightarrow \lim(x_ny_n)=\pm\infty$
        \item $\forall x_n, y_n \lim x_n = \pm\infty \land \exists C: \forall n: y_n \le C < 0  \Rightarrow \lim(x_ny_n)=\mp\infty$
        \item $\forall x_n, y_n \lim x_n = \infty \land \exists C: \forall n: |y_n| \ge C > 0  \Rightarrow \lim(x_ny_n)=\infty$
        \item $\forall x_n, y_n \lim x_n = a \neq 0 \land \forall n: \lim y_n = 0  \Rightarrow \lim \frac{x_n}{y_n}=\infty$
        \item $\forall x_n\text{ --- ограничена}, y_n: \lim y_n = \infty \Rightarrow \lim \frac{x_n}{y_n} = 0$
        \item $\forall x_n, y_n\text{ --- ограничена}: \lim x_n = \infty \land y_n \neq 0 \Rightarrow \lim \frac{x_n}{y_n} = \infty$
    \end{enumerate}
\end{theorem}
\begin{proof}
    \slashn
    \begin{enumerate}
        \item $y_n$ --- ограничена снизу  $\Rightarrow y_n \ge c$. А так как $\lim x_n = +\infty \Rightarrow \forall E \exists N: \forall n \ge N: x_n > E$. Подставим $E-c$ вместо  $E$.  $\exists N \forall n \ge N x_n > E - C \Rightarrow x_n + y_n \ge E - c + y_n \ge E - c + c= E$.
        \item Упражнение.
        \item Упражнение.
        \item $\lim x_n = +\infty \Rightarrow \forall E \exists N \forall n \ge N: x_n > E$. Подставим $\frac{E}{c}$ вместо $E$:  $x_n > \frac{E}{c} \Rightarrow x_ny_n \ge xn_C > \frac{E}{c}\cdot c = E$.
        \item Упражнение.
        \item Упражнение.
        \item $\lim y_n = 0 \Rightarrow y_n$ --- бесконечно малое $\Rightarrow \frac{1}{y_n}$ --- бесконечно большая. Поймем, что $|x_n| \ge C > 0$ при больших $n$. Возьмем картинку и окрестность  $\frac{a}{2}$. Заметим, что начиная с некоторого номер $|x_n| \ge \frac{a}{2} > 0$.
        \item $y_n $ --- бесконечно большая  $\Rightarrow \frac{1}{y_n}$ --- бесконечно малая $\Rightarrow x_n \cdot \frac{1}{y_n}$ --- произведение ограниченное и бесконечно малой.
        \item $x_n$ --- бесконечно большая  $\frac{1}{x_n}$ --- бесконечно малая $\Rightarrow y_n \cdot \frac{1}{x_n}$ --- бесконечно малая $\Rightarrow$  $\frac{y_n}{x_n}$ --- бесконечно малая.
    \end{enumerate}
\end{proof}
\slashn Арифметика с бесконечностями:
\begin{enumerate}
    \item $\pm \infty + c = \pm \infty$ 
    \item $+\infty + \infty = +\infty$
    \item  $-\infty + -\infty = -\infty$
    \item  $\pm \infty \cdot c = \pm \infty$, если  $c>0$
    \item  $\pm \infty \cdot c = \mp \infty$, если $c<0$
    \item  $+\infty\cdot+\infty = +\infty$\\
        $-\infty-\infty=+\infty$\\
        $+\infty-\infty = -\infty$
\end{enumerate}
Запрещенные операции:
\begin{enumerate}
    \item $+\infty - +\infty$ или  $+\infty + -\infty$. Может получиться беспредел, любое число, любая бесконечность.
    \item $+\infty \cdot 0$
    \item $\frac{\pm \infty}{\pm infty}$.Может получиться беспредел, любое число, бесконечность правильного знака.
    \item $\frac{0}{0}$ --- любое число, любая бесконечность, отсутствие предела.
\end{enumerate}
\begin{example}
    \slashn
    \begin{itemize}
        \item $x_n = n + a, y_n=n, x_n-y_n=a: \lim x_n = +\infty, \lim y_n = +\infty, \lim x_n - y_n = a$
        \item  $x_n = 2n \to +\infty, y_n = n \infty + \infty, x_n - y_n = n \to +\infty$
        \item ясно.
        \item  $x_n = n + (-1)^n \to +\infty, y_n = n \to +\infty, x_n - y_n = (-1)^n$ --- нет предела.
    \end{itemize}
\end{example}
\begin{exerc}
    Примеры к остальному.
\end{exerc}
\Subsection{Экспонента}
\begin{theorem}[Неравенство Бернулли]
    $\forall x \ge -1, n \in \N (1+x)^n \ge 1 + nx$. Равенство при $x=0 \lor n=1$.
\end{theorem}
\begin{proof}
    Индукция:
    \begin{itemize}
        \item База $n=1$: $1+x  \ge 1+x$.
        \item Переход $n \to n+1$.
        \item Предположение: $(1+x)^n \ge 1+nx$.
        \item Заметим, что $(1+x)^{n+1} = (1+x)\cdot(1+x)^n \ge (1+x)(1+nx) = 1 + x + nx + nx^2 = 1 + (n+1)x + nx^2 \ge 1 + (n+1)x$. Строгий знак при $x \neq 0$.
    \end{itemize}
\end{proof}
\begin{remark}
    На самом деле $(1+x)^P \ge 1 + Px$, если $x > -1$ и  $P \ge -1$ или $P \le 0$. Иначе верно $(1+x)^P \le 1 + Px$.
\end{remark}
\begin{theorem}
    Пусть $a \in \R$ и  $x_n \coloneqq(1 + \frac{a}{n})^n$. Тогда при $n > -a$ монотонно возрастает и ограничена сверху.
\end{theorem}
\begin{proof}
    $\frac{x_n}{x_{n+1}} = \frac{(1+\frac{a}{n})^n}{(1+\frac{a}{n-1})^{n-1}} = \frac{(n+a)^n}{n^n} \cdot \frac{(n-1)^{n-1}}{(n-1+a)^{n-1}} = \frac{n-1+a}{n-1} \left(\frac{(n+a)(n-1)}{n\cdot (n-1+a)}\right)^n = \frac{n-1+a}{n-1}\left(\frac{n^2+an-n-a}{n^2+an-n}\right)^n = \frac{n-1+a}{n-1}\left(1-\frac{a}{n(n-1+a)} \right)^n \ge \frac{n-1+a}{n-1}(1+n \cdot \frac{-a}{n(n-1+a)}) = \frac{n-1+a}{n-1} \cdot \frac{n-1+a-a}{n-1+a} = 1$

     Убедимся в выполнении условий для неравенства Бернулли. Посмотрим на $\frac{a}{n-1+a}$. Если $a>0$, то очевидно. Если  $a<0$, то  $n_1 > a$, а значит дробь меньше нуля.

    Ограниченность: $y_n \coloneqq (1-\frac{a}{n})^n$ возрастает при $n>a$.  $x_ny_n = \left(\left(1+\frac{a}{n}\right)\left(1-\frac{a}{n}\right)\right)^n = \left(1-\frac{a^2}{n^2}\right)^n \le 1$. Тогда $x_n \le \frac{1}{y_n} \le \frac{1}{y_{n-1}} \le \ldots \le \frac{1}{y_{[a] + 1}}$ 
\end{proof}
\begin{consequence}
    $x_n \coloneqq \left(1 + \frac{a}{n}\right)^n$ имеет предел.
\end{consequence}
\begin{definition}
    $\exp a \coloneqq \lim \left(1+\frac{a}{n}\right)^n$\\
    $e \coloneqq \exp 1 = \lim(1 + \frac{1}{n})^n \approx 2.7182818284590$
\end{definition}
\begin{consequence}
    Последовательность $z_n \coloneqq (1+\frac{1}{n})^{n+1}$ монотонно убывает и стремится к $e$.
\end{consequence}
\begin{proof}
    $\lim z_n = \lim \left(1+\frac{1}{n}\right)^n \lim \left(1+\frac{1}{n}\right) = e \cdot 1 = e$.
    $\frac{1}{z_n} = \frac{1}{\left(\frac{n+1}{n}\right)^{n+1}} = \left(\frac{n}{n+1}\right)^{n+1} = \left(1 - \frac{1}{n+1}\right)^{n+1}$ --- строго монотонно возрастает.
\end{proof}
\slashn
Свойства экспоненты:
\begin{enumerate}
    \item $\exp 0 = 1, \exp 1 = e$.
    \item  $\exp a > 0$
    \item  $\exp a \ge 1+a$. $\left(1+\frac{a}{n}\right)^n \ge 1 + n \frac{a}{n} = 1+a$, при $n > -a$. Далее совершим предельный переход.
    \item  $\exp a \exp (-a) \le 1$. $\left(1+\frac{a}{n}\right)^n\cdot\left(1-\frac{a}{n}\right)^n = \left(1-\frac{a^2}{n^2}\right)^n \le 1$. Далее предельный переход.
    \item $\forall a, b: a \le b \Rightarrow \exp a \le \exp b$. Знаем, что $1 + \frac{a}{n} \le 1 + \frac{b}{n}$ и при больших $n$ они положительны, тогда можно возвести в  $n$-ую степень и совершить предельный переход. 
    \item  $\exp a < \frac{1}{1-a}$ при $a \le 1$. $\exp a \cdot \exp (-a) \le 1 \Rightarrow \exp a \le \frac{1}{\exp(-a)}$. А $\exp(-a) \ge 1 + (-a) = 1 - a$ (применили Бернулли). Тогда можно уменьшить знаменатель, тем самым увеличить дробь.
    \item $\forall n \in \N x_n < e < z_n$. Знаем, что $x_n \uparrow$. Тогда возьмем  $k \ge n + 1: x_n < x_{n+1} < x_k$. Устремляем $k \to \infty: x_n < x_{n+1} \le e \iff x_n < e$.

        С другой стороны $z_n \downarrow$. Тогда по той же технике $z_n > z_{n+1} \ge e \iff z_n \ge e$.
    \item $2 < e < 3$. $2 = x_1$,  $3 = z_{5}$ или  $z_6$.
\end{enumerate}
\begin{remark}
    $z_n - x_n = \frac{x_n}{n} \approx \frac{e}{n}$.
\end{remark}
\begin{lemma}
    $\forall a_n \lim a_n = a \Rightarrow y_n \coloneqq (1+\frac{a_n}{n})^n \to \exp a$.
\end{lemma}
\begin{proof}
    Пусть $x_n = (1+\frac{a}{n})^n$, $A = 1 + \frac{a}{n}$, $B = 1 + \frac{a_n}{n}$. Тогда $|x_n - y_n| = |A^n - B^n| = \underbrace{|A-B|}_{=\frac{|a-a_n|}{n}}\cdot|A^{n-1} + A^{n-2}B + \ldots + B^{n-1}|$.

    Тогда $\lim a_n = a \Rightarrow a_n$ --- ограниченная последовательность ($|a_n| \le M$). Тогда  $|a_n| \le M \Rightarrow A = 1 + \frac{a}{n} \le 1 + \frac{M}{n}, B = 1 + \frac{a}{n} \le 1 + \frac{M}{n}$.Тогда исходное: $\frac{|a-a_n|}{n} n \left(1+\frac{M}{n}\right)^{n-1} \le |a-a_n|(1+\frac{M}{n})^n \le |a-a_n|\exp M \to 0$.

    То есть $\lim x_n - y_n \Rightarrow \lim x_n - \lim (x_n - y_n) = \exp a - 0 = \exp a$.
\end{proof}
\begin{theorem}
    $\exp(a+b) = \exp a \cdot \exp b$.
\end{theorem}
\begin{proof}
    $x_n \coloneqq \left(1+\frac{a}{n}\right)^n \to \exp a$. $y_n \coloneqq (1 + \frac{b}{n})^n \to \exp b$. Тогда $x_ny_n = \left((1+\frac{a}{n})(1+\frac{b}{n})\right)^n = \left(1+\frac{a+b+\frac{ab}{n}}{n}\right)^n \xrightarrow{\text{Лемма}} \exp(a+b)$
\end{proof}
\begin{consequence}
    \slashn
    \begin{enumerate}
        \item $\forall t: |t| < 1 \Rightarrow \lim t^n = 0$
        \item  $\forall t: |t| > 1 \Rightarrow \lim t^n = \infty$
    \end{enumerate}
\end{consequence}
\begin{proof}
    \slashn
    \begin{enumerate}
        \item[2.] Пусть $x = |t| - 1 > 0$. Тогда  $|t^n| = |t|^n = (1+x)^n > 1+nx \to +\infty$
        \item[1.] Если $0<|t|<1$, то  $|\frac{1}{t}| > 1$ и $\left(\frac{1}{t}\right)^n$ --- бесконечно большая $\Rightarrow$ $t^n$ --- бесконечно малая. 
    \end{enumerate}
\end{proof}
\begin{theorem}
    $\forall x_n > 0 \lim \frac{x_{n+1}}{x_n} = a < 1 \Rightarrow \lim x_n = 0$.
\end{theorem}
\begin{proof}
    Картинка. Возьмем окрестность с правой границей $b = \frac{a+1}{2}$. Тогда начиная с некоторого номера $m$ члены последовательности  $\frac{x_{n+1}}{x_n}$ попали в этот интервал. То есть $\frac{x_{n+1}}{x_n} \le b < 1$ при $n \ge m$. 

    Пусть $n > m$.  $x_n = \frac{x_n}{x_{n+1}} \cdot \frac{x_{n-1}{x_{n-2}}}\cdot\ldots\cdot\frac{x_{m+1}}{x_m} \cdot x_m \le b^{n-m}\cdot x_m = b^n \frac{x_m}{b^m}$. Тогда $0 < x_n < \underbrace{b^n}_{\to 0}\frac{x_m}{b^m}$. 
\end{proof}
\begin{consequence}
    \slashn
     \begin{enumerate}
         \item $\lim \frac{n^k}{a^n} = 0$ при $a>1$ и  $k \in \N$. 
         \item $\lim \frac{a^n}{n!} = 0$, при $a \in \R$.
         \item $\lim \frac{n!}{n^n} = 0$.
    \end{enumerate}
\end{consequence}
\begin{proof}
    \slashn
    \begin{enumerate}
        \item $x_n = \frac{n^k}{a^n}$. $\frac{x_{n+1}}{x_n} = \frac{(n+1)^k}{a^{n+1}} \cdot \frac{a^n}{n^k} = \left(\frac{n+1} {n}\right)^k \cdot \frac{1}{a} \to \frac{1}{a} < 1$.
        \item $x_n = \frac{a^n}{n!}$. $\frac{x_{n+1}}{x_n} = \frac{a^{n+1}}{(n+1)!} \cdot \frac{n!}{a^n} = \frac{a}{n+1} \to 0$.
        \item $x_n = \frac{n!}{n^n}$. $\frac{x_{n+1}}{x_n} = \frac{(n+1)!}{(n+1)^{n+1}} \cdot \frac{n^n}{n!} = \frac{n+1}{(n+1)^{n+1}} n^n = \left(\frac{n}{n+1}\right)^n = \frac{1}{\left(1+\frac{1}{n}\right)^n} \to \frac{1}{e} < 1$.
    \end{enumerate}
\end{proof}
\begin{theorem}[Теорема Штольца]
    Пусть $y_1 < y_2 < y_3 < \ldots$ и $\lim y_n = +\infty$. Если  $\lim \frac{x_{n+1}-x_n}{y_{n+1}-y_n} = l \in \overline{\R}$, то  $\lim \frac{x_n}{y_n} = l$.
\end{theorem}
\begin{proof}
    Ключевой случай: $l = 0$. Пусть  $a_n \coloneqq \frac{x_{n+1} - x_n}{y_{n+1} - y_n}$. По условию $\lim a_n = 0$. Зафиксируем  $\varepsilon > 0$. Тогда найдется номер  $m$, такой, что при $n \ge m \Rightarrow |a_n| < \varepsilon$. Тогда $|x_{n+1} -x_n| = |a_n|(y_{n+1} - y_n) < \varepsilon (y_{n+1} - y_n)$.  $|x_n - x_m| \le |x_n - x_{n-1}| + |x_{n-1} + x_{n-2}| + \ldots + |x_{m+1} - x_m| < \varepsilon(y_n - y_{n-1}) + \varepsilon(y_{n-1} + y_{n-2}) + \ldots + \varepsilon(y_{m+1} - y_m) = \varepsilon(y_n - y_m)$.

    Теперь посмотрим на $|\frac{x_n}{y_n}| \le \frac{|x_n -x_m| + |x_m|}{|y_n|} < \frac{\varepsilon(y_n - y_m)}{y_n} + \frac{|x_m|}{y_n} < \varepsilon \frac{1y_n}{y_n} + \frac{|x_m|}{y_n} = \varepsilon + \frac{|x_m|}{y_n} < 2\varepsilon$. Берем такой $N: \forall n \ge N y_n \ge \frac{1}{\varepsilon}|x_m|$. Если $n > \max\{m, N\}$, то  $|\frac{x_n}{y_n}| < 2\varepsilon$. Тогда  $\lim \frac{x_n}{y_n} = 0$

    Рассмотрим случай $l \in \R$. Рассмотрим  $\widetilde{x_n} \coloneqq x_n - l y_n$. Тогда  $\frac{(\widetilde{x_{n+1}} - \widetilde{x_n}) - (x_n - l y_n)}{y_{n+1} - y_n} = \frac{x_{n+1} - x_n}{y_{n+1} - y_n} - l \to 0$, т.к. $\frac{x_{n+1} - x_n}{y_{n+1} - y_n} \to 0$. 

    Рассмотрим случай $l = +\infty$.  $\frac{x_{n+1} - x_n}{y_{n+1} - t_n} \to +\infty \Rightarrow \frac{x_{n+1} - x_n}{y_{n+1} - y_n} > 1$ при $n \ge N \Rightarrow x_{n+1} - x_n > y_{n+1} - y_n > 0 \Rightarrow x_n \uparrow$.

    Из того, что для $x_n - x_{n-1} > y_n - y_{n-1} \land \ldots \land x_{N + 1} - x_{N} > y_{N} - y_{N-1} \Rightarrow x_n - x_N > y_n - y_N \Rightarrow x_n > \underbrace{y_n}_{\to +\infty} + \underbrace{(x_N - y_N)}_{=const} \Rightarrow x_n \to +\infty$. Тогда $\frac{y_{n+1} - y_n}{x_{n+1} - x_n} \to 0 \xRightarrow{\text{Случай 0}} \frac{y_n}{x_n} \to 0 \Rightarrow \frac{x_n}{y_n} \to \infty \frac{x_n}{y_n} \to +\infty$.

    Случай  $l=-\infty$.  $\widetilde{x_n} = -x_n$.
\end{proof}
\begin{example}
    $S_n = 1^k + 2^k + \ldots + n^k < n^nk = n^{k+1}$. Можно еще взять половину: получим $\le \frac{n^{k+1}}{2^{k+1}}$

    Тогда $\lim \frac{S_n}{n^{k+1}} = \lim \frac{S_n - S_{n-1}}{n^{k+1} - (n-1)^{k+1}} = \lim \frac{n^k}{n^{k+1} - (n^{k+1} - k\cdot n^k + \frac{k(k-1)}{2} n^{k-1} - \ldots)} = \lim \frac{1}{k - \ldots n^{-1} + \ldots n^{-2} + \ldots} = \frac{1}{k}$
\end{example}
\begin{remark}
    Тоже самое можно сказать, если $y_n \downarrow -\infty$
\end{remark}
\begin{theorem}[Теорема Штольца 2]
    $\lim x_n = \lim y_n = 0 \land y_n > y_{n+1} > 0$. Тогда  $\lim \frac{x_{n+1} - x_n}{y_{n+1} - y_n} = l \in \overline{\R} \Rightarrow \lim \frac{x_n}{y_n} = l$.
\end{theorem}
\begin{proof}
    Случай $l = 0$.  $a_k \coloneqq \frac{x_{k+1} - x_k}{y_{k+1} - y_k}$.  $\lim a_n = 0$. Возьмем  $\varepsilon > 0$ и найдем  $m \ge N: |a_n| < \varepsilon$ при  $n \ge N$. Тогда $x_n - x_m = (x_n - x_{n-1}) + (x_{n-1} + x_{n-2}) + \ldots + (x_{m+1} - x_m) = a_{n-1}(y_n - y_{n-1}) + a_{n-2}(y_{n-1} - y_{n-2}) + \ldots + a_{m} (y_{n+1} - y_n).$

    Тогда $|x_n - x_m| \le |a_{n-1}|(y_{n-1} - y_n) + \ldots + |a_m|(y_m - y_{m-1}) < \varepsilon ((y_{n-1} - y_n) + (y_{n-2} - y_{n-1}) + \ldots + (y_m - y_{m+1})) = \varepsilon(y_m - y_n)$ 

    $|x_n - x_m| < \varepsilon(y_m - y_n) \to \varepsilon y_m \Rightarrow |x_m| \le \varepsilon y_m \Rightarrow |\frac{x_m}{y_m} \le \varepsilon$. Получили определение предела!!!

    Случай $l \in \R$. См. выше.

    Случай $l = +\infty$ нужна лишь монотонность.

    Случай  $l = -\infty$.
\end{proof}
\Subsection{Подпоследовательность}
\begin{definition}
    Последовательность: последовательность $x_{n_i}$, заданная как набор индексов $n_i: 1 \le n_1 < n_2 < n_3 < \ldots$. 
\end{definition}
\slashn
Свойства.
\begin{enumerate}
    \item $n_k \ge k$. Индукция. $n_{k+1} > n_k \ge k$.
    \item Если последовательность предел в $\overline{\R}$, то подпоследовательность имеет тот же предел.
    \item Две две подпоследовательности  $x_{n_1}, x_{n_2},\ldots$ и $x_{m_1}, x_{m_2},\ldots$ в объединение дают всю последовательность и они имеют один и тот же предел $l \in \overline{R}$, то  $\lim x_n = l$. Доказательство по картинке.
\end{enumerate}
\begin{theorem}[О стягивающихся отрезках]
    Пусть $[a_1; b_1] \supset [a_2; b_2] \supset [a_3, b_3] \supset \ldots$ и $\lim (b_n - a_n) = 0$. Тогда  $\exists! c \in \R$ принадлежащая всем отрезкам и  $\lim a_n = \lim b_n = c$.
\end{theorem}
\begin{proof}
    Существование следует из теоремы о вложенных отрезка. Докажем единственность. Пусть $c, d \in [a_n; b_n]$. Тогда  $c-d \le b_n - a_n \to 0 \Rightarrow c = d$.

    Проверим, что $\lim a_n=c$.  $|a_n - c|$ --- длина подотрезка  $[a_n; b_n]$, тогда $|a_n - c| \le b_n - a_n \to 0 \Rightarrow a_n - c \to 0 \Rightarrow \lim a_n = c$.
\end{proof}
\begin{theorem}[Больцано-Вейерштрасса]
    Из любой ограниченной последовательности можно выделить сходящуюся подпоследовательность. То есть, если $x_n$ --- ограниченная последовательность, то существует  $x_{n_k}$ имеющая конечный предел.
\end{theorem}
\begin{proof}
    $a$ --- нижняя граница,  $b$ --- верхняя для  $x_n$. То есть  $\forall n: x_n \in [a; b]$. В какой-то половине отрезка бесконечное число членов (иначе в сумме конечное число членов). Назовем подходящую $[a_1;b_1]$. Теперь делю эту половинку пополам. В одной из половинок половинки бесконечное число членов. Получили процесс деления отрезков на кусочки. 

    Заметим, что $[a; b] \supset [a_1; b_1] \supset \ldots$. $b_n - a_n = \frac{b-a}{2^n} \to 0$. Тогда $\exists c \in R: \lim a_n = \lim b_n = c$. Берем  $[a_1; b_1]$, там бесконечное число членов. Берем какой-то $x_{n_1}$. Берем $[a_2; b_2]$, там есть  $x_{n_2} > n_1$. Тогда получили возрастание индексов и  $x_{n_k} \in [a_k; b_k]$.  $a_k \le x_{n_k} < b_k$, то тогда по двум милиционерам $\lim x_{n_k} = c$.
\end{proof}
\begin{theorem}
    Несколько пунктов:
    \begin{enumerate}
        \item Монотонная неограниченная последовательность стремится к $\pm\infty$.
        \item Из неограниченной сверху последовательности можно выбрать подпоследовательность, стремящуюся к $+\infty$.
        \item Из неограниченной снизу последовательности можно выбрать подпоследовательность, стремящуюся к $-\infty$.
    \end{enumerate}
\end{theorem}
\begin{proof}
    \slashn
    \begin{enumerate}
        \item Пусть $x_n$ монотонно возрастает. Докажем, что  $\lim x_n = +\infty$. Берем  $E$. Это не верхняя граница, значит найдется  $x_N > E \Rightarrow n \ge N \Rightarrow E < x_N \le X_{N+1} \le \ldots \le x_n$.
        \item 1 --- не верхняя граница $\Rightarrow$ есть  $x_{n_1} > 1$. 2 +  $x_{n_1}$ ---  не верхняя граница $\Rightarrow$ есть  $x_{n_k} > 2 + x_{n_k} + 2$. И так далее \ldots Получилось, что $x_{n_k} > k \to \infty \Rightarrow x_{n_k} \to +\infty$. Дальше переставим члены в порядке возрастания индексов.
    \end{enumerate}
\end{proof}
\begin{definition}
    $l$ --- Частичный предел предел последовательности, если существует подпоследовательность, стремящаяся к $l$.
\end{definition}
\begin{remark}
    Больцано-Виерштрасса + 2/3 пункт предыдущей теоремы говорят о том, что частичный предел точно существует.
\end{remark}
\begin{definition}
    Пусть $l \in \R$. Тогда окрестность  $l$  --- произвольный интервал  $(l-\varepsilon; l+\varepsilon)$.
\end{definition}
\begin{definition}
    Окрестность $+\infty$ --- луч  $(E; +\infty)$,  $-\infty$ --- луч $(-\infty; E)$
\end{definition}
\begin{theorem}
    $l \in \overline{R}$ --- частичный предел последовательности  $\iff $ в любой окрестности  $l$ содержится бесконечно много членов.
\end{theorem}
\begin{proof}
    \slashn
    \begin{itemize}
        \item $\Rightarrow$ $l$ --- частичный предел  $\Rightarrow$ найдется подпоследовательность  $x_{n_k} \to l \Rightarrow$ вне любой окрестности  $l$ конечное число членов последовательности $\Rightarrow$ внутри конечное.
        \item $\Leftarrow$ точка в  $(l-1, l+1)$ бесконечно мало членов последовательности. Если бесконечное число совпало с  $l$, то берем их в качестве подпоследовательности. 

            Если конечно, то берем член, не равный  $l$. Пусть это  $x_{n_1} \neq l$. Рассмотрим  $\varepsilon_2 = \min \{\frac{1}{2}, |x_{n_1} - l|\}>0$. Тогда в $(l-\varepsilon_2; l + \varepsilon_2$ бесконечное число членов. Берем  $n_2 > n_1$ и  $x_{n_2} \neq l$. Берем  $n_2 > n_1$. Рассмотрим $\varepsilon_3$ \ldots. Получилось $n_1 < n_2 < \ldots$ и $0 \le |x_{n_k} - l| < \varepsilon_k \le \frac{1}{k} \to 0$.
    \end{itemize}
\end{proof}
\begin{definition}
    $x_n$ --- фундаментальная (сходящаяся в себе, последовательность Коши), если  $\varepsilon >0 \exists N: m, n \ge N |x_n - x_m| < \varepsilon$.
\end{definition}
\begin{property}
    Сходящаяся последовательность фундаментальна.
\end{property}
\begin{proof}
    Берем $\varepsilon$. Пусть предел  $l = \lim x_n$. Берем  $\varepsilon>0$.  $\exists N \forall n \ge N |x_n - l| < \frac{\varepsilon}{2} \land |x_m - l| < \frac{\varepsilon}{2} \Rightarrow |x_n - x_m| \le |x_n - l| + |l - x_m < \varepsilon$ 
\end{proof}
\begin{property}
    Фундаментальная последовательность ограничена.
\end{property}
\begin{proof}
    Берем $\varepsilon=1$. Тогда  $\exists N \forall m, n \ge N: |x_n - x_m| < 1 \Rightarrow |x_n - x_N| < 1 \Rightarrow |x_n| < 1 + |x_N| \Rightarrow |x_n| \le \max\{|x_1|, \ldots, |x_{N-1}|, 1 + |x_n|$.
\end{proof}
\begin{property}
   Если фундаментальная последовательность содержит сходящуюся последовательность, то она сама сходящаяся.
\end{property}
\begin{proof}
    Берем $\varepsilon >0$.  $\exists N: \forall n, m \ge N |x_n - x_m| < \varepsilon$. $\exists K: \forall k \ge K |x_{n_k} - l| < \varepsilon$. Возьмем $n \ge N$ и рассмотрим $x_n - l \le |x_n - x_{n_k}| + |x_{n_k} -l| < 2 \varepsilon$, если $k = \max\{N, K\}$.
\end{proof}
\begin{theorem}[Критерий Коши]
    Последовательности фундаментальна $\iff$ последовательность сходящаяся. 
\end{theorem}
\begin{proof}
    \slashn
    \begin{itemize}
        \item $\Leftarrow$. Св-во 1.
        \item  $\Rightarrow$ Фундаментальна  $\Rightarrow$ ограничена  $\Rightarrow$ существует сходящаяся подпоследовательность.
    \end{itemize}
\end{proof}
\begin{example}
    $x_n \coloneqq \displaystyle \sum_{k=1}^n \frac{\sin k}{2^k}$. Проверим фундаментальность. Пусть $n > m$.  $|x_n - x_m| = |\sum_{k=m+1}^n \frac{\sin k}{2^k}| \le \sum_{k=m+1}^n \frac{|sin k|}{2^k} \le \sum_{m+1}^n \frac{1}{2^k} = \frac{1}{2^m} - \frac{1}{2^n} < \frac{1}{2^m} < \varepsilon$.
\end{example}
\begin{definition}
    $x_n$ --- последовательность.  $y_n \coloneqq \inf_{k \ge n} x_k, z_n \coloneqq \sup_{k \ge n} x_k$. Тогда верхний предел $x_i$  $\overline{\lim} x_n = \lim z_n$, а нижний предел  $\underline{\lim} x_n = \lim y_n$
\end{definition}
\begin{theorem}
    $\exists \underline{\lim}$ и $\overline{\lim}$.  $\overline{\lim} \ge \underline{\lim}$.
\end{theorem}
\begin{proof}
    $y_n \le z_n \Rightarrow \underline{\lim} \le \overline{\lim}$.

    Существование. $y_n$ монотонно убывает.  $y_{n+1} = \inf \{x_{n+1},\ldots\} \ge \inf\{x_n,\ldots\} = y_n$. Тогда у нее есть предел. Аналогично $z_n$ монотонно убывает.
\end{proof}
\begin{remark}
    $\overline{\lim} = \inf z_n = \inf \sup_{k \ge n} x_n$.

    $\underline{\lim} = \sup y_n = \sup \inf_{k \ge n} x_n$.
\end{remark}

\begin{theorem}
    \slashn
    \begin{enumerate}
        \item Верхний предел --- наибольший из всех частичных пределов.
        \item Нижний предел --- наименьший из всех частичных пределов.
        \item Если $\underline{\lim} = \overline{\lim}$, то последовательность имеет предел и он равен  $\underline{\lim}$.
    \end{enumerate}
\end{theorem}
\begin{proof}
    \slashn
    \begin{enumerate}
        \item
        \item
        \item $y_n = \inf \{x_n, x_{n+1},\ldots\} \le x_n \le \sup \{x_n, x_n+1,\ldots\} = x_n$. Тогда два милиционера.
    \end{enumerate}
\end{proof}

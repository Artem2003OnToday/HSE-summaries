\Subsection{Дифференцируемость и производная}
\begin{definition}
    $f\!:\langle a, b \rangle \to \R \land x_0 \in \langle a, b \rangle$.

    $f$ --- дифференцируема в точке  $x_0$, если существует такое  $k \in \R\!: f(x) = f(x_0) + k(x-x_0) + o(x-x_0)$ при $x \to x_0$. Можно думать, что $\alpha(x) = o(x-x_0)$, где $\alpha\frac{x}{x - x_0} \xrightarrow{x \to x_0} 0$.
\end{definition}
\begin{definition}
    Производная функции $f$ в точке  $x_0$ ---  $\lim_{x \to x_0} = \frac{f(x) - f(x_0)}{x - x_0} = \lim_{h \to 0} \frac{f(x_0+h) - f(x_0)}{h} \eqqcolon f'(x)$.
\end{definition}

\begin{theorem}[Критерий дифференцируемости]
    $f\!: \langle a, b \rangle \to \R, x_0 \in \langle a, b \rangle$. Следующие условия равносильны: 
     \begin{enumerate}
         \item $f$ дифференцируема в точке  $x_0$.
         \item  $f$ имеет в точке  $x_0$ конечную производную.
         \item $\exists \varphi\!: \langle a, b \rangle \to \R: f(x) - f(x_0) = \varphi(x)(x - x_0)$ и $\varphi$ непрерывна в точке  $x_0$.
    \end{enumerate}
    Причем, если выполнены эти условия, то $k = f'(x_0) = \varphi(x_0)$
\end{theorem}
\begin{proof}
    \begin{itemize}
        \item $1. \Rightarrow 2. f(x) = f(x_0) + k(x - x_0) + o(x - x_0) \Rightarrow \frac{f(x) - f(x_0)}{x - x_0} = \frac{k(x - x_0) + o(x - x_0)}{x-x_0} = k + o(1) \Rightarrow \lim_{x \to x_0}\frac{f(x) - f(x_0)}{x-x_0} = k \Rightarrow f'(x_0) = k$ 
        \item $2. \Rightarrow 3.$  $\lim_{x \to x_0} \frac{f(x) - f(x_0)}{x-x_0} = f'(x_0) \in \R$. $\varphi(x) = \begin{cases} \frac{f(x) - f(x_0)}{x - x_0} & x \neq x_0 \\ f'(x_0) & x = x_0 \end{cases} \Rightarrow \varphi$ --- непрерывна в $x_0$. 
        \item  $3. \Rightarrow 1.$ $f(x) - f(x_0) = \varphi(x)(x - x_0)$, причем $\lim_{x \to x_0} \varphi(x) = \varphi(x_0) \Rightarrow f(x) = f(x_0) + \varphi(x_0)(x - x_0) + (\varphi(x) - \varphi(x_0))???$
    \end{itemize}
\end{proof}
\begin{definition}
    Бесконечная производная $\lim_{x\to x_0}\frac{f(x) - f(x_0)}{x - x_0} = \pm \infty$
\end{definition}
\begin{example}
    $f(x) = \sqrt[3]{x}$.  $f'(0) = \lim_{h\to 0} \frac{f(h) - f(0)}{h - 0} = \lim_{h \to 0} \frac{\sqrt[3]{h}}{h} = \lim_{h \to 0} \frac{1}{\sqrt[3]{h^2}} = +\infty$
\end{example}
\begin{definition}
    $f_+' \coloneqq = \lim_{x\to x_0+} \frac{f(x)-f(x_0)}{x-x_0}$
    $f_-' \coloneqq = \lim_{x\to x_0-} \frac{f(x)-f(x_0)}{x-x_0}$
\end{definition}
\begin{remark}
    Существование  $f'(x_0) \iff$ существование $f_{\pm}'(x_0)$ и их равенство.
\end{remark}
\begin{example}
    $f(x) = |x|$.  $f_+'(x) = 1, f_-'(x)=-1$
\end{example}
\begin{definition}
    Касательная --- предельное положение секущей.
\end{definition}
\begin{example}
    Уравнение касательной. Пусть $f$ дифференцируема в точке  $u \in \langle a, b \rangle$.

    $y = f(u) + \frac{f(v) - f(u)}{v - u}(x - u)$. $f'(u) = \lim_{v \to u}\frac{f(v) - f(u)}{v - u}$. То есть $y = f(u) + f'(u)(x-u)$.
\end{example}

\begin{definition}
    Дифференциал функции $f(x_0+h) = f(x_0) + k \cdot h + o(h)$ при  $h \to 0$.  $f(x_0)$ --- константа, $k \cdot h$ --- что-то линейное.
    
    Дифференциал функции ---  линейное отображение $k \cdot$.
\end{definition}
\begin{statement}
    Если $f$ дифференцируема в  $x_0$, то  $f$ непрерывна в  $x_0$.
\end{statement}
\begin{proof}
    $f(x) = f(x_0)+\underbrace{k(x-x_0)}_{\to 0}+\underbrace{o(x-x_0)}_{\to 0} \xrightarrow{x \to x_0} f(x_0)$
\end{proof}

\begin{theorem}[Арифметические действия с дифференцируемыми функциями]
    $f, gg\!: \langle a, b \rangle \to \R, x_0 \in \langle a, b \rangle, f, g$ --- дифференцируемые в  $x_0$. Тогда:
     \begin{enumerate}
         \item $f \pm g$ дифференцируема в  $x_0$ и $(f \pm g)' = f' \pm g'$
         \item $f \cdot g$ дифференцируема в  $x_0$ и $(f \cdot g)' = f'g + fg'$
         \item $cf$ дифференцируема в  $x_0$ и $(cf)' = cf'$
         \item $\alpha f + \beta g$ дифференцируема в  $x_0$ и $(\alpha f + \beta g)' = \alpha f' + \beta g'$
         \item если $g(x_0) \neq 0$, то  $\frac{f}{g}$ дифференцируема в $x_0$ и $\left(\frac{f}{g}\right)' = \frac{f'g - fg'}{g^2}$
    \end{enumerate}
    \begin{proof}
        \slashn
        \begin{enumerate}
            \item $(f+g)'(x_0) = \lim_{x \to x_0} \frac{(f(x) + g(x)) - (f(x_0) + g(x_0))}{x - x_0} = \lim_{x \to x_0} \frac{f(x) - f(x_0)}{x - x_0} + \lim_{x \to x_0} \frac{g(x) - g(x_0)}{x - x_0} = f'(x_0) + g'(x_0)$ 
            \item $(fg)'(x_0) = \lim_{x \to x_0} \frac{f(x)g(x) - f(x_0)g(x_0)}{x - x_0} = \lim_{x \to x_0} = \frac{f(x)g(x) - f(x)g(x_0) + f(x)g(x_0)-f(x_0)g(x_0)} = \lim_{x \to x_0} f(x) \frac{g(x) - g(x_0)}{x - x_0} + \lim_{x \to x_0} g(x) \frac{f(x) - f(x_0)}{x - x_0} = fg' + f'g$
            \item $(cf)' = cf' + c'f = cf'$
            \item  $(\alpha f + \beta g)' = (\alpha f)' + (\beta g)' = \alpha f' + \beta g'$
            \item $\left(\frac{f}{g}\right)' = (f' \cdot \frac{1}{g}) + f \cdot (\frac{1}{g})'$.

                $(\frac{1}{g})'(x_0) = \lim_{x \to x_0} \frac{\frac{1}{g(x)} - \frac{1}{g(x_0)}}{x - x_0} = \lim_{x\to x_0} \frac{1}{g(x_0)g(x_0)}\frac{g(x_0) - g(x)}{x - x_0} = -\frac{g'(x_0)}{g(x_0)^2}$.
        \end{enumerate}
    \end{proof}
\end{theorem}

\begin{theorem}[Дифференцируемость композиции]
    Пусть $f\!: \langle a, b \rangle \to \R,\ g\!:\langle c, d \rangle \to \langle a, g \rangle$,  $x_0 \in \langle c, d \rangle$, $g$ дифференцируема в точке  $x_0$, $f$ дифференцируема в точке $g(x_0)$. 

    Тогда $f\circ g$ дифференцируема в точке $x_0$, и $(f \circ g)'(x_0) = f'(g(x_0))g'(x_0)$. 
\end{theorem}
\begin{proof}
    $g$ дифференцируема в точке  $x_0 \Rightarrow g(x) - g(x_0) = \psi(x)(x - x_0)$, где $\psi$ --- непрерывна в точке  $x_0$. $f$ дифференцируема в точке  $y_0 = g(x_0) \Rightarrow f(y) - f(y_0) = \varphi(y)(y-y_0)$, где $\varphi$ непрерывна в точке  $y_0$.

    Поставим $y=g(x)$, получим $f(g(x)) - f(g(x_0)) = \varphi(g(x))(g(x) - g(x_0) = \underbrace{\varphi(g(x))}_{\mathclap{\text{непрерывна в точке } x_0 \text{ как композиция}}}\psi(x)(x-x_0) \Rightarrow f \circ g$ дифференцируема в точке $x_0$ и $(f\circ g)'(x_0) = \varphi(g(x_0))\psi(x_0) = f'(g(x_0)) g'(x_0)$
\end{proof}
\begin{theorem}[дифференцируемость обратной функции]
    Пусть $f\!:\langle a, b \rangle \to \R$, строго монотонная, непрерывна, $x_0 \in \langle a, b \rangle$, f дифференцируема в точке $x_0$ и $f'(x_0) \neq 0$.

    Тогда $f^{-1}$ дифференцируема в точке $y_0 = f(x_0)$ и $(f^{-1})' = \frac{1}{f'(x_0)}$.
\end{theorem}
\begin{proof}
    $f$ дифференцируема в $x_0 \Rightarrow f(x) - f(x_0) = \varphi(x)(x - x_0)$, где $\varphi$ --- непрерывна в точке  $x_0$. 
    Пусть $y = f(x)$. Тогда предыдущее равенство можно написать как  $y - y_0 = \varphi(f^{-1}(y))(f^{-1}(y) - f^{-1}(y_0))$.

    $\varphi(f^{-1}(y))$ непрерывна в точке $y_0$ как композиция непрерывных и $\varphi(f^{-1}(y_0)) = \varphi(x_0) = f'(x_0) \neq 0$. Тогда $\varphi(x) \neq 0$ в окрестности  $x_0$ и  $f^{-1}(y) - f^{-1}(y_0) = \frac{1}{\varphi(f^{-1}(y))}(y - y_0)$ 

    $f^{-1}(y_0) = \frac{1}{\varphi(f^{-1}(y_0))} = \frac{1}{\varphi(x_0)} = \frac{1}{f'(x_0)}$
\end{proof}
\begin{consequence}
    $f^{-1}(y) = \frac{1}{f'(f^{-1}(y))}$
\end{consequence}
\slashn
\Subsection{Таблица производных}
\begin{enumerate}
    \item $c' = 0$
    \item  $(x^p)' = p x^{p-1}, p \in \R, x > 0$
    \item $(a^x)' = a^x \ln a, a > 0$

        $(e^x)' = e^x$
    \item  $(\log_a x)' = \frac{1}{x \ln a}, a > 0, a \neq 1$ 

        $(ln x)' = \frac{1}{x}$ 
    \item $(\sin x)' = \cos x$
    \item  $(\cos x)' = -\sin x$
    \item  $(\tg x)' = \frac{1}{\cos^2 x}$
    \item $(\ctg x)' = -\frac{1}{\sin^2 x}$
    \item $(\arcsin x)' = \frac{1}{\sqrt{1-x^2}}$ 
    \item $(\arcsin x)' = -\frac{1}{\sqrt{1-x^2}}$ 
    \item $(\arctg x)' = \frac{1}{1 + x^2}$ 
    \item $(\arcctg x)' = -\frac{1}{1+x^2}$
\end{enumerate}
\begin{proof}
    \begin{itemize}
        \item Очевидно.
        \item $(x^p)' = \lim_{h \to 0} \frac{(x + h)^p - x^p}{h} = x^p \lim_{h \to 0} \frac{(1 + \frac{h}{x})^p - 1}{\frac{h}{x}} \frac{1}{x} = x^p \frac{p}{x} = px^{p - 1}$.
        \item $(a^x)' = \lim_{h \to 0} \frac{a^{x+h} - a^x}{h} = a^x \lim_{h \to 0} \frac{a^h - 1}{h} = a^x \ln a$
        \item $\log_a x$ --- обратная к  $a^x$ функция  $f(x) = a^x$, тогда  $(\log_a y)' = \frac{1}{f'(f^{-1}(y))} = \frac{1}{a^{f^{-1}(y)} \ln a} = \frac{1}{y \ln a}$
        \item $(\sin x)' = \lim_{h \to 0} \frac{\sin(x+h) - \sin x}{h} = \lim_{h \to 0} \frac{\sin x \cos h + \sin h \cos x - \sin x}{h} = \sin x \lim_{h\to 0} + \cos x \lim_{h \to 0} \frac{\sin h}{h} = \cos x \cdot 1 + \sin x \cdot 0$ 
        \item Тоже самое.
        \item $(\tg x)' = (\frac{\sin x}{\cos x})' = \frac{(\sin x)' \cos x - (\cos x)' \sin x}{\cos^2 x} = \frac{\cos x \cos x - (-\sin x) \sin x}{\cos ^ x} = \frac{1}{\cos^2 x}$
        \item Тоже самое.
        \item $f(x) = \sin x$.  $\arcsin' (y) = (f^{-1})'(y) = \frac{1}{f'(f^{-1}(y))} = \frac{1}{f'(\arcsin y)} = \frac{1}{\cos(\arcsin y)} = \frac{1}{\sqrt{1 - \sin^2(\arcsin y)}} = \frac{1}{\sqrt{1 - y^2}}$.
        \item $\arccos x = \frac{\pi}{2} - \arcsin x$.
        \item $f(x) = \tg x$,  $\arctg\!: \R \to (-\frac{\pi}{2}; \frac{pi}{2})$.

            $(\arctg)'(y) = (f^{-1})'(y) = \frac{1}{f'(f^{-1}(y))} = \frac{1}{f'(\arctg y)} = \frac{1}{\frac{1}{\cos^2(\arctg y)}} = \cos^2(\arctg y) = \frac{1}{1+\tg^2(\arctg y)} = \frac{1}{1+y^2}$, так как  $\frac{1}{\cos^2 x} = 1 + \tg^2 \Rightarrow \cos^2 x = \frac{1}{1 + \tg^2 x}$
    \end{itemize}
\end{proof}
\Subsection{Теоремы о среднем}
\begin{theorem}[Теорема Ферма]
    $f\!: \langle a, b \rangle \to \R$,  $x_0 \in (a, b), f$ --- дифференцируема в  точке $x_0$.

    $f(x_0) = \max_{x \in \langle a, b \rangle}f(x) \lor f(x_0) = \min_{x \in \langle a, b \rangle}f(x) \Rightarrow f'(x_0) = 0$
\end{theorem}
\begin{proof}
    Пусть $f(x_0) = \max_{x \in \langle a, b \rangle}f(x)$. Тогда  $f'(x_0) = f_+'(x_0) = \lim_{x \to x_0+} \frac{\overbrace{f(x) - f(x_0)}^{\mathclap{\le 0}}}{\underbrace{x - x_0}_{\mathclap{> 0}}} \le 0$. 

    Посмотрим на $f'(x_0) = f_-'(x_0) = \lim_{x \to x_0-}\frac{\overbrace{f(x) - f(x_0)}^{\mathclap{\le 0}}}{\underbrace{x - x_0}_{\mathclap{< 0}}} \ge 0$.
    
    Значит, что $f'(x_0) \le 0$ и $f'(x_0) \ge 0 \Rightarrow f'(x_0) = 0$. 
\end{proof}
\begin{example}
    У $f(x) = x$ на отрезке  $[0, 1]$ максимум в 1, минимум в 0, но производная на концах 1.
\end{example}
\begin{theorem}[Теорема Ролля]
    $f\!: [a, b] \to \R$ непрерывна во всех точках и дифференцируема на $(a, b)$. Если  $f(a) = f(b)$, то $\exists\!: c \in (a, b f'(c) = 0$ 
\end{theorem}
\begin{proof}
    $f$ непрерывна  на $[a, b] \xRightarrow[\text{Вейерштрасса}]{\text{теорема}} f$ достигает наибольшего и наименьшего значения. 

    Пусть $f(p) = \min, f(q) = \max$. Если  $p$ и  $q$ --- концы отрезка, то  $\max = \min \Rightarrow f = \text{const} \Rightarrow f' = 0$ во всех точках. 

    А если же одна из этих точек не является концом отрезка, то по теорема Ферма  $f'$ в этой точке равна 0.
\end{proof}
\begin{remark}
    Геометрический смысл теоремы Ферма: в точках $\min$,  $\max$ касательная горизонтальна.

    Геометрический смысл теоремы Ролля: если значения на концах равны, то можно провести горизонтальную касательную
\end{remark}
\begin{theorem}[Лагранжа, Формула конечных приращений]
    $f\!: [a, b] \to \R$ непрерывна на  $[a, b]$, дифференцируема на  $(a, b)$. Тогда $\exists c \in (a, b)\!: f(b) - f(a) = f'(c)(b-a)$
\end{theorem}
\begin{proof}
    $g(x) = f(x) - kx$. Подберем  $k$ так, что  $g(a) = g(b)$.  $f(a) - ka = g(a) = g(b) = f(b) - kb \Rightarrow k = \frac{f(b) - f(a)}{b - a}$.

    Применим теорему Ролля к функции $g(x)$. $\exists c \in (a, b)\!: g'(c) = 0 \iff 0 = g'(c) = f'(c) = f'(c) - \frac{f(b) - f(a)}{b - a}$ и умножим на $b-a$.
\end{proof}
\begin{remark}
    $\frac{f(b) - f(a)}{b - a}$ --- угловой коэффициент секущей, $f'(c)$ --- угловой коэффициент касательной в точке  $c$. В некоторой точке касательная параллельная секущей. 
\end{remark}
\begin{theorem}[Теорема Коши]
    $f, g\!:[a, b] \to \R$ непрерывны на  $[a, b]$, дифференцируемы на  $(a, b)$,  $g'(x) \neq 0\ \forall x\in(a, b)$. Тогда $\exists c \in (a, b)\!: \frac{f(b) - f(a)}{g(b) - g(a)} = \frac{f'(c)}{g'(c)}$
\end{theorem}
\begin{proof}
    $h(x) \coloneqq f(x) - kg(x)$. Подберем $k$ так, что  $h(a) = h(b)$.

    $f(a) - kg(a) = h(a) = h()b = f(b) - kg(b) \Rightarrow k = \frac{f(b) - f(a)}{g(b) - g(a)}$ (по Роллю у нас $g'(x) \neq 0$, а значит в концах значения точно не равны).

    по т. Ролля для $h$ найдем  $c \in (a, b)\!: h'(c) = 0$. Тогда  $h'(c) = f'(c) - kg'(c) \Rightarrow k = \frac{f'(c)}{g'(c)}$.
\end{proof}
\begin{remark}
    Геометрический смысл. $g(t), f(t))$ координаты точки в момент времени  $t$. Тогда  $k$ --- угловой коэффициент секущей. $(f'(t), g'(t))$ --- вектор скорости в момент времени  $t$. Тогда $\frac{f'(t)}{g'(t)}$ --- угловой коэффициент касательной. 
\end{remark}
\begin{definition}
    $f\!:E\to \R$ --- липшивеца функция с константной $M$, если  $\forall x, y \in E\!: |f(x) - f(y)| \le M|x-y|$.
\end{definition}
\begin{consequence}[Следствия теоремы Лагранжа]
    \slashn
    \begin{enumerate}
        \item $f\!:\langle a, b \rangle \to \R$ непрерывна на  $\langle a, b \rangle$, дифференцируема на $(a, b)$ и  $|f'(x)| \le M\ \forall x \in (a, b)$. 

            Тогда $|f(x) - f(y)| \le M|x-y|\ \forall x, y \in \langle a, b \rangle$ 
        \item  $f\!:\langle a, b \rangle \to \R$ непрерывна на  $\langle a, b \rangle$, дифференцируема на $(a, b)$. Если $f'(x) = 0\ \forall x \in (a, b) \Rightarrow f = \text{const}$. 
        \item  $f\!:\langle a, b \rangle \to \R$ непрерывна на  $\langle a, b \rangle$, дифференцируема на $(a, b)$, $f'(x) > 0\ \forall x \in (a, b) \Rightarrow f$ строго возрастает.
        \item  $f\!:\langle a, b \rangle \to \R$ непрерывна на  $\langle a, b \rangle$, дифференцируема на $(a, b)$, $f'(x) \ge 0\ \forall x \in (a, b) \Rightarrow f$ нестрого возрастает.
        \item  $f\!:\langle a, b \rangle \to \R$ непрерывна на  $\langle a, b \rangle$, дифференцируема на $(a, b)$, $f'(x) < 0\ \forall x \in (a, b) \Rightarrow f$ строго убывает.
        \item  $f\!:\langle a, b \rangle \to \R$ непрерывна на  $\langle a, b \rangle$, дифференцируема на $(a, b)$, $f'(x) \le 0\ \forall x \in (a, b) \Rightarrow f$ нестрого убывает.
    \end{enumerate}
\end{consequence}
\begin{proof}
\slashn
    \begin{enumerate}
        \item $[x, y] \subset \langle a, b \rangle$. Применим теорему Лагранжа к  $[x, y]$. Тогда  $\exists c \in (x, y) \subset (a, b)\!: f(y)-f(x) = f'(c)(y-x) \Rightarrow |f(y) - f(x)| = |f'(c)| \cdot |y-x| \le M|y-x|$ 
        \item Аналогично: $f(y)-f(x) = f'(c)(y - x) = 0$,  $y - x \neq 0$. 
        \item Напишем Лагранжа для $[x, y] \subset \langle a, b \rangle, (x, y) \subset (a, b)$. 
            Тогда  $\exists c \in (a, b)\!: f(y) - f(x) = f'(c)(y-c)$, что  $>0 \iff f(y) >f(x).$
        \item Аналогично.
        \item Аналогично.
        \item Аналогично.
    \end{enumerate}
\end{proof}
\begin{theorem}[Теорема Дорбу]
    $f\!:[a, b] \to \R$ дифференцируема во всех точках. Пусть  $C$ лежит между  $f'(a)$ и  $f'(b)$. Тогда найдется  $c \in (a, b)$, такая что  $f'(c) = C$.
\end{theorem}
\begin{proof}
    Пусть $f'(a) < 0 < f'(b)$. Покажем, что $\exists c \in (a, b)\!: f'(c) = 0$, $f$ непрерывна на  $[a, b]$. По теореме Вейерштрасса  $f$ достигает  $\min$ пусть это в точке $c$. Покажем, что  $c \neq a$ и  $c \neq b$. 

    От противного: пусть $f(a) = \min$. Тогда  $f'(a) = f_+'(a) = \lim \frac{f(x) - f(a)}{x-a}$. Заметим, что числитель  $\ge 0$ и знаменатель $> 0$, тогда по предельному переходу предел  $\ge 0$. Противоречие.

    Пусть $f(b) = \min$. Тогда  $f'(b) = f_-'(b) = \lim_{x \to b-} \frac{f(x) - f(b)}{x - b}$. Заметим, что  числитель  $\ge 0$, а знаменатель $<0$. Противоречие.
    
    Следовательно,  $c \in (a, b)$. Тогда по теореме Ферма  $f'(c) = 0$. Общий случай  $g(x) \coloneqq f(x) - C(x)$.  $g'(x) = f'(x) - C \Rightarrow g'(a)$ и  $g'(b)$ разных знаков, следовательно  $\exists c \in (a, b)\!: g'(c) f'(c) - C = 0 \Rightarrow f'(c) = C$.
\end{proof}
\begin{consequence}
    $f\!: \langle a, b \rangle \to \R$, дифференцируема на  $\langle a, b \rangle, f'(x) \neq 0\ \forall x \in \langle a, b \rangle $. Тогда  $f$ строго монотонна.
\end{consequence}
\begin{proof}
    Очев. 
\end{proof}
\begin{theorem}[Правило Лопиталя]
    $-\infty \le a < b \le +\infty$, $f,g$ дифференцируемы на  $(a, b)$.  $g'(x) \neq 0\ \forall x \in (a, b)$ и $\lim){x \to a+} f(x) = \lim_{x \to a+} = 0$. 

    Если  $\lim_{x \to a+} \frac{f'(x)}{g'(x)} = l \to \overline{R}$, то $\lim_{x \to a+} \frac{f(x)}{g(x)} = l$. 
\end{theorem}
\begin{proof}
    Проверяем по Гейне. Возьмем $x_n \to a$, причем убывающую.  $\frac{f(x_n)}{g(x_n)} \xrightarrow{?} l$. Посчитаем $\lim_{n \to \infty} \frac{f(x_{n+1}) - f(x_n)}{g(x_{n+1}) - g(x_n)} = \lim_{n \to +infty} \frac{f'(c_n)}{g'(c_n)}$, так как $\frac{f'(c_n)}{g'(c_n)}$, где $c_n \to (x_{n+1}, x_n)$. Последовательность  $c_n$ стремится  к $a$ справа. Тогда по Штольцу $\lim \frac{f(x_n)}{g(x_n)} = l$. Надо было проверить, что $g(x_n)$ строго монотонна. Это из последнего следствия и монотонности $x_n$.
\end{proof}
\begin{theorem}[Второе правило Лопиталя]
    $-\infty \le a < b \le +\infty$. $f, g$ дифференцируемы на  $(a, b)$.  $\forall x\in (a, b)\!: g'(x) \neq 0, \lim_{x \to a+}g(x) = +\infty$.  $\lim_{x \to a+} \frac{f'(x)}{g'(x)} =l \to \overline{R} \Rightarrow \lim_{x \to a+} \frac{f(x)}{g(x)} = l$
\end{theorem}
\begin{example}
    $\lim_{x \to +\infty} \frac{\ln x}{x^P} = 0$ при $p > 0$.

    Подставляем в Лопиталя  $f(x) = \ln x, g(x) = x^p$,  $f'(x) = \frac{1}{x}$, $g'(x) = px^{p - 1}$.  $\lim_{x \to +infty}\frac{f'(x)}{g'(x)} = \lim_{x \to +\infty} \frac{\frac{1}{x}}{px^{p-1}} = \lim_{x \to +\infty} \frac{1}{p} \frac{1}{x^p} = 0$
\end{example}
\begin{example}
    $\lim_{x \to +infty} \frac{x^p}{a^x} = a^x$. $f'(x) = px^{-1}, g'(x) = a^x \ln a$.

    $\lim_{x \to +\infty} \frac{f'(x)}{g'(x)} = \lim_{x \to +\infty} \frac{px^{p-1}}{a^x \ln a} =  \frac{p}{\ln a} \lim_{x \to +infty} \frac{x^{p-1}}{a^x} = 0$, при $p \le 1$.
\end{example}
\begin{example}
    $\lim_{x \to 0+} x^x = \lim_{x \to 0+} e^{x \ln x} = e^{\lim_{x \to 0+} x \ln x} = e^0 = 1$.
    
    $\ln x^x = x \ln x, \lim_{x \to 0+} x \ln x = \lim_{x \to +} \frac{\ln x}{\frac{1}{x}} = \lim_{x \to 0+} \frac{(\ln x)'}{(\frac{1}{x})'} = \lim_{x \to 0+} \frac{\frac{1}{x}}{-\frac{1}{x^2}} = \lim_{x \to 0+}(-x) = 0$
\end{example}
\Subsection{Производные высших порядков}
\begin{definition}
    $f: \langle a, b \rangle \to \R$,  $x_0 \in \langle a, b \rangle$, $f$ -- дифференцируема в окрестности  $x_0$.
    И если $f'$ дифференцируема в $x_0$, то $f$ дважды дифференцируема в $x_0$.

    То есть $f''(x_0) \coloneqq (f'(x))'\mid_{x=x_0}$
\end{definition}
\begin{definition}
    $f$ дважды дифференцируема в окрестности $x_0$. Если $f''$ дифференцируема в точке  $x_0$, то $f$ трижды дифференцируема в точке  $x_0$. 
\end{definition}
\begin{definition}
    $f \in C(E) \iff f: E \to \R \land$ непрерывна во всех точках.
\end{definition}
\begin{definition}
    $f \in C^1(\langle a, b \rangle)$  $f\!: \langle a, b \rangle \to \R$, дифференцируема во всех точках и  $f'$ не непрерывна.

    Будем называть такое свойство <<непрерывной дифференцируемостью>>.
\end{definition}
\begin{definition}
    $f \in C^n(\langle a, b \rangle)$. $f\!: \langle a, b \rangle \to \R$  $n$ раз дифференцируема и  $f^{(n)}$ непрерывна.
\end{definition}
\begin{definition}
    $f \in C^{\infty}(\langle a, b \rangle)$ означает, что  $f \in C^n(\langle a, b \rangle)\ \forall n \in \N$.
\end{definition}
\begin{remark}
    $C^n(\langle a, b \rangle) \supsetneqq C^{n+1}(\langle a, b \rangle) \supset C^{\infty}(\langle a, b \rangle)$
\end{remark}
\begin{example}
    $f_n(x) \coloneqq x^{n + \frac{1}{3}}$. Покажем, что $f_n \in C^n(\R), f_n \not in C^{n+1}(\R)$.

    Тогда  $f^{(n)}(x) = (n+\frac{1}{3}(n-\frac{2}{3})\ldots \frac{4}{3}) \eqqcolon cx^{\frac{1}{3}}$. Заметим, что $f^{(n+1)}$ в пределе равна  $\frac{1}{\sqrt[3]{x}}$, что в предел бесконечность.
\end{example}

\begin{theorem}[Арифметические действия с $n$-ми производными]
    $f, g\!: \langle a, b \rangle \to \R, x_0 \in \langle a, b \rangle$, $f, g$  $n$ раз дифференцируема в  $x_0$. Тогда:
    \begin{enumerate} 
        \item $\alpha f + \beta g$  $n$ раз дифференцируема в точке $x_0$ и $(\alpha f + \beta g)^{(n)} = \alpha f^{(n)} + \beta g^{(n)}$
        \item  $fg$  $n$ раз дифференцируема в точке раз и  $(fg)^{(n)} = \sum_{k=0}^n \binom{n}{k}f^{(n)}g^{(n - k)}$.
        \item $f(\alpha x + \beta)^{(n)} = \alpha f^{(n)}(\alpha x + \beta)$ 
        \item (Композиция: формула Фаа-Ди Бруно)
    \end{enumerate}
\end{theorem}
\begin{proof}
    \slashn
    \begin{enumerate}
        \item Индукция по $n$.
        \item Индукция по  $n$. База  $n=1$.  $(fg)' = f'g+ fg'$.

            Переход  $n \to n + 1$.  $(fg)^{(n+1)} = ((fg)^{(n)})' = (\sum_{k=0}^n \binom{n}{k}(f^{(k)}g^{(k-i)})' = \sum_{k=0}^n \binom{n}{k}(f^{(k - 1)}g^{(n - k)} + f^{(k)}g^{(n - k +1)}) = \sum_{k=0}^{n} \binom{n}{k}f^{(k+1)}g^{(n - k)} + \sum_{k = 0}^{n}\binom{n}{k} f^{(k)}g^{(n - k + 1)} = \sum_{k=0}^{n+1}\binom{n+1}{k}f^{(k)}g^{(n - k + 1)}$.
        \item Индукция по  $n$. Очев.
    \end{enumerate}
\end{proof}
\begin{example}
    \begin{enumerate}
        \item $(x^p)^{(n)} = p(p-1)(p-2)\ldots(p-n+1)x^{p-n}$
        \item $(\frac{1}{x})^{(n)} = (-1)(-2)\ldots(-n)x^{-1-n} = \frac{(-1)^n n!}{x^{n+1}}$
        \item $(\ln x)^{(n)} = ((\ln x)')^{(n-1)} = (\frac{1}{x})^{(n-1)} = \frac{(-1)^{n-1} (n-1)!}{x^{n}}$
        \item $a^x = ((a^x)')^{(n-1)} = (\ln a a^x)^{(n-1)} = \ldots = (\ln a)^n a^x$

            $(e^x)^{(n)} = e^x$
        \item  $(\sin x)^{(n)} = \sin(x + \frac{\pi n}{2})$

            $(\sin x)^{(n)} = ((\sin x)^{(n-1)})' = (\sin(x + \frac{\pi n}{2}))' = \sin(x + \frac{\pi (n-1)}{2} + \frac{\pi}{2}) = \sin (x + \frac{\pi n}{2})$
        \item  $(\cos x)^{(n)} = \cos(x + \frac{\pi n}{2})$
    \end{enumerate}
\end{example}
\begin{theorem}[формула Тейлорпа для многочленов]
    $T$ --- многочлен степени  $n$. Тогда  $T(x) = \sum_{k=0}^n \frac{T^{(n)}(x_0)}{k!}(x - x)^k$
\end{theorem}
\begin{lemma}
    $f(x) = (x-x_0)^k \Rightarrow f^{m}(x_0) = \begin{cases} k! = m! & \text{если, }k=m \\ 0 & \text{иначе} \end{cases}$
\end{lemma}
\begin{proof}[Доказательство леммы]
    $f^{(m)}(x) = k(k-1)\ldots(k - m + 1)(x-x_0)^{k-m}$.

    Если $k > m$, то  $f^{(m)}(x_0) = 0$

    Если $k = m$, то  $f^{(m)}(x_0) = k(k-1)\ldots 1 = k!$

    Если $k < m$, то  $f^{(k)}(x) \equiv k!$ и  $f^{(k + 1)}(x) = \equiv 0$
\end{proof}
\begin{proof}[Доказательство теоремы]
    Напишем разложение $T(x) = \sum_{k=0}^n c_k(x-x_0)^k$. (Пояснение: $T(x) = \sum_{k=0}^n a_k x^k = \sum_{k=0}^n a_k(x_0 + (x-x_0))^k$ и раскроем скобки).

    Поймем, что $c_k = \frac{T^{(k)}(x_0)}{k!}$. Заметим, что $T^{(n)}(x_0) = \sum_{k=0}^{n} (c_k(x-x_0)^k)^{(m)} \mid_{x=x_0} = m!$.
\end{proof}
\begin{definition}
    Пусть $f$  $n$ раз дифференцируема в точке  $x_0$. Многочлен Тейлора степени $n$  $T_{n, x_0} f(x) \coloneqq = \sum_{k=0}^n \frac{f^{(k)}(x_0)}{K!}(x-x_0)^k$.

    $R_{n, x_0}f(x) \coloneqq f(x) - T_{n, x_0}f(x)$
\end{definition}

\begin{lemma}
    $g$  $n$ раз дифференцируема в  $x_0$ и  $g(x_0) = g'(x_0) = \ldots = g^{(n)}(x_0) = 0$.

    Тогда $g(x) = o((x-x_0)^n)$ при $x \to x_0$.
\end{lemma}
\begin{proof}
    $\lim_{x \to x_0} \frac{g(x)}{(x-x_0)^n} \stackrel{\text{Лопиталь}}{=} \lim_{x \to x_0}\frac{g'(x)}{n(x-x_0)^{n-1}} = \frac{g''(x)}{n(n-1)(x-x_0)^{n-2}} = \ldots = \lim_{x \to x_0}\frac{g^{(n-1)}(x)}{n! (x-x_0)} = 0$ 

    $g^{(n-1)}$ --- дифференцируема в  $x_0 \Rightarrow g^{(n-1)}(x)= g^{(n-1)}(x_0)+g^{(n)}(x_0)(x - x_0) + o(x-x_0) = o(x-x_0)$ 
\end{proof}
\begin{theorem}[Формула Тейлора с остатком в форме Пеано]
    Пусть $f$  $n$ раз дифференцируема в точке  $x_0$. Тогда $f(x) = T_{n, x_0}f(x) + o((x-x_0)^n) = \sum_{k=0}^n \frac{f^{(k)}(x_0)}{k!}(x-x_0)^k + o((x-x_0)^n)$.
\end{theorem}
\begin{proof}
    $g(x) \coloneqq f(x) - T_{n,x_0}f(x)$. $g^{(m)}(x_0) = f^{(m)}(x_0) - (T_{n,x_0} f)^{(m)}(x_0) = f^{(m)} - f^{(m)}(x_0)$, смотри теорему выше про Тейлора для многочлена.

    Таким образом, $g(x_0) = g'(x_0) = \ldots = g^{n}(x_0) = 0$.

    По лемме $g(x) = o((x-x_0)^n) \Rightarrow f(x) - T_{n, x_0}f(x) = o((x - x_0)^n)$.
\end{proof}
\begin{consequence}
    $f$  $n$ раз дифференцируема в точке  $x_0$ $f(x) = P(x) + o((x-x_0)^n)$ при $x\to x_0$, где $P$ --- многочлен степени $\le n$. Тогда $P(x) = T_{n,x_0}f(x)$ 
\end{consequence}
\begin{proof}
    $Q(x) \coloneqq P(x) - T_{n, x_0}f(x) = o((x-x_0)^n)$ при $x \to x_0$.

    $Q(x) = \sum_{k=0}^n a_k(x-x_0)^k$ пусть $a_m \neq 0$ --- ненулевой коэффициент с наименьшим индексом  $\Rightarrow \frac{Q(x)}{(x-x_0)^m} = a_m + \sum_{k=m+1}^n a_k(x-x_0)^{k-m}$. Левая часть стремится к нулю, справа второе слагаемое стремится к 0. Но значит и $a_m$ должно быть нулю.
\end{proof}
\begin{theorem}[Формула Тейлора с остатком в форме Лагранжа]
    $f\!: \langle a, b \rangle \to \R$,  $x_0 \in \langle a, b \rangle$, $f$  $(n+1)$ раз дифференцируема на  $\langle a, b \rangle$. Тогда  $\exists c$ между  $x$ и  $x_0$, такой что  $f(x) = T_{n, x_0}f(x) + \frac{f^{n+1}(c)}{(n+1)!}(x-x_0)^{n+1} = \sum_{k=0}^n \frac{f^{(k)}(x_0)}{k!}(x-x_0)^k + \frac{f^{(n+1)}(c)}{(n+1)!}(x-x_0)^{n+1}$
\end{theorem}
\begin{proof}
    Зафиксируем $x$ и возьмем  $M \in \R$, такой что  $f(x) = T_{n, x_0}f(x) + M(x-x_0)^{n-1}$. Рассмотрим $g(y) \coloneqq f(t) - T_{n, x_0}f(t) - M(t-x_0)^{n+1}$

    $g^{(n+1)}(t) = f^{(n+1)}(t) - M(n+1)!$, то есть надо доказать, что  $g^{(n+1)}(c) = 0$, в некоторой точке между  $x$ и  $x_0$. Знаем, что $g(x) = 0$, $g(x_0) = g'(x_0) = \ldots = g^{(n)}(x_0) = 0$.
    $g(x) = g(x_0) = 0 \xRightarrow{\text{Ролль}} \exists x_1$ между $x$ и  $x_0$, такая что $g'(x_1) = 0$.

    $g'(x_1) = g'(x_0) =0 \xRightarrow{\text{Ролль}} \exists x_2$ между $x)$ и  $x_1$, такая что $g''(x_2) = 0$.

    И так далее до $g^{(n)}(x_n) = g^{(n)}(x_0) = 0 \xRightarrow{\text{Ролль}} \exists c$ между $x$ и  $x_0$, такая что $g^{(n+1)}(c) = 0$
\end{proof}
\begin{consequence}
    Если  $|f^{(n+1)}(t)| \ge M\ \forall t \in (x_0, x)$, то $|R_{n, x_0}f(x)| \le \frac{M(x-x_0)^{n+1}}{(n+1)!} = \mathcal{O}((x-x_0)^{n+1})$.
\end{consequence}
\begin{consequence}
    Если $|f^{(n)}(t)| \le M\ \forall n\ \forall t \in (a, b)$, то $T_{n, x_0}f(x) \xrightarrow{n \to \infty} f(x)$
\end{consequence}
\begin{proof}
    $|f(x) - T_{n, x_0}f(x)| = |\frac{f^{n+1}(c)}{(n+1)!} (x-x_0)^{n+1}| \le M \frac{(x-x_0)^{n+1}}{(n+1)!} \to 0$. Так как $x - x_0 = h, M \cdot \frac{h^{n+1}}{(n+1)!} \to 0$.
\end{proof}

\textbf{Формулы Тейлора для элементарных функций. Везде $x_0=0$}.
\begin{itemize}
    \item $e^x = 1 + x + \frac{x^2}{2!} + \frac{x^3}{3!} + \ldots + \frac{x^n}{n!} + o(x^n)$
    \item $\sin x = x - \frac{x^3}{3!} + \frac{x^5}{5!} - \ldots + (-1)^n \frac{x^{2n + 1}}{(2n+1)!} + o(x^{2n + 2})$.
    \item $\cos x = 1 - \frac{x^2}{2!} + \frac{x^5}{4!} - \ldots + (-1)^n \frac{x^{2n}}{(2n)!} + o(x^{2n+1})$.
    \item $\ln (1+x) = x - \frac{x^2}{2} + \frac{x^3}{3} - \ldots + (-1)^{n-1} x^\frac{n}{n} + o(x^n)$ 
    \item $(1+x)^P = 1+px + \frac{p(p-1)}{2!}x^2 + \frac{p(p-1)(p-2)}{3!}x^3 + \ldots + \frac{p(p-1)\ldots(p-n+1)}{n!}x^n + o(x^n)$.
\end{itemize}
\begin{proof}
    \begin{itemize}
        \item Для $f(x) = \ln(1+x)$.  $f^{(k)}(x) = \frac{(-1)^{k-1}(k-1)!}{(1+x)^k}$ $f^(k)(0) = ()-1)^{k-1}(k-1)!$

            Тогда $f(x) = \sum_{k=0}^n \frac{f^{(k)}(0)}{k!} = \ldots = $.
        \item ???
    \end{itemize}
\end{proof}

\textbf{Ряды Тейлора для $e^x$,  $\sin x$,  $\cos x$}
\begin{align}
    e^x &= \sum_{n=0}^\infty \frac{x^n}{n!} \\
    \sin x &= \sum_{n = 0}^\infty \frac{(-1)^nx^{2n+1}}{(2n+1)!} \\
    \cos x &= \sum_{n=0}^{\infty} \frac{(-1)^n x^{2n}}{(2n)!}
\end{align}
\begin{proof}
    $\sin x, \cos x$ удовлетворяют следствию 2. $(\sin x)^{(n)} = \sin(x + \frac{\pi n}{2}) \Rightarrow |\sin^{(n)}(x)| \le 1$. $|\cos^{(n)}(x)| \le 1$  

    Следовательно, $T_{n, 0}f(x) \xrightarrow{n\to\infty}f(x)$. А значит по определению ряда  $\sum_{k=0}^\infty \ldots = \sin x$

    Рассмотрим $f(x)=e^x$ на  $[a, b] \Rightarrow f^{(n)}(x) = e^x \le e^b$, тогда $|f^{(n)}(x)| \le e^b$. Тогда по свойству ряда и соображениям выше, получаем, что сумма данного ряда равна $e^x$.
\end{proof}

\Subsection{Предел функции}
\begin{definition}
    $a \in \R$, тогда  $U_a$ --- окрестность точки  $a$  $\Leftarrow U_a = (a-\varepsilon, a + \varepsilon)$.
\end{definition}
\begin{definition}
    $\dot{U_a} = U_a \setminus \{a\}$ --- выколотая окрестность.
\end{definition}
\begin{definition}
    $E \subset \R$ a --- предельная точка  $E$, если любая  $\dot{U_a}$ пересекается с  $E$.
\end{definition}
\begin{theorem}
    Следующие условия равносильны:
    \begin{enumerate}
        \item $a$ --- предельная точка  $E$.
        \item В любой  $U_a$ содержится бесконечное кол-во точек из  $E$.
        \item  $\exists \{a_n\}: \forall n: a_n \in E \land a_n \to a$. Более того, можно выбрать последовательность  $x_n \in E$ так, что  $|x_n - a| \downarrow 0$.
    \end{enumerate}
\end{theorem}
\begin{proof}
    \slashn
    \begin{itemize}
        \item $2 \Rightarrow 1$. $U_a cap E$ содержит бесконечное число точек  $\Rightarrow$ хотя бы одна из них не  $a$ и тогда  $\dot{U_a} \cap E \neq \varnothing$.
        \item $3 \Rightarrow 2$. Берем  $x_n \neq a \in E: \lim x_n = a$. Возьмем  $U_a = (a-\varepsilon, a+\varepsilon)$.  $\exists N: \forall n \ge N\; x_n \in (x_n) \in U_a$.
        \item $1 \Rightarrow 3$. Возьмем $\varepsilon_1 = 1$:  $(a-1; a+1)$ содержит точку из $E \setminus \{a\}$. Назовем такую точку $x_i$.

            Возьмем  $\varepsilon_2 = \min\{\frac{1}{2}, |x_i - a|\} > 0: (a - \varepsilon_2; a + \varepsilon_2)$ содержит точку из $E \setminus \{a\}$. Назовем её  $x_2$.

            Возьмем  $\varepsilon_3 = \min\{\frac{1}{3}, |x_2 - a|\} > 0$ (заметим, что $|x_2 - a| < \varepsilon_2 < |x_1 - a|$). Тогда $(a-\varepsilon_3, a + \varepsilon_3)$ содержит точку из  $E \setminus \{a\}$.

            Получили  $|x_1-a| > |x_2 - a| > \ldots$ причем $|x_k - a| < \varepsilon_k = \min\{\frac{1}{k}, |x_{k-1} - a|\} \le \frac{1}{k} \to 0 \Rightarrow x_k - a \to 0 \Rightarrow x_k \to a$. 
    \end{itemize}
\end{proof}
\begin{definition}
    Пусть $a$ --- предельная точка  $E$.  $f: E \to \R$. Тогда  $A = \lim_{x\to a} f(x)$, если
     \begin{enumerate}
         \item По Коши. $\forall \varepsilon > 0\, \exists \delta >0 \, \forall x \in E: |x-a| < \delta \Rightarrow |f(x) - A| < \varepsilon$.
         \item Окрестности. $\forall U_A \exists U_a: f(\dot{U_a} \cap E) \subset U_A$.
         \item По Гейне. Для любой последовательности  $a \neq x_n \in E: \lim x_n = a \Rightarrow f(x_n) = A$.
    \end{enumerate}
\end{definition}
\begin{proof}[Равносильность 1. и 2.]
    $\forall U_a \exists U_a: f(\dot{U_a} \cap E) \subset U_A$.  

    $\forall U_a \iff \forall \varepsilon > 0: U_A = (A - \varepsilon, A + \varepsilon)$. 

    $\exists U_a \iff \exists \delta > 0: U_a = (a - \delta, a + \delta)$.  

    $x \in \dot{U_a} \in E \iff x \in E \land x \in \dot{U_a} \iff 0 < |x-a|<\delta$.  

    $f(\ldots)\in U_A \iff |f(x) - A| < \varepsilon$.
\end{proof}
\begin{property}
    Определение предела --- локальное свойство. То есть, если $f$ и  $g$ совпадают в  $\dot{V_a}$, то либо оба предела не существуют, либо существуют и равны.
\end{property}
\begin{proof}
    $\lim_{x\to a} f(x) = A$.  $\forall U_A \, \exists U_a: f(\dot{U_a} \cap E) \subset U_A$. Возьмем $U_a \cap V_a$. Тогда все совпадет.
\end{proof}
\begin{property}
    Значение $f$ в точке  $a$ не участвует в определении.
\end{property}
\begin{property}
    В определении по Гейне. Если для любой последовательности $x_n \in E: x_n \to a$  $\lim f(x_n)$ существует, то все эти пределы равны.
\end{property}
\begin{proof}
    Пусть $x_n \in E x_n \to a$ и  $\lim f(x_n) = A$ и  $y_n \to E y_n \to a$ и  $\lim f(y_n) = B$.

    Рассмотрим  $z_n \coloneqq x_1, y_1, x_2, y_2,\ldots \Rightarrow z_n \to a \Rightarrow \lim f(z_n) \eqqcolon C$. Но $\{f(x_n)\}$ --- подпоследовательность  $\{f(z_n)\} \Rightarrow \lim f(x_n) = \lim f(z_n) = C$. Тоже самое для  $y_n$.
\end{proof}
\begin{theorem}
    Определение по Коши и по Гейне равносильны.
\end{theorem}
\begin{proof}
    \slashn
    \begin{itemize}
        \item $C \Rightarrow H$. $\forall \varepsilon > 0\, \exists \delta >0 \, \forall x \in E: |x-a| < \delta \Rightarrow |f(x) - A| < \varepsilon$. Пусть  $x_n \in E: \lim x_n = a$. Проверим, что  $\lim(x_n) = A$. Возьмем $\varepsilon > 0$, берем соответствующий  $\delta$ из определения. Найдется $N: \forall n \ge N: 0 \le \underbrace{|x_n-a|<\delta}_{\text{предел последовательности}} \Rightarrow |f(x_n) - A| < \varepsilon$. 
        \item $H \Rightarrow C$. От противного: нашелся  $\varepsilon > 0$ для которого ни одна  $\delta > 0$ не подходит. Возьмем  $\delta =\frac{1}{n}$. Она не подходит, то есть $\exists x \in E: 0 < |x-a| < \delta$, но  $|f(x) - A| \ge \varepsilon$. Получили $x_n$. 

            Посмотрим на последовательность:  $x_n \neq a \in E\; |x_n-a| < \frac{1}{n} \Rightarrow \lim x_n = a \Rightarrow \lim f(x_n) = A \Rightarrow |f(x_n) - A| < \varepsilon$. Противоречие. 
    \end{itemize}
\end{proof}
\slashn
Свойства пределов:
\begin{enumerate}
    \item Предел единственный.
    \item Если существует $\lim_{x\to a} f(x) = A$, то  $f$ локально ограничена, то есть существует  $U_a$,  $f$ в  $U_a$ ограничена.
    \item (Стабилизация знака). Если  $\lim_{x\to a} f(x) = A \neq 0$, то существует такая окрестность  $U_a$, что  $f(x)$ при  $x \in \dot{U_a}$ имеет тот же знак, что и  $A$.
\end{enumerate}
\begin{proof}
    \slashn
    \begin{enumerate}
        \item Пусть  $\lim_{x\to a} f(x) = A$ и  $\lim_{x\to a} f(x) = B$. Возьмем  $\lim x_n \in E$, такой, что  $x_n \to a$ (рассматриваем только предельные точки $E$). Тогда $\lim f(x_n) = A$ и  $\lim f(x_n) = B$, но предел последовательности единственен $\Rightarrow A=B$.  
        \item Возьмем $\varepsilon = 1$ в определении по Коши.  $\exists \delta > 0\, \forall x \in E 0 < |x-a| < \delta \Rightarrow |f(x) - A| < \varepsilon = 1$. $U_a = (a - \delta, a + \delta)$, тогда  $f$ ограничена на  $U_a \cap E$.  $|f(x)| \le |A| + |f(x) - A| < A + 1$. Аккуратно рассмотрим еще про $x = a$.
        \item Пусть $A > 0$. Возьмем  $\varepsilon = A$.  $\exists \delta > 0: 0 < |x-a| < \delta \land x \in E \Rightarrow |f(x) - A| < A \iff 0 < f(x) < 2A$. Берем  $U_a = (a-\delta, a+\delta)$ для нее значения  $>0$.
    \end{enumerate}
\end{proof}
\begin{theorem}[Теорема о арифметических действиях с пределами]
    Пусть $a$ --- предельная точка  $E$,  $f, g: E \to \R$ и  $\lim_{x\to a} f(x)=A$,  $\lim_{x\to a} g(x)=B$. Тогда 
    \begin{enumerate}
        \item $\lim_{x\to a} (f(x) \pm g(x)) = A \pm B$
        \item $\lim_{x\to a} f(x) \cdot g(x) = A \cdot B$
        \item $\lim_{x\to a} |f(x)| = |A|$
        \item $B \neq 0 \Rightarrow \lim_{x\to a} \frac{f(x)}{g(x)} = \frac{A}{B}$
    \end{enumerate}
\end{theorem}
\begin{proof}
    Проверим определение по Гейне. Берем последовательность $a \neq x_n \in E: \lim x_n = a$. Тогда  $\lim f(x_n) = A$  и  $\lim g(x_n) = B$. Следовательно  $\lim (f(x_n) \pm g(x_n)) = A \pm B \Rightarrow \lim (f(x) \pm g(x)) = A \pm B$.

    Второй и третий пункт доказывается ровно так же.

    Но вот в четвертом пункте надо что-то сказать про $g(x)$. Если  $\lim_{x\to a} g(x) = B \neq 0$, то по теореме о стабилизации знака $\exists \delta > 0: a \neq x_n \in E \land |x-a| < \delta \Rightarrow g(x) \neq 0$. Тогда для $x \in (a-\delta; a + \delta) \cap E$ можно писать  $\frac{f(x)}{g(g)}$.
\end{proof}

\begin{theorem}[О предельном переходе в неравенствах]
    Пусть $a$ --- предельная точка  $E$,  $f, g: E \to \R$ и $f(x) \le g(x)$. Если $\lim_{x\to a} f(x) = A$ и $\lim_{x \to a} g(x) = B$, то  $A \le B$.
\end{theorem}
\begin{proof}
    Возьмем какую-то последовательность $a \neq x_n \in E: \lim x_n = a$ (найдется, так как $a$ --- предельная точка  $E$). Тогда $A = \lim f(x_n)$ и  $B = \lim g(x_n)$.

    Тогда знаем, что  $\forall n: f(x_n) \le g(x_n) \xRightarrow[\text{для послед.}]{\text{пред. переход}} A \le B$
\end{proof}
\begin{theorem}[О двух милиционерах]
    Пусть $a$ --- предельная точка  $E$,  $f, g, h: E \to \R$ и  $f(x) \le g(x) \le h(x)$ при всех $x \in E$. Тогда, $\lim_{x\to a} f(x) = \lim_{x \to a} h(x) \eqqcolon A \Rightarrow \lim_{x\to a} g(x) = A$.
\end{theorem}
\begin{proof}
    Проверим определение по Гейне для $\lim_{x \to a} g(x) = A$. Берем любую последовательность  $a \neq x_n \in E: \lim x_n = a$. Тогда  $\lim f(x_n) = A \land \lim h(x_n) = A \land f(x_n) \le g(x_n) \le h(x_n) \xRightarrow[\text{для послед.}]{\text{Th. о 2 мил.}} \lim g(x_n) = A$.
\end{proof}
\begin{theorem}[Критерий Коши для предела функции]
    $a$ --- предельная точка  $E$,  $f: E \to \R$. Тогда  существует конечный  $\lim_{x\to a} f(x) \iff \forall \varepsilon > 0 \exists delta > 0 \forall x, y \in E:  \begin{array}{l} |x-a| < \delta \\ |y-a| < \delta\end{array} \Rightarrow |f(x) - f(y)| < \varepsilon$
\end{theorem}
\begin{proof}
    \slashn
    \begin{itemize}
        \item $\Rightarrow$. Пусть  $\lim_{x\to a} f(x) = A$. Тогда $\forall \varepsilon > 0 \exists \delta > 0 \begin{array}{l} \forall a \neq x \in E: |x-a| < \delta \Rightarrow |f(x) - A| < \frac{\varepsilon}{2} \\ \forall a \neq y \in E: |y-a| < \delta \Rightarrow |f(y) - A| < \frac{\varepsilon}{2} \end{array}$. 

            Тогда, если сложить получим $|f(x) - f(y)| \le |f(x) - A| + |A - f(y)| < \frac{\varepsilon}{2} + \frac{\varepsilon}{2} = \varepsilon$
        \item $\Leftarrow$. Докажем, что существует конечный  $\lim_{x \to a} f(x)$ по Гейне. Берем последовательно  $a \neq x_n \in E: \lim x_n = a$. Надо доказать, что $\lim f(x_n)$ существует и конечен. Для этого проверим, что $f(x_n)$ фундаментальная последовательность.

            Возьмем $\varepsilon > 0$ и соответствующую ему $\delta > 0$.  $\exists N: \forall n \ge N |x_n - a| < \delta$. Берем $m, n \ge N \begin{array}{l} |x_n - a| < \delta \\ |x_m- a| < \delta \end{array} \land a \neq x_n \in E \land a \neq x_m \in E \Rightarrow |f(x_n) - f(x_m)| < \varepsilon$, т.е. $f(x_т)$ --- фундаментальная последовательность  $\Rightarrow$ существует конечный  $\lim f(x)$.
    \end{itemize}
\end{proof}
\begin{definition}
    $f: E \to \R$,  $E_1 = (-\infty, a) \cap E$. Пусть  $a$ - предельная точка  $E_1$,  $g \coloneqq f$ на  $E_1$.  $\lim_{x\to a} g(x) = A$, то  $\lim_{x \to a-} f(x)= A$ ($\lim_{x \to a-0} f(x) = A$) --- предел слева в точке $A$.
\end{definition}
\begin{definition}
    $f: E \to \R$,  $E_2 = (a, +\infty) \cap E$. Пусть  $a$ - предельная точка  $E_2$,  $g \coloneqq f$ на  $E_2$.  $\lim_{x\to a} g(x) = A$, то  $\lim_{x \to a+} f(x)= A$ ($\lim_{x \to a+0} f(x) = A$) --- предел справа в точке $A$.
\end{definition}
\begin{remark}
    $A = \lim_{x \to a-} f(x) \iff \forall \varepsilon > 0 \, \exists \delta > 0 \, \forall x \in E: a-\delta < x < a \Rightarrow |f(x) - A| < \varepsilon$
\end{remark}
\begin{remark}
    $B = \lim_{x \to a+} f(x) \iff \forall \varepsilon > 0 \, \exists \delta > 0 \, \forall x \in E: a < x < a + \delta \Rightarrow |f(x) - B| < \varepsilon$
\end{remark}
\begin{remark}
    $\lim_{x\to a-} f(x) = \lim_{x \to a+} f(x) = \eqqcolon A \iff \lim_{x \to a} f(x) = A$
\end{remark}
\begin{definition}
    $f: E \to \R$. Тогда  $f$ --- монотонно возрастает $\Leftarrow \forall x, y \in E: x < y \Rightarrow f(x) \le f(y)$.
    Дальше бла-бла-бла.
\end{definition}
\begin{theorem}
    $f: E \to \R, E_1 \coloneqq (-\infty, a) \cap E$,  $a$ --- предельная точка  $E_1$. Тогда
     \begin{enumerate}
         \item Если $f$ монотонно возрастает и ограничена сверху, то существует конечный предел  $\lim_{x \to a-} f(x)$.
         \item Если $f$ монотонно убывает и ограничена снизу, то существует конечный предел  $\lim_{x \to a- f(x)}$.
    \end{enumerate}
\end{theorem}
\begin{remark}
    На самом деле в 1  $\lim_{x \to a-} f(x) = \sup_{x \in E_1} f(x)$, в  2  $\lim_{x \to a-} f(x) = \inf_{x \in E_1} f(x)$.
\end{remark}
\begin{proof}
    \slashn
    \begin{enumerate}
        \item $A \coloneqq \sup_{x \in E_1} f(x)$. Проверим, что  $\lim_{x \to a-} f(x) = A$. Возьмем  $\varepsilon > 0$. Тогда  $A - \varepsilon$ не верхняя граница $\{f(x): x \in E_1\} \Rightarrow$ найдется $x_0 \in E_1: f(x_0) > A - \varepsilon$. $\delta \coloneqq a - x_0 > 0$. Проверим, что он подходит. Возьмем  $x \in E: a - \delta = x_0 < x < a \Rightarrow f(x_0) \le A-\varepsilon < f(x) \le A < A + \varepsilon \Rightarrow |f(x) - A| < \varepsilon$.
    \end{enumerate}
\end{proof}
\Subsection{Непрерывные функции}
\begin{definition}
    $f: E \to \R$ и  $a \in E$.  $f$ непрерывна в точке  $a$, если  $a$ --- не предельная точка или  $a$ --- предельная точка и  $\lim_{x \to a} f(x) = f(a)$.
\end{definition}
\begin{definition}
    с  $\varepsilon$-- $\delta$. Тогда  $\forall \varepsilon > 0 \, \exists \delta > 0 \, \forall x \in E |x-a| < \delta \Rightarrow |f(x) = f(a)| < \varepsilon$
\end{definition}
\begin{definition}
    С окрестностями. $\forall U_{f(a)} \exists U_a f(U_a \cap E) \subset U_{f(a)}$
\end{definition}
\begin{definition}
    $\forall x_n \in E: \lim x_n = a \Rightarrow \lim f(x_n) = f(a)$.
\end{definition}
\begin{example}
    $f(x) = c$ --- непрерывна всегда.
\end{example}
\begin{example}
    $f(x) = x$ --- непрерывна всегда.
\end{example}
\begin{example}
    $f(x) = [x]$. Если  $a \not \in \Z \lim_{x\to a} [x] = n = [a]$. Иначе предела нет.
\end{example}
\begin{example}
    $f(x) = |\{x\} - \frac{1}{2}|$.

    $a \not \in \Z$ --- очевидно.
    $a = n \in \Z$:  $\lim_{x \to n+} |\{x\} - \frac{1}{2} | = \lim_{x \to n+} = |x-n-\frac{1}{2}| = \frac{1}{2} = f(n)$. $\lim_{x \to n-} |\{x\} - \frac{1}{2}| = \lim_{x\to n-} = |x-(n-1)-\frac{1}{2}| = \frac{1}{2} = f(n)$.

    Функция непрерывна!
\end{example}

\begin{theorem}
    $\exp x$ непрерывна во всех точках.
\end{theorem}
\begin{proof}
    $\lim_{x\to a} \exp x = \exp a$. Пусть  $h \coloneqq x - a \to 0$.  $\lim_{h \to 0} \exp(a + h) = \exp a$. Тогда надо доказать, что  $\lim_{h \to 0} \exp h = 1$. Заметим, что  $1 + h \le \exp h \le \frac{1}{1- h}$, при $|h| \le 1$. Значит, по 2 милиционерам $\exp h \to 1$.
\end{proof}
\begin{theorem}[Теорема об арифметических действиях с непрерывными функциями]
    $f, g: E \to \R, a \in E$,  $f, g$ непрерывны в  $a$.
    Тогда:
     \begin{enumerate}
         \item $f \pm g$ непрерывна   $a$.
         \item  $f \cdot g$ непрерывна в  $a$.
         \item  $|f|$ непрерывна в  $a$.
         \item если  $g(a) \neq 0$, то  $\frac{f}{g}$ непрерывна.
    \end{enumerate}
\end{theorem}
\begin{proof}
    Если $a$ не предельная, то очев. Иначе ссылаемся на арифм. действия с пределами.
\end{proof}
\begin{consequence}
    \slashn
   \begin{enumerate}
       \item Многочлены непрерывны во всех точках.
       \item Рациональные функции функции (т.е. отношение двух многочленов) непрерывны на всех области определения.
   \end{enumerate} 
\end{consequence}
\begin{proof}
    \slashn
    \begin{enumerate}
        \item $f(x) = c, g(x) = x \Rightarrow c x^k$ --- непрерывна $\Rightarrow$ многочлены непрерывны.
        \item  $\frac{P(x)}{Q(x)}$ непрерывна в точке $a$, если  $Q(a) \neq 0$.
    \end{enumerate}
\end{proof}
\begin{theorem}[О стабилизации знака]
    $f: E \to \R$,  $a \in E$,  $f$ --- непрерывна в  $a$ и  $f(a) \neq 0$. Тогда  $\exists U_a: \forall x \in U_a \cap E$ знак $f(x)$ совпадает с  $f(a)$.
\end{theorem}
\begin{proof}
    \slashn
    \begin{itemize}
        \item Точка не предельная. Берем окрестность только из $a$.
        \item Иначе ссылаемся на соответствующую теорему из условия.
    \end{itemize}
\end{proof}
\begin{theorem}
    Пусть $f, a$ такие же, как выше.  $g: D \to \R, \lim_{x \to a} f(x) = A$,  $g$ --- непрерывна в  $A$,  $D \supset f(E)$, тогда  $\lim_{x \to a} g(f(x)) = g(A)$.
\end{theorem}
\begin{proof}
    $g$ непрерывна в точке  $A \Rightarrow \forall \varepsilon > 0 \; \exists \delta > 0 \; \forall y \in D |y-A| < \delta \Rightarrow |g(y) - g(A)| < \varepsilon$. 

    Зафиксируем  $\varepsilon > 0$ и по нему возьмем  $\delta > 0$ и подставим его вместо  $\varepsilon$ в определение  $\lim_{x \to a} f(x) = A$. 

    $\forall \delta > 0 \; \exists \gamma > 0 \; \forall x \in E |x - a| < \gamma \Rightarrow |f(x)  - A| < \delta$. Подставим $y = f(x)$:  $|g(f(x)) - g(A)| < \varepsilon$.
\end{proof}
\begin{consequence}[Непрерывность композиции]
    $f, a, g$ бла-бла-бла.  $g \circ f$ непрерывна в  $a$.
\end{consequence}
\begin{proof}
    Если $a$ не предельная, то там неинтересно.

    Если предельная, то предыдущая теорема.
\end{proof}
\begin{remark}
    Без непрерывности $g$ неверно.  $f(x) = x \sin \frac{1}{x}$. $\lim_{x \to 0} f(x) = 0$
    $g(y) = \begin{cases} 0 & y = 0 \\ 1 & y \neq 0\end{cases}$.  $\lim_{y \to 0} g(y) = 1$.  Но  $\lim g(f(x))$ не существует.  $x_n = \frac{1}{\pi n}$, $f(x_n) = 0$.  $g(f(x_n)) = 0$.  $y_n = \frac{1}{2\pi n + \frac{\pi}{2}}$ $f(y_0) = y_n \neq 0$.
\end{remark}
\begin{theorem}
    $0 < x < \frac{\pi}{2} \Rightarrow \sin x < x < \tg x$.
\end{theorem}
\begin{consequence}
    $\forall x: |\sin x| \le |x|$
\end{consequence}
\begin{proof}
\slashn
    \begin{itemize}
        \item $0 < x < \frac{\pi}{2}$ --- уже было.
        \item $x > \frac{\pi}{2} \Rightarrow |\sin x| \le 1 < \frac{\i}{2} \le |x|$.
        \item При $x < 0 \Rightarrow |x| = |-x|$ и  $|\sin x| = |sin(-x)|$.
    \end{itemize}
\end{proof}
\begin{consequence}
\slashn
\begin{enumerate}
    \item $|\sin x - \sin y| \le |x-y|$
    \item $|\cos x - \cos y| \le |x-y|$
\end{enumerate}
\end{consequence}
\begin{proof}
    $|\sin x - \sin y = 2|\sin \frac{x-y}{2}||cos \frac{x+y}{2}| \le 2 |sin \frac{x-y}{2}| \le 2 |\frac{x-y}{2}| \le |x-y|$ 

    Второй так же.
\end{proof}
\begin{theorem}
    $\sin, \cos, \tg, \ctg$ --- непрерывны.
\end{theorem}


\Subsection{Предел функции}
\begin{definition}
    $a \in \R$, тогда  $U_a$ --- окрестность точки  $a$  $\Leftarrow U_a = (a-\varepsilon, a + \varepsilon)$.
\end{definition}
\begin{definition}
    $\dot{U_a} = U_a \setminus \{a\}$ --- выколотая окрестность.
\end{definition}
\begin{definition}
    $E \subset \R$ a --- предельная точка  $E$, если любая  $\dot{U_a}$ пересекается с  $E$.
\end{definition}
\begin{theorem}
    Следующие условия равносильны:
    \begin{enumerate}
        \item $a$ --- предельная точка  $E$.
        \item В любой  $U_a$ содержится бесконечное кол-во точек из  $E$.
        \item  $\exists \{a_n\}: \forall n: a_n \in E \land a_n \to a$. Более того, можно выбрать последовательность  $x_n \in E$ так, что  $|x_n - a| \downarrow 0$.
    \end{enumerate}
\end{theorem}
\begin{proof}
    \slashn
    \begin{itemize}
        \item $2 \Rightarrow 1$. $U_a \cap E$ содержит бесконечное число точек  $\Rightarrow$ хотя бы одна из них не  $a$ и тогда  $\dot{U_a} \cap E \neq \varnothing$.
        \item $3 \Rightarrow 2$. Берем  $x_n \neq a \in E: \lim x_n = a$. Возьмем  $U_a = (a-\varepsilon, a+\varepsilon)$.  $\exists N: \forall n \ge N\; x_n \in (x_n) \in U_a$.
        \item $1 \Rightarrow 3$. Возьмем $\varepsilon_1 = 1$:  $(a-1; a+1)$ содержит точку из $E \setminus \{a\}$. Назовем такую точку $x_i$.

            Возьмем  $\varepsilon_2 = \min\{\frac{1}{2}, |x_i - a|\} > 0: (a - \varepsilon_2; a + \varepsilon_2)$ содержит точку из $E \setminus \{a\}$. Назовем её  $x_2$.

            Возьмем  $\varepsilon_3 = \min\{\frac{1}{3}, |x_2 - a|\} > 0$ (заметим, что $|x_2 - a| < \varepsilon_2 < |x_1 - a|$). Тогда $(a-\varepsilon_3, a + \varepsilon_3)$ содержит точку из  $E \setminus \{a\}$.

            Получили  $|x_1-a| > |x_2 - a| > \ldots$ причем $|x_k - a| < \varepsilon_k = \min\{\frac{1}{k}, |x_{k-1} - a|\} \le \frac{1}{k} \to 0 \Rightarrow x_k - a \to 0 \Rightarrow x_k \to a$. 
    \end{itemize}
\end{proof}
\begin{definition}
    Пусть $a$ --- предельная точка  $E$.  $f: E \to \R$. Тогда  $A = \lim_{x\to a} f(x)$, если
     \begin{enumerate}
         \item По Коши. $\forall \varepsilon > 0\, \exists \delta >0 \, \forall x \in E: |x-a| < \delta \Rightarrow |f(x) - A| < \varepsilon$.
         \item Окрестности. $\forall U_A \exists U_a: f(\dot{U_a} \cap E) \subset U_A$.
         \item По Гейне. Для любой последовательности  $a \neq x_n \in E: \lim x_n = a \Rightarrow \lim f(x_n) = A$.
    \end{enumerate}
\end{definition}
\begin{proof}[Равносильность 1. и 2.]
    $\forall U_a \exists U_a: f(\dot{U_a} \cap E) \subset U_A$.  

    $\forall U_a \iff \forall \varepsilon > 0: U_A = (A - \varepsilon, A + \varepsilon)$. 

    $\exists U_a \iff \exists \delta > 0: U_a = (a - \delta, a + \delta)$.  

    $x \in \dot{U_a} \in E \iff x \in E \land x \in \dot{U_a} \iff 0 < |x-a|<\delta$.  

    $f(\ldots)\in U_A \iff |f(x) - A| < \varepsilon$.
\end{proof}
\begin{property}
    Определение предела --- локальное свойство. То есть, если $f$ и  $g$ совпадают в  $\dot{V_a}$, то либо оба предела не существуют, либо существуют и равны.
\end{property}
\begin{proof}
    $\lim_{x\to a} f(x) = A$.  $\forall U_A \, \exists U_a: f(\dot{U_a} \cap E) \subset U_A$. Возьмем $U_a \cap V_a$. Тогда все совпадет.
\end{proof}
\begin{property}
    Значение $f$ в точке  $a$ не участвует в определении.
\end{property}
\begin{property}
    В определении по Гейне. Если для любой последовательности $x_n \in E: x_n \to a$  $\lim f(x_n)$ существует, то все эти пределы равны.
\end{property}
\begin{proof}
    Пусть $x_n \in E x_n \to a$ и  $\lim f(x_n) = A$ и  $y_n \to E y_n \to a$ и  $\lim f(y_n) = B$.

    Рассмотрим  $z_n \coloneqq x_1, y_1, x_2, y_2,\ldots \Rightarrow z_n \to a \Rightarrow \lim f(z_n) \eqqcolon C$. Но $\{f(x_n)\}$ --- подпоследовательность  $\{f(z_n)\} \Rightarrow \lim f(x_n) = \lim f(z_n) = C$. Тоже самое для  $y_n$.
\end{proof}
\begin{theorem}
    Определение по Коши и по Гейне равносильны.
\end{theorem}
\begin{proof}
    \slashn
    \begin{itemize}
        \item $C \Rightarrow H$. $\forall \varepsilon > 0\, \exists \delta >0 \, \forall x \in E: |x-a| < \delta \Rightarrow |f(x) - A| < \varepsilon$. Пусть  $x_n \in E: \lim x_n = a$. Проверим, что  $\lim(x_n) = A$. Возьмем $\varepsilon > 0$, берем соответствующий  $\delta$ из определения. Найдется $N: \forall n \ge N: 0 \le \underbrace{|x_n-a|<\delta}_{\mathclap{\text{предел последовательности}}} \Rightarrow |f(x_n) - A| < \varepsilon$. 
        \item $H \Rightarrow C$. От противного: нашелся  $\varepsilon > 0$ для которого ни одна  $\delta > 0$ не подходит. Возьмем  $\delta =\frac{1}{n}$. Она не подходит, то есть $\exists x \in E: 0 < |x-a| < \delta$, но  $|f(x) - A| \ge \varepsilon$. Получили $x_n$. 

            Посмотрим на последовательность:  $x_n \neq a \in E\; |x_n-a| < \frac{1}{n} \Rightarrow \lim x_n = a \Rightarrow \lim f(x_n) = A \Rightarrow |f(x_n) - A| < \varepsilon$. Противоречие. 
    \end{itemize}
\end{proof}
\slashn
Свойства пределов:
\begin{enumerate}
    \item Предел единственный.
    \item Если существует $\lim_{x\to a} f(x) = A$, то  $f$ локально ограничена, то есть существует  $U_a$,  $f$ в  $U_a$ ограничена.
    \item (Стабилизация знака). Если  $\lim_{x\to a} f(x) = A \neq 0$, то существует такая окрестность  $U_a$, что  $f(x)$ при  $x \in \dot{U_a}$ имеет тот же знак, что и  $A$.
\end{enumerate}
\begin{proof}
    \slashn
    \begin{enumerate}
        \item Пусть  $\lim_{x\to a} f(x) = A$ и  $\lim_{x\to a} f(x) = B$. Возьмем  $\lim x_n \in E$, такой, что  $x_n \to a$ (рассматриваем только предельные точки $E$). Тогда $\lim f(x_n) = A$ и  $\lim f(x_n) = B$, но предел последовательности единственен $\Rightarrow A=B$.  
        \item Возьмем $\varepsilon = 1$ в определении по Коши.  $\exists \delta > 0\, \forall x \in E\!: 0 < |x-a| < \delta \Rightarrow |f(x) - A| < \varepsilon = 1$. $U_a = (a - \delta, a + \delta)$, тогда  $f$ ограничена на  $U_a \cap E$.  $|f(x)| \le |A| + |f(x) - A| < A + 1$. Аккуратно рассмотрим еще про $x = a$.
        \item Пусть $A > 0$. Возьмем  $\varepsilon = A$.  $\exists \delta > 0: 0 < |x-a| < \delta \land x \in E \Rightarrow |f(x) - A| < A \iff 0 < f(x) < 2A$. Берем  $U_a = (a-\delta, a+\delta)$ для нее значения  $>0$.
    \end{enumerate}
\end{proof}
\begin{theorem}[Теорема о арифметических действиях с пределами]
    Пусть $a$ --- предельная точка  $E$,  $f, g: E \to \R$ и  $\lim_{x\to a} f(x)=A$,  $\lim_{x\to a} g(x)=B$. Тогда 
    \begin{enumerate}
        \item $\lim_{x\to a} (f(x) \pm g(x)) = A \pm B$
        \item $\lim_{x\to a} f(x) \cdot g(x) = A \cdot B$
        \item $\lim_{x\to a} |f(x)| = |A|$
        \item $B \neq 0 \Rightarrow \lim_{x\to a} \frac{f(x)}{g(x)} = \frac{A}{B}$
    \end{enumerate}
\end{theorem}
\begin{proof}
    Проверим определение по Гейне. Берем последовательность $a \neq x_n \in E: \lim x_n = a$. Тогда  $\lim f(x_n) = A$  и  $\lim g(x_n) = B$. Следовательно  $\lim (f(x_n) \pm g(x_n)) = A \pm B \Rightarrow \lim (f(x) \pm g(x)) = A \pm B$.

    Второй и третий пункт доказывается ровно так же.

    Но вот в четвертом пункте надо что-то сказать про $g(x)$. Если  $\lim_{x\to a} g(x) = B \neq 0$, то по теореме о стабилизации знака $\exists \delta > 0: a \neq x_n \in E \land |x-a| < \delta \Rightarrow g(x) \neq 0$. Тогда для $x \in (a-\delta; a + \delta) \cap E$ можно писать  $\frac{f(x)}{g(g)}$.
\end{proof}

\begin{theorem}[О предельном переходе в неравенствах]
    Пусть $a$ --- предельная точка  $E$,  $f, g: E \to \R$ и $f(x) \le g(x)$. Если $\lim_{x\to a} f(x) = A$ и $\lim_{x \to a} g(x) = B$, то  $A \le B$.
\end{theorem}
\begin{proof}
    Возьмем какую-то последовательность $a \neq x_n \in E: \lim x_n = a$ (найдется, так как $a$ --- предельная точка  $E$). Тогда $A = \lim f(x_n)$ и  $B = \lim g(x_n)$.

    Тогда знаем, что  $\forall n: f(x_n) \le g(x_n) \xRightarrow[\text{для послед.}]{\text{пред. переход}} A \le B$
\end{proof}
\begin{theorem}[О двух милиционерах]
    Пусть $a$ --- предельная точка  $E$,  $f, g, h: E \to \R$ и  $f(x) \le g(x) \le h(x)$ при всех $x \in E$. Тогда, $\lim_{x\to a} f(x) = \lim_{x \to a} h(x) \eqqcolon A \Rightarrow \lim_{x\to a} g(x) = A$.
\end{theorem}
\begin{proof}
    Проверим определение по Гейне для $\lim_{x \to a} g(x) = A$. Берем любую последовательность  $a \neq x_n \in E: \lim x_n = a$. Тогда  $\lim f(x_n) = A \land \lim h(x_n) = A \land f(x_n) \le g(x_n) \le h(x_n) \xRightarrow[\text{для послед.}]{\text{Th. о 2 мил.}} \lim g(x_n) = A$.
\end{proof}
\begin{theorem}[Критерий Коши для предела функции]
    $a$ --- предельная точка  $E$,  $f: E \to \R$. Тогда  существует конечный  $\lim_{x\to a} f(x) \iff \forall \varepsilon > 0 \exists \delta > 0 \forall x, y \in E:  \begin{array}{l} |x-a| < \delta \\ |y-a| < \delta\end{array} \Rightarrow |f(x) - f(y)| < \varepsilon$
\end{theorem}
\begin{proof}
    \slashn
    \begin{itemize}
        \item $\Rightarrow$. Пусть  $\lim_{x\to a} f(x) = A$. Тогда $\forall \varepsilon > 0 \exists \delta > 0 \begin{array}{l} \forall a \neq x \in E: |x-a| < \delta \Rightarrow |f(x) - A| < \frac{\varepsilon}{2} \\ \forall a \neq y \in E: |y-a| < \delta \Rightarrow |f(y) - A| < \frac{\varepsilon}{2} \end{array}$. 

            Тогда, если сложить получим $|f(x) - f(y)| \le |f(x) - A| + |A - f(y)| < \frac{\varepsilon}{2} + \frac{\varepsilon}{2} = \varepsilon$
        \item $\Leftarrow$. Докажем, что существует конечный  $\lim_{x \to a} f(x)$ по Гейне. Берем последовательно  $a \neq x_n \in E: \lim x_n = a$. Надо доказать, что $\lim f(x_n)$ существует и конечен. Для этого проверим, что $f(x_n)$ фундаментальная последовательность.

            Возьмем $\varepsilon > 0$ и соответствующую ему $\delta > 0$.  $\exists N: \forall n \ge N |x_n - a| < \delta$. Берем $m, n \ge N \begin{array}{l} |x_n - a| < \delta \\ |x_m- a| < \delta \end{array} \land a \neq x_n \in E \land a \neq x_m \in E \Rightarrow |f(x_n) - f(x_m)| < \varepsilon$, т.е. $f(x_n)$ --- фундаментальная последовательность  $\Rightarrow$ существует конечный  $\lim f(x)$.
    \end{itemize}
\end{proof}
\begin{definition}
    $f: E \to \R$,  $E_1 = (-\infty, a) \cap E$. Пусть  $a$ - предельная точка  $E_1$,  $g \coloneqq f$ на  $E_1$.  $\lim_{x\to a} g(x) = A$, то  $\lim_{x \to a-} f(x)= A$ ($\lim_{x \to a-0} f(x) = A$) --- предел слева в точке $A$.
\end{definition}
\begin{definition}
    $f: E \to \R$,  $E_2 = (a, +\infty) \cap E$. Пусть  $a$ - предельная точка  $E_2$,  $g \coloneqq f$ на  $E_2$.  $\lim_{x\to a} g(x) = A$, то  $\lim_{x \to a+} f(x)= A$ ($\lim_{x \to a+0} f(x) = A$) --- предел справа в точке $A$.
\end{definition}
\begin{remark}
    $A = \lim_{x \to a-} f(x) \iff \forall \varepsilon > 0 \, \exists \delta > 0 \, \forall x \in E: a-\delta < x < a \Rightarrow |f(x) - A| < \varepsilon$
\end{remark}
\begin{remark}
    $B = \lim_{x \to a+} f(x) \iff \forall \varepsilon > 0 \, \exists \delta > 0 \, \forall x \in E: a < x < a + \delta \Rightarrow |f(x) - B| < \varepsilon$
\end{remark}
\begin{remark}
    $\lim_{x\to a-} f(x) = \lim_{x \to a+} f(x) \eqqcolon A \iff \lim_{x \to a} f(x) = A$
\end{remark}
\begin{definition}
    $f: E \to \R$. Тогда  $f$ --- монотонно возрастает $\Leftarrow \forall x, y \in E: x < y \Rightarrow f(x) \le f(y)$.
    Дальше бла-бла-бла.
\end{definition}
\begin{theorem}
    $f: E \to \R, E_1 \coloneqq (-\infty, a) \cap E$,  $a$ --- предельная точка  $E_1$. Тогда
     \begin{enumerate}
         \item Если $f$ монотонно возрастает и ограничена сверху, то существует конечный предел  $\lim_{x \to a-} f(x)$.
         \item Если $f$ монотонно убывает и ограничена снизу, то существует конечный предел  $\lim_{x \to a- f(x)}$.
    \end{enumerate}
\end{theorem}
\begin{remark}
    На самом деле в 1  $\lim_{x \to a-} f(x) = \sup_{x \in E_1} f(x)$, в  2  $\lim_{x \to a-} f(x) = \inf_{x \in E_1} f(x)$.
\end{remark}
\begin{proof}
    \slashn
    \begin{enumerate}
        \item $A \coloneqq \sup_{x \in E_1} f(x)$. Проверим, что  $\lim_{x \to a-} f(x) = A$. Возьмем  $\varepsilon > 0$. Тогда  $A - \varepsilon$ не верхняя граница $\{f(x): x \in E_1\} \Rightarrow$ найдется $x_0 \in E_1: f(x_0) > A - \varepsilon$. $\delta \coloneqq a - x_0 > 0$. Проверим, что он подходит. Возьмем  $x \in E: a - \delta = x_0 < x < a \Rightarrow f(x_0) \le A-\varepsilon < f(x) \le A < A + \varepsilon \Rightarrow |f(x) - A| < \varepsilon$.
    \end{enumerate}
\end{proof}
\Subsection{Непрерывные функции}
\begin{definition}
    $f: E \to \R$ и  $a \in E$.  $f$ непрерывна в точке  $a$, если  $a$ --- не предельная точка или  $a$ --- предельная точка и  $\lim_{x \to a} f(x) = f(a)$.
\end{definition}
\begin{definition}
    с  $\varepsilon$-- $\delta$. Тогда  $\forall \varepsilon > 0 \, \exists \delta > 0 \, \forall x \in E |x-a| < \delta \Rightarrow |f(x) = f(a)| < \varepsilon$
\end{definition}
\begin{definition}
    С окрестностями. $\forall U_{f(a)} \exists U_a f(U_a \cap E) \subset U_{f(a)}$
\end{definition}
\begin{definition}
    $\forall x_n \in E: \lim x_n = a \Rightarrow \lim f(x_n) = f(a)$.
\end{definition}
\begin{example}
    $f(x) = c$ --- непрерывна всегда.
\end{example}
\begin{example}
    $f(x) = x$ --- непрерывна всегда.
\end{example}
\begin{example}
    $f(x) = [x]$. Если  $a \not \in \Z \lim_{x\to a} [x] = n = [a]$. Иначе предела нет.
\end{example}
\begin{example}
    $f(x) = |\{x\} - \frac{1}{2}|$.

    $a \not \in \Z$ --- очевидно.
    $a = n \in \Z$:  $\lim_{x \to n+} |\{x\} - \frac{1}{2} | = \lim_{x \to n+} = |x-n-\frac{1}{2}| = \frac{1}{2} = f(n)$. $\lim_{x \to n-} |\{x\} - \frac{1}{2}| = \lim_{x\to n-} = |x-(n-1)-\frac{1}{2}| = \frac{1}{2} = f(n)$.

    Функция непрерывна!
\end{example}

\begin{theorem}
    $\exp x$ непрерывна во всех точках.
\end{theorem}
\begin{proof}
    $\lim_{x\to a} \exp x = \exp a$. Пусть  $h \coloneqq x - a \to 0$.  $\lim_{h \to 0} \exp(a + h) = \exp a$. Тогда надо доказать, что  $\lim_{h \to 0} \exp h = 1$. Заметим, что  $1 + h \le \exp h \le \frac{1}{1- h}$, при $|h| \le 1$. Значит, по 2 милиционерам $\exp h \to 1$.
\end{proof}
\begin{theorem}[Теорема об арифметических действиях с непрерывными функциями]
    $f, g: E \to \R, a \in E$,  $f, g$ непрерывны в  $a$.
    Тогда:
     \begin{enumerate}
         \item $f \pm g$ непрерывна   $a$.
         \item  $f \cdot g$ непрерывна в  $a$.
         \item  $|f|$ непрерывна в  $a$\.
         \item если  $g(a) \neq 0$, то  $\frac{f}{g}$ непрерывна.
    \end{enumerate}
\end{theorem}
\begin{proof}
    Если $a$ не предельная, то очев. Иначе ссылаемся на арифм. действия с пределами.
\end{proof}
\begin{consequence}
    \slashn
   \begin{enumerate}
       \item Многочлены непрерывны во всех точках.
       \item Рациональные функции функции (т.е. отношение двух многочленов) непрерывны на всех области определения.
   \end{enumerate} 
\end{consequence}
\begin{proof}
    \slashn
    \begin{enumerate}
        \item $f(x) = c, g(x) = x \Rightarrow c x^k$ --- непрерывна $\Rightarrow$ многочлены непрерывны.
        \item  $\frac{P(x)}{Q(x)}$ непрерывна в точке $a$, если  $Q(a) \neq 0$.
    \end{enumerate}
\end{proof}
\begin{theorem}[О стабилизации знака]
    $f: E \to \R$,  $a \in E$,  $f$ --- непрерывна в  $a$ и  $f(a) \neq 0$. Тогда  $\exists U_a: \forall x \in U_a \cap E$ знак $f(x)$ совпадает с  $f(a)$.
\end{theorem}
\begin{proof}
    \slashn
    \begin{itemize}
        \item Точка не предельная. Берем окрестность только из $a$.
        \item Иначе ссылаемся на соответствующую теорему из условия.
    \end{itemize}
\end{proof}
\begin{theorem}
    Пусть $f, a$ такие же, как выше.  $g: D \to \R, \lim_{x \to a} f(x) = A$,  $g$ --- непрерывна в  $A$,  $D \supset f(E)$, тогда  $\lim_{x \to a} g(f(x)) = g(A)$.
\end{theorem}
\begin{proof}
    $g$ непрерывна в точке  $A \Rightarrow \forall \varepsilon > 0 \; \exists \delta > 0 \; \forall y \in D |y-A| < \delta \Rightarrow |g(y) - g(A)| < \varepsilon$. 

    Зафиксируем  $\varepsilon > 0$ и по нему возьмем  $\delta > 0$ и подставим его вместо  $\varepsilon$ в определение  $\lim_{x \to a} f(x) = A$. 

    $\forall \delta > 0 \; \exists \gamma > 0 \; \forall x \in E |x - a| < \gamma \Rightarrow |f(x)  - A| < \delta$. Подставим $y = f(x)$:  $|g(f(x)) - g(A)| < \varepsilon$.
\end{proof}
\begin{consequence}[Непрерывность композиции]
    $f, a, g$ бла-бла-бла.  $g \circ f$ непрерывна в  $a$.
\end{consequence}
\begin{proof}
    Если $a$ не предельная, то там неинтересно.

    Если предельная, то предыдущая теорема.
\end{proof}
\begin{remark}
    Без непрерывности $g$ неверно.  $f(x) = x \sin \frac{1}{x}$. $\lim_{x \to 0} f(x) = 0$
    $g(y) = \begin{cases} 0 & y = 0 \\ 1 & y \neq 0\end{cases}$.  $\lim_{y \to 0} g(y) = 1$.  Но  $\lim g(f(x))$ не существует.  $x_n = \frac{1}{\pi n}$, $f(x_n) = 0$.  $g(f(x_n)) = 0$.  $y_n = \frac{1}{2\pi n + \frac{\pi}{2}}$ $f(y_0) = y_n \neq 0$.
\end{remark}
\begin{theorem}
    $0 < x < \frac{\pi}{2} \Rightarrow \sin x < x < \tg x$.
\end{theorem}
\begin{consequence}
    $\forall x: |\sin x| \le |x|$
\end{consequence}
\begin{proof}
\slashn
    \begin{itemize}
        \item $0 < x < \frac{\pi}{2}$ --- уже было.
        \item $x > \frac{\pi}{2} \Rightarrow |\sin x| \le 1 < \frac{\i}{2} \le |x|$.
        \item При $x < 0 \Rightarrow |x| = |-x|$ и  $|\sin x| = |sin(-x)|$.
    \end{itemize}
\end{proof}
\begin{consequence}
\slashn
\begin{enumerate}
    \item $|\sin x - \sin y| \le |x-y|$
    \item $|\cos x - \cos y| \le |x-y|$
\end{enumerate}
\end{consequence}
\begin{proof}
    $|\sin x - \sin y = 2|\sin \frac{x-y}{2}||cos \frac{x+y}{2}| \le 2 |sin \frac{x-y}{2}| \le 2 |\frac{x-y}{2}| \le |x-y|$ 

    Второй так же.
\end{proof}
\begin{theorem}
    $\sin, \cos, \tg, \ctg$ --- непрерывны.
\end{theorem}
\begin{proof}
    А где пруф???
\end{proof}

\begin{theorem}[Первый замечательный предел]
    \[
        \lim_{x\to 0} \frac{\sin x}{x} = 1
    .\] 
\end{theorem}
\begin{proof}
    Нам известно неравенство $\sin x \le x \le \tg x = \frac{\sin x}{\cos x}$, при $0 < x < \frac{\pi}{2}$. 

    Тогда $\sin x \le x \le \tg x \iff \frac{1}{\tg x} \le \frac{1}{x} \le \frac{1}{\sin x}$

    Тогда $\cos x \le \frac{\sin x}{x} \le 1$ при $x \in \left(\frac{\pi}{2}, \frac{\pi}{2}\right) \setminus \{0\}$.

    Тогда применим двух милиционеров: $\cos x \to 1$,  $1 \to 1$ при $x\to 0$, а значит  $\frac{\sin x}{x} \to 1$.
\end{proof}
\begin{theorem}[Теорема Вейерштрасса]
\slashn
\begin{enumerate}
    \item Непрерывная на отрезке функция ограничена.
    \item Непрерывная на отрезке функция принимает наибольшее и наименьшее значение.
\end{enumerate}
\end{theorem}
\begin{proof}
    Введем обозначения: $f: [a, b] \to \R$ непрерывна во всех точках.
    \begin{enumerate}
        \item От противного. Число $n$ не является верхней границей для  $|f|$. Значит  $\exists x_n \in [a, b]\!: |f(x_n)| > n$.  

            Применим теорему Больцано-Вейерштрасса. Выберем $x_{n_k}$ имеющий предел  $c \in \R$. Причем, так как  $a \le x_{n_k} \le b \Rightarrow a \le c \le b$. А значит, $f$ непрерывна в точке  $c$, тогда  $\displaystyle \lim_{x\to c}|f(x)| = |f(c)| \Rightarrow \lim_{k \to \infty} |f(x_{n_k})| = |f(c)|$, но $|f(x_{n_k}) > n_k \to +\infty$. Получили противоречие.  $+\infty \neq c \in \R$.
        \item Будем доказывать максимум.  $\displaystyle M \coloneqq \sup_{x \in [a ,b]} f(x) \in \R$. Пусть он не достигается, то есть $f(x) < M\ \forall x \in [a, b]$. 

            Тогда рассмотрим вспомогательную функцию $g(x) \coloneqq \frac{1}{M - f(x)}$ --- непрерывна на $[a, b] \Rightarrow g$ ограничена  $\Rightarrow g(x) \le \widehat{M}\ \forall x \in [a, b]$.

            Тогда $M - f(x) \ge \frac{1}{\widehat{M}} \Rightarrow f(x) \le M - \frac{1}{\widehat{M}} < M$. Получили противоречие, так как $M$ --- супремум.
    \end{enumerate}
\end{proof}
\begin{remark}
    Отрезок нельзя менять на интервал или полуинтервал. $f(x) = \frac{1}{x}$ на $(0, 1]$ --- непрерывность есть, ограниченности нет.
\end{remark}
\begin{remark}
    Непрерывность должна быть на всех точках. $f(x) = \begin{cases} \frac{1}{x} & x \in (0, 1] \\ 0 & x = 0\end{cases}$ на $[0, 1]$. Ограниченности нет, непрерывность во всех точках, кроме одной. 
\end{remark}
\begin{theorem}[Теорема Больцано-Коши, теорема о промежуточных значениях]
    \slashn
    \begin{enumerate}
        \item $f$ непрерывна на  $[a, b]$ и значение  $f(a)$ и  $f(b)$ разных знаков. Тогда  $\exists c \in (a, b)\!: f(c) = 0$.
        \item $f$ непрерывна на  $[a, b]$ и  $y$ лежит между  $f(a)$ и $f(b)$. Тогда  $\exists c \in (a, b)\!: f(c) = y$ 
    \end{enumerate}
\end{theorem}
\begin{proof}
    \slashn
    \begin{enumerate}
        \item Считаем, что $f(a) < 0 \land f(b) > 0$. Будем делить отрезок на части. $a_0 \coloneqq a, b_0 \coloneqq b$. Тогда: \[
        [a_{n+1}, b_{n+1}] = \begin{cases} 
            [\frac{a_n+b_n}{2}, b_n] & \text{если}\ f(\frac{a_n + b_n}{2}) < 0 \\
            [a_n, \frac{a_n+b_n}{2}] & \text{если}\ f(\frac{a_n + b_n}{2}) > 0
        \end{cases}
    .\] 
    Тогда, если процесс не оборвался, получили $[a, b] = [a_0,b_0] \supset [a_1, b_1] \subset [a_2, b_2] \supset \ldots$. Стягивающиеся отрезки $b_n - a_n = \frac{b-a}{2^n} \to 0$. Тогда найдется $c$, лежащая во всех отрезках, причем  $\lim a_n = \lim b_n = c$. Функция непрерывна в  $c \Rightarrow \lim \overbrace{f(a_n)}^{<0} = f(c) = \lim \overbrace{f(b_n)}^{>0} \Rightarrow 0 \ge \lim f(a_n) = f(c) \land \lim f(b_n) \ge 0 \Rightarrow f(c) = 0$.
\item $g(x) \coloneqq f(x) - y$. Тогда $g(a)$ и $g(b)$ разных знаков  $\Rightarrow \exists c \in [a, b]\!:f(c) - y = g(c) = 0$
        \end{enumerate}
\end{proof}
\begin{remark}
    Нужна непрерывность во всех точках. $f(x) = \begin{cases} 1 & \text{при}\ x \in [0, 1] \\ -1 & \text{при}\ x \in [-1;0) \end{cases}$ непрерывна на $[-1, 1]$, за исключением 0.
\end{remark}
\begin{remark}
    Бывают разрывные функции, удовлетворяющие теореме о промежуточных значениях. $f(x) = \begin{cases} \sin \frac{1}{x} & \text{при}\ x \in (0; 1] \\ 0 & \text{при}\ x = 0\end{cases}$. Если $a, b \in [0, 1]$, то на  $[a, b]$  $f$ принимает все значения между  $f(a)$ и  $f(b)$. Случай  $0 < a < b$ --- теорема Больцано-Коши. Случай  $a=0 < b$ на  $(0, b)$ принимает все значения. Так как на $[\frac{1}{2\pi(n+1)}; \frac{1}{2\pi n}]$ $\sin \frac{1}{x}$  принимает все значения от $-1$ до  $1$.
\end{remark}
\begin{theorem}
    Непрерывный образ отрезка --- отрезок. То есть $f\!: [a, b] \to \R$ непрерывная $\Rightarrow f([a, b])$ --- отрезок. 
\end{theorem}
\begin{proof}
    $\displaystyle m \coloneqq \min_{x \in [a, b]}f(x) \quad M \coloneqq \max_{x \in [a, b]}f(x)$. По теореме Вейерштрасса $\exists p, q \in [a, b]\!: m = f(p), M = f(q)$. Рассмотрим отрезок с концами в $p$ и  $q$ по теореме Больцано-Коши на  $(p, q)$ функция принимает все значения между  $f(p)$ и  $f(q)$, то есть все значения из $[m, M]$ достигаются  $\Rightarrow f([a, b]) = [m, M]$.
\end{proof}
\begin{definition}
    $\langle a, b \rangle$ означает один из промежутков  $[a, b]; (a, b]; [a, b); (a, b)$.
\end{definition}
\begin{theorem}
    Непрерывный образ промежутка --- промежуток (но, возможно, промежуток другого типа.)
\end{theorem}
\begin{proof}
    $\langle a, b \rangle$ --- промежуток.  $m \coloneqq \inf_{x \in \langle a, b \rangle} f(x)$ (возможно $-\infty$). $M \coloneqq \sup_{x \in \langle a, b \rangle} f(x)$.

    Тогда  $m \le f(x) \le M\ \forall x \in \langle a, b \rangle$. Докажем, что $f(\langle a, b \rangle) \supset (m; M)$, но при этом  $f(\langle a, b \rangle) \subset [m, M]$. 

    Возьмем  $y \in (m, M)$, такой что  $m < y < M$.  

    $m = \inf \Rightarrow$ $y$ ---  не нижняя граница $\Rightarrow \exists p \in \langle a, b \rangle\!: f(p) < y$. 

    $M = \sup \Rightarrow$ $y$ ---  не верхняя граница $\Rightarrow \exists p \in \langle a, b \rangle\!: f(q) > y$.

    Применим теорему Больцано-Коши для отрезка с концами $p$ и  $q \Rightarrow \exists c \in (p, q) \subset \langle a, b \rangle\!: f(c)=y$. 

    Тогда $(m, M) \subset f(\langle a, b \rangle) \subset [m, M] \Rightarrow f(\langle a, b \rangle)$ --- промежуток. 
\end{proof}
\begin{remark}
    $f(x) = x^2$ на  $(-1, 1)$, образ  $[0, 1)$.

    $f(x) = \sin x$ на  $(0, 2\pi)$, образ $[-1, 1]$.

    $f(x) = \frac{\sin \frac{1}{x}}{x}$ на $(0, 1]$, образ  $\R$.
\end{remark}

\begin{definition}
    Пусть $f\!: \langle a, b \rangle \to \R$ и инъективна. Тогда  $g$ --- обратная к  $f$  $\iff g\!: f(\langle a, b\rangle) \to \langle a, b \rangle $ и  $f(g(y)) = y \land g(f(x)) = x$.

    $g$ обозначатся как  $f^{-1}$.
\end{definition}
\begin{remark}
    Обратная функция существует, так как $f$ --- биекция между  $\langle a, b \rangle$ и  $f(\langle a, b \rangle)$.
\end{remark}
\begin{remark}
    График непрерывной функции симметричен относительно прямой $y=x$.
\end{remark}
\begin{theorem}
    $f\!: \langle a, b\rangle \to \R$ непрерывна и строго монотонна. $m \coloneqq \inf_{x \in \langle a, b \rangle} f(x)$ (возможно $-\infty$). $M \coloneqq \sup_{x \in \langle a, b \rangle} f(x)$. Тогда:
    \begin{enumerate}
        \item $f$ обратима и  $f^{-1}\!: \langle m, M \rangle \to \langle a, b \rangle$
        \item  $f^{-1}$ строго монотонна.
        \item  $f^{-1}$ непрерывна на  $\langle m, M \rangle$.
    \end{enumerate}
\end{theorem}
\begin{proof}
    \slashn
    \begin{enumerate}
        \item Строгая монотонность $\Rightarrow f$ --- инъекция  $\Rightarrow f$ --- обратима.
        \item  Пусть  $f$ строго возрастает  $\Rightarrow f(x) < f(y) \iff x < y$.  Тогда $x < y \Rightarrow f^{-1}(x) < f^{-1}(y)$.
        \item Непрерывность. Возьмем  $y_0 \in \langle m, M \rangle$. Докажем непрерывность в точке $y_0$.  $A \coloneqq \sup_{y < y_0} f^{-1}(y) = \lim_{y \to y_0-} f^{-1}(y)$ (функция строго монотонная) $\le f^{-1}(y_0) \le \lim_{y \to y_0+} f^{-1}(y) = \inf_{y > y_0} f^{-1}(y) \eqqcolon B$. Докажем, что $A = B$.

            Пусть $A < B$. Рассмотрим множество значений  $f^{-1}$ : $(-\infty; A] \cup \{f^{-1}(y_0)\} \cup [B; +\infty) \supset f^{-1}(\langle m, M \rangle) = \langle a, b \rangle$. Противоречие. 
    \end{enumerate}
\end{proof}

\Subsection{Элементарные функции}
\textbf{Обратные тригонометрические функции}.
\begin{itemize}
    \item $\sin\!: \left[-\frac{\pi}{2};\frac{\pi}{2}\right] \to [-1, 1]$ непрерывна и строго возрастает. По теории у него есть обратная функция.

\item$\arcsin\!:[-1, 1] \to [-\frac{\pi}{2}; \frac{\pi}{2}]$ непрерывна и строго возрастает.

\item$\cos\!:[0;\pi] \to [-1, 1]$ непрерывна и строго убывает.

\item$\arccos\!: [-1, 1] \to [0, \pi]$ непрерывна и строго убывает.

\item$\tg\!: \left(-\frac{\pi}{2};\frac{\pi}{2}\right) \to \R$ непрерывна и строго возрастает. По теории у него есть обратная функция.

\item$\arctg\!:\R \to (-\frac{\pi}{2}; \frac{\pi}{2})$ непрерывна и строго возрастает.

\item$\ctg\!:(0;\pi) \to \R$ непрерывна и строго убывает.

\item$\arcctg\!: \R \to (0, \pi)$ непрерывна и строго убывает.
\end{itemize}
\textbf{Логарифм.}

    $\exp\!: \R \to (0; +\infty)$ непрерывна и строго возрастает.
    Обратная функция $\ln\!:(0, +\infty) \to \R$ непрерывна и строго возрастает.
     \begin{properties}
         \begin{itemize}
             \item $\lim_{x\to 0+} \ln x = -\infty$, $\lim_{x\to +\infty} \ln x = +\infty$ 
             \item $\ln(ab) = \ln a + \ln b$
             \item  $\ln(1+x) \le x$, при $x > -1$.
             \item $\ln(1+x) \ge 1 - \frac{1}{1 + x}$, при $x> -1$.
         \end{itemize}
    \end{properties}
    \begin{proof}
        \begin{itemize}
            \item Предел существует из монотонности, они такие, т.е. множество значений у $\ln$ ---  $\R$.
            \item  $\exp(u+v) = \exp u \cdot \exp v$. Если  $\exp u = a, \exp v = b$, то  $u = \ln a, v = \ln b$, тогда  $\exp(u + v) = a \cdot b \Rightarrow \ln(ab) = u + v = \ln a + \ln b$
            \item  $\exp u \ge 1 + u \Rightarrow u = \ln(\exp u) \ge \ln(1+u)$
            \item $y \coloneqq \ln(1+x)$.  $\exp y = 1 + x \Rightarrow 1+x = \exp y \le \frac{1}{1 - y} \Rightarrow 1-y \le \frac{1}{1+x}$.
        \end{itemize}
    \end{proof}
\begin{theorem}
    \[
        \lim_{x \to 0} \frac{\ln(1+x)}{x} = 1
    .\] 
\end{theorem}
\begin{proof}
    $\frac{x}{1+x} = 1 - \frac{1}{1+x} \le \ln(1+x) \le x$, при $-1 < x < 1$.

    При  $x > 0$:  $\underbrace{\frac{1}{1 + x}}_{\to 1} \le \frac{\ln(1+x)}{x} \le 1$.

    При $x < 0$:  $1 \le \frac{\ln(1+x)}{x} \le \frac{1}{1+x} \to 1$
\end{proof}

\begin{definition}
    $a^b \coloneqq \exp(b \ln a)$, при  $a>0, b \in \R$
\end{definition}
\slashn
Если $b \in \N\!: a^n = \exp(\underbrace{\ln a + \ln a + \ldots + \ln a}_{n\ \text{штук}}) = \exp(\ln a) \cdot \ldots \exp(\ln a)$ --- штук.

$a^{-n} = \exp(-n \ln a) = \frac{1}{\exp(n \ln a)} = \frac{1}{a^n}$.

$a^{\frac{m}{n}} = (\sqrt[n]{a})^m$. $\exp(\frac{m}{n} \ln a) = \exp(m \frac{\ln a}{n}) = (\exp(\frac{\ln a}{n}))^m$.

\begin{exerc}
    Доказать, что $a^b = \lim a^b$, где  $b_n \in \Q$ и  $b_n \to b$.
\end{exerc}

\begin{consequence}
    \begin{equation}
         \lim_{x\to 0} (1+x)^{\frac{1}{x}} = e
     \end{equation}

    \begin{equation}
        \lim_{x\to +\infty} (1+\frac{1}{x})^{x} = \lim_{x\to -\infty} (1+\frac{1}{x})^{x} = e
     \end{equation}
\end{consequence}
\begin{proof}
    \begin{enumerate}
        \item $(1+x)^{\frac{1}{x}} = \exp(\frac{1}{x} \ln(1+x))$ 

            $\lim_{x\to 0}(1+x)^{\frac{1}{x}} = \lim_{x \to 0} \exp(\frac{\ln(1+x)}{x}) = \exp(\lim \ldots) = \exp 1 = e$ 
        \item $y = \frac{1}{x}$. $(1+x)^{\frac{1}{x}} = (1 + \frac{1}{y})^y$. При $x \to 0+$  $y \to +\infty$. При  $x \to 0-$  $y \to -\infty$
    \end{enumerate}
\end{proof}

\textbf{Показательная функция}
\begin{definition}
    Показательная функция $a^x \coloneqq \exp(x \ln a): \R \to (0, +\infty)$, при $a > 0$.
\end{definition}
\begin{properties}
    \begin{enumerate}
        \item При $a > 1$ строго возрастает. Так как $\ln a > 0$. 
        \item При $a < 1$ строго убывает.
        \item  $a^x \ge 1 + x \ln a$ по свойству экспоненты.
    \end{enumerate}
\end{properties}
\begin{theorem}
     \[
         \lim_{x \to 0} \frac{a^x - 1}{x} = \ln a
    .\] 
\end{theorem}
\begin{proof}
    \begin{align*}
        a^x \ge 1 + x \ln a \Rightarrow  a^x - 1 \ge x \ln a \\
        a^{-x} \ge 1 - x \ln a \Rightarrow a^x = \frac{1}{a^{-x}} \le \frac{1}{1 - x \ln a}\ \text{при малых}\ x \\
        a^x - 1 \le -1 + \frac{1}{1 - x \ln a} = \frac{x \ln a}{1 - x \ln a}.
        \end{align*}
        Тогда $\frac{a^x - 1}{x}$ зажато между $\ln a$ и  $\frac{\ln a}{1 - x \ln a} \xrightarrow[x \to 0]{} \ln a$. 
\end{proof}

\textbf{Степенная функция}
\begin{definition}
    Степенная функция: $x^p \coloneqq \exp(p \ln x)\!: (0, +\infty) \to \R$, когда  $p \in \R$.

    Функция непрерывна и, если  $p \neq 0$, то строго монотонная.
\end{definition}

\begin{theorem}
    \[
        \lim_{x\to 0} \frac{(1+x)^p - 1}{x} = p
    .\] 
\end{theorem}
\begin{proof}
    \begin{align*}
        (1+x)^p &= \exp(p \ln(1+x))  \ge 1 + p \ln(1+x) \\
        (1+x)^p &= \frac{1}{(1+x)^{-p}} \le \frac{1}{1 - p \ln (1+x)}, \text{при}\ x\ \text{близких к нулю}\\
        p \ln (1+x) &= (1+x)^p - 1 \le \frac{1}{1-p\ln(1+x)} - 1 = \frac{p\ln(1+x)}{1 - p \ln(1+x)}
    \end{align*}
    Тогда $\frac{(1+x)^p - 1}{x}$ зажато между $p \cdot \frac{\ln(1+x)}{x} \xrightarrow{\text{зам. предел}}p$ и $p \cdot \frac{\ln(1+x)}{x} \cdot \frac{1}{1 - p \ln (1+x)} \xrightarrow [\text{непрерывность}]{\text{зам. предел}} p$. Дальше два милиционера.
\end{proof}

\Subsection{Сравнение функций}
\begin{definition}
$f, g\!: E \to \R$.  $x_0$ --- предельная точка  $E$. Если $\exists \varphi\!:E \to \R\!: f = \varphi g$, при $x \in \dot{U}_{x_0} \cap E$ и 
     \begin{enumerate}
         \item $\varphi$ --- ограниченная:  $f = \mathcal{O}(g)$. $|f(x)| \le C|g(x)|$ в окрестности $x_0$.
         \item  $\varphi(x) \xrightarrow{x \to x_0} 0$:  $f = o(g)$. $\lim_{x \to x_0} \frac{f(x)}{g(x)} = 0$
         \item  $\varphi(x) \xrightarrow{x \to x_0} 1$:  $f \sim g$. $\lim_{x \to x_0} \frac{f(x)}{g(x)} = 1$
    \end{enumerate}
\end{definition}
\begin{definition}
    $f = \mathcal{O}(g)$ на множестве  $E$  $\xLeftrightarrow{\text{def}} \exists C > 0\!: |f(x)| \le C|g(x)|\ \forall x \in E$.
\end{definition}
\begin{definition}
    $f = \mathcal{O}(g) \quad f \prec g \quad g \succ f$. Если  $f = \mathcal{O}(g)$ и  $g = \mathcal{O}(f)$, то  $f \asymp g \iff \exists C_1, C_2 > 0\!: C_1|g(x)| \le |f(x)| \le C_2|g(x)|$.
\end{definition}
\begin{remark}
    $g(x) \neq 0 \Rightarrow \varphi(x) = \frac{f(x)}{g(x)}$ и $g(x) = 0 \Rightarrow f(x) = 0$ (иначе $\varphi(x)$ не существует.)
\end{remark}
\begin{properties}
    \begin{enumerate}
        \item $\sim$ --- отношение эквивалентности.
        \item  $f_1 \sim g_1 \land f_2 \sim g_2 \Rightarrow f_1f_2 \sim g_1g_2$.
        \item $f_2$ и  $g_2$ не обращаются в ноль в  $\dot{U_{x_0}} \Rightarrow f_1 \sim g_1 \land f_2 \sim g_2 \Rightarrow \frac{f_1}{f_2} \sim \frac{g_1}{g_2}$
        \item $f \sim g \iff f = g + o(g) \iff f = g + o(f)$.
    \end{enumerate}
\end{properties}
\begin{proof}
    \begin{enumerate}
        \item Рефлективность $f \sim f$:  $\varphi = 1$.

            Симметричность $f \sim g \Rightarrow g \sim f \quad f \sim g \Rightarrow f = \varphi g \Rightarrow g = \frac{1}{\varphi} f$, где $\lim_{x \to x_0} \varphi(x) = 1 \Rightarrow \lim_{x \to x_0} \frac{1}{\varphi(x)} = 1$.

            Транзитивность: $f \sim g \land g \sim h \Rightarrow f \sum h$.  $f = \varphi_1 g \land g = \varphi_2 h \Rightarrow = f \varphi_1 \varphi_2 h$. И еще пределы (очевидно).
        \item $f_i \sim g_i \Rightarrow f_i = \varphi_i g_i$, и  $\lim_{x \to x_0}\varphi_i = 1$. Можно перемножить  $\varphi_i$, все будет ок.
        \item  $\frac{f_1}{f_2} = \frac{\varphi_1}{\varphi_2} \frac{g_1}{g_2}$.
        \item $f \sim g \iff f = \varphi g \iff f = g + (\varphi  - 1) g$. Так как  $\varphi \to 1 \Rightarrow \varphi - 1 \to 0$.
    \end{enumerate}
\end{proof}
\begin{properties}
   \begin{enumerate}
       \setcounter{enumi}{5}
   \item $f = o(g) \Rightarrow f = \mathcal{O}(g)$ в точке  $x_0$.
   \item $f \sim g \Rightarrow f = \mathcal{O}(g)$ в точке $x_0$.
   \item  $f \cdot o(g) = o(fg)$
   \item  $o(f) + o(f) = o(f)$ и  $\mathcal{O}(f) + \mathcal{O}(f) = \mathcal{O}(f)$.
   \item $\lim_{x\to x_0} f(x) = a \iff f(x) = a + o(1)$
   \end{enumerate} 
\end{properties}
\begin{proof}
    \begin{enumerate}
        \setcounter{enumi}{5}
    \item $f = o(g) \Rightarrow f = \varphi g$, где  $\lim_{x \to x_0} \varphi = 0.$
        Для $\sim\ \lim_{x \to x_0} \varphi = 1 \Rightarrow \varphi$ ограничена в окрестности.
    \item $h = f \cdot o(g) \iff h = f \varphi g$, где  $\lim_{x \to x_0} \varphi = 0 \iff h = \varphi f g \iff h = o(fg)$.
    \item  $g = o(f), h = o(f) \Rightarrow g + h = o(f)$.  $g = \varphi f, h = \psi f$, Где  $\lim_{x \to x_0} \varphi = \lim_{x \to x_0} \psi = 0 \Rightarrow g+h = (\varphi + \psi)f$ и предел $=0$.f

        $g = \mathcal{O}(f) \Rightarrow |g| \le C|f| \land h = \mathcal{O}(f) \Rightarrow |h| \le C'|f| \Rightarrow |g+h| \le |g| + |h| \le (C + C') |F|$.
    \item $f(x) = a + o(1)$, где  $o(1)$ --- что-то, стремящееся к 0.  $\lim_{x \to x_0} (f(x) - a) = 0 \iff \lim_{x \to x_0} f(x) = a$
    \end{enumerate}
\end{proof}
\begin{example}
    $\sin x \sim x, \ln(1+x) \sim x, \tg x \sim x$ при $x \to 0$.
\end{example}
\begin{example}
    $\sin x = x + o(1)$.

    $\ln(1+x) = x + o(1)$
    
    $\tg x = x + o(x)$ 

    $\frac{(1+x)^p - 1}{x} \to p \iff \frac{(1+x)^p - 1}{x} = p+o(1) = (1+x)^p = 1 + px + o(1)$ 

    $\frac{a^x - 1}{x} \to \ln a \iff \frac{a^x - 1}{x} = \ln a + o(1) \iff a^x = 1 + x \ln a + o(x)$.

    $\frac{1-\cos x}{x^2} = \frac{2\sin^2 \frac{x}{2}}{x^2} \sim \frac{2\left(\frac{x}{2}\right)^2}{x^2} = \frac{1}{2}$. Получается, что $\frac{1 - \cos x}{x^2} = \frac{1}{2} + o(1)$. А значит $1-\cos x = \frac{x^2}{2} + o(x^2) \iff \cos x = 1 - \frac{x^2}{2} + o(x^2)$.
\end{example}
